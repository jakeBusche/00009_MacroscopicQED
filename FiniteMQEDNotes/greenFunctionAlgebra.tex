%!TEX root = finiteMQED.tex

\section{Helmholtz Expansion of the Dyadic Green's Function}

\subsection{Helmholtz expansion of Dirac-delta dyads}\label{sec:diracDyadExpansion}

Beginning with the ``self-field'' term of the dyadic Green's function, $\hat{\mathbf{r}}\hat{\mathbf{r}}\delta(\mathbf{r} - \mathbf{r}')/k^2$, we can conduct an expansion similar to that of Appendix \ref{sec:truncatedHarmonicExpansion} to produce
\begin{equation}
\begin{split}
\frac{\hat{\mathbf{r}}\hat{\mathbf{r}}}{k^2}\delta(\mathbf{r} - \mathbf{r}') &= \int\frac{\hat{\mathbf{s}}\hat{\mathbf{s}}}{k^2}\delta(\mathbf{s} - \mathbf{r}')\cdot\bm{1}_2\delta(\mathbf{s} - \mathbf{r})\;\mathrm{d}^3\mathbf{s}\\
&= \int\frac{\hat{\mathbf{s}}\hat{\mathbf{s}}}{k^2}\delta(\mathbf{s} - \mathbf{r}')\cdot\left[\frac{1}{2\pi^2}\sum_{\bm{\alpha}}\int_0^\infty\mathbf{X}_{\bm{\alpha}}(\mathbf{s},\kappa)\mathbf{X}_{\bm{\alpha}}(\mathbf{r},\kappa)\kappa^2\;\mathrm{d}\kappa\right.\\
&\qquad\left.+\frac{1}{4\pi}\sum_{\bm{\alpha}}\nabla_{\mathbf{s}}\nabla\left\{f_{\bm{\alpha}}^>(\mathbf{s})f_{\bm{\alpha}}^<(\mathbf{r})\Theta(s - r) + f_{\bm{\alpha}}^<(\mathbf{s})f_{\bm{\alpha}}^>(\mathbf{r})\Theta(r - s)\right\} \vphantom{\int_0^\infty} \right]\mathrm{d}^3\mathbf{s}\\
&= \frac{1}{2\pi^2k^2}\sum_{\bm{\alpha}}\int_0^\infty\mathbf{X}_{\bm{\alpha}}(\mathbf{r},\kappa)\left[\hat{\mathbf{r}}'\cdot\mathbf{X}_{\bm{\alpha}}(\mathbf{r}',\kappa)\right]\hat{\mathbf{r}}'\kappa^2\;\mathrm{d}\kappa\\
&\qquad+\frac{1}{4\pi k^2}\sum_{\bm{\alpha}}\nabla\left\{\frac{\partial f_{\bm{\alpha}}^>(\mathbf{r}')}{\partial r'}f_{\bm{\alpha}}^<(\mathbf{r})\Theta(r' - r) + \frac{\partial f_{\bm{\alpha}}^<(\mathbf{r}')}{\partial r'}f_{\bm{\alpha}}^>(\mathbf{r})\Theta(r - r')\right\}\hat{\mathbf{r}}'.
\end{split}
\end{equation}







\subsection{Finite radial integrals of products of spherical Bessel functions}\label{sec:finiteRadialIntegrals}

Letting $z_\ell(x)$ be a spherical Bessel ($j_\ell(x)$, $y_\ell(x)$) or Hankel ($h_\ell^{(1)}(x)$, $h_\ell^{(2)}(x)$) function, one can see that the Sturm-Liouville equation
\begin{equation}
\frac{\mathrm{d}}{\mathrm{d}r}\left\{r^2\frac{\mathrm{d}z_\ell(kr)}{\mathrm{d}r}\right\} - \ell(\ell + 1)z_\ell(kr) = -k^2r^2z_\ell(kr). 
\end{equation} 
One can then see that a function
\begin{equation}
v_\ell(r) = r^2\left[\frac{\mathrm{d}z_{1\ell}(kr)}{\mathrm{d}r}z_{2\ell}(k'r) - z_{1\ell}(kr)\frac{\mathrm{d}z_{2\ell}(k'r)}{\mathrm{d}r}\right]
\end{equation}
has a derivative
\begin{equation}
\begin{split}
\frac{\mathrm{d}v_\ell(r)}{\mathrm{d}r} &= \frac{\mathrm{d}}{\mathrm{d}r}\left\{r^2\frac{\mathrm{d}z_{1\ell}(kr)}{\mathrm{d}r}\right\}z_{2\ell}(k'r) - z_{1\ell}(kr)\frac{\mathrm{d}}{\mathrm{d}r}\left\{r^2\frac{\mathrm{d}z_{2\ell}(k'r)}{\mathrm{d}r}\right\}\\
&= r^2(k'^2 - k^2)z_{1\ell}(kr)z_{2\ell}(k'r)
\end{split}
\end{equation}
where $z_{1\ell}$ and $z_{2\ell}$ can be any two spherical Bessel or Hankel functions. Therefore,
\begin{equation}
\begin{split}
\int_a^bz_{1\ell}(kr)z_{2\ell}(k'r)r^2 &= \left.\frac{1}{k'^2 - k^2}v_\ell(r)\right|_a^b\\
&= \left.\frac{r^2}{k'^2 - k^2}\left[\frac{\mathrm{d}z_{1\ell}(kr)}{\mathrm{d}r}z_{2\ell}(k'r) - z_{1\ell}(kr)\frac{\mathrm{d}z_{2\ell}(k'r)}{\mathrm{d}r}\right]\right|_a^b.
\end{split}
\end{equation}
We can use the identity 
\begin{equation}
\frac{\mathrm{d}z_\ell(kr)}{\mathrm{d}r} = k\left[z_{\ell-1}(kr) - \frac{\ell + 1}{kr}z_\ell(kr)\right]
\end{equation}
to see that
\begin{equation}
\begin{split}
\frac{\mathrm{d}z_{1\ell}(kr)}{\mathrm{d}r}z_{2\ell}(k'r) - z_{1\ell}(kr)\frac{\mathrm{d}z_{2\ell}(k'r)}{\mathrm{d}r} &= k\left[z_{1,\ell-1}(kr) - \frac{\ell + 1}{kr}z_{1\ell}(kr)\right]z_{2\ell}(k'r)\\
&\qquad- z_{1\ell}(kr)k'\left[z_{2,\ell-1}(k'r) - \frac{\ell + 1}{k'r}z_{2\ell}(k'r)\right]\\[0.5em]
&= kz_{1,\ell-1}(kr)z_{2\ell}(k'r) - k'z_{2,\ell-1}(k'r)z_{1\ell}(kr).
\end{split}
\end{equation}
Therefore, we can simplify our integral to
\begin{equation}
\int_a^bz_{1\ell}(kr)z_{2\ell}(k'r)r^2 = \left.\frac{r^2}{k'^2 - k^2}\left[kz_{1,\ell - 1}(kr)z_{2\ell}(k'r) - k'z_{2,\ell - 1}(k'r)z_{1\ell}(kr)\right]\right|_a^b.
\end{equation}








\subsection{Helmholtz expansion of piecewise-truncated vector spherical harmonics}\label{sec:truncatedHarmonicExpansion}

A useful mathematical exercise that appears in the treatment of quantum optical problems using dyadic Green's functions in spherical coordinates is the expansion of functions of the types
\begin{equation}
\begin{split}
&\mathbf{X}_{\bm{\alpha}}(\mathbf{r},k)\Theta(\pm r\mp r'),\\
&\bm{\mathcal{X}}_{\bm{\alpha}}(\mathbf{r},k)\Theta(\pm r\mp r')
\end{split}
\end{equation}
into separated products of functions of the variables $r$ and $r'$. To do so, we can use the expansion of the dyadic Dirac delta $\bm{1}_2\delta(\mathbf{r} - \mathbf{r}')$ from Appendix \ref{app:helmholtzDelta}. This allows us to say
\begin{equation}\label{eq:truncatedVectorHarmonicExpansion1}
\begin{split}
\mathbf{X}_{\bm{\alpha}}(\mathbf{r},k)\Theta(r - r') &= \int\mathbf{X}_{\bm{\alpha}}(\mathbf{s},k)\Theta(s - r')\cdot\bm{1}_2\delta(\mathbf{s} - \mathbf{r})\;\mathrm{d}^3\mathbf{s}\\
&= \int\mathbf{X}_{\bm{\alpha}}(\mathbf{s},k)\Theta(s - r')\cdot\left[\frac{1}{2\pi^2}\int_0^\infty\sum_{\bm{\beta}}\kappa^2\mathbf{X}_{\bm{\beta}}(\mathbf{s},\kappa)\mathbf{X}_{\bm{\beta}}(\mathbf{r},\kappa)\;\mathrm{d}\kappa\right.\\
&\qquad\left.+ \frac{1}{4\pi}\sum_{\bm{\beta}}\nabla_\mathbf{r}\nabla_\mathbf{s}\left\{ f_{\bm{\beta}}^>(\mathbf{s})f_{\bm{\beta}}^<(\mathbf{r})\Theta(s - r) +  f_{\bm{\beta}}^<(\mathbf{s})f_{\bm{\beta}}^>(\mathbf{r})\Theta(r - s)\right\}\right]\mathrm{d}^3\mathbf{s}.\\
\end{split}
\end{equation}
We can evaluate the projection integrals using the orthogonality identities of Eqs. \eqref{eq:vectorSphericalHarmonicOrthogonalityConditionFinite} and \eqref{eq:angularMixedOrthogonalityNf} as well as the generalizations of the product rule and divergence theorem
\begin{equation}\label{eq:generalizedProductRule}
g(\mathbf{r})\nabla\cdot\left\{\mathbf{F}(\mathbf{r})f(\mathbf{r})\mathbf{c}\right\} = \left[\mathbf{F}(\mathbf{r})\cdot\nabla \left\{f(\mathbf{r})\mathbf{c}\right\} + \nabla\cdot\mathbf{F}(\mathbf{r})f(\mathbf{r})\mathbf{c}\right]g(\mathbf{r})
\end{equation}
and
\begin{equation}\label{eq:generalizedDivergenceTheorem}
\begin{split}
\int_\mathbb{V}\nabla\cdot\left\{\mathbf{F}(\mathbf{r})f(\mathbf{r})\mathbf{c}\right\}\;\mathrm{d}^3\mathbf{r} &= \mathbf{c}\int_\mathbb{V}\nabla\cdot\left\{\mathbf{F}(\mathbf{r})f(\mathbf{r})\right\}\;\mathrm{d}^3\mathbf{r}\\
&= \mathbf{c}\oint_{\partial\mathbb{V}} f(\mathbf{r})\mathbf{F}(\mathbf{r})\cdot\mathrm{d}^2\mathbf{a},
\end{split}
\end{equation} 
respectively. Explicitly, we can see that
\begin{equation}
\int\mathbf{X}_{\bm{\alpha}}(\mathbf{s},k)\cdot\mathbf{X}_{\bm{\beta}}(\mathbf{s},\kappa)\Theta(s - r')\;\mathrm{d}^3\mathbf{s} = 4\pi(1 - \delta_{p1}\delta_{m0})\delta_{\bm{\alpha}\bm{\beta}}R^\ll_{T\ell}(k,\kappa;r',\infty)
\end{equation}
such that the transverse part of the truncated vector harmonics are
\begin{equation}
\begin{split}
\left[\mathbf{X}_{\bm{\alpha}}(\mathbf{r},k)\Theta(r - r')\right]^\perp &= \int\mathbf{X}_{\bm{\alpha}}(\mathbf{s},k)\Theta(s - r')\cdot\frac{1}{2\pi^2}\sum_{\bm{\beta}}\int_0^\infty\kappa^2\mathbf{X}_{\bm{\beta}}(\mathbf{s},\kappa)\mathbf{X}_{\bm{\beta}}(\mathbf{r},\kappa)\;\mathrm{d}\kappa\,\mathrm{d}^3\mathbf{s}\\
&= \sum_{\bm{\beta}}\int_0^\infty \frac{\kappa^2}{2\pi^2}4\pi(1 - \delta_{p1}\delta_{m0})\delta_{\bm{\alpha}\bm{\beta}}R_{T\ell}^\ll(k,\kappa;r',\infty)\mathbf{X}_{\bm{\beta}}(\mathbf{r},\kappa)\;\mathrm{d}\kappa\\
&= \frac{2}{\pi}(1 - \delta_{p1}\delta_{m0})\int_0^\infty \kappa^2R_{T\ell}^\ll(k,\kappa;r',\infty)\mathbf{X}_{\bm{\alpha}}(\mathbf{r},\kappa)\;\mathrm{d}\kappa.
\end{split}
\end{equation}


The longitudinal parts are more difficult to calculate, but we can start by passing through the $\mathbf{r}$-gradient $\nabla_{\mathbf{r}}$ in eq. \eqref{eq:truncatedVectorHarmonicExpansion1} such that
\begin{equation}
\begin{split}
&\left[\mathbf{X}_{\bm{\alpha}}(\mathbf{r},k)\Theta(r - r')\right]^\parallel\\
&\quad = \int\mathbf{X}_{\bm{\alpha}}(\mathbf{s},k)\Theta(s - r')\cdot\frac{1}{4\pi}\sum_{\bm{\beta}}\nabla_{\mathbf{s}}\nabla_{\mathbf{r}}\left\{f_{\bm{\beta}}^>(\mathbf{s}) f_{\bm{\beta}}^<(\mathbf{r})\Theta(s - r) + f_{\bm{\beta}}^<(\mathbf{s})f_{\bm{\beta}}^>(\mathbf{r})\Theta(r - s)\right\}\mathrm{d}^3\mathbf{s}\\
&\quad = \int\mathbf{X}_{\bm{\alpha}}(\mathbf{s},k)\Theta(s - r')\cdot\frac{1}{4\pi}\sum_{\bm{\beta}}\nabla_{\mathbf{s}}\left\{f_{\bm{\beta}}^>(\mathbf{s})\nabla_{\mathbf{r}} f_{\bm{\beta}}^<(\mathbf{r})\Theta(s - r) + f_{\bm{\beta}}^<(\mathbf{s})\nabla_{\mathbf{r}}f^>(\mathbf{r})\Theta(r - s)\right\}\mathrm{d}^3\mathbf{s},
\end{split}
\end{equation}
wherein terms proportional to gradients of the Heaviside functions cancel as shown in Appendix \ref{app:helmholtzDelta}. From here, we can use the generalized product rule of Eq. \eqref{eq:generalizedProductRule} to replace terms that go like $\mathbf{X}_{\bm{\alpha}}(\mathbf{s},k)\cdot\nabla_{\mathbf{s}}\{f_{\bm{\beta}}^i(\mathbf{s})\nabla_{\mathbf{r}}f^j(\mathbf{r})\Theta(\pm r \mp s)\}$ with terms that involve divergences, such that
\begin{equation}
\begin{split}
\int&\mathbf{X}_{\bm{\alpha}}(\mathbf{s},k)\Theta(s - r')\cdot\frac{1}{4\pi}\sum_{\bm{\beta}}\nabla_{\mathbf{s}}\left\{f_{\bm{\beta}}^>(\mathbf{s})\nabla_{\mathbf{r}} f_{\bm{\beta}}^<(\mathbf{r})\Theta(s - r) + f_{\bm{\beta}}^<(\mathbf{s})\nabla_{\mathbf{r}}f_{\bm{\beta}}^>(\mathbf{r})\Theta(r - s)\right\}\mathrm{d}^3\mathbf{s}\\
&= \frac{1}{4\pi}\sum_{\bm{\beta}}\int\Theta(s - r')\nabla_{\mathbf{s}}\cdot\left\{\mathbf{X}_{\bm{\alpha}}(\mathbf{s},k)\left[f_{\bm{\beta}}^>(\mathbf{s})\nabla_{\mathbf{r}}f_{\bm{\beta}}^<(\mathbf{r})\Theta(s - r) + f_{\bm{\beta}}^<(\mathbf{s})\nabla_{\mathbf{r}}f_{\bm{\beta}}^>(\mathbf{r})\Theta(r - s)\right]\right\}\mathrm{d}^3\mathbf{s}\\
&\qquad- \frac{1}{4\pi}\sum_{\bm{\beta}}\int\Theta(s - r')\nabla_{\mathbf{s}}\cdot\mathbf{X}_{\bm{\alpha}}(\mathbf{s},k)\left[f_{\bm{\beta}}^>(\mathbf{s})\nabla_{\mathbf{r}}f_{\bm{\beta}}^<(\mathbf{r})\Theta(s - r) + f_{\bm{\beta}}^<(\mathbf{s})\nabla_{\mathbf{r}}f_{\bm{\beta}}^>(\mathbf{r})\Theta(r - s)\right]\mathrm{d}^3\mathbf{s}.
\end{split}
\end{equation}
Because the transverse vector spherical harmonics $\mathbf{X}_{\bm{\alpha}}(\mathbf{s},k)$ are transverse in $\mathbf{s}$, the second term on the right-hand side above is zero. We can then use the generalized divergence theorem of Eq. \eqref{eq:generalizedDivergenceTheorem} to replace the first term of the right-hand side with an integral across the boundaries of the region set by $\Theta(s - r')$:
\begin{equation}
\begin{split}
\frac{1}{4\pi}&\sum_{\bm{\beta}}\int\Theta(s - r')\nabla_{\mathbf{s}}\cdot\left\{\mathbf{X}_{\bm{\alpha}}(\mathbf{s},k)\left[f_{\bm{\beta}}^>(\mathbf{s})\nabla_{\mathbf{r}}f_{\bm{\beta}}^<(\mathbf{r})\Theta(s - r) + f_{\bm{\beta}}^<(\mathbf{s})\nabla_{\mathbf{r}}f_{\bm{\beta}}^>(\mathbf{r})\Theta(r - s)\right]\right\}\mathrm{d}^3\mathbf{s}\\
&= \frac{1}{4\pi}\sum_{\bm{\beta}}\oint_{\substack{\partial\mathbb{V}_{s\to r'}\\ + \partial\mathbb{V}_{s\to\infty}}}\mathrm{d}^2\mathbf{a}\cdot\mathbf{X}_{\bm{\alpha}}(\mathbf{s},k)\left[f_{\bm{\beta}}^>(\mathbf{s})\nabla_{\mathbf{r}}f_{\bm{\beta}}^<(\mathbf{r})\Theta(s - r) + f_{\bm{\beta}}^<(\mathbf{s})\nabla_{\mathbf{r}}f_{\bm{\beta}}^>(\mathbf{r})\Theta(r - s)\right]
\end{split}
\end{equation}
One of these boundaries ($\partial\mathbb{V}_{s\to\infty}$) is at infinity, at which the integrand above is zero. In more detail, $\Theta(r - s)$ is clearly zero for all finite $r$ and infinite $s$ as $\Theta(r - s) = 0$ for $s > r$. Additionally, it is zero at infinite $r$ due to the $\sim r^{-\ell_{\bm{\beta}} - 2}$ fall-off of $\nabla_{\mathbf{r}}f_{\bm{\beta}}^>(\mathbf{r})$ (see Eq. \eqref{eq:scalarHarmonics} for details). 

Therefore, the only surface integral that contributes to the total is the integral over $\partial\mathbb{V}_{s\to r'}$. Along this surface, $\mathrm{d}^2\mathbf{a} = -\hat{\mathbf{r}}'r'^2\sin\theta\mathrm{d}\theta\,\mathrm{d}\phi$ such that
\begin{equation}
\begin{split}
\left[\mathbf{X}_{\bm{\alpha}}(\mathbf{r},k)\Theta(r - r')\right]^\parallel &= 
-\frac{1}{4\pi}\sum_{\bm{\beta}} \int_0^{2\pi}\int_0^\pi\hat{\mathbf{r}}'\cdot\mathbf{X}_{\bm{\alpha}}(r',\theta,\phi;k) \left[f_{\bm{\beta}}^>(r',\theta,\phi)\nabla f_{\bm{\beta}}^<(\mathbf{r})\Theta(r' - r)\right.\\
&\qquad\left.+ f_{\bm{\beta}}^<(r',\theta,\phi)\nabla f_{\bm{\beta}}^>(\mathbf{r})\Theta(r - r')\right]r'^2\sin\theta\mathrm{d}\theta\,\mathrm{d}\phi.
\end{split}
\end{equation}
We have dopped the subscript $\mathbf{r}$ from the gradient symbol as the meaning of $\nabla$ now clearly implies differentiation in $\mathbf{r}$. Further, we can use the orthogonality condition of Eq. \eqref{eq:angularMixedOrthogonalityNf} to substitute
\begin{equation}
\begin{split}
\int_0^{2\pi}\int_0^\pi\hat{\mathbf{r}}'\cdot\mathbf{X}_{\bm{\alpha}}(r',\theta,\phi;k)f_{\bm{\beta}}^>(r',\theta,\phi)\sin\theta\;\mathrm{d}\theta\,\mathrm{d}\phi &= 4\pi(1 - \delta_{p1}\delta_{m0})\delta_{TE}\delta_{\bm{\alpha}\bm{\beta}}\sqrt{\frac{\ell(\ell + 1)}{2\ell + 1}}\frac{j_\ell(kr')}{kr'}\frac{1}{r'^{\ell + 1}},\\
\int_0^{2\pi}\int_0^\pi\hat{\mathbf{r}}'\cdot\mathbf{X}_{\bm{\alpha}}(r',\theta,\phi;k)f_{\bm{\beta}}^<(r',\theta,\phi)\sin\theta\;\mathrm{d}\theta\,\mathrm{d}\phi &= 4\pi(1 - \delta_{p1}\delta_{m0})\delta_{TE}\delta_{\bm{\alpha}\bm{\beta}}\sqrt{\frac{\ell(\ell + 1)}{2\ell + 1}}\frac{j_\ell(kr')}{kr'}r'^\ell,
\end{split}
\end{equation}
such that
\begin{equation}
\begin{split}
\left[\mathbf{X}_{\bm{\alpha}}(\mathbf{r},k)\Theta(r - r')\right]^\parallel &= -(1 - \delta_{p1}\delta_{m0})\delta_{TE}\sqrt{\frac{\ell(\ell + 1)}{2\ell + 1}}\frac{j_\ell(kr')}{kr'}\\
&\qquad\times\left[r'^{-\ell + 1}\nabla f_{\bm{\alpha}}^<(\mathbf{r})\Theta(r' - r) + r'^{\ell + 2}\nabla f_{\bm{\alpha}}^>(\mathbf{r})\Theta(r - r')\right]\\[0.5em]
&= -(1 - \delta_{p1}\delta_{m0})\delta_{TE}\sqrt{\frac{\ell(\ell + 1)}{2\ell + 1}}\frac{j_\ell(kr')}{kr'}\mathbf{Z}_{\bm{\alpha}}(\mathbf{r};r').
\end{split}
\end{equation}

Here, we have defined the regularized longitudinal spherical vector harmonics 
\begin{equation}\label{eq:regularizedLongitudinalHarmonics}
\begin{split}
\mathbf{Z}_{\bm{\alpha}}(\mathbf{r};s) &= \left[s^{-\ell + 1}\nabla f_{\bm{\alpha}}^<(\mathbf{r})\Theta(s - r) + s^{\ell + 2}\nabla f_{\bm{\alpha}}^>(\mathbf{r})\Theta(r - s)\right]\\
&= \nabla\left\{s^{-\ell + 1} f_{\bm{\alpha}}^<(\mathbf{r})\Theta(s - r) + s^{\ell + 2} f_{\bm{\alpha}}^>(\mathbf{r})\Theta(r - s)\right\}.
\end{split}
\end{equation}
These obey the orthogonality relation
\begin{equation}
\int\mathbf{Z}_{\bm{\alpha}}(\mathbf{r};s)\cdot\mathbf{Z}_{\bm{\beta}}(\mathbf{r};s)\;\mathrm{d}^3\mathbf{r} = 4\pi s^3\delta_{TE}(1 - \delta_{p1}\delta_{m0})\delta_{\bm{\alpha}\bm{\beta}}
\end{equation}
and, due to the null Laplacian of the scalar harmonics, $\nabla\cdot\nabla f_{\bm{\alpha}}^i(\mathbf{r}) = 0$, have the useful divergence
\begin{equation}
\begin{split}
\nabla\cdot\mathbf{Z}_{\bm{\alpha}}(\mathbf{r};s) &= \nabla\cdot\left\{a^{\ell + 2}\nabla f_{\bm{\alpha}}^>(\mathbf{r})\Theta(r - s) + s^{-\ell + 1}\nabla f_{\bm{\alpha}}^<(\mathbf{r})\Theta(s - r)\right\}\\[0.5em]
&= \left[s^{\ell + 2}\nabla f_{\bm{\alpha}}^>(\mathbf{r}) - s^{-\ell + 1}\nabla f_{\bm{\alpha}}^<(\mathbf{r})\right]\cdot\hat{\mathbf{r}}\delta(r - s)\\
&= \nabla\left\{s^{\ell + 2} f_{\bm{\alpha}}^>(\mathbf{r}) - s^{-\ell + 1} f_{\bm{\alpha}}^<(\mathbf{r})\right\}\cdot\hat{\mathbf{r}}\delta(r - s)
\end{split}
\end{equation}

In conclusion, we have
\begin{equation}
\begin{split}
&\mathbf{X}_{\bm{\alpha}}(\mathbf{r},k)\Theta(r - r') = \left[\mathbf{X}_{\bm{\alpha}}(\mathbf{r},k)\Theta(r - r')\right]^\perp + \left[\mathbf{X}_{\bm{\alpha}}(\mathbf{r},k)\Theta(r - r')\right]^\parallel\\
&\quad= \frac{2}{\pi}(1 - \delta_{p1}\delta_{m0})\int_0^\infty\kappa^2R_{T\ell}^\ll(k,\kappa;r',\infty)\mathbf{X}_{\bm{\alpha}}(\mathbf{r},\kappa)\;\mathrm{d}\kappa - (1 - \delta_{p1}\delta_{m0})\delta_{TE}\sqrt{\frac{\ell(\ell + 1)}{2\ell + 1}}\frac{j_\ell(kr')}{kr'}\mathbf{Z}_{\bm{\alpha}}(\mathbf{r};r').
\end{split}
\end{equation}
We can repeat this process for both truncated exterior transverse spherical harmonics and truncations using $\Theta(r' - r)$. The mathematics is cumbersome but straightforward, and results in
\begin{equation}
\begin{split}
\mathbf{X}_{\bm{\alpha}}(\mathbf{r},k)\Theta(r' - r) &= \frac{2}{\pi}(1 - \delta_{p1}\delta_{m0})\int_0^\infty\kappa^2R_{T\ell}^\ll(k,\kappa;0,r')\mathbf{X}_{\bm{\alpha}}(\mathbf{r},\kappa)\;\mathrm{d}\kappa\\
&\qquad + (1 - \delta_{p1}\delta_{m0})\delta_{TE}\sqrt{\frac{\ell(\ell + 1)}{2\ell + 1}}\frac{j_\ell(kr')}{kr'}\mathbf{Z}_{\bm{\alpha}}(\mathbf{r};r'),\\[1.0em]
\bm{\mathcal{X}}_{\bm{\alpha}}(\mathbf{r},k)\Theta(r - r') &= \frac{2}{\pi}(1 - \delta_{p1}\delta_{m0})\int_0^\infty\kappa^2R_{T\ell}^{><}(k,\kappa;r',\infty)\mathbf{X}_{\bm{\alpha}}(\mathbf{r},\kappa)\;\mathrm{d}\kappa\\
&\qquad- (1 - \delta_{p1}\delta_{m0})\delta_{TE}\sqrt{\frac{\ell(\ell + 1)}{2\ell + 1}}\frac{h_\ell(kr')}{kr'}\mathbf{Z}_{\bm{\alpha}}(\mathbf{r};r'),\\[1.0em]
\bm{\mathcal{X}}_{\bm{\alpha}}(\mathbf{r},k)\Theta(r' - r) &= \frac{2}{\pi}(1 - \delta_{p1}\delta_{m0})\int_0^\infty\kappa^2R_{T\ell}^{><}(k,\kappa;0,r')\mathbf{X}_{\bm{\alpha}}(\mathbf{r},\kappa)\;\mathrm{d}\kappa\\
&\qquad+ (1 - \delta_{p1}\delta_{m0})\delta_{TE}\sqrt{\frac{\ell(\ell + 1)}{2\ell + 1}}\frac{h_\ell(kr')}{kr'}\mathbf{Z}_{\bm{\alpha}}(\mathbf{r};r').
\end{split}
\end{equation}







\subsection{Helmholtz expansion of the dyadic Green's function of free space}\label{sec:helmholtzDecompositionFreeSpaceGreenFunction}

In this subsection, we provide the identity given in Eq. \eqref{eq:helmholtzDecompositionFreeSpaceGreenFunction}. Using the results of Appendices \ref{sec:truncatedHarmonicExpansion} and \ref{sec:diracDyadExpansion}, we can see that
\begin{equation}
\begin{split}
\mathbf{G}_0(\mathbf{r},\mathbf{r}';\omega) &= \frac{\mathrm{i}k}{4\pi}\sum_{\bm{\alpha}}\left[\bm{\mathcal{X}}_{\bm{\alpha}}(\mathbf{r},k)\mathbf{X}_{\bm{\alpha}}(\mathbf{r}',k)\Theta(r - r') + \mathbf{X}_{\bm{\alpha}}(\mathbf{r},k) \bm{\mathcal{X}}_{\bm{\alpha}}(\mathbf{r}',k)\Theta(r' - r)\right] - \frac{\hat{\mathbf{r}}\hat{\mathbf{r}}}{k^2}\delta(\mathbf{r} - \mathbf{r}')\\[1.0em]
&= \frac{\mathrm{i}k}{4\pi}\sum_{\bm{\alpha}}\left[\mathbf{X}_{\bm{\alpha}}(\mathbf{r}',k)\frac{2}{\pi}(1 - \delta_{p1}\delta_{m0})\int_0^\infty\kappa^2R_{T\ell}^{><}(k,\kappa;r',\infty)\mathbf{X}_{\bm{\alpha}}(\mathbf{r},\kappa)\;\mathrm{d}\kappa\right.\\
&\qquad- \mathbf{X}_{\bm{\alpha}}(\mathbf{r}',k)(1 - \delta_{p1}\delta_{m0})\delta_{TE}\sqrt{\frac{\ell(\ell + 1)}{2\ell + 1}}\frac{h_\ell(kr')}{kr'}\mathbf{Z}_{\bm{\alpha}}(\mathbf{r};r')\\
&\qquad+ \bm{\mathcal{X}}_{\bm{\alpha}}(\mathbf{r}',k)\frac{2}{\pi}(1 - \delta_{p1}\delta_{m0})\int_0^\infty\kappa^2R_{T\ell}^\ll(k,\kappa;0,r')\mathbf{X}_{\bm{\alpha}}(\mathbf{r},\kappa)\;\mathrm{d}\kappa\\
&\qquad+\left.\bm{\mathcal{X}}_{\bm{\alpha}}(\mathbf{r}',k)(1 - \delta_{p1}\delta_{m0})\delta_{TE}\sqrt{\frac{\ell(\ell + 1)}{2\ell + 1}}\frac{j_\ell(kr')}{kr'}\mathbf{Z}_{\bm{\alpha}}(\mathbf{r};r')\right]\\
&- \frac{1}{2\pi^2k^2}\sum_{\bm{\alpha}}\int_0^\infty\mathbf{X}_{\bm{\alpha}}(\mathbf{r},\kappa)\left[\hat{\mathbf{r}}'\cdot\mathbf{X}_{\bm{\alpha}}(\mathbf{r}',\kappa)\right]\hat{\mathbf{r}}'\kappa^2\;\mathrm{d}\kappa\\
&\qquad - \frac{1}{4\pi k^2}\sum_{\bm{\alpha}}\nabla\left\{\frac{\partial f_{\bm{\alpha}}^>(\mathbf{r}')}{\partial r'}f_{\bm{\alpha}}^<(\mathbf{r})\Theta(r' - r) + \frac{\partial f_{\bm{\alpha}}^<(\mathbf{r}')}{\partial r'}f_{\bm{\alpha}}^>(\mathbf{r})\Theta(r - r')\right\}\hat{\mathbf{r}}'.
\end{split}
\end{equation}
We can use the identity $\sum_{\bm{\alpha}}\delta_{TE}\mathbf{X}_{\bm{\alpha}}(\mathbf{r},k) = \sum_{p\ell m}\mathbf{N}_{p\ell m}(\mathbf{r},k)$ and group terms to rewrite this as
\begin{equation}
\begin{split}
\mathbf{G}_0(&\mathbf{r},\mathbf{r}';\omega) = \frac{\mathrm{i}k}{2\pi^2}\sum_{\bm{\alpha}}\int_0^\infty\mathbf{X}_{\bm{\alpha}}(\mathbf{r},\kappa)\left( \vphantom{\frac{1}{4\pi k^2}} \mathbf{X}_{\bm{\alpha}}(\mathbf{r}',k)R_{T\ell}^{><}(k,\kappa;r',\infty) + \bm{\mathcal{X}}_{\bm{\alpha}}(\mathbf{r}',k)R_{T\ell}^\ll(k,\kappa;0,r')\right.\\
&\left. - \frac{1}{\mathrm{i} k^3}\left[\hat{\mathbf{r}}'\cdot\mathbf{X}_{\bm{\alpha}}(\mathbf{r}',\kappa)\right]\hat{\mathbf{r}}'\right)\kappa^2\mathrm{d}\kappa + \frac{\mathrm{i}k}{4\pi}\sum_{\bm{\alpha}}\left(\delta_{TE}\sqrt{\frac{\ell(\ell + 1)}{2\ell + 1}}\mathbf{Z}_{\bm{\alpha}}(\mathbf{r},r')\left[-\frac{h_\ell(kr')}{kr'}\mathbf{N}_{p\ell m}(\mathbf{r}',k)\right.\right.\\
&\left.\left.+ \frac{j_\ell(kr')}{kr'}\bm{\mathcal{N}}_{p\ell m}(\mathbf{r}',k)\right] -\frac{1}{\mathrm{i} k^3}\nabla\left\{\frac{\partial f_{\bm{\alpha}}^>(\mathbf{r}')}{\partial r'}f_{\bm{\alpha}}^<(\mathbf{r})\Theta(r' - r) + \frac{\partial f_{\bm{\alpha}}^<(\mathbf{r}')}{\partial r'}f_{\bm{\alpha}}^>(\mathbf{r})\Theta(r - r')\right\}\hat{\mathbf{r}}' \vphantom{\sqrt{\frac{\ell(\ell + 1)}{2\ell + 1}}} \right).
\end{split}
\end{equation}
The first sum over $\bm{\alpha}$ (accompanied by a $\kappa$-integral) is a sum over functions that vary in the observer coordinate $\mathbf{r}$ like $\mathbf{X}_{\bm{\alpha}}(\mathbf{r},\kappa)$ while the second sum (notably integral-free) is a sum over functions that vary as $\nabla f_{\bm{\alpha}}^<(\mathbf{r})$ or $\nabla f_{\bm{\alpha}}^>(\mathbf{r})$, depending on the relative magnitudes of $r$ and $r'$. Therefore, it is clear that the first sum is divergence-free in $\mathbf{r}$ while the second is curl free. What is nontrivial to show, however, is that the first sum is also divergence-free in $\mathbf{r}'$ and the second is analogously also curl-free in $\mathbf{r}'$.

Showing that the $\mathbf{r}'$-behavior mirrors the $\mathbf{r}$ is important, as we must reproduce the Helmholtz decomposition
\begin{equation}\label{eq:waveEquationDerivativeHelmholtzDecomp}
\begin{split}
&\nabla\times\nabla\times\mathbf{A}(\mathbf{r},\omega) - \frac{\omega^2}{c^2}\mathbf{A}(\mathbf{r},\omega) = \frac{4\pi}{c}\mathbf{J}(\mathbf{r},\omega)\\[0.5em]
\implies&
\begin{dcases}
\nabla\times\nabla\times\mathbf{A}^\perp(\mathbf{r},\omega) - \frac{\omega^2}{c^2}\mathbf{A}^\perp(\mathbf{r},\omega) = \frac{4\pi}{c}\mathbf{J}^\perp(\mathbf{r},\omega),\\
\nabla\times\nabla\times\mathbf{A}^\parallel(\mathbf{r},\omega) - \frac{\omega^2}{c^2}\mathbf{A}^\parallel(\mathbf{r},\omega) = \frac{4\pi}{c}\mathbf{J}^\parallel(\mathbf{r},\omega)
\end{dcases}
\end{split}
\end{equation}
within the integral formulation of the wave equation. In other words, we require
\begin{equation}
\mathbf{A}(\mathbf{r},\omega) = \int\mathbf{G}_0(\mathbf{r},\mathbf{r}';\omega)\cdot\frac{\mathbf{J}(\mathbf{r}',\omega)}{c}\;\mathrm{d}^3\mathbf{r}'
\end{equation}
to imply that
\begin{equation}\label{eq:waveEquationIntegralHelmholtzDecomp}
\begin{split}
\mathbf{A}^\perp(\mathbf{r},\omega) &= \int\mathbf{G}_0^\perp(\mathbf{r},\mathbf{r}';\omega)\cdot\frac{\mathbf{J}^\perp(\mathbf{r}',\omega)}{c}\;\mathrm{d}^3\mathbf{r}',\\
\mathbf{A}^\parallel(\mathbf{r},\omega) &= \int\mathbf{G}_0^\parallel(\mathbf{r},\mathbf{r}';\omega)\cdot\frac{\mathbf{J}^\parallel(\mathbf{r}',\omega)}{c}\;\mathrm{d}^3\mathbf{r}',
\end{split}
\end{equation}
where $\mathbf{G}_0^{\perp,\parallel}(\mathbf{r},\mathbf{r}';\omega)$ are transverse and longitudinal in $\mathbf{r}'$, respectively, such that the integrals in the source coordinate are not null, and are also transverse and longitudinal in $\mathbf{r}$ such that the transverse (longitudinal) part of the vector potential can only be driven by the transverse (longitudinal) part of the current density.

The following proofs of these required properties are long and arduous but necessary. Beginning with the $\mathbf{r}$-transverse part of the Green's function of free space, we can use the product rule $\nabla\cdot\left\{\mathbf{c}\mathbf{F}(\mathbf{r})g(\mathbf{r})\right\} = \mathbf{c}\left[g(\mathbf{r})\nabla\cdot\mathbf{F}(\mathbf{r}) + \mathbf{F}(\mathbf{r})\cdot\nabla g(\mathbf{r})\right]$ and the fundamental theorem of calculus $(\partial/\partial x)\int_x^a f(y)\;\mathrm{d}y = -f(x)$ to say
\begin{equation}
\begin{split}
\nabla'\cdot&\left\{\mathbf{X}_{\bm{\alpha}}(\mathbf{r},\kappa)\mathbf{X}_{\bm{\alpha}}(\mathbf{r}',k)R_{T\ell}^{><}(k,\kappa;r',\infty)\right\}\\[0.5em]
&= \mathbf{X}_{\bm{\alpha}}(\mathbf{r},\kappa)\left[\nabla'\cdot\mathbf{X}_{\bm{\alpha}}(\mathbf{r}',k)R_{T\ell}^{><}(k,\kappa;r',\infty) + \mathbf{X}_{\bm{\alpha}}(\mathbf{r}',k)\cdot\nabla'R_{M\ell}^{><}(k,\kappa;r',\infty)\right]\\
&= \mathbf{X}_{\bm{\alpha}}(\mathbf{r},\kappa)\left[ \vphantom{\frac{\ell}{2\ell + 1}} 0 - \delta_{TM}\mathbf{M}_{p\ell m}(\mathbf{r}',k)\cdot\hat{\mathbf{r}}'h_\ell(kr')j_\ell(\kappa r')r'^2\right.\\
&\left.\qquad- \delta_{TE}\mathbf{N}_{p\ell m}(\mathbf{r}',k)\cdot\hat{\mathbf{r}}'\left(\frac{\ell + 1}{2\ell + 1}h_{\ell - 1}(kr')j_{\ell - 1}(\kappa r') + \frac{\ell}{2\ell + 1}h_{\ell + 1}(kr')j_{\ell + 1}(\kappa r')\right)r'^2\right]\\
&= -\mathbf{X}_{\bm{\alpha}}(\mathbf{r},\kappa)\delta_{TE}\left(\sqrt{K_{\ell m}}\frac{\ell(\ell + 1)}{2\ell + 1}\left[j_{\ell - 1}(kr') + j_{\ell + 1}(kr')\right]P_{\ell m}(\cos\theta')S_p(m\phi')\right)\\
&\qquad\times\left(\frac{\ell + 1}{2\ell + 1}h_{\ell - 1}(kr')j_{\ell - 1}(\kappa r') + \frac{\ell}{2\ell + 1}h_{\ell + 1}(kr')j_{\ell + 1}(\kappa r')\right)r'^2,
\end{split}
\end{equation}
wherein we have also used the fact that the magnetic-type transverse vector spherical harmonics have no radial component and that the electric-type transverse vector spherical harmonics have a radial component that goes like $j_{\ell}(kr')/kr' = [j_{\ell - 1}(kr') + j_{\ell + 1}(kr')]/(2\ell + 1)$. Similarly,
\begin{equation}
\begin{split}
\nabla'\cdot&\left\{\mathbf{X}_{\bm{\alpha}}(\mathbf{r},\kappa)\bm{\mathcal{X}}_{\bm{\alpha}}(\mathbf{r}',k)R_{T\ell}^\ll(k,\kappa;0,r')\right\}\\
&= \mathbf{X}_{\bm{\alpha}}(\mathbf{r},\kappa)\delta_{TE}\left(\sqrt{K_{\ell m}}\frac{\ell(\ell + 1)}{2\ell + 1}\left[h_{\ell - 1}(kr') + h_{\ell + 1}(kr')\right]P_{\ell m}(\cos\theta')S_p(m\phi')\right)\\
&\qquad\times\left(\frac{\ell + 1}{2\ell + 1}j_{\ell - 1}(kr')j_{\ell - 1}(\kappa r') + \frac{\ell}{2\ell + 1}j_{\ell + 1}(kr')j_{\ell + 1}(\kappa r')\right)r'^2.
\end{split}
\end{equation}
The third derivative identity we will need is the divergence of the term of the (putatively) transverse Green's function that arises from the dyadic Dirac delta. We can succinctly write this as
\begin{equation}
\begin{split}
\nabla'\cdot&\left\{-\frac{1}{\mathrm{i}k^3}\mathbf{X}_{\bm{\alpha}}(\mathbf{r},\kappa)\left[\hat{\mathbf{r}}'\cdot\mathbf{X}_{\bm{\alpha}}(\mathbf{r}',\kappa)\right]\hat{\mathbf{r}}'\right\}\\
&= \delta_{TE}\nabla'\cdot\left\{-\frac{1}{\mathrm{i}k^3}\mathbf{X}_{\bm{\alpha}}(\mathbf{r},\kappa)\left[\hat{\mathbf{r}}'\cdot\mathbf{N}_{p\ell m}(\mathbf{r}',\kappa)\right]\hat{\mathbf{r}}'\right\}\\
&= -\delta_{TE}\frac{1}{\mathrm{i}k^3}\mathbf{X}_{\bm{\alpha}}(\mathbf{r},\kappa)\frac{\partial}{\partial r'}\left\{\sqrt{K_{\ell m}}\frac{j_\ell(\kappa r')}{\kappa r'}P_{\ell m}(\cos\theta')S_p(m\phi')\right\}\\
&= \delta_{TE}\frac{\mathrm{i}}{k^3}\mathbf{X}_{\bm{\alpha}}(\mathbf{r},\kappa)\sqrt{K_{\ell m}}\ell(\ell + 1)P_{\ell m}(\cos\theta')S_p(m\phi')\left[\frac{1}{\kappa r'}\frac{\partial j_\ell(\kappa r')}{\partial r'} - \frac{j_\ell(\kappa r')}{\kappa r'^2}\right]\\
&= \delta_{TE}\mathrm{i}\mathbf{N}_{p\ell m}(\mathbf{r},\kappa)\sqrt{K_{\ell m}}\ell(\ell + 1)P_{\ell m}(\cos\theta')S_p(m\phi')\frac{1}{k^3\kappa r'^2}\left[r'\frac{\partial j_\ell(\kappa r')}{\partial r'} - j_\ell(\kappa r')\right]
\end{split}
\end{equation}
using the identity $\mathbf{X}_{\bm{\alpha}}(\mathbf{r}',\kappa)\cdot\hat{\mathbf{r}}' = \delta_{TE}\mathbf{N}_{p\ell m}(\mathbf{r}',\kappa)$. 

Combining the first two derivatives above and using the identity
\begin{equation}
j_{\ell - 1}(kr')h_{\ell + 1}(kr') - j_{\ell + 1}(kr')h_{\ell - 1}(kr') = -\mathrm{i}\frac{2\ell + 1}{(kr')^3},
\end{equation}
one finds, after distribution and reorganization of the products of spherical Bessel and Hankel functions, 
\begin{equation}
\begin{split}
\nabla'\cdot&\left\{\mathbf{X}_{\bm{\alpha}}(\mathbf{r},\kappa)\mathbf{X}_{\bm{\alpha}}(\mathbf{r}',k)R_{T\ell}^{><}(k,\kappa;r',\infty)\right\} + \nabla'\cdot\left\{\mathbf{X}_{\bm{\alpha}}(\mathbf{r},\kappa)\bm{\mathcal{X}}_{\bm{\alpha}}(\mathbf{r}',k)R_{T\ell}^\ll(k,\kappa;0,r')\right\}\\
&=\delta_{TE}\mathbf{X}_{\bm{\alpha}}(\mathbf{r},\kappa)\sqrt{K_{\ell m}}\frac{\ell(\ell + 1)}{2\ell + 1}P_{\ell m}(\cos\theta')S_p(m\phi')r'^2\\
&\qquad\times\left[j_{\ell - 1}(kr')h_{\ell + 1}(kr')\left(\frac{\ell + 1}{2\ell + 1}j_{\ell - 1}(\kappa r') - \frac{\ell}{2\ell + 1}j_{\ell + 1}(\kappa r')\right)\right.\\
&\qquad-\left.j_{\ell + 1}(kr')h_{\ell - 1}(kr')\left(\frac{\ell + 1}{2\ell + 1}j_{\ell - 1}(\kappa r') - \frac{\ell}{2\ell + 1}j_{\ell + 1}(\kappa r')\right)\right]\\
&= \delta_{TE}\mathbf{X}_{\bm{\alpha}}(\mathbf{r},\kappa)\sqrt{K_{\ell m}}\frac{\ell(\ell + 1)}{2\ell + 1}P_{\ell m}(\cos\theta')S_p(m\phi')r'^2\\
&\qquad\times\left[j_{\ell - 1}(kr')h_{\ell + 1}(kr') - j_{\ell + 1}(kr')h_{\ell - 1}(kr')\right]\frac{1}{\kappa r'}\frac{\partial\{r'j_\ell(\kappa r')\}}{\partial r'}\\
&= -\delta_{TE}\mathrm{i}\mathbf{X}_{\bm{\alpha}}(\mathbf{r},\kappa)\sqrt{K_{\ell m}}\ell(\ell + 1)P_{\ell m}(\cos\theta')S_p(m\phi')\frac{r'^2}{(kr')^3\kappa r'}\left[j_\ell(\kappa r') + r'\frac{\partial\{j_\ell(\kappa r')\}}{\partial r'}\right]
\end{split}
\end{equation}
Therefore, one can see that the sum of the three terms is
\begin{equation}
\begin{split}
\nabla'\cdot&\left\{ \vphantom{\frac{1}{k^3}} \mathbf{X}_{\bm{\alpha}}(\mathbf{r},\kappa)\mathbf{X}_{\bm{\alpha}}(\mathbf{r}',k)R_{T\ell}^{><}(k,\kappa;r',\infty)\right.\\
&\left.+ \mathbf{X}_{\bm{\alpha}}(\mathbf{r},\kappa)\bm{\mathcal{X}}_{\bm{\alpha}}(\mathbf{r}',k)R_{T\ell}^\ll(k,\kappa;0,r') - \frac{1}{\mathrm{i}k^3}\mathbf{X}_{\bm{\alpha}}(\mathbf{r},\kappa)\left[\hat{\mathbf{r}}'\cdot\mathbf{X}_{\bm{\alpha}}(\mathbf{r}',\kappa)\right]\hat{\mathbf{r}}'\right\}\\
&= \delta_{TE}\mathrm{i}\mathbf{X}_{\bm{\alpha}}(\mathbf{r},\kappa)\sqrt{K_{\ell m}}\ell(\ell + 1)P_{\ell m}(\cos\theta')S_p(m\phi')\frac{1}{k^3\kappa r'^2}\left[r'\frac{\partial j_\ell(\kappa r')}{\partial r'} - j_\ell(\kappa r') - j_\ell(\kappa r') - r'\frac{\partial\{j_\ell(\kappa r')\}}{\partial r'}\right]\\
&= -\delta_{TE}\frac{2\mathrm{i}}{k^3r'}\mathbf{X}_{\bm{\alpha}}(\mathbf{r},\kappa)\sqrt{K_{\ell m}}\ell(\ell + 1)\frac{j_\ell(\kappa r')}{\kappa r'}P_{\ell m}(\cos\theta')S_p(m\phi')\\
&= -\delta_{TE}\frac{2\mathrm{i}}{k^3r'}\mathbf{N}_{p\ell m}(\mathbf{r},\kappa)\mathbf{N}_{p\ell m}(\mathbf{r}',\kappa)\cdot\hat{\mathbf{r}}'.
\end{split}
\end{equation}
Further, from Appendices \ref{app:helmholtzDelta} and \ref{sec:diracDyadExpansion}, we can see that
\begin{equation}
\begin{split}
\hat{\mathbf{r}}'\delta(\mathbf{r} - \mathbf{r}') &= \left[\hat{\mathbf{r}}'\hat{\mathbf{r}}'\delta(\mathbf{r} - \mathbf{r}')\right]\cdot\hat{\mathbf{r}}'\\
&= \left(\left[\bm{1}_2\delta(\mathbf{r} - \mathbf{r}')\right]_\perp\cdot\hat{\mathbf{r}}'\hat{\mathbf{r}}'\right)\cdot\hat{\mathbf{r}}'\\
&= \left(\frac{1}{2\pi^2}\int_0^\infty\sum_{\bm{\alpha}}k^2\mathbf{X}_{\bm{\alpha}}(\mathbf{r},k)\mathbf{X}_{\bm{\alpha}}(\mathbf{r}',k)\;\mathrm{d}k\cdot\hat{\mathbf{r}}'\hat{\mathbf{r}}'\right)\cdot\hat{\mathbf{r}}'\\
&= \frac{1}{2\pi^2}\int_0^\infty\sum_{p\ell m}\mathbf{N}_{p\ell m}(\mathbf{r},k)\mathbf{N}_{p\ell m}(\mathbf{r}',k)\cdot\hat{\mathbf{r}}'k^2\;\mathrm{d}k,
\end{split}
\end{equation}
such that we can see that the $\mathbf{r}'$-divergence of the $\mathbf{r}$-transverse part of $\mathbf{G}_0(\mathbf{r},\mathbf{r}';\omega)$ is 
\begin{equation}
\begin{split}
&\nabla'\cdot\frac{\mathrm{i}k}{2\pi^2}\sum_{\bm{\alpha}}\int_0^\infty\mathbf{X}_{\bm{\alpha}}(\mathbf{r},\kappa)\left( \vphantom{\frac{1}{4\pi k^2}} \mathbf{X}_{\bm{\alpha}}(\mathbf{r}',k)R_{T\ell}^{><}(k,\kappa;r',\infty)\right.\\
&\qquad\left.+ \bm{\mathcal{X}}_{\bm{\alpha}}(\mathbf{r}',k)R_{T\ell}^\ll(k,\kappa;0,r') - \frac{1}{\mathrm{i} k^3}\left[\hat{\mathbf{r}}'\cdot\mathbf{X}_{\bm{\alpha}}(\mathbf{r}',\kappa)\right]\hat{\mathbf{r}}'\right)\kappa^2\mathrm{d}\kappa\\
&= -\frac{2\mathrm{i}}{k^3r'}\frac{\mathrm{i}k}{2\pi^2}\sum_{p\ell m}\int_0^\infty\mathbf{N}_{p\ell m}(\mathbf{r},\kappa)\mathbf{N}_{p\ell m}(\mathbf{r}',\kappa)\cdot\hat{\mathbf{r}}'\kappa^2\;\mathrm{d}\kappa\\
&= \frac{2\hat{\mathbf{r}}'}{k^2r'}\delta(\mathbf{r} - \mathbf{r}'),
\end{split}
\end{equation}
such that the $\mathbf{r}$-transverse part of $\mathbf{G}_0(\mathbf{r},\mathbf{r}';\omega)$ is also transverse in $\mathbf{r}'$ for $\mathbf{r}'\neq\mathbf{r}$. We can also see that the $\mathbf{r}'$-divergence of the $\mathbf{r}$-transverse part is irregular at $r' = 0$. 

Both of these cases need to be handled with care, however neither causes insurmountable problems. In the case $\mathbf{r}'\to\mathbf{r}$, the fields are, in general, irregular, and we can safely ignore these points as our solution is meaningless anyway. For $r'\to0$, our expansion into mode functions $\mathbf{X}_{\bm{\alpha}}(\mathbf{r},\kappa)$ and $f_{\bm{\alpha}}^{>,<}(\mathbf{r})$ breaks due to the irregularity of the function $R_{T\ell}^{><}(k,\kappa;r',\infty)$ as $r'$ approaches the origin. However, in this case the Green's function simply becomes
\begin{equation}
\lim_{r'\to0}\mathbf{G}(\mathbf{r},\mathbf{r}';\omega) = \lim_{r'\to0}\left(\frac{ik}{4\pi}\sum_{\bm{\alpha}}\bm{\mathcal{X}}_{\bm{\alpha}}(\mathbf{r},k)\mathbf{X}_{\bm{\alpha}}(\mathbf{r}',k) - \frac{\hat{\mathbf{r}}\hat{\mathbf{r}}}{k^2}\delta(\mathbf{r} - \mathbf{r}')\right),
\end{equation}
such that (if we allow derivatives to pass through the limit operator) $\nabla'\cdot\lim_{r'\to0}\mathbf{G}(\mathbf{r},\mathbf{r}';\omega) = \lim_{r'\to0}\nabla'\cdot\hat{\mathbf{r}}'\hat{\mathbf{r}}\delta(\mathbf{r} - \mathbf{r}')/k^2 = \lim_{r'\to0}\hat{\mathbf{r}}(\partial/\partial r')\{\delta(\mathbf{r} - \mathbf{r}')\}/k^2$ as $\mathbf{X}_{\bm{\alpha}}(\mathbf{r}',k)$ is divergence-free everywhere. We have also let one factor of $\hat{\mathbf{r}}$ become $\hat{\mathbf{r}}'$ due to the equivalence enforced by the Dirac delta $\delta(\mathbf{r} - \mathbf{r}')$. The resulting $r'$-derivative operator acting on $\delta(\mathbf{r} - \mathbf{r}')$ produces more expressions proportional to $\delta(\mathbf{r} - \mathbf{r}')$, such that, even for $r'\to 0$, the $\mathbf{r}'$-divergence of the Green's function is zero except when the observer and source are co-located. Therefore, for $\mathbf{r} \neq \mathbf{r}'$ and $r'>0$, we can see that
\begin{equation}\label{eq:G0transverse}
\begin{split}
\mathbf{G}_0^\perp(\mathbf{r},\mathbf{r}';\omega) &= \frac{\mathrm{i}k}{2\pi^2}\sum_{\bm{\alpha}}\int_0^\infty\mathbf{X}_{\bm{\alpha}}(\mathbf{r},\kappa)\left( \vphantom{\frac{1}{4\pi k^2}} \mathbf{X}_{\bm{\alpha}}(\mathbf{r}',k)R_{T\ell}^{><}(k,\kappa;r',\infty)\right.\\
&\qquad\left.+ \bm{\mathcal{X}}_{\bm{\alpha}}(\mathbf{r}',k)R_{T\ell}^\ll(k,\kappa;0,r') - \frac{1}{\mathrm{i} k^3}\left[\hat{\mathbf{r}}'\cdot\mathbf{X}_{\bm{\alpha}}(\mathbf{r}',\kappa)\right]\hat{\mathbf{r}}'\right)\kappa^2\mathrm{d}\kappa
\end{split}
\end{equation}
is divergence-free in both $\mathbf{r}$ and $\mathbf{r}'$, and the case of $r' = 0$ requires more thought but produces similarly null divergences.

The $\mathbf{r}$-longitudinal part of the free-space Green's function is easier to analyze than $\mathbf{G}_0^\perp(\mathbf{r},\mathbf{r}';\omega)$. We can quickly use the vector calculus identity $\nabla'\times\left\{\nabla f(\mathbf{r})g(\mathbf{r}')\mathbf{F}(\mathbf{r}')\right\} = \nabla f(\mathbf{r})\left[g(\mathbf{r}')\nabla'\times\mathbf{F}(\mathbf{r}') + \nabla'g(\mathbf{r}')\times\mathbf{F}(\mathbf{r}')\right]$ to analyze its three main terms:
\begin{equation}
\begin{split}
\nabla'&\times\left\{\mathbf{Z}_{\bm{\alpha}}(\mathbf{r},r')\left(-\frac{h_\ell(kr')}{kr'}\right)\mathbf{N}_{p\ell m}(\mathbf{r}',k)\right\}\\
&= \left[r'^{-\ell + 1}\nabla f_{\bm{\alpha}}^<(\mathbf{r})\Theta(r' - r) + r'^{\ell + 2}\nabla f_{\bm{\alpha}}^>(\mathbf{r})\Theta(r - r')\right]\left(-\frac{h_\ell(kr')}{kr'}\right)\nabla'\times\mathbf{N}_{p\ell m}(\mathbf{r}',k)\\
&\qquad+ \nabla'\left\{\left[r'^{-\ell + 1}\nabla f_{\bm{\alpha}}^<(\mathbf{r})\Theta(r' - r) + r'^{\ell + 2}\nabla f_{\bm{\alpha}}^>(\mathbf{r})\Theta(r - r')\right]\left(-\frac{h_\ell(kr')}{kr'}\right)\right\}\times\mathbf{N}_{p\ell m}(\mathbf{r}',k)\\
&= -\left[r'^{-\ell + 1}\nabla f_{\bm{\alpha}}^<(\mathbf{r})\Theta(r' - r) + r'^{\ell + 2}\nabla f_{\bm{\alpha}}^>(\mathbf{r})\Theta(r - r')\right]\frac{h_\ell(kr')}{kr'}k\mathbf{M}_{p\ell m}(\mathbf{r}',k)\\
&\qquad - \left[r'^{-\ell + 1}\nabla f_{\bm{\alpha}}^<(\mathbf{r})\Theta(r' - r) + r'^{\ell + 2}\nabla f_{\bm{\alpha}}^>(\mathbf{r})\Theta(r - r')\right]\frac{\partial}{\partial r'}\left\{\frac{h_\ell(kr')}{kr'}\right\}\hat{\mathbf{r}}'\times\mathbf{N}_{p\ell m}(\mathbf{r}',k)\\
&\qquad - \left[(-\ell + 1)r'^{-\ell}\nabla f_{\bm{\alpha}}^<(\mathbf{r})\Theta(r' - r) + (\ell + 2)r'^{\ell + 1}\nabla f_{\bm{\alpha}}^>(\mathbf{r})\Theta(r - r')\right]\frac{h_\ell(kr')}{kr'}\hat{\mathbf{r}}'\times\mathbf{N}_{p\ell m}(\mathbf{r}',k)\\
&= -\left[r'^{-\ell + 1}\nabla f_{\bm{\alpha}}^<(\mathbf{r})\Theta(r' - r) + r'^{\ell + 2}\nabla f_{\bm{\alpha}}^>(\mathbf{r})\Theta(r - r')\right]\frac{h_\ell(kr')}{kr'}k\mathbf{M}_{p\ell m}(\mathbf{r}',k)\\
&\qquad - \left[r'^{-\ell}\nabla f_{\bm{\alpha}}^<(\mathbf{r})\Theta(r' - r) + r'^{\ell + 1}\nabla f_{\bm{\alpha}}^>(\mathbf{r})\Theta(r - r')\right]\\
&\qquad\qquad\times\frac{1}{kr'}\left(-2h_\ell(kr') + \frac{\partial\left\{r'h_\ell(kr')\right\}}{\partial r'}\right)\hat{\mathbf{r}}'\times\mathbf{N}_{p\ell m}(\mathbf{r}',k)\\
&\qquad - \left[(-\ell + 1)r'^{-\ell}\nabla f_{\bm{\alpha}}^<(\mathbf{r})\Theta(r' - r) + (\ell + 2)r'^{\ell + 1}\nabla f_{\bm{\alpha}}^>(\mathbf{r})\Theta(r - r')\right]\\
&\qquad\qquad\times\frac{1}{kr'}\left(\frac{\partial}{\partial r'}\left\{r'h_\ell(kr')\right\} - r'\frac{\partial h_\ell(kr')}{\partial r'}\right)\hat{\mathbf{r}}'\times\mathbf{N}_{p\ell m}(\mathbf{r}',k),
\end{split}
\end{equation}
and
\begin{equation}
\begin{split}
\nabla'&\times\left\{\mathbf{Z}_{\bm{\alpha}}(\mathbf{r},r')\left(\frac{j_\ell(kr')}{kr'}\right)\bm{\mathcal{N}}_{p\ell m}(\mathbf{r}',k)\right\}\\
&= \left[r'^{-\ell + 1}\nabla f_{\bm{\alpha}}^<(\mathbf{r})\Theta(r' - r) + r'^{\ell + 2}\nabla f_{\bm{\alpha}}^>(\mathbf{r})\Theta(r - r')\right]\frac{j_\ell(kr')}{kr'}k\bm{\mathcal{M}}_{p\ell m}(\mathbf{r}',k)\\
&\qquad + \left[r'^{-\ell}\nabla f_{\bm{\alpha}}^<(\mathbf{r})\Theta(r' - r) + r'^{\ell + 1}\nabla f_{\bm{\alpha}}^>(\mathbf{r})\Theta(r - r')\right]\\
&\qquad\qquad\times\frac{1}{kr'}\left(-2j_\ell(kr') + \frac{\partial\left\{r'j_\ell(kr')\right\}}{\partial r'}\right)\hat{\mathbf{r}}'\times\bm{\mathcal{N}}_{p\ell m}(\mathbf{r}',k)\\
&\qquad + \left[(-\ell + 1)r'^{-\ell}\nabla f_{\bm{\alpha}}^<(\mathbf{r})\Theta(r' - r) + (\ell + 2)r'^{\ell + 1}\nabla f_{\bm{\alpha}}^>(\mathbf{r})\Theta(r - r')\right]\\
&\qquad\qquad\times\frac{1}{kr'}\left(\frac{\partial}{\partial r'}\left\{r'j_\ell(kr')\right\} - r'\frac{\partial j_\ell(kr')}{\partial r'}\right)\hat{\mathbf{r}}'\times\bm{\mathcal{N}}_{p\ell m}(\mathbf{r}',k),
\end{split}
\end{equation}
and
\begin{equation}\label{eq:term3}
\begin{split}
&\nabla'\times\left\{\nabla\left\{\frac{\partial f_{\bm{\alpha}}^>(\mathbf{r}')}{\partial r'}f_{\bm{\alpha}}^<(\mathbf{r})\Theta(r' - r) + \frac{\partial f_{\bm{\alpha}}^<(\mathbf{r}')}{\partial r'}f_{\bm{\alpha}}^>(\mathbf{r})\Theta(r - r')\right\}\hat{\mathbf{r}}'\right\}\\
&= -\hat{\mathbf{r}}'\times\nabla\nabla'\left\{\frac{\partial f_{\bm{\alpha}}^>(\mathbf{r}')}{\partial r'}f_{\bm{\alpha}}^<(\mathbf{r})\Theta(r' - r) + \frac{\partial f_{\bm{\alpha}}^<(\mathbf{r}')}{\partial r'}f_{\bm{\alpha}}^>(\mathbf{r})\Theta(r - r')\right\}\\
&= \delta_{TE}\sqrt{(2 - \delta_{m0})\frac{(\ell - m)!}{(\ell + m)!}}\left(-\hat{\bm{\phi}}'\frac{1}{r'}\frac{\partial}{\partial\theta'} + \hat{\bm{\theta}}'\frac{1}{r'\sin\theta'}\frac{\partial}{\partial \phi'}\right)\\
&\quad\times\nabla\left\{(-\ell - 1)\frac{1}{r'^{\ell + 2}}P_{\ell m}(\cos\theta')S_p(m\phi')f_{\bm{\alpha}}^<(\mathbf{r})\Theta(r' - r) + \ell r'^{\ell - 1}P_{\ell m}(\cos\theta')S_p(m\phi')f_{\bm{\alpha}}^>(\mathbf{r})\Theta(r - r')\right\}\\
&= \delta_{TE}\sqrt{(2 - \delta_{m0})\frac{(\ell - m)!}{(\ell + m)!}}\left(-\frac{\partial P_{\ell m}(\cos\theta')}{\partial \theta'}S_p(m\phi')\hat{\bm{\phi}}' + (-1)^{p+1}m\frac{P_{\ell m}(\cos\theta')}{\sin\theta'}S_{p+1}(m\phi')\hat{\bm{\theta}}'\right)\\
&\quad\times\nabla\left\{-\frac{\ell + 1}{r'^{\ell + 3}}f_{\bm{\alpha}}^<(\mathbf{r})\Theta(r' - r) + \ell r'^{\ell - 2}f_{\bm{\alpha}}^>(\mathbf{r})\Theta(r - r')\right\}.
\end{split}
\end{equation}
The first two of these can be combined to give
\begin{equation}
\begin{split}
&\frac{\mathrm{i}k}{4\pi}\delta_{TE}\sqrt{\frac{\ell(\ell + 1)}{2\ell + 1}}\nabla'\times\left\{\mathbf{Z}_{\bm{\alpha}}(\mathbf{r};r')\left(\frac{j_\ell(kr')}{kr'}\bm{\mathcal{N}}_{p\ell m}(\mathbf{r}',k) - \frac{h_\ell(kr')}{kr'}\mathbf{N}_{p\ell m}(\mathbf{r}',k)\right)\right\}\\
=& \frac{\mathrm{i}k}{4\pi}\delta_{TE}\sqrt{\frac{\ell(\ell + 1)}{2\ell + 1}}\left( \vphantom{\frac{1}{kr'}} \left[r'^{-\ell + 1}\nabla f_{\bm{\alpha}}^<(\mathbf{r})\Theta(r' - r) + r'^{\ell + 2}\nabla f_{\bm{\alpha}}^>(\mathbf{r})\Theta(r - r')\right]\right.\\
&\qquad\times\left[-\frac{h_\ell(kr')}{kr'}k\mathbf{M}_{p\ell m}(\mathbf{r}',k) + \frac{j_\ell(kr')}{kr'}k\bm{\mathcal{M}}_{p\ell m}(\mathbf{r}',k)\right]\\
&\;+ \left[r'^{-\ell}\nabla f_{\bm{\alpha}}^<(\mathbf{r})\Theta(r' - r) + r'^{\ell + 1}\nabla f_{\bm{\alpha}}^>(\mathbf{r})\Theta(r - r')\right]\\
&\qquad\times\left[-\frac{1}{kr'}\frac{\partial\{r'h_\ell(kr')\}}{\partial r'}\hat{\mathbf{r}}'\times\mathbf{N}_{p\ell m}(\mathbf{r}',k) + \frac{1}{kr'}\frac{\partial\{r'j_\ell(kr')\}}{\partial r'}\hat{\mathbf{r}}'\times\bm{\mathcal{N}}_{p\ell m}(\mathbf{r}',k)\right]\\
&\;+ \left[(-\ell + 1)r'^{-\ell}\nabla f_{\bm{\alpha}}^<(\mathbf{r})\Theta(r' - r) + (\ell + 2)r'^{\ell + 1}\nabla f_{\bm{\alpha}}^>(\mathbf{r})\Theta(r - r')\right]\\
&\qquad\times \left[-\frac{1}{kr'}\frac{\partial\{r'h_\ell(kr')\}}{\partial r'}\hat{\mathbf{r}}'\times\mathbf{N}_{p\ell m}(\mathbf{r}',k) + \frac{1}{kr'}\frac{\partial\{r'j_\ell(kr')\}}{\partial r'}\hat{\mathbf{r}}'\times\bm{\mathcal{N}}_{p\ell m}(\mathbf{r}',k)\right]\\
&\;+ \left[r'^{-\ell}\nabla f_{\bm{\alpha}}^<(\mathbf{r})\Theta(r' - r) + r'^{\ell + 1}\nabla f_{\bm{\alpha}}^>(\mathbf{r})\Theta(r - r')\right]\\
&\qquad\times \left[\left(\frac{2}{kr'}h_\ell(kr') + \frac{2}{kr'}r'\frac{\partial h_\ell(kr')}{\partial r'}\right)\hat{\mathbf{r}}'\times\mathbf{N}_{p\ell m}(\mathbf{r}',k)\right.\\
&\qquad\qquad\left.- \left(\frac{2}{kr'}j_\ell(kr') + \frac{2}{kr'}r'\frac{\partial j_\ell(kr')}{\partial r'}\right)\hat{\mathbf{r}}'\times\bm{\mathcal{N}}_{p\ell m}(\mathbf{r}',k)\right]\\
&\;+\left[(-\ell - 1)r'^{-\ell}\nabla f_{\bm{\alpha}}^<(\mathbf{r})\Theta(r' - r) + \ell r'^{\ell + 1}\nabla f_{\bm{\alpha}}^>(\mathbf{r})\Theta(r - r')\right]\\
&\qquad\times\left.\left[\frac{1}{kr'}r'\frac{h_\ell(kr')}{\partial r'}\hat{\mathbf{r}}'\times\mathbf{N}_{p\ell m}(\mathbf{r}',k) - \frac{1}{kr'}r'\frac{j_\ell(kr')}{\partial r'}\hat{\mathbf{r}}'\times\bm{\mathcal{N}}_{p\ell m}(\mathbf{r}',k)\right]\right).
\end{split}
\end{equation}
The first term in the sum on the right-hand side is zero as $\mathbf{M}_{p\ell m}(\mathbf{r}',k)$ and $\bm{\mathcal{M}}_{p\ell m}(\mathbf{r}',k)$ vary in $r'$ as $j_\ell(kr')$ and $h_\ell(kr')$, respectively, and have identical $\theta'$ and $\phi'$ dependence. The second and third terms can be seen to zero via a similar cancellation, as the $\hat{\bm{\theta}}'$ and $\hat{\bm{\phi}}'$ terms of $\mathbf{N}_{p\ell m}(\mathbf{r}',k)$ and $\bm{\mathcal{N}}_{p\ell m}(\mathbf{r}',k)$, i.e. the terms that survive the cross-product with $\hat{\mathbf{r}}'$, vary in $r'$ as $(\partial\{r' j_\ell(kr')\}/\partial r')/kr'$ and $(\partial\{r' h_\ell(kr')\}/\partial r')/kr'$, respectively, and have the same angular dependence. The fourth term is zero by the same logic, as can be seen using the identities $\partial\{r'j_\ell(kr')\}\partial r' = j_\ell(kr') + r'\partial j_\ell(kr')/\partial r'$ and $\partial\{r'h_\ell(kr')\}\partial r' = h_\ell(kr') + r'\partial h_\ell(kr')/\partial r'$ to simplify the factors in parentheses. The fifth term is the only nonzero contribution to the total and gives, after expansion of the vector harmonics, use of the identities $\partial j_\ell(kr')/\partial r' = -kj_{\ell + 1}(kr') + \ell k j_\ell(kr')/kr'$ and $\partial h_\ell(kr')/\partial r' = -kh_{\ell + 1}(kr') + \ell k h_\ell(kr')/kr'$, and some cancellation,
\begin{equation}
\begin{split}
&\frac{\mathrm{i}k}{4\pi}\delta_{TE}\sqrt{\frac{\ell(\ell + 1)}{2\ell + 1}}\nabla'\times\left\{\mathbf{Z}_{\bm{\alpha}}(\mathbf{r};r')\left(\frac{j_\ell(kr')}{kr'}\bm{\mathcal{N}}_{p\ell m}(\mathbf{r}',k) - \frac{h_\ell(kr')}{kr'}\mathbf{N}_{p\ell m}(\mathbf{r}',k)\right)\right\}\\
=& -\frac{1}{4\pi k^2r'^3}\delta_{TE}\sqrt{(2 - \delta_{m0})\frac{(\ell - m)!}{(\ell + m)!}}\left[(-\ell - 1)r'^{-\ell}\nabla f_{\bm{\alpha}}^<(\mathbf{r})\Theta(r - r') + \ell r'^{\ell + 1}\nabla f_{\bm{\alpha}}^>(\mathbf{r})\Theta(r' - r)\right]\\
&\qquad\times\left(\frac{\partial P_{\ell m}(\cos\theta')}{\partial \theta'}S_p(m\phi')\hat{\bm{\phi}}' - \frac{(-1)^{p+1}m}{\sin\theta'}P_{\ell m}(\cos\theta')S_{p+1}(m\phi')\hat{\bm{\theta}}'\right).
\end{split}
\end{equation}
We can see that the above sum is the negative of Eq. \eqref{eq:term3} times a factor of $-1/4\pi k^2$, such that
\begin{equation}
\begin{split}
\nabla'&\times\frac{\mathrm{i}k}{4\pi}\sum_{\bm{\alpha}}\left(\delta_{TE}\sqrt{\frac{\ell(\ell + 1)}{2\ell + 1}}\mathbf{Z}_{\bm{\alpha}}(\mathbf{r},r')\left[-\frac{h_\ell(kr')}{kr'}\mathbf{N}_{p\ell m}(\mathbf{r}',k) + \frac{j_\ell(kr')}{kr'}\bm{\mathcal{N}}_{p\ell m}(\mathbf{r}',k)\right]\right.\\
&\left. -\frac{1}{\mathrm{i} k^3}\nabla\left\{\frac{\partial f_{\bm{\alpha}}^>(\mathbf{r}')}{\partial r'}f_{\bm{\alpha}}^<(\mathbf{r})\Theta(r' - r) + \frac{\partial f_{\bm{\alpha}}^<(\mathbf{r}')}{\partial r'}f_{\bm{\alpha}}^>(\mathbf{r})\Theta(r - r')\right\}\hat{\mathbf{r}}' \vphantom{\sqrt{\frac{\ell(\ell + 1)}{2\ell + 1}}} \right) = 0
\end{split}
\end{equation}
and the $\mathbf{r}$-longitudinal part of the Green's function is also $\mathbf{r}'$-longitudinal. Therefore, we can easily define
\begin{equation}\label{eq:G0longitudinal}
\begin{split}
\mathbf{G}_0^\parallel(\mathbf{r},\mathbf{r}';\omega) &= \frac{\mathrm{i}k}{4\pi}\sum_{\bm{\alpha}}\left(\delta_{TE}\sqrt{\frac{\ell(\ell + 1)}{2\ell + 1}}\mathbf{Z}_{\bm{\alpha}}(\mathbf{r},r')\left[-\frac{h_\ell(kr')}{kr'}\mathbf{N}_{p\ell m}(\mathbf{r}',k) + \frac{j_\ell(kr')}{kr'}\bm{\mathcal{N}}_{p\ell m}(\mathbf{r}',k)\right]\right.\\
&\left. -\frac{1}{\mathrm{i} k^3}\nabla\left\{\frac{\partial f_{\bm{\alpha}}^>(\mathbf{r}')}{\partial r'}f_{\bm{\alpha}}^<(\mathbf{r})\Theta(r' - r) + \frac{\partial f_{\bm{\alpha}}^<(\mathbf{r}')}{\partial r'}f_{\bm{\alpha}}^>(\mathbf{r})\Theta(r - r')\right\}\hat{\mathbf{r}}' \vphantom{\sqrt{\frac{\ell(\ell + 1)}{2\ell + 1}}} \right),
\end{split}
\end{equation}
which, alongside the definition of the transverse Green's function in Eq. \eqref{eq:G0transverse}, provides
\begin{equation}
\mathbf{G}_0(\mathbf{r},\mathbf{r}';\omega) = \mathbf{G}_0^\perp(\mathbf{r},\mathbf{r}';\omega) + \mathbf{G}_0^\parallel(\mathbf{r},\mathbf{r}';\omega)
\end{equation}
which satisfies the originally desired conditions of Eqs. \eqref{eq:waveEquationDerivativeHelmholtzDecomp} and \eqref{eq:waveEquationIntegralHelmholtzDecomp}.