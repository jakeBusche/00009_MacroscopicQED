%!TEX root = finiteMQED.tex

\section{Dependent matter: the dielectric picture}\label{app:dependentMatter}

We have touched on this picture already in the preceding sections, but it is useful to reiterate that the Euler-Lorentz equations derivable from the system Lagrangian $L$ state clearly that the degrees of freedom of the matter and electromagnetic field are coupled. More explicitly, there are four Euler-Lorentz equations of the form
\begin{equation}
\frac{\mathrm{d}}{\mathrm{d}t}\left\{\frac{\partial\mathcal{L}'}{\partial\dot{F}(\mathbf{r},t)}\right\} = \frac{\partial\mathcal{L}'}{\partial F(\mathbf{r},t)} - \sum_{i = 1}^3\frac{\partial}{\partial r_i}\frac{\partial\mathcal{L}'}{\partial\left(\frac{\partial F(\mathbf{r},t)}{\partial r_i}\right)},
\end{equation}
where $F(\mathbf{r},t)$ takes the place of the scalar potential or a component of the vector potential, matter displacement field, or reservoir displacement field. The modified Lagrangian density $\mathcal{L}' = \mathcal{L} - (1/4\pi c)\nabla\Phi(\mathbf{r},t)\cdot\dot{\mathbf{A}}(\mathbf{r},t)$ provides the same Lagrangian, i.e. $\int\mathcal{L}\;\mathrm{d}^3\mathbf{r} = \int\mathcal{L}'\;\mathrm{d}^3\mathbf{r} = L$, due to the orthogonality under integration (see Eq. [\ref{eq:helmholtzOrthogonality}]) of the longitudinal gradient of the scalar potential and transverse (Coulomb gauge) vector potential. Here, the use of $\mathcal{L}'$ provides mathematical simplicity.

The equations of motion the four Euler-Lagrange equations lead to are
\begin{equation}\label{eq:fourEOMs}
\begin{split}
\eta_m(\mathbf{r})\ddot{\mathbf{Q}}_m(\mathbf{r},t) + \int_0^\infty\tilde{v}(\nu)\dot{\tilde{\mathbf{Q}}}_\nu(\mathbf{r},t)\;&\mathrm{d}\nu + \eta_m(\mathbf{r})\omega_0^2\mathbf{Q}_m(\mathbf{r},t)\\
+ \eta_m(\mathbf{r})\sigma_0\mathbf{Q}_m(\mathbf{r},t)\cdot\bm{1}_3&\cdot\mathbf{Q}_m(\mathbf{r},t) = \frac{e}{\Delta}\left(-\nabla\Phi(\mathbf{r},t) - \frac{1}{c}\dot{\mathbf{A}}(\mathbf{r})\right),\\[1.0em]
\eta_\nu(\mathbf{r})\ddot{\tilde{\mathbf{Q}}}_\nu(\mathbf{r},t) + \eta_\nu(\mathbf{r})\nu^2\tilde{\mathbf{Q}}_\nu(\mathbf{r},t) &= \tilde{v}(\nu)\dot{\mathbf{Q}}_m(\mathbf{r},t),\\[0.5em]
-\nabla\cdot\nabla\Phi(\mathbf{r},t) &= 4\pi\rho_f(\mathbf{r},t) - \frac{4\pi e}{\Delta}\nabla\cdot\mathbf{Q}_m(\mathbf{r},t),\\
\nabla\times\nabla\times\mathbf{A}(\mathbf{r},t) - \frac{1}{c^2}\ddot{\mathbf{A}}(\mathbf{r},t) &= \frac{4\pi}{c}\left[\frac{e}{\Delta}\dot{\mathbf{Q}}_m(\mathbf{r},t) + \mathbf{J}_f(\mathbf{r},t)\right] - \frac{\nabla\dot{\Phi}(\mathbf{r},t)}{c}\\
&= \frac{4\pi}{c}\left[\frac{e}{\Delta}\dot{\mathbf{Q}}_m^\perp(\mathbf{r},t) + \mathbf{J}_f^\perp(\mathbf{r},t)\right].
\end{split}
\end{equation}
The most pertient equation to analyze here is the second. Using the Green's function (valid for $r < a$)
\begin{equation}
G(t - t') = \frac{\Theta(t - t')}{\eta_\nu(\mathbf{r})\nu}\sin(\nu[t - t'])
\end{equation}
of the oscillator differential equation
\begin{equation}
\eta_\nu(\mathbf{r})\ddot{G}(t - t') + \eta_\nu(\mathbf{r})\nu^2G(t - t') = 
\begin{cases}
\delta(t - t'), & t > t';\\
0, & t < t';
\end{cases}
\end{equation}
we can see that
\begin{equation}
\begin{split}
\dot{\tilde{\mathbf{Q}}}_\nu(\mathbf{r},t) &= \dot{\tilde{\mathbf{Q}}}_\nu^{(0)}(\mathbf{r},t) + \frac{\mathrm{d}}{\mathrm{d}t}\int_{-\infty}^\infty G(t - t')\tilde{v}(\nu)\dot{\mathbf{Q}}_m(\mathbf{r},t')\;\mathrm{d}t'\\
&= \dot{\tilde{\mathbf{Q}}}_\nu^{(0)}(\mathbf{r},t) + \int_{-\infty}^\infty\frac{\Theta(t - t')}{\eta_\nu(\mathbf{r})}\cos(\nu[t - t'])\tilde{v}(\nu)\dot{\mathbf{Q}}_m(\mathbf{r},t')\;\mathrm{d}t'.
\end{split}
\end{equation}
where $\dot{\tilde{\mathbf{Q}}}^{(0)}_\nu(\mathbf{r},t)$ is the $\nu^\mathrm{th}$ velocity field at $t$ generated by stimuli far in the past (formally, at the initial time $-\infty$). The inverse of the reservoir mass-density, $1/\eta_\nu(\mathbf{r}) = \eta_\nu^{-1}(\mathbf{r})$, is taken to imply use of the function $\eta_\nu^{-1}(\mathbf{r}) = \Theta(a - r)\Delta/\sum_{i = 1}^{N_r}\mu_{ri}\Theta[\mathbf{r}\in\mathbb{V}(\mathbf{r}_i)]$, i.e. is assumed to be equal to $\Delta/\mu_{ri}$ within each small region $\mathbb{V}(\mathbf{r}_i)$ inside the sphere and zero for all $r > a$. We can then define a loss function
\begin{equation}
\gamma_0(t - t') = \frac{1}{\eta_m(\mathbf{r})}\int_0^\infty\frac{\tilde{v}^2(\nu)}{\eta_\nu(\mathbf{r})}\cos(\nu[t - t'])\;\mathrm{d}\nu
\end{equation}
and a thermal noise field (dimensions of force per unit volume)
\begin{equation}
\mathbf{N}_\mathrm{therm}(\mathbf{r},t) = -\int_0^\infty\tilde{v}(\nu)\dot{\tilde{\mathbf{Q}}}^{(0)}_\nu(\mathbf{r},t)\;\mathrm{d}\nu
\end{equation}
using an identical assumption for the inverse of the mass-density of the matter field (simply let $\mu_{ri}\to\mu_{mi}$, etc.) such that the first equation of Eq. \eqref{eq:fourEOMs} becomes
\begin{equation}\label{eq:matterEOM}
\begin{split}
\eta_m(\mathbf{r})\ddot{\mathbf{Q}}_m(\mathbf{r},t) &+ \eta_m(\mathbf{r})\int_{-\infty}^t\gamma_0(t - t')\dot{\mathbf{Q}}_m(\mathbf{r},t')\;\mathrm{d}t' + \eta_m(\mathbf{r})\omega_0^2\mathbf{Q}_m(\mathbf{r},t)\\
&+ \eta_m(\mathbf{r})\sigma_0\mathbf{Q}_m(\mathbf{r},t)\cdot\bm{1}_3\cdot\mathbf{Q}_m(\mathbf{r},t) = \frac{e}{\Delta}\left(-\nabla\Phi(\mathbf{r},t) - \frac{1}{c}\dot{\mathbf{A}}(\mathbf{r})\right) + \mathbf{N}_\mathrm{therm}(\mathbf{r},t).
\end{split}
\end{equation}

Solutions to Eq. \eqref{eq:matterEOM} are simplest in the case where $\sigma_0$ is a ``small'' quantity. We will not belabor the exact definition of ``small'' here, other than to say that we will assume $\sigma_0$ to be linearly proportional to a characteristically small number $\lambda$ that can be used to expand the matter displacement field as a perturbation series:
\begin{equation}
\begin{split}
\mathbf{Q}_m(\mathbf{r},t) &= \mathbf{Q}_m^{(1)}(\mathbf{r},t) + \mathbf{Q}_m^{(2)}(\mathbf{r},t) + \ldots\\
&= \sum_{n = 1}^\infty\lambda^{n-1}\bm{\mathcal{Q}}_m^{(n)}(\mathbf{r},t).
\end{split}
\end{equation}
Here, the functions $\bm{\mathcal{Q}}_m^{(n)}(\mathbf{r},t)$ are of roughly the same order of magnitude at each order $n$ such that it is the different powers of $\lambda$ that separate the terms in the series. Using this expansion, we find that
\begin{equation}\label{eq:matterEOM2}
\begin{split}
\eta_m(\mathbf{r})\ddot{\mathbf{Q}}_m^{(1)}(\mathbf{r},t) + \eta_m(\mathbf{r})\int_{-\infty}^t\gamma_0(t - t')\dot{\mathbf{Q}}_m^{(1)}(\mathbf{r},t')\;\mathrm{d}t' &+ \eta_m(\mathbf{r})\omega_0^2\mathbf{Q}_m^{(1)}(\mathbf{r},t) = \frac{e}{\Delta}\mathbf{E}(\mathbf{r},t) + \mathbf{N}_\mathrm{therm}(\mathbf{r},t),\\
\eta_m(\mathbf{r})\ddot{\mathbf{Q}}_m^{(2)}(\mathbf{r},t) + \eta_m(\mathbf{r})\int_{-\infty}^t\gamma_0(t - t')\dot{\mathbf{Q}}_m^{(2)}(\mathbf{r},t')\;\mathrm{d}t'
&+ \eta_m(\mathbf{r})\omega_0^2\mathbf{Q}_m^{(2)}(\mathbf{r},t)\\
&= -\eta_m(\mathbf{r})\sigma_0\mathbf{Q}_m^{(1)}(\mathbf{r},t)\cdot\bm{1}_3\cdot\mathbf{Q}_m^{(1)}(\mathbf{r},t)
\end{split}
\end{equation}
where we have let $\mathbf{E}(\mathbf{r},t) = -\nabla\Phi(\mathbf{r},t) - \dot{\mathbf{A}}(\mathbf{r},t)/c$.

Further, we can simplify our analysis by choosing the simple loss function $\gamma_0(t - t') = \gamma_0\delta(t - t')$. We can see that this is achievable with the choice of coupling function
\begin{equation}
\tilde{v}(\nu) = \sqrt{\frac{\gamma_0}{\pi}\eta_m(\mathbf{r})\eta_\nu(\mathbf{r})},
\end{equation}
as can be seen through the application of the identity $\delta(t - t') = \int_0^\infty 2\cos(\nu[t - t'])\;\mathrm{d}\nu/2\pi$. Using this simplification, we can take the Fourier transform of both lines of Eq. \eqref{eq:matterEOM2} to see
\begin{equation}\label{eq:matterEOM3}
\begin{split}
\mathbf{Q}_m^{(1)}(\mathbf{r},\omega)\eta_m(\mathbf{r})\left[-\omega^2 - \mathrm{i}\omega\gamma_0 + \omega_0^2\right] &= \frac{e}{\Delta}\mathbf{E}(\mathbf{r},\omega) + \mathbf{N}_\mathrm{therm}(\omega),\\
\mathbf{Q}_m^{(2)}(\mathbf{r},\omega)\eta_m(\mathbf{r})\left[-\omega^2 - \mathrm{i}\omega\gamma_0 + \omega_0^2\right] &= -\eta_m(\mathbf{r})\sigma_0\int_{-\infty}^\infty\mathbf{Q}_m^{(1)}(\mathbf{r},\omega')\cdot\bm{1}_3\cdot\mathbf{Q}_m^{(1)}(\mathbf{r},\omega - \omega')\;\mathrm{d}\omega'.
\end{split}
\end{equation}
This form of the equations of motion of the matter displacement fields is simple to connect back to a dielectric picture through the definition of the polarization field $\mathbf{P}(\mathbf{r},\omega) = (e/\Delta)\mathbf{Q}_m(\mathbf{r},\omega)$. Generally, this is done without thermal fluctuations, so we'll let $\mathbf{N}_\mathrm{therm}(\mathbf{r},\omega)\to0$ for now. The polarization field can be expanded as a perturbation series analogous to that of the displacement field, such that $\mathbf{P}(\mathbf{r},\omega) = \sum_n\mathbf{P}^{(n)}(\mathbf{r},\omega)$. This expansion can further be connected to a dielectric model through the definitions $\mathbf{P}^{(1)}(\mathbf{r},\omega) = \chi^{(1)}(\mathbf{r},\omega)\mathbf{E}(\mathbf{r},\omega)$ and $\mathbf{P}^{(2)}(\mathbf{r},\omega) = \int_{-\infty}^\infty\mathbf{E}(\mathbf{r},\omega')\cdot\bm{\chi}^{(2)}(\mathbf{r};\omega',\omega - \omega')\cdot\mathbf{E}(\mathbf{r},\omega - \omega')\;\mathrm{d}\omega'$. Explicitly, we can use these definitions along with a substitution within Eq. \eqref{eq:matterEOM3} to see that
\begin{equation}
\begin{split}
\chi^{(1)}(\mathbf{r},\omega) &= \frac{e^2}{\Delta^2\eta_m(\mathbf{r})}\frac{1}{\omega_0^2 - \omega^2 - \mathrm{i}\omega\gamma_0},\\
\bm{\chi}^{(2)}(\mathbf{r};\omega',\omega - \omega') &= -\frac{3e^3\sigma_0}{\Delta^3\eta_m^2(\mathbf{r})}\frac{\bm{1}_3}{(\omega_0^2 - \omega^2 - \mathrm{i}\omega\gamma_0)(\omega_0^2 - \omega'^2 - \mathrm{i}\omega'\gamma_0)(\omega_0^2 - [\omega - \omega']^2 - \mathrm{i}[\omega - \omega']\gamma_0)}.
\end{split}
\end{equation}

We are now in a position to connect our microscopic model back to a Green's-function solution to the system. Explicitly, all we need to do is redefine the last two lines of Eq. \eqref{eq:fourEOMs} in Fourier space,
\begin{equation}
\begin{split}
-\nabla\cdot\nabla\Phi(\mathbf{r},\omega) &= 4\pi\rho_f(\mathbf{r},\omega) - 4\pi\nabla\cdot\mathbf{P}(\mathbf{r},\omega),\\
\nabla\times\nabla\times\mathbf{A}(\mathbf{r},\omega) - \frac{\omega^2}{c^2}\mathbf{A}(\mathbf{r},\omega) &= -\frac{4\pi\omega^2}{c}\mathbf{P}^\perp(\mathbf{r},t) - \frac{4\pi\mathrm{i}\omega}{c}\mathbf{J}_f^\perp(\mathbf{r},\omega),
\end{split}
\end{equation}
then use our definitions for the susceptibilities to build the canonical wave Helmholtz equations for inhomogeneous dielectrics. 
