%!TEX root = finiteMQED.tex

\section{Harmonic Definitions}\label{app:harmonicAlgebra}

The dyadic Green's function of a sphere is composed of transverse vector spherical harmonics
\begin{equation}
\begin{split}
\mathbf{M}_{p\ell m}(\mathbf{r},k) &= \sqrt{K_{\ell m}}\left[\frac{(-1)^{p + 1}m}{\sin\theta}j_\ell(kr)P_{\ell m}(\cos\theta)S_{p+1}(m\phi)\hat{\bm{\theta}} - j_\ell(kr)\frac{\partial P_{\ell m}(\cos\theta)}{\partial\theta}S_p(m\phi)\hat{\bm{\phi}}\right],\\[0.5em]
\mathbf{N}_{p\ell m}(\mathbf{r},k) &= \sqrt{K_{\ell m}}\left[\frac{\ell(\ell + 1)}{kr}j_\ell(kr)P_{\ell m}(\cos\theta)S_p(m\phi)\hat{\mathbf{r}}\right.\\
&+ \left.\frac{1}{kr}\frac{\partial\{rj_\ell(kr)\}}{\partial r}\left(\frac{\partial P_{\ell m}(\cos\theta)}{\partial\theta}S_p(m\phi)\hat{\bm{\theta}} + \frac{(-1)^{p+1}m}{\sin\theta}P_{\ell m}(\cos\theta)S_{p + 1}(m\phi)\hat{\bm{\phi}}\right)\right],
\end{split}
\end{equation}
which each have zero divergence, i.e. $\nabla\cdot\mathbf{M}_{p\ell m}(\mathbf{r},k) = \nabla\cdot\mathbf{N}_{p\ell m}(\mathbf{r},k) = 0$, and longitudinal vector spherical harmonics
\begin{equation}
\begin{split}
\mathbf{L}_{p\ell m}(\mathbf{r},k) &= \frac{1}{k}\nabla\left\{\sqrt{K_{\ell m}}\sqrt{\ell(\ell + 1)}j_\ell(kr)P_{\ell m}(\cos\theta)S_p(m\phi)\right\}\\
&= \sqrt{K_{\ell m}}\sqrt{\ell(\ell + 1)}\left[\frac{1}{k}\frac{\partial\{j_\ell(kr)\}}{\partial r}P_{\ell m}(\cos\theta)S_p(m\phi)\hat{\mathbf{r}} + \frac{1}{kr}j_\ell(kr)\frac{\partial P_{\ell m}(\cos\theta)}{\partial\theta}S_p(m\phi)\hat{\bm{\theta}}\right.\\
&\left. + \frac{(-1)^{p+1}m}{kr\sin\theta}j_\ell(kr)P_{\ell m}(\cos\theta)S_{p+1}(m\phi)\hat{\bm{\phi}}\right]
\end{split}
\end{equation}
which obey $\nabla\times\mathbf{L}_{p\ell m}(\mathbf{r},k) = 0$. Here $j_\ell(x)$ are spherical Bessel functions with $\ell = 1,2,3,\ldots,\infty$; $P_{\ell m}(x)$ are associated Legendre polynomials with the same range in $\ell$; $m = 0,1,2,\ldots,\ell$; and 
\begin{equation}
S_p(mx) = 
\begin{cases}
\cos(mx), & p\;\;\mathrm{even},\\
\sin(mx), & p \;\;\mathrm{odd}
\end{cases}
\end{equation}
and the same range in $m$. Due to the binary nature of $p$, we will in general need to use special notation to assert that $S_p(x) = S_{p + 2}(x) = S_{p+4}(x) = \cdots$. However, to save space, we will use the convention that any nonzero even value of $p$ is automatically replaced by 0 and any non-one odd value of $p$ is automatically replaced by 1 in what follows. Further, the prefactors $K_{\ell m}$, which are set to $1$ in most texts, are set to
\begin{equation}
K_{\ell m} = (2 - \delta_{m0})\frac{2\ell + 1}{\ell(\ell + 1)}\frac{(\ell - m)!}{(\ell + m)!}
\end{equation}
in an effort to regularize the transverse harmonics, i.e. ensure that their maximum magnitudes in $\mathbf{r}$ for fixed $k$ are similar for all index combinations $T,p,\ell,m$. The prefactors of the Laplacian harmonics are defined similarly, but with an added factor of $\delta_{TE}$. Here, $T$ is a fourth ``type'' index, similar to the parity ($p$), order ($\ell$), and degree ($m$) indices, that takes values of $M$ (magnetic type) or $E$ (electric type) to describe whether the characteristic field profiles of each mode resemble those of magnetic or electric multipoles. Only the electric longitudinal modes $\mathbf{L}_{p\ell m}(\mathbf{r})$ are nonzero such that we won't bother to invent a second symbol for magnetic longitudinal modes, but the transverse harmonics clearly come in two different types: magnetic modes $\mathbf{M}_{p\ell m}(\mathbf{r},k)$ ($T = M$) and electric modes $\mathbf{N}_{p\ell m}(\mathbf{r},k)$ ($T = E$).

Both the transverse and Laplacian modes also come with a fifth modifier, the ``region'' index, taking values of $<$ for \textit{interior} harmonics and $>$ for \textit{exterior} harmonics, although we have neglected append our definitions with yet another symbol. The interior harmonics, listed above as $\mathbf{L}$, $\mathbf{M}$, and $\mathbf{N}$, are generally used for $\mathbf{r}$ confined within some closed spherical surface such that $r<\infty$ always and $r = 0$ somwhere within the region. To describe fields in regions that do not include the origin, one can find the second set of solutions to the second-order wave equation PDE, which are the so-called \textit{exterior} vector spherical harmonics $\bm{\mathcal{L}}_{p\ell m}(\mathbf{r})$, $\bm{\mathcal{M}}_{p\ell m}(\mathbf{r},k)$, and $\bm{\mathcal{N}}_{p\ell m}(\mathbf{r},k)$. These harmonics are defined with the simple substitutions $j_\ell(kr)\to h_\ell(kr)$ and are explicitly given by
\begin{equation}
\begin{split}
\bm{\mathcal{L}}_{p\ell m}(\mathbf{r},k) &= \frac{1}{k}\nabla\left\{\sqrt{K_{\ell m}}\sqrt{\ell(\ell + 1)}h_\ell(kr)P_{\ell m}(\cos\theta)S_p(m\phi)\right\},\\
&= \sqrt{K_{\ell m}}\sqrt{\ell(\ell + 1)}\left[\frac{1}{k}\frac{\partial\{h_\ell(kr)\}}{\partial r}P_{\ell m}(\cos\theta)S_p(m\phi)\hat{\mathbf{r}} + \frac{1}{kr}h_\ell(kr)\frac{\partial P_{\ell m}(\cos\theta)}{\partial\theta}S_p(m\phi)\hat{\bm{\theta}}\right.\\
&\left. + \frac{(-1)^{p+1}m}{kr\sin\theta}h_\ell(kr)P_{\ell m}(\cos\theta)S_{p+1}(m\phi)\hat{\bm{\phi}}\right],\\[1.0em]
\bm{\mathcal{M}}_{p\ell m}(\mathbf{r},k) &= \sqrt{K_{\ell m}}\left[\frac{(-1)^{p + 1}m}{\sin\theta}h_\ell(kr)P_{\ell m}(\cos\theta)S_{p+1}(m\phi)\hat{\bm{\theta}} - h_\ell(kr)\frac{\partial P_{\ell m}(\cos\theta)}{\partial\theta}S_p(m\phi)\hat{\bm{\phi}}\right],\\[0.5em]
\bm{\mathcal{N}}_{p\ell m}(\mathbf{r},k) &= \sqrt{K_{\ell m}}\left[\frac{\ell(\ell + 1)}{kr}h_\ell(kr)P_{\ell m}(\cos\theta)S_p(m\phi)\hat{\mathbf{r}}\right.\\
&+ \left.\frac{1}{kr}\frac{\partial\{rh_\ell(kr)\}}{\partial r}\left(\frac{\partial P_{\ell m}(\cos\theta)}{\partial\theta}S_p(m\phi)\hat{\bm{\theta}} + \frac{(-1)^{p+1}m}{\sin\theta}P_{\ell m}(\cos\theta)S_{p + 1}(m\phi)\hat{\bm{\phi}}\right)\right].
\end{split}
\end{equation}
The functions $h_\ell(kr) = j_\ell(kr) + \mathrm{i}y_\ell(kr)$ are the spherical Hankel functions of the first kind and involve the spherical Bessel functions of both the first kind ($j_\ell[kr]$) and the second kind, $y_\ell(kr)$, the latter of which are irregular at $r\to0$, justifying their exclusion from use in all regions that include the point $r = 0$.

We would like the notation for the harmonics to be as succinct as possible. Therefore, for shorthand, we can define
\begin{equation}
\begin{split}
\mathbf{X}_{Tp\ell m}(\mathbf{r},k) &= 
\begin{cases}
\mathbf{L}_{p\ell m}(\mathbf{r},k), & T=L,\\
\mathbf{M}_{p\ell m}(\mathbf{r},k), & T=M,\\
\mathbf{N}_{p\ell m}(\mathbf{r},k), & T=E;
\end{cases}\\[1.0em]
\bm{\mathcal{X}}_{Tp\ell m}(\mathbf{r},k) &=
\begin{cases}
\bm{\mathcal{L}}_{p\ell m}(\mathbf{r},k), & T=L,\\
\bm{\mathcal{M}}_{p\ell m}(\mathbf{r},k), & T=M,\\
\bm{\mathcal{N}}_{p\ell m}(\mathbf{r},k), & T=E.
\end{cases}
\end{split}
\end{equation}
Here, we have finally introduced the fourth ``type'' index to the harmonic notation, with $L$, $M$, and $E$ denoting the $L$ongitudinal, transverse $M$agnetic, and transverse $E$lectric harmonics, respectively. We can condense the notation further by writing the string of four indices as $\bm{\alpha} = (T,p,\ell,m)$, such that our harmonics can now be written as $\mathbf{X}_{\bm{\alpha}}(\mathbf{r},k)$ and $\bm{\mathcal{X}}_{\bm{\alpha}}(\mathbf{r},k)$.

The symmetries of the vector spherical harmonics are such that
\begin{equation}
\begin{split}
\mathbf{L}_{p\ell m}(\mathbf{r},k) &= (-1)^{\ell+1}\mathbf{L}_{p\ell m}(\mathbf{r},-k),\\
\mathbf{M}_{p\ell m}^*(\mathbf{r},k) &= (-1)^\ell\mathbf{M}_{p\ell m}(\mathbf{r},-k),\\
\mathbf{N}_{p\ell m}^*(\mathbf{r},k) &= (-1)^{\ell + 1}\mathbf{N}_{p\ell m}(\mathbf{r},-k),
\end{split}
\end{equation}
with identical relations for the respective exterior harmonics. Further, the transverse harmonics obey the duality
\begin{equation}
\begin{split}
\nabla\times\mathbf{M}_{p\ell m}(\mathbf{r},k) &= k\mathbf{N}_{p\ell m}(\mathbf{r},k),\\
\nabla\times\mathbf{N}_{p\ell m}(\mathbf{r},k) &= k\mathbf{M}_{p\ell m}(\mathbf{r},k),
\end{split}
\end{equation}
and similar for the exterior transverse harmonics, such that a second set of harmonics can be defined as 
\begin{equation}
\begin{split}
\mathbf{Y}_{Lp\ell m}(\mathbf{r},t) &= 0,\\
\mathbf{Y}_{Mp\ell m}(\mathbf{r},k) &= \mathbf{N}_{p\ell m}(\mathbf{r},t),\\
\mathbf{Y}_{Ep\ell m}(\mathbf{r},k) &= \mathbf{M}_{p\ell m}(\mathbf{r},t).
\end{split}
\end{equation}
More succinctly,
\begin{equation}
\begin{split}
\mathbf{Y}_{\bm{\alpha}}(\mathbf{r},k) &= \frac{1}{k}\nabla\times\mathbf{X}_{\bm{\alpha}}(\mathbf{r},k),\\
\bm{\mathcal{Y}}_{\bm{\alpha}}(\mathbf{r},k) &= \frac{1}{k}\nabla\times\bm{\mathcal{X}}_{\bm{\alpha}}(\mathbf{r},k),
\end{split}
\end{equation}
where $\bm{\mathcal{Y}}(\mathbf{r},k)$ are the corresponding exterior harmonics to $\mathbf{Y}_{\bm{\alpha}}(\mathbf{r},k)$.

Due to their regularity at all points $\mathbf{r}$, the interior harmonics obey the orthogonality condition
\begin{equation}\label{eq:vectorSphericalHarmonicOrthogonality}
\int \mathbf{X}_{\bm{\alpha}}(\mathbf{r},k)\cdot\mathbf{X}_{\bm{\alpha}'}(\mathbf{r},k')\;\mathrm{d}^3\mathbf{r} = \frac{2\pi^2}{k^2}\delta(k - k')(1 - \delta_{p1}\delta_{m0})\delta_{TT'}\delta_{pp'}\delta_{\ell\ell'}\delta_{mm'}\\
\end{equation}
for $k,k' > 0$ and integration across the entire universe. The transverse magnetic harmonics also satisfy convenient orthogonality relations when integrated over finite regions in $r$, with
\begin{equation}\label{eq:vectorSphericalHarmonicOrthogonalityConditionFinite}
\begin{split}
\int_0^{2\pi}\int_0^\pi\int_a^b\mathbf{L}_{p\ell m}(\mathbf{r},k)\cdot\mathbf{M}_{p'\ell'm'}(\mathbf{r},k')r^2\sin\theta\;\mathrm{d}r\,\mathrm{d}\theta\,\mathrm{d}\phi &= 0,\\
\int_0^{2\pi}\int_0^\pi\int_a^b\mathbf{N}_{p\ell m}(\mathbf{r},k)\cdot\mathbf{M}_{p'\ell'm'}(\mathbf{r},k')r^2\sin\theta\;\mathrm{d}r\,\mathrm{d}\theta\,\mathrm{d}\phi &= 0.
\end{split}
\end{equation}
However, finite spherical integrals of the longitudinal and transverse electric harmonics produce
\begin{equation}\label{eq:vectorSphericalHarmonicOrthogonalityConditionFiniteMixed}
\begin{split}
\int_0^{2\pi}\int_0^{\pi}\int_a^b\mathbf{L}_{p\ell m}(\mathbf{r},k)\cdot\mathbf{N}_{p'\ell'm'}(\mathbf{r},k')r^2\sin\theta\;\mathrm{d}r\,\mathrm{d}\theta\,\mathrm{d}\phi &= 4\pi(1 - \delta_{p1}\delta_{m0})\sqrt{\ell(\ell + 1)}\delta_{pp'}\delta_{\ell\ell'}\delta_{mm'}\\[-0.5em]
&\qquad\times\frac{1}{kk'}\left[bj_\ell(kb)j_\ell(k'b) - aj_\ell(ka)j_\ell(k'a)\right],
\end{split}
\end{equation}
which can be seen to produce zero in the limits $a\to0$, $b\to\infty$ as $\lim_{r\to0}rj_\ell(kr)j_\ell(k'r) = \lim_{r\to\infty}rj_\ell(kr)j_\ell(k'r) = 0$ for all real, positive $k$ and $k'$ if $\ell$ is an integer greater than 0. Finally, the harmonics of the same type are normalized such that integrals over finite spherical regions produce
\begin{equation}
\int_0^{2\pi}\int_0^\pi\int_a^b\mathbf{X}_{Tp\ell m}(\mathbf{r},k)\cdot\mathbf{X}_{Tp'\ell'm'}(\mathbf{r},k')r^2\sin\theta\;\mathrm{d}r\,\mathrm{d}\theta\,\mathrm{d}\phi = 4\pi(1 - \delta_{p1}\delta_{m0})\delta_{pp'}\delta_{\ell\ell'}\delta_{mm'}R_{T\ell}^{\ll}(k,k';a,b),
\end{equation}
where $R_{T\ell}^\ll(k,k';a,b)$ are radial integral functions defined by
\begin{equation}\label{eq:finiteRadialIntegrals}
\begin{split}
R_{M\ell}^\ll(k,k';a,b) &= \int_a^bj_\ell(kr)j_\ell(k'r)r^2\;\mathrm{d}r\\
&= \left.\frac{r^2}{k^2 - k'^2}\left[k'j_{\ell - 1}(k'r)j_\ell(kr) - kj_{\ell - 1}(kr)j_\ell(k'r)\right]\right|_a^b,\\[0.5em]
R_{E\ell}^\ll(k,k';a,b) &= \frac{\ell + 1}{2\ell + 1}R_{M,\ell - 1}^\ll(k,k';a,b) + \frac{\ell}{2\ell + 1}R_{M,\ell + 1}^\ll(k,k';a,b),\\
R_{L\ell}^\ll(k,k';a,b) &= \frac{\ell + 1}{2\ell + 1}R_{M,\ell + 1}^\ll(k,k';a,b) + \frac{\ell}{2\ell + 1}R_{M,\ell - 1}^\ll(k,k';a,b).
\end{split}
\end{equation}
The superscript $\ll$ indicates that the integral involves two spherical Bessel functions of the first kind and thus arises from the integration of two interior harmonics. 

An important case of this integral occurs when $b\to a$ and $a\to0$, i.e. during integration over a finite sphere. In this case, the characteristic wavenumbers are often defined such that $k = z_{\ell n}/a$ and $k' = z_{\ell n'}/a$, where $z_{\ell n}$ is the $n^\mathrm{th}$ unitless root of $j_\ell(x)$ with $n = 1,2,\ldots$, or as $k = w_{\ell n}/a$ and $k' = w_{\ell n'}/a$, where $w_{\ell n}$ is the $n^\mathrm{th}$ unitless root of $\partial j_\ell(x)/\partial x$ with $n = 1,2,\ldots$. With these values of $k$ and $k'$ fixed, we can use the identity $j_{\ell - 1}(w_{\ell n}) = (\ell + 1)j_\ell(w_{\ell n})/w_{\ell n}$ to see that $R_{M\ell}^\ll(z_{\ell n}/a,z_{\ell n'}/a;0,a) = R_{M\ell}^\ll(w_{\ell n}/a,w_{\ell n'}/a;0,a) = 0$ for $n\neq n'$. Further, with
\begin{equation}
\lim_{k'\to k}R_{M\ell}^\ll(k,k';0,a) = \frac{a^3}{2}\left[j_\ell^2(ka) - j_{\ell - 1}(ka)j_{\ell + 1}(ka)\right]
\end{equation}
by L'Hopital's rule, we can use the identity $j_{\ell - 1}(z_{n\ell}) = -j_{\ell + 1}(z_{n\ell})$ to say
\begin{equation}
R_{M\ell}^\ll\left(\frac{z_{\ell n}}{a},\frac{z_{\ell n'}}{a};0,a\right) = R_{M,\ell-1}^\ll\left(\frac{z_{\ell n}}{a},\frac{z_{\ell n'}}{a};0,a\right) = R_{M,\ell+1}^\ll\left(\frac{z_{\ell n}}{a},\frac{z_{\ell n'}}{a};0,a\right) = \delta_{nn'}\frac{a^3}{2}j_{\ell + 1}^2(z_{\ell n}).
\end{equation}
Further, we can use the identities $j_{\ell + 1}(w_{\ell n}) = \ell j_\ell(w_{\ell n})/w_{\ell n}$, $j_{\ell - 2}(x) = (2\ell - 1)j_{\ell - 1}(x)/x - j_\ell(x)$, and $j_{\ell + 2}(x) = (2\ell + 3)j_{\ell + 1}(x)/x - j_\ell(x)$ to say
\begin{equation}
\begin{split}
R_{M\ell}^\ll\left(\frac{w_{\ell n}}{a},\frac{w_{\ell n'}}{a};0,a\right) &= \delta_{nn'}\frac{a^3}{2}j_\ell^2(w_{\ell n})\left(1 - \frac{\ell(\ell + 1)}{w_{\ell n}^2}\right),\\
R_{M,\ell - 1}^\ll\left(\frac{w_{\ell n}}{a},\frac{w_{\ell n'}}{a};0,a\right) &= \delta_{nn'}\frac{a^3}{2}j_{\ell}^2(w_{\ell n})\left(1 - \frac{\ell^2 - \ell-2}{w_{\ell n}^2}\right),\\
R_{M,\ell - 1}^\ll\left(\frac{w_{\ell n}}{a},\frac{w_{\ell n'}}{a};0,a\right) &= \delta_{nn'}\frac{a^3}{2}j_{\ell}^2(w_{\ell n})\left(1 - \frac{\ell(\ell + 3)}{w_{\ell n}^2}\right).
\end{split}
\end{equation}
These expressions can be used to construct the useful identities
\begin{equation}
\begin{split}
R_{E\ell}^\ll\left(\frac{z_{\ell n}}{a},\frac{z_{\ell n'}}{a};0,a\right) &= \delta_{nn'}\frac{a^3}{2}j_{\ell + 1}^2(z_{\ell n}),\\[0.5em]
R_{L\ell}^\ll\left(\frac{w_{\ell n}}{a},\frac{w_{\ell n'}}{a};0,a\right) &= \delta_{nn'}\frac{a^3}{2}\left(1 - \frac{\ell(\ell + 1)}{w_{\ell n}^2}\right)j_\ell^2(w_{\ell n}).
\end{split}
\end{equation}
Therefore, letting $k_{Mp\ell mn} = k_{Ep\ell mn} = z_{\ell n}/a$ and $k_{Lp\ell m n} = w_{\ell n}/a$, we have
\begin{equation}
\begin{split}
\int_0^{2\pi}\int_0^\pi\int_0^a\mathbf{X}_{\bm{\alpha}}(\mathbf{r},k_{\bm{\alpha}n})&\cdot\mathbf{X}_{\bm{\alpha}'}(\mathbf{r},k_{\bm{\alpha}'n'})r^2\sin\theta\;\mathrm{d}r\;\mathrm{d}\theta\;\mathrm{d}\phi = 2\pi a^3(1 - \delta_{p1}\delta_{m0})\delta_{\bm{\alpha}\bm{\alpha}'}\delta_{nn'}\\
&\qquad\times\left[(\delta_{TM} + \delta_{TE})j_{\ell + 1}^2(z_{\ell n}) + \delta_{TL}\left(1 - \frac{\ell(\ell + 1)}{w_{\ell n}^2}\right)j_{\ell}^2(w_{\ell n})\right].
\end{split}
\end{equation}
In related fashion, using the identity
\begin{equation}
\nabla\cdot\mathbf{L}_{p\ell m}(\mathbf{r},k) = -k\sqrt{K_{\ell m}}\sqrt{\ell(\ell + 1)}j_\ell(kr)P_{\ell m}(\cos\theta)S_p(m\phi),
\end{equation}
we have
\begin{equation}
\begin{split}
\int_0^{2\pi}\int_0^\pi\int_0^a&\left[\nabla\cdot\mathbf{L}_{p\ell m}(\mathbf{r},k_{L\ell n})\right]\left[\nabla\cdot\mathbf{L}_{p'\ell'm'}(\mathbf{r},k_{L\ell'n'})\right]r^2\sin\theta\;\mathrm{d}r\;\mathrm{d}\theta\;\mathrm{d}\phi\\
&= 2\pi a^3(1 - \delta_{p1}\delta_{m0})\delta_{pp'}\delta_{\ell\ell'}\delta_{mm'}\delta_{nn'}k_{L\ell n}^2j_\ell^2(k_{L\ell n}a)\left(1 - \frac{\ell(\ell + 1)}{(k_{L\ell n}a)^2}\right).
\end{split}
\end{equation}

In addition to our vector spherical harmonics, the expansion of the Coulomb Green's function
\begin{equation}
1/|\mathbf{r} - \mathbf{r}'| = \sum_{p\ell m}[f_{p\ell m}^>(\mathbf{r})f_{p\ell m}^<(\mathbf{r}')\Theta(r - r') + f_{p\ell m}^<(\mathbf{r})f_{p\ell m}^>(\mathbf{r}')\Theta(r' - r)]
\end{equation}
into \textit{scalar} spherical harmonics
\begin{equation}\label{eq:scalarHarmonics}
f_{p\ell m}^i(\mathbf{r}) = \sqrt{K_{\ell m}}\sqrt{\frac{\ell(\ell + 1)}{2\ell + 1}}P_{\ell m}(\cos\theta)S_p(m\phi)\times
\begin{cases}
r^\ell, & i = \, <,\\
\dfrac{1}{r^{\ell + 1}}, & i = \,>
\end{cases}
\end{equation}
motivates the calculation of a further set of integral identities. In particular, the radial integrals
\begin{equation}
\begin{split}
\int_a^b j_\ell(kr)r^{\ell+2}\;\mathrm{d}r &= \frac{b^{\ell + 2}}{k}j_{\ell + 1}(kb) - \frac{a^{\ell + 2}}{k}j_{\ell + 1}(ka),\\
\int_a^b j_\ell(kr)r^{-\ell + 1}\;\mathrm{d}r &=  \frac{1}{ka^{\ell - 1}}j_{\ell - 1}(ka) - \frac{1}{kb^{\ell -1}}j_{\ell - 1}(kb).\\
\end{split}
\end{equation}
are useful to define, as are the volume integrals
\begin{equation}
\int_{r<a}\mathbf{X}_{\bm{\beta}}(\mathbf{r},k_{\bm{\beta}n})\cdot\nabla f_{p'\ell'm'}^<(\mathbf{r})\;\mathrm{d}^3\mathbf{r} = 4\pi\delta_{TL}(1 - \delta_{p1}\delta_{m0})\delta_{pp'}\delta_{\ell\ell'}\delta_{mm'}\frac{a^{\ell + 2}}{\sqrt{2\ell + 1}}j_{\ell + 1}(w_{\ell n}).
\end{equation}
and
\begin{equation}\label{eq:laplacianHarmonicOrthogonality}
\begin{split}
\int_0^{2\pi}\int_0^\pi\int_a^b\nabla f_{p\ell m}^<(\mathbf{r})\cdot\nabla f_{p'\ell'm'}^<(\mathbf{r})r^2\sin\theta\;\mathrm{d}r\,\mathrm{d}\theta\,\mathrm{d}\phi &= 4\pi\frac{\ell}{2\ell + 1}\left(b^{2\ell + 1} - a^{2\ell + 1}\right)\delta_{pp'}\delta_{\ell\ell'}\delta_{mm'},\\
\int_0^{2\pi}\int_0^\pi\int_a^b\nabla f_{p\ell m}^>(\mathbf{r})\cdot\nabla f_{p'\ell'm'}^>(\mathbf{r})r^2\sin\theta\;\mathrm{d}r\,\mathrm{d}\theta\,\mathrm{d}\phi &= -4\pi\frac{\ell + 1}{2\ell + 1}\left(b^{-2\ell - 1} - a^{-2\ell - 1}\right)\delta_{pp'}\delta_{\ell\ell'}\delta_{mm'},\\
\int_0^{2\pi}\int_0^\pi\int_a^b\nabla f_{p\ell m}^<(\mathbf{r})\cdot\nabla f_{p'\ell'm'}^>(\mathbf{r})r^2\sin\theta\;\mathrm{d}r\,\mathrm{d}\theta\,\mathrm{d}\phi &= 0.
\end{split}
\end{equation}










% \noindent\rule{\textwidth}{0.5pt}

% \begin{equation}
% \begin{split}
% \mathbf{Y}_{\bm{\alpha}}(\mathbf{r},k) &= \frac{1}{k}\nabla\times\mathbf{X}_{\bm{\alpha}}(\mathbf{r},k),\\
% \bm{\mathcal{Y}}_{\bm{\alpha}}(\mathbf{r},k) &= \frac{1}{k}\nabla\times\bm{\mathcal{X}}_{\bm{\alpha}}(\mathbf{r},k).
% \end{split}
% \end{equation}
% These obey the same conjugation symmetry as $\mathbf{X}_{\bm{\alpha}}$ and $\bm{\mathcal{X}}_{\bm{\alpha}}$ but with an extra factor of $-1$ such that
% \begin{equation}
% \begin{split}
% \bm{\mathcal{X}}_{\bm{\alpha}}^*(\mathbf{r},k) &= 
% \begin{cases}
% (-1)^{\ell}\bm{\mathcal{X}}_{\bm{\alpha}}(\mathbf{r},-k), & T = M,\\
% (-1)^{\ell + 1}\bm{\mathcal{X}}_{\bm{\alpha}}(\mathbf{r},-k), & T = E;\\
% \end{cases}\\
% \bm{\mathcal{Y}}_{\bm{\alpha}}^*(\mathbf{r},k) &= 
% \begin{cases}
% (-1)^{\ell + 1}\bm{\mathcal{Y}}_{\bm{\alpha}}(\mathbf{r},-k), & T = M,\\
% (-1)^{\ell}\bm{\mathcal{Y}}_{\bm{\alpha}}(\mathbf{r},-k), & T = E;\\
% \end{cases}
% \end{split}
% \end{equation}
% and similar for the interior transverse harmonics. The longitudinal vector spherical harmonics obey the symmetries
% \begin{equation}
% \begin{split}
% \mathbf{L}_{p\ell m}(\mathbf{r},k) &= (-1)^{\ell+1}\mathbf{L}_{p\ell m}(\mathbf{r},-k),\\
% \bm{\mathcal{L}}_{p\ell m}^*(\mathbf{r},k) &= (-1)^{\ell+1}\bm{\mathcal{L}}_{p\ell m}(\mathbf{r},-k).
% \end{split}
% \end{equation}

% Due to their regularity at all points $\mathbf{r}$, the transverse interior harmonics obey the orthogonality condition
% \begin{equation}\label{eq:vectorSphericalHarmonicOrthogonality}
% \int \mathbf{X}_{\bm{\alpha}}(\mathbf{r},k)\cdot\mathbf{X}_{\bm{\alpha}'}(\mathbf{r},k')\;\mathrm{d}^3\mathbf{r} = \frac{2\pi^2}{k^2}\delta(k - k')(1 - \delta_{p1}\delta_{m0})\delta_{TT'}\delta_{pp'}\delta_{\ell\ell'}\delta_{mm'}\\
% \end{equation}
% for $k,k' > 0$ and integration across the entire universe. The simpler orthogonality relations that contribute to this simple result are given in Appendix \ref{sec:simpleOrthogonality}. The longitudinal interior harmonics obey the orthogonality condition
% \begin{equation}\label{eq:longitudinalHarmonicOrthogonality}
% \int\mathbf{L}_{\bm{\alpha}}(\mathbf{r},k)\cdot\mathbf{L}_{\bm{\alpha}'}(\mathbf{r},k')\;\mathrm{d}^3\mathbf{r} = \frac{2\pi^2}{k^2}\delta(k - k')(1 - \delta_{p1}\delta_{m0})\delta_{TE}\delta_{TT'}\delta_{pp'}\delta_{\ell\ell'}\delta_{mm'}.
% \end{equation}
% Accordingly, the transverse and longitudinal harmonics also satisfy convenient orthogonality relations when integrated over finite regions in $r$. Explicitly,
% \begin{equation}\label{eq:vectorSphericalHarmonicOrthogonalityConditionFinite}
% \begin{split}
% \int_0^{2\pi}\int_0^\pi\int_a^b\mathbf{X}_{\bm{\alpha}}(\mathbf{r},k)\cdot\mathbf{X}_{\bm{\alpha}'}(\mathbf{r},k')r^2\sin\theta\;\mathrm{d}r\,\mathrm{d}\theta\,\mathrm{d}\phi &= 4\pi(1 - \delta_{p1}\delta_{m0})\delta_{\bm{\alpha}\bm{\alpha}'}R_{T\ell}^{\ll}(k,k';a,b),\\
% \int_0^{2\pi}\int_0^\pi\int_a^b\mathbf{L}_{\bm{\alpha}}(\mathbf{r},k)\cdot\mathbf{L}_{\bm{\alpha}'}(\mathbf{r},k')r^2\sin\theta\;\mathrm{d}r\,\mathrm{d}\theta\,\mathrm{d}\phi &= 4\pi(1 - \delta_{p1}\delta_{m0})\delta_{\bm{\alpha}\bm{\alpha}'}R_{L\ell}^{\ll}(k,k';a,b)
% \end{split}
% \end{equation}
% where $R_{T\ell}^\ll(k,k';a,b)$ and $R_{L\ell}^\ll(k,k';a,b)$ are radial integral functions defined by
% \begin{equation}\label{eq:finiteRadialIntegrals}
% \begin{split}
% R_{M\ell}^\ll(k,k';a,b) &= \int_a^bj_\ell(kr)j_\ell(k'r)r^2\;\mathrm{d}r\\
% &= \left.\frac{r^2}{k^2 - k'^2}\left[k'j_{\ell - 1}(k'r)j_\ell(kr) - kj_{\ell - 1}(kr)j_\ell(k'r)\right]\right|_a^b,\\[0.5em]
% R_{E\ell}^\ll(k,k';a,b) &= \frac{\ell + 1}{2\ell + 1}R_{M,\ell - 1}^\ll(k,k';a,b) + \frac{\ell}{2\ell + 1}R_{M,\ell + 1}^\ll(k,k';a,b),\\
% R_{L\ell}^\ll(k,k';a,b) &= \frac{\ell + 1}{2\ell + 1}R_{M,\ell + 1}^\ll(k,k';a,b) + \frac{\ell}{2\ell + 1}R_{M,\ell - 1}^\ll(k,k';a,b).
% \end{split}
% \end{equation}
% The superscript $\ll$ indicates that the integral involves two spherical Bessel functions of the first kind and thus arises from the integration of two interior harmonics. 

% From their opposing Helmholtz symmetries, we can conclude that integrals of the inner product of $\mathbf{M}$ or $\mathbf{N}$ and $\mathbf{L}$ over all space are null. However, the same is not true when integrating over finite regions of space. Again choosing the region of integration to be finite in $r$, we can see that
% \begin{equation}\label{eq:vectorSphericalHarmonicOrthogonalityConditionFiniteMixed}
% \begin{split}
% \int_0^{2\pi}\int_0^{\pi}\int_a^b\mathbf{L}_{\bm{\alpha}}(\mathbf{r},k)\cdot\mathbf{X}_{\bm{\alpha}'}(\mathbf{r},k')r^2\sin\theta\;\mathrm{d}r\,\mathrm{d}\theta\,\mathrm{d}\phi &= 4\pi(1 - \delta_{p1}\delta_{m0})\sqrt{\ell(\ell + 1)}\delta_{TE}\delta_{T'E}\delta_{pp'}\delta_{\ell\ell'}\delta_{mm'}\\[-0.5em]
% &\qquad\times\frac{1}{kk'}\left[bj_\ell(kb)j_\ell(k'b) - aj_\ell(ka)j_\ell(k'a)\right].
% \end{split}
% \end{equation}
% Note that integration of the longitudinal and transverse magnetic harmonics always produces zero, even when carried out over finite regions in $r$. Integration of the longitudinal and transverse electric harmonics can be seen to produce zero in the limit where the integration region is allowed to become infinite. Explicitly, with $a\to0$ and $b\to\infty$, one can see that $\lim_{r\to0}rj_\ell(kr)j_\ell(k'r) = \lim_{r\to\infty}rj_\ell(kr)j_\ell(k'r) = 0$ for all real, positive $k$ and $k'$ if $\ell$ is an integer greater than 0.

% An important case of this integral occurs when $b\to a$ and $a\to0$, i.e. during integration over a finite sphere. In this case, the characteristic wavenumbers are often defined such that $k = z_{\ell n}/a$ and $k' = z_{\ell n'}/a$, where $z_{\ell n}$ is the $n^\mathrm{th}$ unitless root of $j_\ell(x)$ with $n = 1,2,\ldots$. With these values of $k$ and $k'$ fixed, we can easily see that $R_{M\ell}^\ll(z_{\ell n}/a,z_{\ell n'}/a;0,a) = 0$ for $n\neq n'$. Further, with
% \begin{equation}
% \lim_{k'\to k}R_{M\ell}^\ll(k,k';0,a) = \frac{a^3}{2}\left[j_\ell^2(ka) - j_{\ell - 1}(ka)j_{\ell + 1}(ka)\right]
% \end{equation}
% by L'Hopital's rule, we can see that
% \begin{equation}
% R_{M\ell}^\ll\left(\frac{z_{\ell n}}{a},\frac{z_{\ell n'}}{a};0,a\right) = -\delta_{nn'}\frac{a^3}{2}j_{\ell - 1}(z_{\ell n})j_{\ell + 1}(z_{\ell n}).
% \end{equation}
% Finally, using the recursion identity $j_\ell(x) = [x/(2\ell + 1)][j_{\ell - 1}(x) + j_{\ell 1}(x)]$, we can see that $j_{\ell - 1}(z_{n\ell}) = -j_{\ell + 1}(z_{n\ell})$. Therefore, letting $k_{\bm{\alpha}n} = z_{\ell n}/a$ and $k_{\bm{\alpha}'n'} = z_{\ell'n'}/a$, we have
% \begin{equation}
% \begin{split}
% \int_0^{2\pi}\int_0^\pi\int_0^a\mathbf{X}_{\bm{\alpha}}(\mathbf{r},k_{\bm{\alpha}n})\cdot\mathbf{X}_{\bm{\alpha}'}(\mathbf{r},k_{\bm{\alpha}'n'})r^2\sin\theta\;\mathrm{d}r\;\mathrm{d}\theta\;\mathrm{d}\phi &= 2\pi a^3(1 - \delta_{p1}\delta_{m0})\delta_{\bm{\alpha}\bm{\alpha}'}\delta_{nn'}j_{\ell + 1}^2(z_{\ell n}),\\
% \int_0^{2\pi}\int_0^\pi\int_0^a\mathbf{L}_{\bm{\alpha}}(\mathbf{r},k_{\bm{\alpha}n})\cdot\mathbf{L}_{\bm{\alpha}'}(\mathbf{r},k_{\bm{\alpha}'n'})r^2\sin\theta\;\mathrm{d}r\;\mathrm{d}\theta\;\mathrm{d}\phi &= 2\pi a^3(1 - \delta_{p1}\delta_{m0})\delta_{\bm{\alpha}\bm{\alpha}'}\delta_{nn'}j_{\ell + 1}^2(z_{\ell n}).
% \end{split}
% \end{equation}

% \noindent\rule{\textwidth}{0.5pt}\\

% For calculations involving one or more \textit{exterior} harmonics, one simply replaces one or both of the spherical Bessel functions with spherical Hankel functions. More explicitly, if we define 
% \begin{equation}
% z_\ell^i(x) = 
% \begin{cases}
% j_\ell(x), & i = \,<,\\
% h_\ell(x), & i = \,>,
% \end{cases}
% \end{equation}
% then we can generalize Eq. \eqref{eq:finiteRadialIntegrals} with
% \begin{equation}
% \begin{split}
% R_{M\ell}^{ij}(k,k';a,b) &= \int_a^bz_\ell^i(kr)z_\ell^j(k'r)r^2\;\mathrm{d}r\\
% &= \lim_{\kappa\to k'}\frac{r^2}{k^2 - \kappa^2}\left[\kappa z_{\ell-1}^j(\kappa r)z_{\ell}^i(kr) - kz_{\ell-1}^i(kr)z_{\ell}^j(\kappa r)\right]_a^b
% \end{split}
% \end{equation}
% and $R_{E\ell}^{ij}(k,k';a,b) = (\ell+1)R_{M,\ell - 1}^{ij}(k,k';a,b)/(2\ell + 1) + \ell R_{M,\ell + 1}^{ij}(k,k';a,b)/(2\ell + 1)$. For $k'\neq k$, the limits within both the magnetic and electric radial integrals can be evaluated with simple substitution, but for $k'=k$ it is necessary to evaluate the limit carefully.




% and Laplacian vector spherical harmonics
% \begin{equation}
% \begin{split}
% \mathbf{F}_{Tp\ell m}(\mathbf{r}) &= \nabla\left\{\delta_{TE}\sqrt{(2 - \delta_{m0})\frac{(\ell - m)!}{(\ell + m)!}}r^\ell P_{\ell m}(\cos\theta)S_p(m\phi)\right\},\\
% % \nabla f_{Tp\ell m}^{>}(\mathbf{r}) &= \nabla\left\{\delta_{TE}\sqrt{(2 - \delta_{m0})\frac{(\ell - m)!}{(\ell + m)!}}\frac{1}{r^{\ell + 1}}P_{\ell m}(\cos\theta)S_p(m\phi)\right\},
% &= \delta_{TE}\sqrt{(2 - \delta_{m0})\frac{(\ell - m)!}{(\ell + m)!}}\left[ \vphantom{\frac{(-1)^{p+1}}{\sin\theta}} \ell r^{\ell - 1}P_{\ell m}(\cos\theta)S_p(m\phi)\hat{\mathbf{r}}\right.\\
% &\qquad+ \left. r^{\ell - 1}\frac{\partial P_{\ell m}(\cos\theta)}{\partial \theta}S_p(m\phi)\hat{\bm{\theta}} + \frac{(-1)^{p+1}m}{\sin\theta}r^{\ell - 1}P_{\ell m}(\cos\theta)S_{p+1}(m\phi)\hat{\bm{\phi}} \right]
% \end{split}
% \end{equation}
% which have neither curl ($\nabla\times\mathbf{F}_{Tp\ell m}(\mathbf{r}) = 0$) or divergence ($\nabla\cdot\mathbf{F}_{Tp\ell m}(\mathbf{r}) = 0$). 

% \begin{equation}
% \begin{split}
% \mathbf{M}_{p\ell m}(\mathbf{r},k) &= \sqrt{K_{\ell m}}\left[\frac{(-1)^{p + 1}m}{\sin\theta}j_\ell(kr)P_{\ell m}(\cos\theta)S_{p+1}(m\phi)\hat{\bm{\theta}} - j_\ell(kr)\frac{\partial P_{\ell m}(\cos\theta)}{\partial\theta}S_p(m\phi)\hat{\bm{\phi}}\right],\\[0.5em]
% \mathbf{N}_{p\ell m}(\mathbf{r},k) &= \sqrt{K_{\ell m}}\left[\frac{\ell(\ell + 1)}{kr}j_\ell(kr)P_{\ell m}(\cos\theta)S_p(m\phi)\hat{\mathbf{r}}\right.\\
% &+ \left.\frac{1}{kr}\frac{\partial\{rj_\ell(kr)\}}{\partial r}\left(\frac{\partial P_{\ell m}(\cos\theta)}{\partial\theta}S_p(m\phi)\hat{\bm{\theta}} + \frac{(-1)^{p+1}m}{\sin\theta}P_{\ell m}(\cos\theta)S_{p + 1}(m\phi)\hat{\bm{\phi}}\right)\right],
% \end{split}
% \end{equation}


% \begin{equation}
% \begin{split}
% \bm{\mathcal{F}}_{Tp\ell m}(\mathbf{r},k) &= \nabla\left\{\delta_{TE}\sqrt{(2 - \delta_{m0})\frac{(\ell - m)!}{(\ell + m)!}}\frac{1}{r^{\ell + 1}} P_{\ell m}(\cos\theta)S_p(m\phi)\right\},\\
% &= \delta_{TE}\sqrt{(2 - \delta_{m0})\frac{(\ell - m)!}{(\ell + m)!}}\left[(-\ell - 1)\frac{1}{r^{\ell + 2}}P_{\ell m}(\cos\theta)S_p(m\phi)\hat{\mathbf{r}}\right.\\
% &\qquad+ \left.\frac{1}{r^{\ell + 2}}\frac{\partial P_{\ell m}(\cos\theta)}{\partial \theta}S_p(m\phi)\hat{\bm{\theta}} + \frac{(-1)^{p+1}m}{\sin\theta}\frac{1}{r^{\ell + 2}}P_{\ell m}(\cos\theta)S_{p+1}(m\phi)\hat{\bm{\phi}} \right]
% \end{split}
% \end{equation}
 





%  The Laplacian harmonics are either irregular at infinity ($r^\ell$) or at the origin ($1/r^{\ell + 1}$) for $\ell > 0$, such that their orthogonality relations can only be defined across finite regions in $r$:
% \begin{equation}\label{eq:laplacianHarmonicOrthogonality}
% \begin{split}
% \int_0^{2\pi}\int_0^\pi\int_a^b\mathbf{F}_{\bm{\alpha}}(\mathbf{r})\cdot\mathbf{F}_{\bm{\alpha}'}(\mathbf{r})r^2\sin\theta\;\mathrm{d}r\,\mathrm{d}\theta\,\mathrm{d}\phi &= 4\pi\frac{\ell}{2\ell + 1}\left(b^{2\ell + 1} - a^{2\ell + 1}\right)\delta_{T\!E}\delta_{TT'}\delta_{pp'}\delta_{\ell\ell'}\delta_{mm'},\\
% \int_0^{2\pi}\int_0^\pi\int_a^b\bm{\mathcal{F}}_{\bm{\alpha}}(\mathbf{r})\cdot\bm{\mathcal{F}}_{\bm{\alpha}'}(\mathbf{r})r^2\sin\theta\;\mathrm{d}r\,\mathrm{d}\theta\,\mathrm{d}\phi &= -4\pi\frac{\ell + 1}{2\ell + 1}\left(b^{-2\ell - 1} - a^{-2\ell - 1}\right)\delta_{TE}\delta_{TT'}\delta_{pp'}\delta_{\ell\ell'}\delta_{mm'},\\
% \int_0^{2\pi}\int_0^\pi\int_a^b\mathbf{F}_{\bm{\alpha}}(\mathbf{r})\cdot\bm{\mathcal{F}}_{\bm{\alpha}'}(\mathbf{r})r^2\sin\theta\;\mathrm{d}r\,\mathrm{d}\theta\,\mathrm{d}\phi &= 0.
% \end{split}
% \end{equation}

% A useful mixed Laplacian-transverse orthogonality conditions also exists:
% \begin{equation}\label{eq:angularMixedOrthogonalityNf}
% \begin{split}
% \int_0^{2\pi}\int_0^\pi&\left[\hat{\mathbf{r}}\cdot\mathbf{N}_{p\ell m}(r,\theta,\phi;k)\right]f_{p'\ell'm'}^i(r,\theta,\phi)\sin\theta\;\mathrm{d}\theta\,\mathrm{d}\phi\\
% &= 4\pi\delta_{pp'}\delta_{\ell\ell'}\delta_{mm'}(1 - \delta_{p1}\delta_{m0})\sqrt{\frac{\ell(\ell + 1)}{2\ell + 1}}\frac{j_\ell(kr)}{kr}\left[r^\ell\delta_{i,<} + \frac{\delta_{i,>}}{r^{\ell + 1}}\right],
% \end{split}
% \end{equation}
% as does the mixed Laplacian-longitudinal orthogonality condition
% \begin{equation}\label{eq:angularMixedOrthogonalityLf}
% \begin{split}
% \int_0^{2\pi}\int_0^\pi&\left[\hat{\mathbf{r}}\cdot\mathbf{L}_{p\ell m}(r,\theta,\phi;k)\right]f_{p'\ell'm'}^i(r,\theta,\phi)\sin\theta\;\mathrm{d}\theta\,\mathrm{d}\phi\\
% &= 4\pi\delta_{pp'}\delta_{\ell\ell'}\delta_{mm'}(1 - \delta_{p1}\delta_{m0})\frac{1}{\sqrt{2\ell + 1}}\frac{1}{k}\frac{\partial j_\ell(kr)}{\partial r}\left[r^\ell\delta_{i,<} + \frac{\delta_{i,>}}{r^{\ell + 1}}\right].
% \end{split}
% \end{equation}

% \begin{equation}
% \begin{split}
% \int_{r<a}\mathbf{F}_{\bm{\alpha}}(\mathbf{r})\cdot\mathbf{X}_{\bm{\beta}}(\mathbf{r},k)\;\mathrm{d}^3\mathbf{r} &= \int_{r < a}\mathbf{X}_{\bm{\beta}}(\mathbf{r},k)\cdot\nabla f_{\bm{\alpha}}^<(\mathbf{r})\;\mathrm{d}^3\mathbf{r}\\
% &= \int_{r < a}\nabla\cdot\left\{\mathbf{X}_{\bm{\beta}}(\mathbf{r},k) f_{\bm{\alpha}}^<(\mathbf{r})\right\}\;\mathrm{d}^3\mathbf{r} - \int_{r < a}\nabla\cdot\mathbf{X}_{\bm{\beta}}(\mathbf{r},k)f_{\bm{\alpha}}^<(\mathbf{r})\;\mathrm{d}^3\mathbf{r}\\
% &= \int_0^{2\pi}\int_0^\pi\mathbf{X}_{\bm{\beta}}(a,\theta,\phi;k)f_{\bm{\alpha}}^<(a,\theta,\phi)\cdot \hat{\mathbf{r}}a^2\sin\theta\;\mathrm{d}\theta\,\mathrm{d}\phi - 0\\
% &= \delta_{TE}\delta_{T'E}\int_0^{2\pi}\int_0^\pi\left[\mathbf{N}_{p\ell m}(a,\theta,\phi;k)\cdot\hat{\mathbf{r}}\right]f_{p'\ell' m'}^<(a,\theta,\phi)a^2\sin\theta\;\mathrm{d}\theta\,\mathrm{d}\phi\\
% &= 4\pi\delta_{TE}\delta_{T'E}\delta_{pp'}\delta_{\ell\ell'}\delta_{mm'}(1 - \delta_{p1}\delta_{m0})\sqrt{\frac{\ell(\ell + 1)}{2\ell + 1}}\frac{j_\ell(ka)}{ka}a^{\ell+2}
% \end{split}
% \end{equation}

% \begin{equation}
% \begin{split}
% \int_{a<r<b}\mathbf{F}_{\bm{\alpha}}(\mathbf{r})\cdot\mathbf{L}_{\bm{\beta}}(\mathbf{r},k)\;\mathrm{d}^3\mathbf{r} &= 4\pi\delta_{TE}\delta_{T'E}\delta_{pp'}\delta_{\ell\ell'}\delta_{mm'}(1 - \delta_{p1}\delta_{m0})\frac{\ell}{\sqrt{2\ell+1}}\left(\frac{j_\ell(kb)}{kb}b^{\ell+2} - \frac{j_\ell(ka)}{ka}a^{\ell+2}\right),\\
% \int_{a<r<b}\bm{\mathcal{F}}_{\bm{\alpha}}(\mathbf{r})\cdot\mathbf{L}_{\bm{\beta}}(\mathbf{r},k)\;\mathrm{d}^3\mathbf{r} &= 4\pi\delta_{TE}\delta_{T'E}\delta_{pp'}\delta_{\ell\ell'}\delta_{mm'}(1 - \delta_{p1}\delta_{m0})\frac{\ell+1}{\sqrt{2\ell+1}}\frac{1}{k}\left(\frac{1}{a^\ell}j_\ell(ka) - \frac{1}{b^\ell}j_\ell(kb)\right).
% \end{split}
% \end{equation}

% The associated completeness relations to these orthogonality conditions can be derived from the Helmholtz expansion of $\bm{1}_2\delta(\mathbf{r} - \mathbf{r}')$, which states that 
% \begin{equation}\label{eq:vectorSphericalHarmonicCompleteness}
% \begin{split}
% \bm{1}_2\delta(\mathbf{r} - \mathbf{r}') &= \frac{1}{2\pi^2}\int_0^\infty\sum_{\bm{\alpha}}k^2\mathbf{X}_{\bm{\alpha}}(\mathbf{r},k)\mathbf{X}_{\bm{\alpha}}(\mathbf{r}',k)\;\mathrm{d}k\\
% &+ \frac{1}{4\pi}\sum_{\bm{\alpha}}\nabla\nabla'\left\{ f_{\bm{\alpha}}^>(\mathbf{r})f_{\bm{\alpha}}^<(\mathbf{r}')\Theta(r - r') +  f_{\bm{\alpha}}^<(\mathbf{r})f_{\bm{\alpha}}^>(\mathbf{r}')\Theta(r' - r)\right\}.
% \end{split}
% \end{equation}
% Here, the scalar harmonics 
% \begin{equation}\label{eq:scalarHarmonics}
% f_{\bm{\alpha}}^i(\mathbf{r}) = \delta_{TE}\sqrt{(2 - \delta_{m0})\frac{(\ell - m)!}{(\ell + m)!}}P_{\ell m}(\cos\theta)S_p(m\phi)\times
% \begin{cases}
% r^\ell, & i = \, <,\\
% \dfrac{1}{r^{\ell + 1}}, & i = \,>
% \end{cases}
% \end{equation}
% are defined such that $\mathbf{F}_{\bm{\alpha}}(\mathbf{r}) = \nabla f_{\bm{\alpha}}^<(\mathbf{r})$ and $\bm{\mathcal{F}}_{\bm{\alpha}}(\mathbf{r}) = \nabla f_{\bm{\alpha}}^>(\mathbf{r})$.
% Note that, as is clear form the derivation of Section \ref{app:helmholtzDelta}, the second term is curl-free in both the $\mathbf{r}$ and $\mathbf{r}'$ and the first term is likewise transverse in both coordinate systems.

% The transverse exterior harmonics also satisfy the convenient relation at large radii
% \begin{equation}
% \begin{split}
% \lim_{r\to\infty}\int_0^{2\pi}\int_0^\pi \bm{\mathcal{X}}_{\bm{\alpha}}(\mathbf{r},k)&\times\bm{\mathcal{Y}}_{\bm{\alpha}'}(\mathbf{r},k)\cdot\hat{\mathbf{r}}r^2\sin\theta\;\mathrm{d}\theta\,\mathrm{d}\phi = \\
% &(1 + \delta_{p0}\delta_{m0} - \delta_{p1}\delta_{m0})2\pi\frac{(-1)^{\ell + 1}\mathrm{i}^{2\ell + 1}}{k^2}\frac{\ell(\ell + 1)}{2\ell + 1}\frac{(\ell + m)!}{(\ell - m)!}\delta_{\bm{\alpha}\bm{\alpha}'}.
% \end{split}
% \end{equation}

% It is also useful from time to time to use the unitless regularized longitudinal harmonics
% \begin{equation}
% \begin{split}
% \mathbf{Z}_{\bm{\alpha}}(\mathbf{r};s) &= s^{-\ell + 1}\nabla f_{\bm{\alpha}}^<(\mathbf{r})\Theta(s - r) + s^{\ell + 2}\nabla f_{\bm{\alpha}}^>(\mathbf{r})\Theta(r - s)\\
% &= \nabla\left\{s^{-\ell + 1} f_{\bm{\alpha}}^<(\mathbf{r})\Theta(s - r) + s^{\ell + 2} f_{\bm{\alpha}}^>(\mathbf{r})\Theta(r - s)\right\}\\
% &= s^{-\ell + 1}\mathbf{F}_{\bm{\alpha}}(\mathbf{r})\Theta(s - r) + s^{\ell + 2}\bm{\mathcal{F}}_{\bm{\alpha}}(\mathbf{r})\Theta(r - s)
% \end{split}
% \end{equation}
% which obey the orthogonality conditions
% \begin{equation}
% \begin{split}
% \int\mathbf{Z}_{\bm{\alpha}}(\mathbf{r};a)\cdot\mathbf{Z}_{\bm{\beta}}(\mathbf{r};b)\;\mathrm{d}^3\mathbf{r} &= 4\pi\delta_{TE}\delta_{\bm{\alpha}\bm{\beta}}(1 - \delta_{p1}\delta_{m0})\left[\Theta(a - b)\frac{b^{\ell + 2}}{a^{\ell - 1}} + \Theta(b - a)\frac{a^{\ell + 2}}{b^{\ell - 1}}\right],\\
% \int\mathbf{Z}_{\bm{\alpha}}(\mathbf{r};s)\cdot\mathbf{Z}_{\bm{\beta}}(\mathbf{r};s)\;\mathrm{d}^3\mathbf{r} &= 4\pi s^3\delta_{TE}\delta_{\bm{\alpha}\bm{\beta}}(1 - \delta_{p1}\delta_{m0}).
% \end{split}
% \end{equation}








\subsection{Useful Orthogonality Relations}\label{sec:simpleOrthogonality}

A list of the useful and simple orthogonality relations useful to our calculations begins with the angular integral
\begin{equation}
\int_0^{2\pi}S_p(m\phi)S_{p'}(m'\phi)\;\mathrm{d}\phi = \pi(1 + \delta_{p0}\delta_{m0} - \delta_{p1}\delta_{m0})\delta_{pp'}\delta_{mm'},
\end{equation}
wherein the convention is used that the replacements $p + 2n\to 0$ for even $p$ and $p + 2n\to1$ for odd $p$ are taken automatically (here $n = 0,1,2,\ldots$ is a nonnegative integer). Similarly, $p + 2n + 1\to 1$ for even $p$ and $p + 2n + 1\to0$ for odd $p$ are implied. 

Two more angular integrals, this time in $\theta$, are of use:
\begin{equation}
\int_0^\pi P_{\ell m}(\cos\theta)P_{\ell'm}(\cos\theta)\sin\theta\;\mathrm{d}\theta = \frac{2}{(2\ell + 1)}\frac{(\ell + m)!}{(\ell - m)!}\delta_{\ell\ell'}\\
\end{equation}
and
\begin{equation}
\begin{split}
\int_0^\pi\left(\frac{\partial P_{\ell m}(\cos\theta)}{\partial\theta}\frac{\partial P_{\ell'm}(\cos\theta)}{\partial\theta}\right. &+ \left.m^2\frac{P_{\ell m}(\cos\theta)P_{\ell'm}(\cos\theta)}{\sin^2\theta}\right)\sin\theta\;\mathrm{d}\theta\\
&= \int_0^\pi\ell(\ell + 1)P_{\ell m}(\cos\theta)P_{\ell'm}(\cos\theta)\sin\theta\;\mathrm{d}\theta\\
&= \frac{2\ell(\ell + 1)}{(2\ell + 1)}\frac{(\ell + m)!}{(\ell - m)!}\delta_{\ell\ell'},
\end{split}
\end{equation}
for $\ell > 0$, $0 \leq m \leq \ell$. The second identity can be derived using the identity
\begin{equation}
\begin{split}
&\frac{\partial}{\partial\theta}\left\{\sin\theta\frac{\partial P_{\ell'm}(\cos\theta)}{\partial\theta}P_{\ell m}(\cos\theta) + \sin\theta\frac{\partial P_{\ell m}(\cos\theta)}{\partial \theta}P_{\ell' m}(\cos\theta)\right\}\\
&= 2\sin\theta\left(\frac{\partial P_{\ell m}(\cos\theta)}{\partial\theta}\frac{\partial P_{\ell'm}(\cos\theta)}{\partial\theta} + m^2\frac{P_{\ell m}(\cos\theta)P_{\ell'm}(\cos\theta)}{\sin^2\theta}\right) - 2\ell(\ell + 1)\sin\theta P_{\ell m}(\cos\theta)P_{\ell' m}(\cos\theta),
\end{split}
\end{equation}
which can in turn be derived from the generalized Legendre equation defining the functions $P_{\ell m}(\cos\theta)$,
\begin{equation}
\frac{\partial^2}{\partial\theta^2}P_{\ell m}(\cos\theta) + \frac{\cos\theta}{\sin\theta}\frac{\partial}{\partial\theta}P_{\ell m}(\cos\theta) + \left(\ell(\ell + 1) - \frac{m^2}{\sin^2\theta}\right)P_{\ell m}(\cos\theta) = 0.
\end{equation}

Further,
\begin{equation}
\begin{split}
\int_0^\pi\left[\frac{P_{\ell m}(\cos\theta)}{\sin\theta}\frac{\partial P_{\ell'm}(\cos\theta)}{\partial\theta} + \frac{P_{\ell'm}(\cos\theta)}{\sin\theta}\frac{\partial P_{\ell m}(\cos\theta)}{\partial\theta}\right]\sin\theta\,\mathrm{d}\theta &= \int_0^\pi\frac{\partial}{\partial \theta}\left\{P_{\ell m}(\cos\theta)P_{\ell' m}(\cos\theta)\right\}\mathrm{d}\theta\\
&= \left.P_{\ell m}(\cos\theta)P_{\ell' m}(\cos\theta)\right|_0^\pi\\
&= 2\delta_{m0}\delta_{\mathrm{par}(\ell),1},
\end{split}
\end{equation}
where
\begin{equation}
\mathrm{par}(n) = 
\begin{cases}
0, & |n|\;\mathrm{even},\\
1, & |n|\;\mathrm{odd}
\end{cases}
\end{equation}
is the parity function that returns 0 for even integers and 1 for odd integers.








\subsection{Useful recursion relations}
It is sometimes useful to know that
\begin{equation}
\begin{split}
z_\ell(x) &= \frac{x}{2\ell + 1}\left[z_{\ell - 1}(x) + z_{\ell + 1}(x)\right],\\
\frac{\partial z_\ell(x)}{\partial x} &= z_{\ell - 1}(x) - \frac{\ell + 1}{x}z_\ell(x),\\
\frac{\partial z_\ell(x)}{\partial x} &= -z_{\ell + 1}(x) + \frac{\ell}{x}z_\ell(x),
\end{split}
\end{equation}
and
\begin{equation}
\frac{\partial\{xz_\ell(x)\}}{\partial x} = \frac{x}{2\ell + 1}\left[(\ell + 1)z_{\ell - 1}(x) - \ell z_{\ell + 1}(x)\right]
\end{equation}
where $z_\ell(x)$ is any spherical Bessel or Hankel function. These identities can be combined to give
\begin{equation}
\begin{split}
\ell(\ell + 1)\frac{w_\ell(x_1)z_\ell(x_2)}{x_1x_2} &+ \frac{1}{x_1x_2}\frac{\partial\{x_1w_\ell(x_1)\}}{\partial x_1}\frac{\partial\{x_2z_\ell(x_2)\}}{\partial x_2}\\
&= \frac{1}{2\ell + 1}\left[(\ell + 1)w_{\ell - 1}(x_1)z_{\ell - 1}(x_2) + \ell w_{\ell + 1}(x_1)z_{\ell + 1}(x_2)\right]
\end{split}
\end{equation}
and
\begin{equation}
\frac{\partial z_\ell(x_1)}{\partial x_1}\frac{\partial w_\ell(x_2)}{\partial x_2} = \frac{\ell + 1}{2\ell + 1}z_{\ell + 1}(x_1)w_{\ell + 1}(x_2) + \frac{\ell}{2\ell + 1}z_{\ell - 1}(x_1)w_{\ell - 1}(x_2) - \frac{\ell(\ell + 1)}{x_1x_2}z_\ell(x_1)w_\ell(x_2)
\end{equation}
where $w_\ell$ and $z_\ell$ are any two spherical Bessel or Hankel functions.
