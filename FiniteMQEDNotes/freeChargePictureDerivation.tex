%!TEX root = finiteMQED.tex

\section{No matter: the free-charge picture}\label{app:noMatter}

Allowing the charges is a system to all be free charges is the simplest version of charge-inclusive QED that exists. Rather than deal with the various expansions and specifications of the previous section, we can simply start with Eq. \eqref{eq:lagrangianDensity1} and define current and charge densities $\sum_ie_i\dot{\mathbf{x}}_i(t)\delta[\mathbf{r} - \mathbf{x}_i(t)] = \mathbf{J}(\mathbf{r},t)$ and $\sum_ie_i\delta[\mathbf{r} - \mathbf{x}_i(t)] = \rho(\mathbf{r},t)$, respectively. Further, we can assume the vector potential is transverse (the Coulomb gauge) and drop terms that go like $\mathbf{A}(\mathbf{r},t)\cdot\nabla\Phi(\mathbf{r},t)$ or $\mathbf{J}^\parallel(\mathbf{r},t)\cdot\mathbf{A}(\mathbf{r},t) = [\mathbf{J}(\mathbf{r},t) - \mathbf{J}^\perp(\mathbf{r},t)]\cdot\mathbf{A}(\mathbf{r},t)$ such that
\begin{equation}
\begin{split}
\mathcal{L}' &= \sum_{i = 1}^N\frac{1}{2}\mu_i\dot{\mathbf{x}}_i^2(t)\delta[\mathbf{r} - \mathbf{x}_i(t)] + \frac{1}{8\pi}\left(\left[-\nabla\Phi(\mathbf{r},t)\right]^2 + \frac{1}{c}\dot{\mathbf{A}}^2(\mathbf{r},t) - \left[\nabla\times\mathbf{A}(\mathbf{r},t)\right]^2\right)\\
&+ \mathbf{J}^\perp(\mathbf{r},t)\cdot\mathbf{A}(\mathbf{r},t) - \rho(\mathbf{r},t)\Phi(\mathbf{r},t).
\end{split}
\end{equation}
The Euler-Lagrange equations associated with this Lagrangian density (see Section \ref{app:dependentMatter}) give us
\begin{equation}
\begin{split}
&\nabla\times\nabla\times\mathbf{A}(\mathbf{r},t) + \frac{1}{c^2}\ddot{\mathbf{A}}(\mathbf{r},t) = \frac{4\pi}{c}\mathbf{J}^\perp(\mathbf{r},t),\\
\implies &\mathbf{A}(\mathbf{r},t) = 4\pi\int_{-\infty}^{\infty}\int\mathbf{G}_0(\mathbf{r},\mathbf{r}';\omega)\cdot\frac{\mathbf{J}^\perp(\mathbf{r}',\omega)}{c}\mathrm{e}^{-\mathrm{i}\omega t}\;\mathrm{d}^3\mathbf{r}'\,\frac{\mathrm{d}\omega}{2\pi}
\end{split}
\end{equation}
where the dyadic Green's function $\mathbf{G}_0(\mathbf{r},\mathbf{r}';\omega)$ is given in Eq. \eqref{eq:G0} as
\begin{equation}
\mathbf{G}_0(\mathbf{r},\mathbf{r}';\omega) = \frac{\mathrm{i}k}{4\pi}\sum_{\bm{\alpha}}\left[\bm{\mathcal{X}}_{\bm{\alpha}}(\mathbf{r},k)\mathbf{X}_{\bm{\alpha}}(\mathbf{r}',k)\Theta(r - r') + \mathbf{X}_{\bm{\alpha}}(\mathbf{r},k)\bm{\mathcal{X}}_{\bm{\alpha}}(\mathbf{r}',k)\Theta(r' - r)\right] - \frac{\hat{\mathbf{r}}\hat{\mathbf{r}}}{k^2}\delta(\mathbf{r} - \mathbf{r}').
\end{equation}

The Green's function definition of $\mathbf{A}(\mathbf{r},t)$ tells us that the transverse vector potential is generated by the projection integral of the transverse current $\mathbf{J}^\perp(\mathbf{r},t)$ with the dyadic Green's function. In its usual form, however, the Green's function is not obviously transverse or longitudinal in $\mathbf{r}$, due to the presence of the functions $\Theta(\pm r \mp r')$ and $\hat{\mathbf{r}}\hat{\mathbf{r}}\delta(\mathbf{r} - \mathbf{r}')$ that are not separated into products of functions of $\mathbf{r}$ and $\mathbf{r}'$. We can produce this separation and clarify the role of $\mathbf{G}_0(\mathbf{r},\mathbf{r}';\omega)$, however, through use the analyses of Appendices \ref{sec:truncatedHarmonicExpansion} and \ref{sec:diracDyadExpansion}. First focusing on the terms of $\mathbf{G}_0(\mathbf{r},\mathbf{r}';\omega)$ that appear under the sum in $\bm{\alpha}$, we can let
\begin{equation}
\begin{split}
\bm{\mathcal{X}}_{\bm{\alpha}}(\mathbf{r},k)\Theta(r - r') &= \frac{2}{\pi}(1 - \delta_{p1}\delta_{m0})\int_0^\infty\kappa^2R_{T\ell}^{><}(k,\kappa;r',\infty)\mathbf{X}_{\bm{\alpha}}(\mathbf{r},\kappa)\;\mathrm{d}\kappa\\
&\qquad- (1 - \delta_{p1}\delta_{m0})\delta_{TE}\sqrt{\frac{\ell(\ell + 1)}{2\ell + 1}}\frac{h_\ell(kr')}{kr'}\mathbf{Z}_{\bm{\alpha}}(\mathbf{r};r'),\\[1.0em]
\mathbf{X}_{\bm{\alpha}}(\mathbf{r},k)\Theta(r' - r) &= \frac{2}{\pi}(1 - \delta_{p1}\delta_{m0})\int_0^\infty\kappa^2R_{T\ell}^\ll(k,\kappa;0,r')\mathbf{X}_{\bm{\alpha}}(\mathbf{r},\kappa)\;\mathrm{d}\kappa\\
&\qquad + (1 - \delta_{p1}\delta_{m0})\delta_{TE}\sqrt{\frac{\ell(\ell + 1)}{2\ell + 1}}\frac{j_\ell(kr')}{kr'}\mathbf{Z}_{\bm{\alpha}}(\mathbf{r};r'),
\end{split}
\end{equation}
where in both equations the first term on the right-hand side is transverse in $\mathbf{r}$ and the second is longitudinal. Although the longitudinal terms involve the non-separable longitudinal harmonics $\mathbf{Z}_{\bm{\alpha}}(\mathbf{r};r') = \nabla\left\{r'^{-\ell + 1} f_{\bm{\alpha}}^<(\mathbf{r})\Theta(r' - r) + r'^{\ell + 2} f_{\bm{\alpha}}^>(\mathbf{r})\Theta(r - r')\right\}$ and are more unwieldy to work with, one can see that any integration in $\mathbf{r}'$ involving these harmonics can be carried out under the gradient in $\mathbf{r}$, thus guaranteeing that the resulting contributions to the vector potential will be curl-free.

The ``self-field'' term of $\mathbf{G}_0(\mathbf{r},\mathbf{r}';\omega)$, i.e. the Driac-delta dyad $\hat{\mathbf{r}}\hat{\mathbf{r}}\delta(\mathbf{r} - \mathbf{r}')/k^2$, can be expanded similarly, providing
\begin{equation}
\begin{split}
-\frac{\hat{\mathbf{r}}\hat{\mathbf{r}}}{k^2}\delta(\mathbf{r} - \mathbf{r}') &= -\frac{1}{2\pi^2k^2}\sum_{\bm{\alpha}}\int_0^\infty\mathbf{X}_{\bm{\alpha}}(\mathbf{r},\kappa)\left[\hat{\mathbf{r}}'\cdot\mathbf{X}_{\bm{\alpha}}(\mathbf{r}',\kappa)\right]\hat{\mathbf{r}}'\kappa^2\;\mathrm{d}\kappa\\
&\qquad-\frac{1}{4\pi k^2}\sum_{\bm{\alpha}}\nabla\left\{\frac{\partial f_{\bm{\alpha}}^>(\mathbf{r}')}{\partial r'}f_{\bm{\alpha}}^<(\mathbf{r})\Theta(r' - r) + \frac{\partial f_{\bm{\alpha}}^<(\mathbf{r}')}{\partial r'}f_{\bm{\alpha}}^>(\mathbf{r})\Theta(r - r')\right\}\hat{\mathbf{r}}'.
\end{split}
\end{equation}
As expected, only the radially-oriented components of the current contribute to the vector potential through the self-field term, and both longitudinal and transverse self-fields exist, in general. We can see now that the Green's function is decomposable as $\mathbf{G}_0(\mathbf{r},\mathbf{r}';\omega) = \mathbf{G}_0^\perp(\mathbf{r},\mathbf{r}';\omega) + \mathbf{G}_0^\parallel(\mathbf{r},\mathbf{r}';\omega)$, wherein the transverse ($\perp$) and longitudinal ($\parallel$) components of the Green's function are
\begin{equation}\label{eq:helmholtzDecompositionFreeSpaceGreenFunction}
\begin{split}
\mathbf{G}_0^\perp(\mathbf{r},\mathbf{r}';\omega) &= \frac{\mathrm{i}k}{2\pi^2}\sum_{\bm{\alpha}}\int_0^\infty\left( \vphantom{\frac{1}{4\pi k^2}} \mathbf{X}_{\bm{\alpha}}(\mathbf{r},\kappa)\mathbf{X}_{\bm{\alpha}}(\mathbf{r}',k)R_{T\ell}^{><}(k,\kappa;r',\infty)\right.\\
&\left.+ \mathbf{X}_{\bm{\alpha}}(\mathbf{r},\kappa)\bm{\mathcal{X}}_{\bm{\alpha}}(\mathbf{r}',k)R_{T\ell}^\ll(k,\kappa;0,r') - \frac{1}{\mathrm{i} k^3}\mathbf{X}_{\bm{\alpha}}(\mathbf{r},\kappa)\left[\hat{\mathbf{r}}'\cdot\mathbf{X}_{\bm{\alpha}}(\mathbf{r}',\kappa)\right]\hat{\mathbf{r}}'\right)\kappa^2\mathrm{d}\kappa,\\[1.0em]
\mathbf{G}_0^\parallel(\mathbf{r},\mathbf{r}';\omega) &= \frac{\mathrm{i}k}{4\pi}\sum_{\bm{\alpha}}\left(\delta_{TE}\sqrt{\frac{\ell(\ell + 1)}{2\ell + 1}}\mathbf{Z}_{\bm{\alpha}}(\mathbf{r};r')\left[-\frac{h_\ell(kr')}{kr'}\mathbf{N}_{\bm{\alpha}}(\mathbf{r}',k) + \frac{j_\ell(kr')}{kr'}\bm{\mathcal{N}}_{\bm{\alpha}}(\mathbf{r}',k)\right]\right.\\
&\left.\qquad-\frac{1}{\mathrm{i} k^3}\nabla\left\{\frac{\partial f_{\bm{\alpha}}^>(\mathbf{r}')}{\partial r'}f_{\bm{\alpha}}^<(\mathbf{r})\Theta(r' - r) + \frac{\partial f_{\bm{\alpha}}^<(\mathbf{r}')}{\partial r'}f_{\bm{\alpha}}^>(\mathbf{r})\Theta(r - r')\right\}\hat{\mathbf{r}}' \vphantom{\sqrt{\frac{\ell(\ell + 1)}{2\ell + 1}}} \right).
\end{split}
\end{equation}
We have dropped factors of $1 - \delta_{p1}\delta_{m0}$ as they are redundant when multiplied by vector harmonics that are already zero at $(p,m) = (1,0)$. The $\mathbf{r}$-transverse and -longitudinal nature of these functions is explicit from the properties of $\mathbf{X}_{\bm{\alpha}}(\mathbf{r},\kappa)$, $\mathbf{Z}_{\bm{\alpha}}(\mathbf{r};r')$, and $\nabla f_{\bm{\alpha}}^{>,<}(\mathbf{r})$. It is less obvious but, nonetheless, true that these components are also transverse and longitudinal, respectively, in the source coordinate $\mathbf{r}'$ as long as $\mathbf{r}'\neq\mathbf{r}$ and $r' > 0$. The proofs of these latter conditions are given in Appendix \ref{sec:helmholtzDecompositionFreeSpaceGreenFunction}.

It is now clear that we can rewrite the integral definition of the vector potential as
\begin{equation}
\begin{split}
\mathbf{A}(\mathbf{r},t) &= 4\pi\int_{-\infty}^{\infty}\int\mathbf{G}_0^\perp(\mathbf{r},\mathbf{r}';\omega)\cdot\frac{\mathbf{J}^\perp(\mathbf{r}',\omega)}{c}\mathrm{e}^{-\mathrm{i}\omega t}\;\mathrm{d}^3\mathbf{r}'\,\frac{\mathrm{d}\omega}{2\pi}\\
&= 4\pi\int_{-\infty}^\infty\int\left[\frac{\mathrm{i}k}{2\pi^2}\sum_{\bm{\alpha}}\int_0^\infty\mathbf{X}_{\bm{\alpha}}(\mathbf{r},\kappa)\left( \vphantom{\frac{1}{4\pi k^2}} \mathbf{X}_{\bm{\alpha}}(\mathbf{r}',k)R_{T\ell}^{><}(k,\kappa;r',\infty) + \bm{\mathcal{X}}_{\bm{\alpha}}(\mathbf{r}',k)R_{T\ell}^\ll(k,\kappa;0,r')\right.\right.\\
&\quad\left.\left.- \frac{1}{\mathrm{i} k^3}\left[\hat{\mathbf{r}}'\cdot\mathbf{X}_{\bm{\alpha}}(\mathbf{r}',\kappa)\right]\hat{\mathbf{r}}'\right)\kappa^2\mathrm{d}\kappa \vphantom{\int_0^\infty} \right]\cdot\frac{\mathbf{J}^\perp(\mathbf{r}',\omega)}{c}\mathrm{e}^{-\mathrm{i}\omega t}\;\mathrm{d}^3\mathbf{r}'\,\frac{\mathrm{d}\omega}{2\pi}\\
&= \sum_{\bm{\alpha}}\int_{-\infty}^\infty\int_0^\infty\mathcal{A}_{\bm{\alpha}}(\kappa,k)\mathbf{X}_{\bm{\alpha}}(\mathbf{r},\kappa)\mathrm{e}^{-\mathrm{i}\omega t}\;\mathrm{d}\kappa\,\frac{\mathrm{d}\omega}{2\pi},
\end{split}
\end{equation}
where
\begin{equation}\label{eq:AconnectionFunction}
\begin{split}
\mathcal{A}_{\bm{\alpha}}(\kappa,k) &= \frac{2\mathrm{i}k\kappa^2}{\pi c}\int\left( \vphantom{\frac{1}{k^3}} \mathbf{X}_{\bm{\alpha}}(\mathbf{r}',k)R_{T\ell}^{><}(k,\kappa;r',\infty) + \bm{\mathcal{X}}_{\bm{\alpha}}(\mathbf{r}',k)R^\ll_{T\ell}(k,\kappa;0,r')\right.\\
&\qquad\left.- \frac{1}{\mathrm{i}k^3}\left[\hat{\mathbf{r}}'\cdot\mathbf{X}_{\bm{\alpha}}(\mathbf{r}',\kappa)\right]\hat{\mathbf{r}}'\right)\cdot\mathbf{J}^\perp(\mathbf{r}',\omega)\;\mathrm{d}^3\mathbf{r}'
\end{split}
\end{equation}
are the amplitudes of the modes of the field that have angular indices $\bm{\alpha}$, a radial index $\kappa$, and oscillate at $\omega$. In contrast to waves moving in free space, the excited modes here do not display a fixed dispersion relation between $\kappa$ and $k = \omega/c$: modes of radial index (spatial wavenumber?) $\kappa$ can oscillate in time at any frequency $\omega$ (temporal wavenumber $k$?). However, both $R_{T\ell}^><(k,\kappa;r',\infty)$ and $R_{T\ell}^\ll(k,\kappa;0,r')$ vary as $1/(k^2 - \kappa^2)$ (see Appendix \ref{app:harmonicAlgebra}) such that modes in which $|k|\approx\kappa$ are more readily excited than those in which $|k|$ and $\kappa$ are dissimilar. Finally, these amplitudes have dimensions of $[\mathrm{charge}][\mathrm{length}][\mathrm{time}]/[\mathrm{length}] = [\mathrm{charge}][\mathrm{time}]$.

The transverse and longitudinal electric fields are also straightforward to calculate from the Helmholtz components of $\mathbf{G}_0(\mathbf{r},\mathbf{r}';\omega)$. Specifically, 
\begin{equation}
\mathbf{E}^{\perp,\parallel}(\mathbf{r},t) = \int_{-\infty}^\infty\int\mathbf{G}_0^{\perp,\parallel}(\mathbf{r},\mathbf{r}';\omega)\cdot\frac{4\pi\mathrm{i}\omega}{c^2}\mathbf{J}^{\perp,\parallel}(\mathbf{r}',\omega)\mathrm{e}^{-\mathrm{i}\omega t}\;\mathrm{d}^3\mathbf{r}'\,\frac{\mathrm{d}\omega}{2\pi},
\end{equation}
such that
\begin{equation}
\mathbf{E}^\perp(\mathbf{r},t) = \sum_{\bm{\alpha}}\int_{-\infty}^\infty\int_0^\infty\mathrm{i}k\mathcal{A}(\kappa,k)\mathbf{X}_{\bm{\alpha}}(\mathbf{r},\kappa)\mathrm{e}^{-\mathrm{i}\omega t}\;\mathrm{d}\kappa\,\frac{\mathrm{d}\omega}{2\pi}
\end{equation}
in agreement with the condition $\mathbf{E}^\perp(\mathbf{r},t) = -\dot{\mathbf{A}}(\mathbf{r},t)/c$ and
\begin{equation}\label{eq:Elongitudinal1}
\begin{split}
\mathbf{E}&^\parallel(\mathbf{r},t) = \int_{-\infty}^\infty\int\frac{\mathrm{i}k}{4\pi}\sum_{\bm{\alpha}}\left(\delta_{TE}\sqrt{\frac{\ell(\ell + 1)}{2\ell + 1}}\mathbf{Z}_{\bm{\alpha}}(\mathbf{r};r')\left[-\frac{h_\ell(kr')}{kr'}\mathbf{N}_{\bm{\alpha}}(\mathbf{r}',k) + \frac{j_\ell(kr')}{kr'}\bm{\mathcal{N}}_{\bm{\alpha}}(\mathbf{r}',k)\right]\right.\\
&\left.-\frac{1}{\mathrm{i} k^3}\nabla\left\{\frac{\partial f_{\bm{\alpha}}^>(\mathbf{r}')}{\partial r'}f_{\bm{\alpha}}^<(\mathbf{r})\Theta(r' - r) + \frac{\partial f_{\bm{\alpha}}^<(\mathbf{r}')}{\partial r'}f_{\bm{\alpha}}^>(\mathbf{r})\Theta(r - r')\right\}\hat{\mathbf{r}}' \vphantom{\sqrt{\frac{\ell(\ell + 1)}{2\ell + 1}}} \right) \cdot\frac{4\pi\mathrm{i}\omega}{c^2}\mathbf{J}^\parallel(\mathbf{r}',\omega)\mathrm{e}^{-\mathrm{i}\omega t}\;\mathrm{d}^3\mathbf{r}'\,\frac{\mathrm{d}\omega}{2\pi}.
\end{split}
\end{equation}
The longitudinal electric field can be simplified through the expansion of the harmonics $\mathbf{Z}_{\bm{\alpha}}(\mathbf{r};r')$ and the application of the gradient in the second line above. In more detail, use of the closure relation
\begin{equation}
\sum_{p\ell m}(2 - \delta_{m0})(2\ell + 1)\frac{(\ell - m)!}{(\ell + m)!}P_{\ell m}(\cos\theta)P_{\ell m}(\cos\theta')S_p(m\phi)S_p(m\phi') = \frac{4\pi}{\sin\theta}\delta(\theta - \theta')\delta(\phi - \phi')
\end{equation}
in addition to the product rule and the Dirac delta definitions $\nabla\Theta(\pm r \mp a) = \pm\hat{\mathbf{r}}\delta(r - a)$ and $\delta(\mathbf{r} - \mathbf{r}') = \delta(r - r')\delta(\theta - \theta')\delta(\phi - \phi')/r^2\sin\theta$ provides
\begin{equation}
\begin{split}
&\sum_{\bm{\alpha}}\nabla\left\{\frac{\partial f_{\bm{\alpha}}^>(\mathbf{r}')}{\partial r'}f_{\bm{\alpha}}^<(\mathbf{r})\Theta(r' - r) + \frac{\partial f_{\bm{\alpha}}^<(\mathbf{r}')}{\partial r'}f_{\bm{\alpha}}^>(\mathbf{r})\Theta(r - r')\right\}\hat{\mathbf{r}}'\\
=&\sum_{\bm{\alpha}}\left[\frac{\partial f_{\bm{\alpha}}^>(\mathbf{r}')}{\partial r'}\nabla f_{\bm{\alpha}}^<(\mathbf{r})\Theta(r' - r) + \frac{\partial f_{\bm{\alpha}}^<(\mathbf{r}')}{\partial r'} \nabla f_{\bm{\alpha}}^>(\mathbf{r})\Theta(r - r')\right]\hat{\mathbf{r}}' + 4\pi\delta(\mathbf{r} - \mathbf{r}')\hat{\mathbf{r}}\hat{\mathbf{r}}'.
\end{split}
\end{equation}
Therefore,
\begin{equation}\label{eq:Elongitudinal2}
\mathbf{E}^\parallel(\mathbf{r},t) = \sum_{\bm{\alpha}}\int_{-\infty}^\infty\left[\mathcal{E}_{\bm{\alpha}}^<(k;r)\nabla f_{\bm{\alpha}}^<(\mathbf{r}) + \mathcal{E}_{\bm{\alpha}}^>(k;r)\nabla f_{\bm{\alpha}}^>(\mathbf{r})\right]\mathrm{e}^{-\mathrm{i}\omega t}\;\frac{\mathrm{d}\omega}{2\pi} - \int_{-\infty}^\infty\frac{4\pi\mathrm{i}}{\omega}\left[\mathbf{J}^\parallel(\mathbf{r},\omega)\cdot\hat{\mathbf{r}}\right]\hat{\mathbf{r}}\mathrm{e}^{-\mathrm{i}\omega t}\;\frac{\mathrm{d}\omega}{2\pi},
\end{equation}
wherein
\begin{equation}
\begin{split}
\mathcal{E}_{\bm{\alpha}}^<(k;r) &= -\frac{k^2}{c}\int\left[\delta_{TE}\sqrt{\frac{\ell(\ell + 1)}{2\ell + 1}}r'^{-\ell + 1}\left(\frac{j_\ell(kr')}{kr'}\bm{\mathcal{N}}_{p\ell m}(\mathbf{r}',k) - \frac{h_\ell(kr')}{kr'}\mathbf{N}_{p\ell m}(\mathbf{r}',k)\right)\right.\\
&\qquad+\left.\frac{\mathrm{i}}{k^3}\frac{\partial f_{\bm{\alpha}}^>(\mathbf{r}')}{\partial r'}\hat{\mathbf{r}}' 
\vphantom{\sqrt{\frac{\ell}{\ell}}} \right]\cdot\mathbf{J}^\parallel(\mathbf{r}',\omega)\Theta(r' - r)\;\mathrm{d}^3\mathbf{r}',\\
\mathcal{E}_{\bm{\alpha}}^>(k;r) &= -\frac{k^2}{c}\int\left[\delta_{TE}\sqrt{\frac{\ell(\ell + 1)}{2\ell + 1}}r'^{\ell + 2}\left(\frac{j_\ell(kr')}{kr'}\bm{\mathcal{N}}_{p\ell m}(\mathbf{r}',k) - \frac{h_\ell(kr')}{kr'}\mathbf{N}_{p\ell m}(\mathbf{r}',k)\right)\right.\\
&\qquad+\left.\frac{\mathrm{i}}{k^3}\frac{\partial f_{\bm{\alpha}}^<(\mathbf{r}')}{\partial r'}\hat{\mathbf{r}}' 
\vphantom{\sqrt{\frac{\ell}{\ell}}} \right]\cdot\mathbf{J}^\parallel(\mathbf{r}',\omega)\Theta(r - r')\;\mathrm{d}^3\mathbf{r}'.
\end{split}
\end{equation}
The scalar potential defined by $-\nabla \Phi(\mathbf{r},t) = \mathbf{E}^\parallel(\mathbf{r},t)$ can be found straightforwardly from Eq. \eqref{eq:Elongitudinal1} by removing, rather than applying, the gradient operators in $\mathbf{r}$. Explicitly,
\begin{equation}
\begin{split}
\Phi(\mathbf{r},t) &= -\int_{-\infty}^\infty\int\frac{\mathrm{i}k}{4\pi}\sum_{\bm{\alpha}}\left(\delta_{TE}\sqrt{\frac{\ell(\ell + 1)}{2\ell + 1}}\left[r'^{-\ell + 1}f_{\bm{\alpha}}^<(\mathbf{r})\Theta(r' - r) + r'^{\ell + 2}f_{\bm{\alpha}}^>(\mathbf{r})\Theta(r - r')\right]\right.\\
&\qquad\times\left[-\frac{h_\ell(kr')}{kr'}\mathbf{N}_{\bm{\alpha}}(\mathbf{r}',k) + \frac{j_\ell(kr')}{kr'}\bm{\mathcal{N}}_{\bm{\alpha}}(\mathbf{r}',k)\right] - \frac{1}{\mathrm{i} k^3}\left[\frac{\partial f_{\bm{\alpha}}^>(\mathbf{r}')}{\partial r'}f_{\bm{\alpha}}^<(\mathbf{r})\Theta(r' - r)\right.\\
&\qquad\left.\left. + \frac{\partial f_{\bm{\alpha}}^<(\mathbf{r}')}{\partial r'}f_{\bm{\alpha}}^>(\mathbf{r})\Theta(r - r')\right]\hat{\mathbf{r}}' \vphantom{\sqrt{\frac{\ell(\ell + 1)}{2\ell + 1}}} \right)\cdot\frac{4\pi\mathrm{i}\omega}{c^2}\mathbf{J}^\parallel(\mathbf{r}',\omega)\mathrm{e}^{-\mathrm{i}\omega t}\;\mathrm{d}^3\mathbf{r'}\,\frac{\mathrm{d}\omega}{2\pi}.
\end{split}
\end{equation}
Finally, the magnetic fields can be calculated from $\mathbf{B}(\mathbf{r},t) = \nabla\times\mathbf{A}(\mathbf{r},t)$ and give
\begin{equation}
\mathbf{B}(\mathbf{r},t) = \sum_{\bm{\alpha}}\int_{-\infty}^\infty\int_0^\infty\kappa\mathcal{A}_{\bm{\alpha}}(\kappa,k)\mathbf{Y}_{\bm{\alpha}}(\mathbf{r},\kappa)\mathrm{e}^{-\mathrm{i}\omega t}\;\mathrm{d}\kappa\,\frac{\mathrm{d}\omega}{2\pi}.
\end{equation}