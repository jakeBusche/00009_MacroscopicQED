\documentclass{article}
%%%%%%%%%%%%%%%%%%%%%%%%%%%%%%%%%%%%%%%%%%%%%%%%%%%%%%%%%%%%%%%%%%%%%%
% Useful packages
%%%%%%%%%%%%%%%%%%%%%%%%%%%%%%%%%%%%%%%%%%%%%%%%%%%%%%%%%%%%%%%%%%%%%%

%% General text flexibility packages
\usepackage[english]{babel}
\usepackage{color}

%% Mathematics typesetting packages
\usepackage[intlimits]{amsmath}
\usepackage{amssymb}
\usepackage{mathtools}
\usepackage{bm}
\usepackage{dutchcal}
\usepackage[cal=dutchcal]{mathalfa}

%% Graphics handling packages
% \usepackage[final]{graphics}
\usepackage{capt-of}
% \usepackage{caption}

%% Page management packages
\usepackage{multicol}
\usepackage[UKenglish]{isodate}
\usepackage[top=1in, bottom=1in, left=1in, right=1in]{geometry}
% %\usepackage{fancyhdr}
\usepackage{lipsum}

%% Miscellaneous packages (each needs commenting separately)
% csquotes is useful for orienting leading quotations properly
\usepackage[autostyle, english = american]{csquotes} 

%%%%%%%%%%%%%%%%%%%%%%%%%%%%%%%%%%%%%%%%%%%%%%%%%%%%%%%%%%%%%%%%%%%%%%
% Useful custom short commands
%%%%%%%%%%%%%%%%%%%%%%%%%%%%%%%%%%%%%%%%%%%%%%%%%%%%%%%%%%%%%%%%%%%%%%

% Companion command to the csquotes package
\MakeOuterQuote{"}

%Useful figure referenceing updates
\newcommand{\reff}[1]{Figure \!\ref{#1}}
\newcommand{\Rref}[1]{Eq. \!\ref{#1}}

% Useful math symbol typesetting shortcuts

\renewcommand{\d}[2]{
	\;\mathrm{d}^{#1}#2
}

\newcommand{\der}[2]{
	\dfrac{\text{d}#1}{\text{d}#2}
}

\newcommand{\pder}[2]{
	\dfrac{\partial#1}{\partial#2}
}

\newcommand{\ppder}[2]{
	\dfrac{\partial^2#1}{\partial{#2}^2}
}

\newcommand{\pppder}[2]{
	\dfrac{{\partial}^3#1}{\partial{#2}^3}
}

\newcommand{\bra}[1]{\langle{#1}|}
\newcommand{\ket}[1]{|{#1}\rangle}
\newcommand{\jay}{\mathcal{j}}
\newcommand{\kay}{\mathcal{k}}
\newcommand{\pprime}{{\prime\prime}}


%%%%%%%%%%%%%%%%%%%%%%%%%%%%%%%%%%%%%%%%%%%%%%%%%%%%%%%%%%%%%%%%%%%%%%
% Useful custom long commands
%%%%%%%%%%%%%%%%%%%%%%%%%%%%%%%%%%%%%%%%%%%%%%%%%%%%%%%%%%%%%%%%%%%%%%


% Building a \widebar command to mimic the behavior of \widetilde
\makeatletter
\let\save@mathaccent\mathaccent
\newcommand*\if@single[3]{%
  \setbox0\hbox{${\mathaccent"0362{#1}}^H$}%
  \setbox2\hbox{${\mathaccent"0362{\kern0pt#1}}^H$}%
  \ifdim\ht0=\ht2 #3\else #2\fi
  }
%The bar will be moved to the right by a half of \macc@kerna, which is computed by amsmath:
\newcommand*\rel@kern[1]{\kern#1\dimexpr\macc@kerna}
%If there's a superscript following the bar, then no negative kern may follow the bar;
%an additional {} makes sure that the superscript is high enough in this case:
\newcommand*\widebar[1]{\@ifnextchar^{{\wide@bar{#1}{0}}}{\wide@bar{#1}{1}}}
%Use a separate algorithm for single symbols:
\newcommand*\wide@bar[2]{\if@single{#1}{\wide@bar@{#1}{#2}{1}}{\wide@bar@{#1}{#2}{2}}}
\newcommand*\wide@bar@[3]{%
  \begingroup
  \def\mathaccent##1##2{%
%Enable nesting of accents:
    \let\mathaccent\save@mathaccent
%If there's more than a single symbol, use the first character instead (see below):
    \if#32 \let\macc@nucleus\first@char \fi
%Determine the italic correction:
    \setbox\z@\hbox{$\macc@style{\macc@nucleus}_{}$}%
    \setbox\tw@\hbox{$\macc@style{\macc@nucleus}{}_{}$}%
    \dimen@\wd\tw@
    \advance\dimen@-\wd\z@
%Now \dimen@ is the italic correction of the symbol.
    \divide\dimen@ 3
    \@tempdima\wd\tw@
    \advance\@tempdima-\scriptspace
%Now \@tempdima is the width of the symbol.
    \divide\@tempdima 10
    \advance\dimen@-\@tempdima
%Now \dimen@ = (italic correction / 3) - (Breite / 10)
    \ifdim\dimen@>\z@ \dimen@0pt\fi
%The bar will be shortened in the case \dimen@<0 !
    \rel@kern{0.6}\kern-\dimen@
    \if#31
      \overline{\rel@kern{-0.6}\kern\dimen@\macc@nucleus\rel@kern{0.4}\kern\dimen@}%
      \advance\dimen@0.4\dimexpr\macc@kerna
%Place the combined final kern (-\dimen@) if it is >0 or if a superscript follows:
      \let\final@kern#2%
      \ifdim\dimen@<\z@ \let\final@kern1\fi
      \if\final@kern1 \kern-\dimen@\fi
    \else
      \overline{\rel@kern{-0.6}\kern\dimen@#1}%
    \fi
  }%
  \macc@depth\@ne
  \let\math@bgroup\@empty \let\math@egroup\macc@set@skewchar
  \mathsurround\z@ \frozen@everymath{\mathgroup\macc@group\relax}%
  \macc@set@skewchar\relax
  \let\mathaccentV\macc@nested@a
%The following initialises \macc@kerna and calls \mathaccent:
  \if#31
    \macc@nested@a\relax111{#1}%
  \else
%If the argument consists of more than one symbol, and if the first token is
%a letter, use that letter for the computations:
    \def\gobble@till@marker##1\endmarker{}%
    \futurelet\first@char\gobble@till@marker#1\endmarker
    \ifcat\noexpand\first@char A\else
      \def\first@char{}%
    \fi
    \macc@nested@a\relax111{\first@char}%
  \fi
  \endgroup
}
\makeatother

%%%%%%%%%%%%%%%%%%%%%%%%% Additional Packages %%%%%%%%%%%%%%%%%%%%%%%%%%%%%%%%%%

% Used to get bibliogrpahy options that I want
\usepackage[comma,super,sort&compress]{natbib}

% Used to get nice physics macros
\usepackage{physics}

% Used to get tables that span page breaks
\usepackage{longtable}

%%%%%%%%%%%%%%%%%%%%%%%%%%% New Commands %%%%%%%%%%%%%%%%%%%%%%%%%%%%%%%%%%%%%%%
                  
% Used to place references in text
\newcommand{\citen}[1]{
	\begingroup
		\hspace{-18pt}
		\setcitestyle{numbers} % cite style is no longer superscript
		\cite{#1}
		\hspace{-6.5pt}
	\endgroup
}

\begin{document}

\title{Macroscopic QED in Finite Particles}
\author{Jacob A Busche}
\maketitle











\section{Introduction}

\subsection{Definitions of the coordinate fields}\label{sec:coordinateFields}

The canonical Lagrangian density of quantum electrodynamics is given by\cite{cohen-tannoudji2004photons}
\begin{equation}\label{eq:lagrangianDensityQED}
\begin{split}
\mathcal{L}_\mathrm{QED}&\left[\mathbf{x}_1,\ldots,\mathbf{x}_N,\dot{\mathbf{x}}_1,\ldots,\dot{\mathbf{x}}_N;\mathbf{A},\dot{\mathbf{A}}\right] = \sum_{i = 1}^N\frac{1}{2}m_i\dot{\mathbf{x}}_i^2(t)\delta[\mathbf{r} - \mathbf{x}_i(t)] + \frac{1}{8\pi}\left[\mathbf{E}^2(\mathbf{r},t) - \mathbf{B}^2(\mathbf{r},t)\right]\\
&+ \sum_{i = 1}^N\frac{e_i}{c}\dot{\mathbf{x}}_i(t)\delta[\mathbf{r} - \mathbf{x}_i(t)]\cdot\mathbf{A}(\mathbf{r},t) - \sum_{i = 1}^Ne_i\delta[\mathbf{r} - \mathbf{x}_i(t)]\Phi(\mathbf{r},t),
\end{split}
\end{equation}
and is appropriate for use in systems showcasing well-separated point-like charges interacting with electromagnetic field. However, in general, the charges that make up a material are not confined to small regions of space but are instead spread over larger volumes. In this sense, the usual charge densities $e_i\delta[\mathbf{r} - \mathbf{x}_i(t)]$ present in $\mathcal{L}_\mathrm{QED}$ are a bad approximation and must be replaced by a more general description. 

In our model, we assume that four separate charge distributions $\rho_\phi(\mathbf{r},t)$ exist within the volume $\mathbb{V}$ occupied by the material of our system, where $\phi = \{m,b,r,f\}$ denotes the ``family'' the charges belong to. The first is the distribution of the mobile ($\phi = m$) charges, which are assumed to be responsible for the material's interactions with the electromagnetic field. 

% Giving this charge density a characteristic charge magnitude $e$, we can define a number density of mobile charges via $n(\mathbf{r},t) = \rho_m(\mathbf{r},t)/e$. Moreover, we assume that this charge density can be expanded as $n(\mathbf{r},t) = \bar{n}(\mathbf{r}) + \delta n(\mathbf{r},t)$, with $\bar{n}(\mathbf{r})$ the equilibrium charge density that exists in the absence of driving and $\delta n(\mathbf{r},t)$ a perturbation density the encodes the material's response to electromagnetic stimuli.

These mobile charges experience motion described by a velocity field $\dot{\mathbf{Q}}(\mathbf{r},t)$ and form a bound current density $\mathbf{J}_m(\mathbf{r})$ given by
\begin{equation}
\begin{split}
\mathbf{J}_m(\mathbf{r},t) &= \sum_i\mathbf{j}_i(\mathbf{r},t)\\
% &= e\bar{n}(\mathbf{r})\dot{\mathbf{Q}}(\mathbf{r},t)
&= e\eta\Theta(\mathbf{r}\in\mathbb{V})\dot{\mathbf{Q}}(\mathbf{r},t)
\end{split}
\end{equation}
that captures the contributions of the microscopic currents $\mathbf{j}_i(\mathbf{r},t)$ and can be calculated through atomistic models. Here $e$ is the elementary charge and is used as a characteristic charge scale, while $\eta$ is a characteristic number density. In general, we will assume $\mathbb{V}$ is a finite domain, such that $\dot{\mathbf{Q}}(\mathbf{r},t)$ can be mathematically defined over all of $\mathbb{R}^3$ as a periodic function in whichever sense best fits the system geometry. Thus, $\Theta(\mathbf{r}\in\mathbb{V})\dot{\mathbf{Q}}(\mathbf{r},t)$ represents the first ``unit-cell'' of $\dot{\mathbf{Q}}(\mathbf{r},t)$ and is the physical quantity of interest. 

% The mobile charge current obeys the hard-wall boundary condition $\mathbf{J}_m(\mathbf{r},t)\cdot\hat{\mathbf{n}}(\mathbf{r})|_{\partial \mathbb{V}} = 0$ with $\hat{\mathbf{n}}(\mathbf{r})$ the outward-oriented surface-normal unit vector, such that no mobile charge enters or exits the material when the system is excited. More explicitly, the continuity equation

The related continuity equation states that
\begin{equation}
\begin{split}
\frac{\partial \rho_m(\mathbf{r},t)}{\partial t} 
%&= e\,\delta\dot{n}(\mathbf{r},t)\\
&= -\nabla\cdot\mathbf{J}_m(\mathbf{r},t)\\
&= -e\eta\left[\Theta(\mathbf{r}\in\mathbb{V})\nabla\cdot\dot{\mathbf{Q}}(\mathbf{r},t) - \delta(\mathbf{r}\in\partial\mathbb{V})\hat{\mathbf{n}}(\mathbf{r})\cdot\dot{\mathbf{Q}}(\mathbf{r},t)\right],
\end{split}
\end{equation}
wherein $\rho_m(\mathbf{r},t)$ is the mobile charge density and $\hat{\mathbf{n}}(\mathbf{r})$ is the outward-facing surface-normal unit vector.

% contains no contribution from $\nabla\bar{n}(\mathbf{r})\cdot\dot{\mathbf{Q}}(\mathbf{r},t)|_{\mathbf{r}\in\partial\mathbb{V}} = 0$, as $\nabla\bar{n}(\mathbf{r})|_{\mathbf{r}\in\partial\mathbb{V}}\sim\hat{\mathbf{n}}(\mathbf{r})$. This boundary condition is imposed here in addition to the usual Maxwell boundary conditions, as is generally done in material models that include a mobile charge pressure term, here understood to be the third term in $\mathcal{L}_m$ [Eq. \eqref{eq:lagrangianDensityM}] characterized by a hydrodynamic parameter $D$.\cite{raza2011unusual,ciraci2013hydrodynamic,ruppin1979mie}

The connection between the velocity field and the mobile charge coordinate field is now straightforward:
\begin{equation}
\mathbf{Q}(\mathbf{r},t) = \int_{-\infty}^t\dot{\mathbf{Q}}(\mathbf{r},t')\;\mathrm{d}t',
\end{equation}
where the system is assumed to be at rest at $t\to-\infty$ such that $\lim_{t\to-\infty}\mathbf{Q}(\mathbf{r},t) = 0$. In parallel fashion, the mobile charge polarization field $\mathbf{P}(\mathbf{r},t)$ is related to the current density via
\begin{equation}
\begin{split}
\mathbf{P}(\mathbf{r},t) &= \int_{-\infty}^t\mathbf{J}_m(\mathbf{r},t')\;\mathrm{d}t'\\
&= e\eta\Theta(\mathbf{r}\in\mathbb{V})\mathbf{Q}(\mathbf{r},t),
\end{split}
\end{equation}
such that
\begin{equation}
\begin{split}
\nabla\cdot\mathbf{P}(\mathbf{r},t) 
%&= -e\,\delta n(\mathbf{r},t)\\
&= -\rho_m(\mathbf{r},t)\\
&= e\eta\left[\Theta(\mathbf{r}\in\mathbb{V})\nabla\cdot\mathbf{Q}(\mathbf{r},t) - \delta(\mathbf{r}\in\partial\mathbb{V})\hat{\mathbf{n}}(\mathbf{r})\cdot\mathbf{Q}(\mathbf{r},t)\right] - \bar{\rho}_m(\mathbf{r}).
\end{split}
\end{equation}
Here, $\bar{\rho}_m(\mathbf{r})$ is the equilibrium mobile charge density, defined by the limit $\lim_{t\to-\infty}\rho_m(\mathbf{r},t) = \bar{\rho}_m(\mathbf{r})$.

% Interestingly, this suggests that $\nabla\bar{n}(\mathbf{r})\cdot\mathbf{Q}(\mathbf{r},t)|_{\mathbf{r}\in\partial\mathbb{V}}$ cannot be time-varying. Via our initial conditions, we have disallowed the existence of a static coordinate field, such that the polarization field has no surface-normal component at the material boundaries.

The background ($\phi = b$) charge density is assumed to be static, such that $\rho_b(\mathbf{r},t) = \rho_b(\mathbf{r})$. We assume it has an equal number density to that of the mobile charges and an equal and opposite characteristic charge, such that $\rho_b(\mathbf{r}) = -\bar{\rho}_m(\mathbf{r})$.

The reservoir charges ($\phi = r$) are treated in the same manner, with the exception that they are assumed to form a globally neutral system comprised of many independent densities $\rho_\nu(\mathbf{r},t)$. Each reservoir charge density is given a characteristic number density $\eta_\nu$  such that reservoir currents $\mathbf{J}_\nu(\mathbf{r},t) = e\eta_\nu\Theta(\mathbf{r}\in\mathbb{V})\dot{\mathbf{Q}}_\nu(\mathbf{r},t)$ can be defined via the coordinate fields $\mathbf{Q}_\nu(\mathbf{r},t)$.

Finally, the free charges ($\phi = f$) are left as the general charge distribution $\rho_f(\mathbf{r},t) = \sum_{i = 1}^{N_f}e_{fi}\delta[\mathbf{r} - \mathbf{x}_{fi}(t)]$ with associated charges $e_{fi}$ and coordinates $\mathbf{x}_{fi}(t)$ such that a more specific driving charge source can be specified later.












\subsection{The First-Principles Lagrangian}\label{sec:LagrangianExpansion}

The Lagrangian density of our system can now be clearly defined. In total,
\begin{equation}\label{eq:lagrangianDensityMQED}
\begin{split}
\mathcal{L}&\left[\{\mathbf{x}_{fi}(t)\},\{\dot{\mathbf{x}}_{fi}(t)\};\mathbf{Q}(\mathbf{r},t),\dot{\mathbf{Q}}(\mathbf{r},t);\{\mathbf{Q}_\nu(\mathbf{r},t)\},\{\dot{\mathbf{Q}}_\nu(\mathbf{r},t)\};\mathbf{A}(\mathbf{r},t),\dot{\mathbf{A}}(\mathbf{r},t)\right]\\
&= \mathcal{L}_f + \mathcal{L}_m + \mathcal{L}_r + \mathcal{L}_\mathrm{int} + \mathcal{L}_\mathrm{bind} + \mathcal{L}_\mathrm{EM}
% + \mathcal{L}_\mathrm{NL},
\end{split}
\end{equation}
where
\begin{equation}
\mathcal{L}_f = \sum_{i = 1}^{N_f}\frac{1}{2}m_{fi}\dot{\mathbf{x}}_{fi}^2(t)\delta[\mathbf{r} - \mathbf{x}_{fi}(t)]
\end{equation}
describes the kinetic energy of the free charges with masses $m_{fi}$, coordinates $\mathbf{x}_{fi}(t)$, and velocities $\dot{\mathbf{x}}_{fi}(t)$,
\begin{equation}\label{eq:lagrangianDensityM}
\mathcal{L}_m = \frac{\mu}{2}\left[\dot{\mathbf{Q}}^2(\mathbf{r},t) - \omega_0^2\mathbf{Q}^2(\mathbf{r},t)\right]\Theta(\mathbf{r}\in\mathbb{V})
%+ \frac{D\bar{n}(\mathbf{r})}{2}\left[\nabla\cdot\mathbf{Q}(\mathbf{r},t)\right]^2
\end{equation}
describes the kinetic and harmonic potential energies experienced by the mobile material charges with characteristic mass-densities $\mu$, coordinate fields $\mathbf{Q}(\mathbf{r},t)$, velocity fields $\dot{\mathbf{Q}}(\mathbf{r},t)$, and natural frequencies $\omega_0$, and
\begin{equation}
\mathcal{L}_r = \int_0^\infty\frac{\mu}{2}\left[\dot{\mathbf{Q}}_\nu^2(\mathbf{r},t) - \nu^2\mathbf{Q}_\nu^2(\mathbf{r},t)\right]\Theta(\mathbf{r}\in\mathbb{V})\;\mathrm{d}\nu
%+ \int_0^\infty\frac{D_\nu\bar{n}_\nu(\mathbf{r})}{2}\left[\nabla\cdot\mathbf{Q}_\nu(\mathbf{r},t)\right]^2\mathrm{d}\nu
\end{equation}
describes the kinetic and potential energies of reservoir charges that build in the system's damping and have mass-densities $\mu$, coordinate fields $\mathbf{Q}_\nu(\mathbf{r},t)$, velocity fields $\dot{\mathbf{Q}}_\nu(\mathbf{r},t)$, and natural frequencies $\nu$. These forms for the mobile charge and reservoir Lagrangian densities impose the restriction on our model that the natural frequencies of each coordinate field are spatially isotropic, but are otherwise general. Further,
\begin{equation}
\begin{split}
\mathcal{L}_\mathrm{int} &= -\left[-e\eta\nabla\cdot\left\{\Theta(\mathbf{r}\in\mathbb{V})\mathbf{Q}(\mathbf{r},t)\right\} + \rho_f(\mathbf{r},t)\right]\Phi(\mathbf{r},t) + \frac{1}{c}\left[e\eta\Theta(\mathbf{r}\in\mathbb{V})\dot{\mathbf{Q}}(\mathbf{r},t) + \mathbf{J}_f(\mathbf{r},t)\right]\cdot\mathbf{A}(\mathbf{r},t)\\
&- \eta\int_0^\infty  v(\nu)\mathbf{Q}(\mathbf{r},t)\cdot\dot{\mathbf{Q}}_\nu(\mathbf{r},t)\Theta(\mathbf{r}\in\mathbb{V})\;\mathrm{d}\nu
\end{split}
\end{equation}
describes the interactions between the scalar potential $\Phi(\mathbf{r},t)$, the vector potential $\mathbf{A}(\mathbf{r},t)$, the mobile charges, and the free charges. Here $ v(\nu)$ is a characteristic coupling strength between the mobile charges and reservoir, and is chosen to be spatially invariant to provide the simplest realistic model of damping in real materials. Finally,
\begin{equation}
\mathcal{L}_\mathrm{bind} = -\sum_{\alpha,\beta = m,b,r}\left[u_{\alpha\beta}^\infty(\mathbf{r}) + u_{\beta\alpha}^\infty(\mathbf{r})\right] - \sum_{\alpha = m,b,r,f}u_\alpha^\mathrm{self}
\end{equation}
describes the binding potentials $u_{\alpha\beta}^\infty(\mathbf{r})$ and self-energies $u_\alpha^\mathrm{self}$ experienced by each charge, and
\begin{equation}
\mathcal{L}_\mathrm{EM} = \frac{1}{8\pi}\left[\left[\nabla\Phi(\mathbf{r},t)\right]^2 + \frac{1}{c^2}\dot{\mathbf{A}}^2(\mathbf{r},t) - \left[\nabla\times\mathbf{A}(\mathbf{r},t)\right]^2 + \frac{2}{c}\nabla\Phi(\mathbf{r},t)\cdot\dot{\mathbf{A}}(\mathbf{r},t)\right]
\end{equation}
describes the energy bound in the electromagnetic fields. 

% Finally, 
% \begin{equation}
% \mathcal{L}_\mathrm{NL} = - \frac{\sigma_0}{3}\mu(\mathbf{r})\mathbf{Q}(\mathbf{r},t)\cdot\left[\mathbf{Q}(\mathbf{r},t)\cdot\bm{1}_3\cdot\mathbf{Q}(\mathbf{r},t)\right]
% \end{equation}
% describes the anharmonic portion of the potential energy experienced by the material charges. This term can be generalized as needed to describe the interesting nonlinear dynamics of the system.











\subsection{Equations of motion}

The derivations of the equations of motion of the generalized coordinates of $\mathcal{L}$ are straightforward. The equation of motion for the free charges is given by
\begin{equation}
\frac{\mathrm{d}}{\mathrm{d}t}\frac{\partial L}{\partial \dot{\mathbf{x}}_{fi}(t)} = \frac{\partial L}{\partial \mathbf{x}_{fi}(t)},
\end{equation}
where $L = \int\mathcal{L}\,\mathrm{d}^3\mathbf{r}$. Before we can evaluate these variational derivatives, we need to first define the free current $\mathbf{J}_f(\mathbf{r},t)$ that appears in $\mathcal{L}_\mathrm{int}$. It is straightforward to prove that, with the Dirac delta ``softened'' to be a very sharply-peaked but finite even function,
\begin{equation}
\frac{\partial}{\partial t}\left\{\delta[\mathbf{r} - \mathbf{x}_{fi}(t)]\right\} = -\dot{\mathbf{x}}_{fi}(t)\cdot\nabla\left\{\delta[\mathbf{r} - \mathbf{x}_i(t)]\right\},
\end{equation}
such that, with the definition
\begin{equation}
\mathbf{J}_{fi}(\mathbf{r},t) = \sum_{i = 1}^{N_f}e_{fi}\dot{\mathbf{x}}_{fi}(t)\delta[\mathbf{r} - \mathbf{x}_{fi}(t)],
\end{equation}
the continuity equation $(\partial/\partial t)\rho_f(\mathbf{r},t) + \nabla\cdot\mathbf{J}_f(\mathbf{r},t) = 0$ is satisfied for all $(\mathbf{r},t)$. Using this description and noting that $\partial/\partial \mathbf{x}_{fi}(t) = \sum_{k = x,y,z}\hat{\mathbf{e}}_k\partial/\partial [\hat{\mathbf{e}}_k\cdot\mathbf{x}_{fi}(t)]$ is simply a gradient operator with respect to $\mathbf{x}_{fi}(t)$ rather than the usual $\mathbf{r}$, we can see that the Euler-Lagrange equations for the free charges give
\begin{equation}
\begin{split}
\frac{\mathrm{d}}{\mathrm{d}t}\frac{\partial}{\partial \dot{\mathbf{x}}_{fi}(t)}&\left\{\sum_j\frac{1}{2}m_{fj}\dot{\mathbf{x}}_{fj}^2(t) + \frac{1}{c}\sum_je_{fj}\dot{\mathbf{x}}_{fj}(t)\cdot\mathbf{A}[\mathbf{x}_{fj}(t),t] + \ldots\right\} = m_{fi}\ddot{\mathbf{x}}_{fi}^2(t) + \frac{1}{c}e_{fi}\frac{\mathrm{d}}{\mathrm{d}t}\{\mathbf{A}[\mathbf{x}_{fi}(t),t]\}\\
&= m_{fi}\ddot{\mathbf{x}}_{fi}^2(t) + \frac{e_{fi}}{c}\left(\dot{\mathbf{A}}[\mathbf{x}_{fi}(t),t] + \left[\dot{\mathbf{x}}_{fi}(t)\cdot\nabla\right]\{\mathbf{A}[\mathbf{x}_{fi}(t),t]\}\right)
\end{split}
\end{equation}
and
\begin{equation}
\begin{split}
\frac{\partial}{\partial \mathbf{x}_{fi}(t)}&\left\{-\sum_je_{fj}\Phi[\mathbf{x}_{fj}(t),t] + \frac{1}{c}\sum_je_{fj}\dot{\mathbf{x}}_{fj}(t)\cdot\mathbf{A}[\mathbf{x}_{fj}(t),t] + \ldots\right\} = -e_{fi}\nabla\Phi[\mathbf{x}_{fi}(t),t]\\
&\qquad + \frac{1}{c}\left(\dot{\mathbf{x}}_{fi}(t)\times\nabla\times\mathbf{A}[\mathbf{x}_{fi}(t),t] + \left[\dot{\mathbf{x}}_{fi}(t)\cdot\nabla\right]\{\mathbf{A}[\mathbf{x}_{fi}(t),t]\}\right),
\end{split}
\end{equation}
wherein we have used the identities $\nabla\{\mathbf{a}\cdot\mathbf{B}(\mathbf{r})\} = \mathbf{a}\times\nabla\times\mathbf{B}(\mathbf{r}) + (\mathbf{a}\cdot\nabla)\{\mathbf{B}(\mathbf{r})\}$ and $\mathrm{d}/\mathrm{d}t\{\mathbf{F}[\mathbf{x}(t),t]\} = \partial\mathbf{F}[\mathbf{x}(t),t]/\partial t + \sum_i[\partial F_i[\mathbf{x}(t),t]/\partial x_i(t)]\partial x_i(t)/\partial t $ and assumed that the gradient operators are taken with respect to the coordinate that each field depends on. Therefore,
\begin{equation}
m_{fi}\ddot{\mathbf{x}}_{fi}^2(t) = e_{fi}\left(-\nabla\Phi[\mathbf{x}_{fi}(t),t] - \frac{1}{c}\dot{\mathbf{A}}[\mathbf{x}_{fi}(t),t] + \frac{1}{c}\dot{\mathbf{x}}_{fi}(t)\times\nabla\times\mathbf{A}[\mathbf{x}_{fi}(t),t]\right),
\end{equation}
which is precisely the Maxwell-Lorentz force with $\mathbf{E}(\mathbf{r},t) = -\nabla\Phi(\mathbf{r},t) - (1/c)\dot{\mathbf{A}}(\mathbf{r},t)$ and $\mathbf{B}(\mathbf{r},t) = \nabla\times\mathbf{A}(\mathbf{r},t)$.

The equations of motion for the mobile and reservoir charge coordinate fields are derived from $\mathcal{L}$ directly, with
\begin{equation}
\frac{\mathrm{d}}{\mathrm{d}t}\left\{\frac{\partial \mathcal{L}}{\partial \dot{F}_j(\mathbf{r},t)}\right\} = \frac{\partial \mathcal{L}}{\partial F_j(\mathbf{r},t)} - \sum_{k = 1}^3\frac{\partial}{\partial r_k}\frac{\partial \mathcal{L}}{\partial\!\left(\frac{\partial F_j(\mathbf{r},t)}{\partial r_k}\right)}
\end{equation}
the Euler-Lagrange equation for the $j^\mathrm{th}$-component of the dummy vector field $\mathbf{F}(\mathbf{r},t)$. Beginning with the mobile charges, we can let $\Theta(\mathbf{r}\in\mathbb{V})Q_i(\mathbf{r},t)$ be the field components used as independent variables in our variational derivatives such that, using the property $\Theta^2(\mathbf{r}\in\mathbb{V}) = \Theta(\mathbf{r}\in\mathbb{V})$,
\begin{equation}
\begin{split}
\sum_j\hat{\mathbf{e}}_j\frac{\mathrm{d}}{\mathrm{d}t}\left\{\frac{\partial \mathcal{L}}{\partial\left(\Theta(\mathbf{r}\in\mathbb{V})\dot{Q}_{j}(\mathbf{r},t)\right)}\right\} &= \left[\mu\ddot{\mathbf{Q}}(\mathbf{r},t) + \frac{e\eta}{c}\dot{\mathbf{A}}(\mathbf{r},t)\right]\Theta(\mathbf{r}\in\mathbb{V}),\\
\sum_j\hat{\mathbf{e}}_j\frac{\partial \mathcal{L}}{\partial \left[\Theta(\mathbf{r}\in\mathbb{V})Q_{j}(\mathbf{r},t)\right]} &= \left[-\mu\omega_0^2\mathbf{Q}(\mathbf{r},t)
- \eta\int_0^\infty v(\nu)\dot{\mathbf{Q}}_\nu(\mathbf{r},t)\;\mathrm{d}\nu\right]\Theta(\mathbf{r}\in\mathbb{V}),\\
-\sum_j\hat{\mathbf{e}}_j\sum_{k = 1}^3\frac{\partial}{\partial r_k}\frac{\partial \mathcal{L}}{\partial\!\left(\frac{\partial \left[\Theta(\mathbf{r}\in\mathbb{V})Q_{j}(\mathbf{r},t)\right]}{\partial r_k}\right)} &= -e\eta\nabla\Phi(\mathbf{r},t)\Theta(\mathbf{r}\in\mathbb{V}),
\end{split}
\end{equation}
and
\begin{equation}\label{eq:eqOfMotionQm}
\begin{split}
\left[\mu\ddot{\mathbf{Q}}(\mathbf{r},t) + \int_0^\infty v(\nu)\dot{\mathbf{Q}}_\nu(\mathbf{r},t)\;\mathrm{d}\nu + \mu\omega_0^2\mathbf{Q}(\mathbf{r},t)\right]\Theta(\mathbf{r}\in\mathbb{V}) = e\eta\left[-\nabla\Phi(\mathbf{r},t) - \frac{1}{c}\dot{\mathbf{A}}(\mathbf{r},t)\right]\Theta(\mathbf{r}\in\mathbb{V}).
\end{split}
\end{equation}
The reservoir charges have a similarly simple results, with
\begin{equation}
\begin{split}
\sum_j\hat{\mathbf{e}}_j\frac{\mathrm{d}}{\mathrm{d}t}\left\{\frac{\partial \mathcal{L}}{\partial \left(\Theta(\mathbf{r}\in\mathbb{V})\dot{\tilde{Q}}_{\nu j}(\mathbf{r},t)\right)}\right\} &= \mu(\mathbf{r})\ddot{\mathbf{Q}}_\nu(\mathbf{r},t) -  \eta v(\nu)\dot{\mathbf{Q}}(\mathbf{r},t),\\
\sum_j\hat{\mathbf{e}}_j\frac{\partial \mathcal{L}}{\partial \left[\Theta(\mathbf{r}\in\mathbb{V})\tilde{Q}_{\nu j}(\mathbf{r},t)\right]} &= -\mu(\mathbf{r})\nu^2\mathbf{Q}_\nu(\mathbf{r},t),\\
-\sum_j\hat{\mathbf{e}}_j\sum_{k = 1}^3\frac{\partial}{\partial r_k}\frac{\partial \mathcal{L}}{\partial\!\left(\frac{\partial \left[\Theta(\mathbf{r}\in\mathbb{V})\tilde{Q}_{\nu j}(\mathbf{r},t)\right]}{\partial r_k}\right)} &= 0
\end{split}
\end{equation}
such that
\begin{equation}\label{eq:eqOfMotionQnu}
\left[\mu\ddot{\mathbf{Q}}_\nu(\mathbf{r},t) + \mu\nu^2\mathbf{Q}_\nu(\mathbf{r},t)\right]\Theta(\mathbf{r}\in\mathbb{V}) =  \eta v(\nu)\dot{\mathbf{Q}}(\mathbf{r},t)\Theta(\mathbf{r}\in\mathbb{V}).
\end{equation}
Together, Eqs. \eqref{eq:eqOfMotionQm} and \eqref{eq:eqOfMotionQnu} can be combined to produce a simple oscillator dielectric picture with internal losses determined by the details of the coupling function $ v(\nu)$. For details, see Appendix \ref{app:dependentMatter}.

Finally, we can use $\mathcal{L}$ to define the equations of motion for the potentials $\Phi(\mathbf{r},t)$ and $\mathbf{A}(\mathbf{r},t)$. First, we can see that 
\begin{equation}
\begin{split}
\frac{\partial\mathcal{L}}{\partial\dot{\Phi}(\mathbf{r},t)} &= 0,\\
\frac{\partial \mathcal{L}}{\partial \Phi(\mathbf{r},t)} &= e\eta\nabla\cdot\left\{\Theta(\mathbf{r}\in\mathbb{V})\mathbf{Q}(\mathbf{r},t)\right\} - \rho_f(\mathbf{r},t) = -\rho_m(\mathbf{r},t) + \bar{\rho}_m(\mathbf{r}) - \rho_f(\mathbf{r},t),\\
-\sum_{k = 1}^3\frac{\partial}{\partial r_k}\frac{\partial \mathcal{L}}{\partial\left(\frac{\partial\Phi(\mathbf{r},t)}{\partial r_k}\right)} &= -\frac{1}{4\pi}\nabla\cdot\nabla\Phi(\mathbf{r},t),\\
\end{split}
\end{equation}
such that
\begin{equation}\label{eq:poisson1}
-\nabla\cdot\nabla\Phi(\mathbf{r},t) = 4\pi\left[\rho_m(\mathbf{r},t) - \bar{\rho}_m(\mathbf{r}) + \rho_f(\mathbf{r},t)\right].
\end{equation}
In the limit where the equilibrium charge densities $\bar{\rho}_m(\mathbf{r})$ and $\rho_b(\mathbf{r})$ are equal and opposite, our Euler-Lagrange equation for the scalar potential therefore correctly reproduces the potential formulation of Gauss' law, which can quickly be rewritten as
\begin{equation}
\nabla\cdot\mathbf{E}(\mathbf{r},t) = 4\pi\left[\rho_m(\mathbf{r},t) + \rho_b(\mathbf{r}) + \rho_f(\mathbf{r},t)\right].
\end{equation}
Further, using the identity
\begin{equation}
\begin{split}
\sum_{jk}\frac{\partial}{\partial r_k}\frac{\partial}{\partial\!\left(\frac{\partial F_j(\mathbf{r},t)}{\partial r_k}\right)}\left\{\left[\nabla\times\mathbf{F}(\mathbf{r},t)\right]^2\right\}\hat{\mathbf{e}}_j &= -2\nabla\times\nabla\times\mathbf{F}(\mathbf{r},t),
\end{split}
\end{equation}
we can see that
\begin{equation}
\begin{split}
\sum_j\hat{\mathbf{e}}_j\frac{\mathrm{d}}{\mathrm{d}t}\left\{\frac{\partial\mathcal{L}}{\partial \dot{A}_j(\mathbf{r},t)}\right\} &= \frac{1}{4\pi c^2}\ddot{\mathbf{A}}(\mathbf{r},t) + \frac{1}{4\pi c}\nabla\dot{\Phi}(\mathbf{r},t),\\
\sum_j\hat{\mathbf{e}}_j\frac{\partial\mathcal{L}}{\partial A_j(\mathbf{r},t)} &= \frac{1}{c}\left[\mathbf{J}_m(\mathbf{r},t) + \mathbf{J}_f(\mathbf{r},t)\right],\\
-\sum_j\hat{\mathbf{e}}_j\sum_{k = 1}^3\frac{\partial}{\partial r_k}\frac{\partial \mathcal{L}}{\partial\!\left(\frac{\partial A_{j}(\mathbf{r},t)}{\partial r_k}\right)} &= -\frac{1}{4\pi}\nabla\times\nabla\times\mathbf{A}(\mathbf{r},t),
\end{split}
\end{equation}
such that
\begin{equation}
\nabla\times\nabla\times\mathbf{A}(\mathbf{r},t) + \frac{1}{c^2}\ddot{\mathbf{A}}(\mathbf{r},t) = \frac{4\pi}{c}\left[\mathbf{J}_m(\mathbf{r},t) + \mathbf{J}_f(\mathbf{r},t)\right] - \frac{1}{c}\nabla\dot{\Phi}(\mathbf{r},t).
\end{equation}
From our analysis of the scalar potential we know that $-\nabla\cdot\nabla\dot{\Phi}(\mathbf{r},t) = 4\pi\left[\dot{\rho}_m(\mathbf{r},t) + \dot{\rho}_f(\mathbf{r},t)\right] = -4\pi\nabla\cdot\{\mathbf{J}_m^\parallel(\mathbf{r},t) + \mathbf{J}_f^\parallel(\mathbf{r},t)\}$, where the superscipt $\parallel$ denotes the longitudinal component of a vector field. Therefore,
\begin{equation}
-\frac{1}{c}\nabla\dot{\Phi}(\mathbf{r},t) = -\frac{4\pi}{c}\left[\mathbf{J}_m^\parallel(\mathbf{r},t) + \mathbf{J}_f^\parallel(\mathbf{r},t)\right]
\end{equation}
and, noting that $\mathbf{J}_\phi(\mathbf{r},t) = \mathbf{J}_\phi^\parallel(\mathbf{r},t) + \mathbf{J}_\phi^\perp(\mathbf{r},t)$, the correct wave equation for the vector potential in the Coulomb gauge,
\begin{equation}\label{eq:vectorPotentialWaveEq}
\nabla\times\nabla\times\mathbf{A}(\mathbf{r},t) + \frac{1}{c^2}\ddot{\mathbf{A}}(\mathbf{r},t) = \frac{4\pi}{c}\left[\mathbf{J}^\perp_{m}(\mathbf{r},t) + \mathbf{J}^\perp_{f}(\mathbf{r},t)\right],
\end{equation}
is reproduced.



















\section{Construction of a Mechanical Picture for MQED}

From this point, we will simplify the mobile-charge equilibrium density to a constant with a sharp cutoff at a spherical boundary $r = a$, such that $\Theta(\mathbf{r}\in\mathbb{V}) = \Theta(a - r)$. Our goal is then to map the dynamics of this system onto a set of coupled oscillators such that the standard tricks of quantization function normally. We can do this via the expansions of the vector potential, mobile charge field, and reservoir field into sets of normal modes and analyzing the time-dependence of those modes' amplitudes. These expansions are given by
\begin{equation}\label{eq:fieldExpansions}
\begin{split}
\mathbf{A}(\mathbf{r},t) &= \sum_{\bm{\alpha}}\int_{0}^\infty \frac{e\omega^\frac{3}{2}}{c^2}\mathcal{A}_{\bm{\alpha}}(\omega,t)\mathbf{X}_{\bm{\alpha}}(\mathbf{r},k)\;\mathrm{d}\omega,\\
\mathbf{Q}(\mathbf{r},t) &= \sum_{\bm{\beta}}\sum_n\mathcal{B}_{\bm{\beta}n}(t)\mathbf{X}_{\bm{\beta}}(\mathbf{r},k_{\bm{\beta} n}),\\
\mathbf{Q}_\nu(\mathbf{r},t) &= \sum_{\bm{\gamma}}\sum_n\mathcal{C}_{\bm{\gamma}n}(\nu,t)\mathbf{X}_{\bm{\gamma}}(\mathbf{r},k_{\bm{\gamma} n}).
\end{split}
\end{equation}
The mode functions $\mathbf{X}_{\bm{\alpha}}$ are chosen to match the symmetry of our problem (details in Appendix \ref{app:harmonicAlgebra}) such that the definition of the vector potential can be extrapolated from any introductory quantum optics text. Specifically, the vector potential is, in the Coulomb gauge, comprised of the excitation of a series of photon modes\cite{cohen-tannoudji2004photons}, each of which has angular indices $\bm{\alpha} = (T',p',\ell',m')$ and a radial index $k = \omega/c$ that determine its wave pattern in space. The `type' index $T$ of the vector potential can take the values $M$ and $E$, in which cases the mode functions are transverse vector spherical harmonics of magnetic and electric type, respectively. The other three indices are the parity $p$, spherical harmonic order $\ell$, and spherical harmonic degree $m$ that determine the mode's angular symmetries. In these notes, we will generally used primed index labels for the vector potential modes and related quantities.

% Finally, in free space without sources, the radial index is linked to the mode's time evolution through the relation $k = \omega/c$ with $\omega$ the Fourier conjugate to $t$. However, with both free and bound charges present, this relationship will not hold in general such that each mode's time-variations will need to be accounted for separately.

The expansions for the matter and bath fields are similar, with angular indices given by $\bm{\beta} = (T,p,\ell,m)$ and $\bm{\gamma} = (T'',p'',\ell'',m'')$, respectively. However, their definitions involve sums over discrete wavenumbers $k_{\bm{\beta}n}$ and $k_{\bm{\gamma}n}$, rather than the continuously-varying $k$, and have an additional allowed value of the type index, $T = L$. The additional allowed mode functions are the longitudinal vector spherical harmonics that describe longitudinal matter waves

The discrete wavenumber behavior is generated by the fact that these mode functions are being used to reconstruct periodic fields with a finite unit cell. The proof is complicated\cite{titchmarsh1946eigenfunction} and will not be shown here, but relies on the as afact that the vector spherical harmonics $\mathbf{X}_{\bm{\beta}}(\mathbf{r},k)$ are vector eigenfunctions of the Sturm-Liouville operator $\nabla^2 + k^2$. 

A consequence of this proof is that, in order to form a complete set on a finite region (in this case, the interior of the sphere), one of the spherical vector harmonics' components, the derivative of one of their components, or a linear combination of the two must be equal to zero at the boundary $r = a$. Within this restriction, we have nearly complete freedom to define which discrete set of wavenumbers $k_{\bm{\beta}n}$ we use in our expansion.






\subsection{Satisfying Boundary Conditions}

The main caveat to this last point is that our expansion must produce fields that satisfy Maxwell's boundary conditions for nonmagnetic media,
\begin{equation}
\begin{split}
\hat{\mathbf{r}}\times\left[\mathbf{E}^{(+)}(\mathbf{r},t) - \mathbf{E}^{(-)}(\mathbf{r},t)\right]_{r\to a} &= 0,\\
\hat{\mathbf{r}}\cdot\left[\mathbf{E}^{(+)}(\mathbf{r},t) + 4\pi\mathbf{P}^{(+)}(\mathbf{r},t) - \mathbf{E}^{(-)}(\mathbf{r},t) - 4\pi\mathbf{P}^{(-)}(\mathbf{r},t)\right]_{r\to a} &= 0,\\
\hat{\mathbf{r}}\times\left[\mathbf{B}^{(+)}(\mathbf{r},t) - \mathbf{B}^{(-)}(\mathbf{r},t)\right]_{r\to a} &= 0,\\
\hat{\mathbf{r}}\cdot\left[\mathbf{B}^{(+)}(\mathbf{r},t) - \mathbf{B}^{(-)}(\mathbf{r},t)\right]_{r\to a} &= 0.
\end{split}
\end{equation}
Here, vector and fields outside and inside the boundary are labeled with superscripts $(\pm)$, respectively. Letting the fields be broken into their Helmholtz components such that $\mathbf{E}(\mathbf{r},t) = \mathbf{E}^\perp(\mathbf{r},t) + \mathbf{E}^\parallel(\mathbf{r},t)$ and $\mathbf{B}(\mathbf{r},t) = \mathbf{B}^\perp(\mathbf{r},t)$, we can first notice that the transverse components $\perp$ are entirely comprised of the photon field, i.e. $\mathbf{E}^\perp(\mathbf{r},t) = -(1/c)\dot{\mathbf{A}}(\mathbf{r},t)$ and $\mathbf{B}^\perp(\mathbf{r},t) = \nabla\times\mathbf{A}(\mathbf{r},t)$. The photon field is smooth everywhere, such that the transverse components of the fields cancel in each of the above equations. The latter two boundary conditions, involving only the (transverse) magnetic fields, are therefore satisfied.

The longitudinal part of the electric field, however, is comprised of the matter-generated fields such that $\mathbf{E}^\parallel(\mathbf{r},t) = -\nabla\Phi_m(\mathbf{r},t) - \nabla\Phi_f(\mathbf{r},t)$. The latter field, $-\nabla\Phi_f(\mathbf{r},t)$, is smooth everywhere except at the locations of the free charges $\rho_f(\mathbf{r},t)$ which are not near $r = a$. Therefore, $-\nabla\Phi_f(\mathbf{r},t)$ cancels in each of the above equations. We are left with the task, then, of showing that $-\nabla\Phi_m(\mathbf{r},t)$ satisfies the simplified boundary conditions
\begin{equation}
\begin{split}
\hat{\mathbf{r}}\times\left[-\nabla\Phi_m^{(+)}(\mathbf{r},t) + \nabla\Phi_m^{(-)}(\mathbf{r},t)\right] &= 0,\\
\hat{\mathbf{r}}\cdot\left[-\nabla\Phi_m^{(+)}(\mathbf{r},t) + \nabla\Phi_m^{(-)}(\mathbf{r},t)\right] &= \hat{\mathbf{r}}\cdot4\pi\mathbf{P}(\mathbf{r},t),
\end{split}
\end{equation}
wherein we have noted that $\mathbf{P}^{(+)}(\mathbf{r},t) = 0$ such that $\mathbf{P}^{(-)}(\mathbf{r},t) = \mathbf{P}(\mathbf{r},t)$. To do so, we can begin by expanding the mobile-charge scalar potential as
\begin{equation}
\begin{split}
\Phi_m(\mathbf{r},t) &= -\int\frac{\nabla'\cdot\mathbf{P}(\mathbf{r}',t)}{|\mathbf{r} - \mathbf{r}'|}\;\mathrm{d}^3\mathbf{r}'\\
% &= -\int\frac{e\eta\Theta(a - r')\nabla'\cdot\mathbf{Q}(\mathbf{r}',t)}{|\mathbf{r} - \mathbf{r}'|}\;\mathrm{d}^3\mathbf{r}' - \int\frac{e\eta\nabla'\Theta(a - r')\cdot\mathbf{Q}(\mathbf{r}',t)}{|\mathbf{r} - \mathbf{r}'|}\;\mathrm{d}^3\mathbf{r}'\\
% &= -e\eta\int\frac{1}{|\mathbf{r} - \mathbf{r}'|}\left(-e\eta\delta(r' - a)\hat{\mathbf{r}}'\cdot\sum_{\bm{\beta}n}\mathcal{B}_{\bm{\beta}n}(t)\mathbf{X}_{\bm{\beta}}(\mathbf{r},k_{\bm{\beta}n})\right.\\
% &\qquad\left. + e\eta\Theta(a - r')\sum_{p\ell mn}\mathcal{B}_{Lp\ell mn}(t)\nabla'\cdot\mathbf{L}_{p\ell m}(\mathbf{r}',k_{L\ell n})\right)\mathrm{d}^3\mathbf{r}'\\
% &= -\int_0^{2\pi}\int_0^\pi\int_0^\infty\sum_{p'\ell'm'}\left[f_{p'\ell'm'}^>(\mathbf{r})f_{p'\ell'm'}^<(\mathbf{r}')\Theta(r - r') + f_{\bm{\alpha}}^<(\mathbf{r})f_{\bm{\alpha}}^>(\mathbf{r}')\Theta(r' - r)\right]\\
% &\qquad\times\left(-e\eta\delta(r' - a)\hat{\mathbf{r}}'\cdot\sum_{\bm{\beta}n}\mathcal{B}_{\bm{\beta}n}(t)\mathbf{X}_{\bm{\beta}}(\mathbf{r},k_{\bm{\beta}n})\right.\\
% &\qquad\qquad\left. + e\eta\Theta(a - r')\sum_{p\ell mn}\mathcal{B}_{Lp\ell mn}(t)\nabla'\cdot\mathbf{L}_{p\ell m}(\mathbf{r}',k_{L\ell n})\right)r'^2\sin\theta'\;\mathrm{d}r'\,\mathrm{d}\theta'\,\mathrm{d}\phi'\\
% &= \int_0^{2\pi}\int_0^\pi e\eta a^2\sum_{p'\ell'm'}\left[f_{p'\ell'm'}^>(\mathbf{r})f_{p'\ell'm'}^<(a,\theta',\phi')\Theta(r - a) + f_{\bm{\alpha}}^<(\mathbf{r})f_{\bm{\alpha}}^>(a,\theta',\phi')\Theta(a - r)\right]\\
% &\qquad\times\sum_{\bm{\beta}n}\mathcal{B}_{\bm{\beta}n}(t)\hat{\mathbf{r}}'\cdot\mathbf{X}_{\bm{\beta}}(a,\theta',\phi';k_{\bm{\beta}n})\sin\theta'\;\mathrm{d}\theta'\,\mathrm{d}\phi'\\
% &\qquad - \int_0^{2\pi}\int_0^\pi\int_0^a\sum_{p'\ell'm'}\left[f_{p'\ell'm'}^>(\mathbf{r})f_{p'\ell'm'}^<(\mathbf{r}')\Theta(r - r') + f_{\bm{\alpha}}^<(\mathbf{r})f_{\bm{\alpha}}^>(\mathbf{r}')\Theta(r' - r)\right]\\
% &\qquad\qquad\times e\eta\sum_{p\ell mn}\mathcal{B}_{Lp\ell mn}(t)(-k_{L\ell n})\sqrt{K_{\ell m}}\sqrt{\ell(\ell + 1)}j_\ell(k_{L\ell n}r')P_{\ell m}(\cos\theta')S_p(m\phi')\\
% &\qquad\qquad\times r'^2\sin\theta'\;\mathrm{d}r'\,\mathrm{d}\theta'\,\mathrm{d}\phi'\\
% &= \int_0^{2\pi}\int_0^\pi e\eta a^2\sum_{p'\ell'm'}\sqrt{K_{\ell'm'}}\sqrt{\frac{\ell'(\ell' + 1)}{2\ell' + 1}}P_{\ell'm'}(\cos\theta')S_{p'}(m'\phi')\left[f_{p'\ell'm'}^>(\mathbf{r})a^{\ell'}\Theta(r - a)\right.\\
% &\qquad\left.+ f_{p'\ell'm'}^<(\mathbf{r})\frac{1}{a^{\ell' + 1}}\Theta(a - r)\right]\sum_{p\ell mn}\sqrt{K_{\ell m}}P_{\ell m}(\cos\theta')S_p(m\phi')\\
% &\qquad\qquad\times\left(\mathcal{B}_{Ep\ell m n}(t)\frac{\ell(\ell + 1)}{k_{E\ell n}a}j_\ell(k_{E\ell n}a) + \mathcal{B}_{Lp\ell mn}(t)\sqrt{\ell(\ell + 1)}\frac{1}{k_{L\ell n}}\frac{\partial j_\ell(k_{L\ell n}a)}{\partial a}\right)\\
% &\qquad\qquad\times\sin\theta'\;\mathrm{d}\theta'\,\mathrm{d}\phi'\\
% &\qquad + \int_0^{2\pi}\int_0^\pi\int_0^a e\eta\sum_{p'\ell'm'}\sqrt{K_{\ell'm'}}\sqrt{\frac{\ell'(\ell' + 1)}{2\ell' + 1}}P_{\ell'm'}(\cos\theta')S_{p'}(m'\phi')\left[f_{p'\ell'm'}^>(\mathbf{r})r'^\ell\Theta(r - r')\right.\\
% &\qquad\left. + f_{p'\ell'm'}^<(\mathbf{r})\frac{1}{r'^{\ell' + 1}}\Theta(r' - r)\right]\sum_{p\ell mn}\sqrt{K_{\ell m}}\sqrt{\ell(\ell + 1)}\mathcal{B}_{Lp\ell mn}(t)k_{L\ell n}j_\ell(k_{L\ell n}r')\\
% &\qquad\times P_{\ell m}(\cos\theta')S_p(m\phi')r'^2\sin\theta'\;\mathrm{d}r'\,\mathrm{d}\theta'\,\mathrm{d}\phi'\\
% &= e\eta a^2\sum_{p\ell mn}\sum_{p'\ell'm'}\sqrt{K_{\ell m}K_{\ell'm'}}\sqrt{\frac{\ell'(\ell' + 1)}{2\ell' + 1}}\left(f_{p'\ell'm'}^>(\mathbf{r})a^{\ell'}\Theta(r - a) + f_{p'\ell'm'}^<(\mathbf{r})\frac{1}{a^{\ell' + 1}}\Theta(a - r)\right)\\
% &\qquad\times\left(\mathcal{B}_{Ep\ell mn}(t)\frac{\ell(\ell + 1)}{k_{E\ell n}a}j_\ell(k_{E\ell n}a) + \mathcal{B}_{Lp\ell mn}(t)\sqrt{\ell(\ell + 1)}\frac{1}{k_{L\ell n}}\frac{\partial j_\ell(k_{L\ell n}a)}{\partial a}\right)\\
% &\qquad\times\int_0^\pi P_{\ell'm'}(\cos\theta')P_{\ell m}(\cos\theta')\sin\theta'\;\mathrm{d}\theta'\int_0^{2\pi}S_{p'}(m'\phi')S_p(m\phi')\;\mathrm{d}\phi'\\
% &\qquad + e\eta\sum_{p\ell mn}\sum_{p'\ell'm'}\sqrt{K_{\ell m}K_{\ell'm'}}\sqrt{\frac{\ell'(\ell' + 1)}{2\ell' + 1}}\sqrt{\ell(\ell + 1)}\mathcal{B}_{Lp\ell m}(t)k_{L\ell n}\\
% &\qquad\qquad\times\int_0^a\left(f_{p'\ell'm'}^>(\mathbf{r})r'^{\ell'}\Theta(r - r') + f_{p'\ell'm'}^<(\mathbf{r})\frac{1}{r'^{\ell' + 1}}\Theta(r' - r)\right)j_\ell(k_{L\ell n}r')r'^2\;\mathrm{d}r'\\
% &\qquad\qquad\times\int_0^\pi P_{\ell'm'}(\cos\theta')P_{\ell m}(\cos\theta')\sin\theta'\;\mathrm{d}\theta'\int_0^{2\pi}S_{p'}(m'\phi')S_p(m\phi')\;\mathrm{d}\phi'\\
% &= e\eta a^2\sum_{p\ell mn}\sum_{p'\ell'm'}\sqrt{K_{\ell m}K_{\ell'm'}}\sqrt{\frac{\ell'(\ell' + 1)}{2\ell' + 1}}\left(f_{p'\ell'm'}^>(\mathbf{r})a^{\ell'}\Theta(r - a) + f_{p'\ell'm'}^<(\mathbf{r})\frac{1}{a^{\ell' + 1}}\Theta(a - r)\right)\\
% &\qquad\times\left(\mathcal{B}_{Ep\ell mn}(t)\frac{\ell(\ell + 1)}{k_{E\ell n}a}j_\ell(k_{E\ell n}a) + \mathcal{B}_{Lp\ell mn}(t)\sqrt{\ell(\ell + 1)}\frac{1}{k_{L\ell n}}\frac{\partial j_\ell(k_{L\ell n}a)}{\partial a}\right)\\
% &\qquad\times\pi(1 + \delta_{p0}\delta_{m0} - \delta_{p1}\delta_{m0})\delta_{pp'}\delta_{mm'}\frac{2}{2\ell + 1}\frac{(\ell + m)!}{(\ell - m)!}\delta_{\ell\ell'}\\
% &\qquad + e\eta\sum_{p\ell mn}\sum_{p'\ell'm'}\sqrt{K_{\ell m}K_{\ell'm'}}\sqrt{\frac{\ell'(\ell' + 1)}{2\ell' + 1}}\sqrt{\ell(\ell + 1)}\mathcal{B}_{Lp\ell m}(t)k_{L\ell n}\\
% &\qquad\qquad\times\int_0^a\left(f_{p'\ell'm'}^>(\mathbf{r})r'^{\ell'}\Theta(r - r') + f_{p'\ell'm'}^<(\mathbf{r})\frac{1}{r'^{\ell' + 1}}\Theta(r' - r)\right)j_\ell(k_{L\ell n}r')r'^2\;\mathrm{d}r'\\
% &\qquad\qquad\times\pi(1 + \delta_{p0}\delta_{m0} - \delta_{p1}\delta_{m0})\delta_{pp'}\delta_{mm'}\frac{2}{2\ell + 1}\frac{(\ell + m)!}{(\ell - m)!}\delta_{\ell\ell'}\\
% &= e\eta a^2\sum_{p\ell mn}4\pi(1 - \delta_{p1}\delta_{m0})\frac{1}{\sqrt{\ell(\ell + 1)}\sqrt{2\ell + 1}}\left(f_{p\ell m}^>(\mathbf{r})a^\ell\Theta(r - a) + f_{p\ell m}^<(\mathbf{r})\frac{1}{a^{\ell + 1}}\Theta(a - r)\right)\\
% &\qquad\times\left(\mathcal{B}_{Ep\ell mn}(t)\frac{\ell(\ell + 1)}{k_{E\ell n}a}j_\ell(k_{E\ell n}a) + \mathcal{B}_{Lp\ell mn}(t)\sqrt{\ell(\ell + 1)}\frac{1}{k_{L\ell n}}\frac{\partial j_{\ell}(k_{L\ell n}a)}{\partial a}\right)\\
% &\qquad + e\eta\sum_{p\ell mn}4\pi(1 - \delta_{p1}\delta_{m0})\frac{k_{L\ell n}}{\sqrt{2\ell + 1}}\mathcal{B}_{Lp\ell mn}(t)\left(f_{p\ell m}^>(\mathbf{r})\Theta(a - r)\int_0^r j_\ell(k_{L\ell n}r')r'^{\ell + 2}\;\mathrm{d}r'\right.\\
% &\qquad\qquad\left.+ f_{p\ell m}^>(\mathbf{r})\Theta(r - a)\int_0^aj_\ell(k_{L\ell n}r')r'^{\ell' + 2}\;\mathrm{d}r' + \Theta(a - r)f_{p\ell m}^<(\mathbf{r})\int_r^a j_\ell(k_{L\ell n}r')r'^{-\ell + 1}\;\mathrm{d}r'\right)\\
&= e\eta a^2\sum_{p\ell mn}\frac{4\pi(1 - \delta_{p1}\delta_{m0})}{\sqrt{\ell(\ell + 1)}\sqrt{2\ell + 1}}\left(f_{p\ell m}^>(\mathbf{r})a^\ell\Theta(r - a) + f_{p\ell m}^<(\mathbf{r})\frac{1}{a^{\ell + 1}}\Theta(a - r)\right)\\
&\qquad\times\left(\mathcal{B}_{Ep\ell mn}(t)\frac{\ell(\ell + 1)}{k_{E\ell n}a}j_\ell(k_{E\ell n}a) + \mathcal{B}_{Lp\ell mn}(t)\frac{\sqrt{\ell(\ell + 1)}}{k_{L\ell n}}\frac{\partial j_{\ell}(k_{L\ell n}a)}{\partial a}\right)\\
&\qquad + e\eta\sum_{p\ell mn}4\pi(1 - \delta_{p1}\delta_{m0})\frac{k_{L\ell n}}{\sqrt{2\ell + 1}}\mathcal{B}_{Lp\ell mn}(t)\left(f_{p\ell m}^>(\mathbf{r})\frac{r^{\ell + 2}}{k_{L\ell n}}j_{\ell + 1}(k_{L\ell n}r)\Theta(a - r)\right.\\
&\qquad\qquad + f_{p\ell m}^>(\mathbf{r})\frac{a^{\ell + 2}}{k_{L\ell n}}j_{\ell + 1}(k_{L\ell n}a)\Theta(r - a) + f_{p\ell m}^<(\mathbf{r})\frac{1}{k_{L\ell n}r^{\ell - 1}}j_{\ell - 1}(k_{L\ell n}r)\Theta(a - r)\\
&\qquad\qquad\left. - f_{p\ell m}^<(\mathbf{r})\frac{1}{k_{L\ell n}a^{\ell - 1}}j_{\ell - 1}(k_{L\ell n}a)\Theta(a - r)\right).
\end{split}
\end{equation}
Noting that
\begin{equation}
\begin{split}
\nabla\left\{f_{p\ell m}^>(\mathbf{r})a^\ell\Theta(r - a) + \frac{f_{p\ell m}^<(\mathbf{r})}{a^{\ell + 1}}\Theta(a - r)\right\} &= \nabla f_{p\ell m}^>(\mathbf{r})a^\ell\Theta(r - a) + \frac{\nabla f_{p\ell m}^<(\mathbf{r})}{a^{\ell + 1}}\Theta(a - r)\\
&= \frac{1}{a^2}\mathbf{Z}_{p\ell m}(\mathbf{r};a),
\end{split}
\end{equation}
with the vector harmonics $\mathbf{Z}_{p\ell m}(\mathbf{r};a)$ defined implicitly for convenience, we can see that
\begin{equation}
\begin{split}
-\nabla\Phi_m(\mathbf{r},t) 
% &= -e\eta\sum_{p\ell mn}\frac{4\pi(1 - \delta_{p1}\delta_{m0})}{\sqrt{\ell(\ell + 1)}\sqrt{2\ell + 1}}\mathbf{Z}_{p\ell m}(\mathbf{r};a)\left(\mathcal{B}_{Ep\ell mn}(t)\frac{\ell(\ell + 1)}{k_{Ep\ell mn}a}j_\ell(k_{E\ell n}a) \right.\\
% &\qquad\left. + \mathcal{B}_{Lp\ell mn}(t)\frac{\sqrt{\ell(\ell + 1)}}{k_{L\ell n}}\frac{\partial j_{\ell}(k_{L\ell n}a)}{\partial a}\right)+ e\eta\sum_{p\ell mn}4\pi(1 - \delta_{p1}\delta_{m0})\frac{k_{L\ell n}}{\sqrt{2\ell + 1}}\mathcal{B}_{Lp\ell mn}(t)\\
% &\qquad\qquad\times\left(\nabla f_{p\ell m}^>(\mathbf{r})\Theta(a - r)\frac{r^{\ell + 2}}{k_{L\ell n}}j_{\ell + 1}(k_{L\ell n}r) - \hat{\mathbf{r}}f_{p\ell m}^>(\mathbf{r})\delta(r - a)\frac{r^{\ell + 2}}{k_{L\ell n}}j_{\ell + 1}(k_{L\ell n}r)\right.\\
% &\qquad\qquad + f_{p\ell m}^>(\mathbf{r})\Theta(a - r)(\ell + 2)\frac{r^{\ell + 1}}{k_{L\ell n}}j_{\ell + 1}(k_{L\ell n}r)\hat{\mathbf{r}} + f_{p\ell m}^>(\mathbf{r})\Theta(a - r)\frac{r^{\ell + 2}}{k_{L\ell n}}\frac{\partial j_{\ell + 1}(k_{L\ell n}r)}{\partial r}\hat{\mathbf{r}}\\
% &\qquad\qquad + \nabla f_{p\ell m}^>(\mathbf{r})\Theta(r - a)\frac{a^{\ell + 2}}{k_{L\ell n}}j_{\ell + 1}(k_{L\ell n}a) + \hat{\mathbf{r}}f_{p\ell m}^>(\mathbf{r})\delta(r - a)\frac{a^{\ell + 2}}{k_{L\ell n}}j_{\ell + 1}(k_{L\ell n}a)\\
% &\qquad\qquad + \nabla f_{p\ell m}^<(\mathbf{r})\Theta(a - r)\frac{j_{\ell - 1}(k_{L\ell n}r)}{k_{L\ell n}r^{\ell - 1}} - \hat{\mathbf{r}}f_{p\ell m}^<(\mathbf{r})\delta(r - a)\frac{j_{\ell - 1}(k_{L\ell n}r)}{k_{L\ell n}r^{\ell - 1}}\\
% &\qquad\qquad + (-\ell + 1)f_{p\ell m}^<(\mathbf{r})\Theta(a - r)\frac{j_{\ell - 1}(k_{L\ell n}r)}{k_{L\ell n}r^\ell}\hat{\mathbf{r}} + f_{p\ell m}^<(\mathbf{r})\Theta(a - r)\frac{1}{k_{L\ell n}r^{\ell - 1}}\frac{\partial j_{\ell - 1}(k_{L\ell n}r)}{\partial r}\hat{\mathbf{r}}\\
% &\qquad\qquad\left. - \nabla f_{p\ell m}^<(\mathbf{r})\Theta(a - r)\frac{j_{\ell - 1}(k_{L\ell n}a)}{k_{L\ell n}a^{\ell - 1}} + \hat{\mathbf{r}}f_{p\ell m}^<(\mathbf{r})\delta(r - a)\frac{j_{\ell - 1}(k_{L\ell n}a)}{k_{L\ell n}a^{\ell - 1}}\right)\\
&= -e\eta\sum_{p\ell mn}\frac{4\pi(1 - \delta_{p1}\delta_{m0})}{\sqrt{\ell(\ell + 1)}\sqrt{2\ell + 1}}\mathbf{Z}_{p\ell m}(\mathbf{r};a)\left(\mathcal{B}_{Ep\ell mn}(t)\frac{\ell(\ell + 1)}{k_{E\ell n}a}j_\ell(k_{E\ell n}a) \right.\\
&\qquad\left. + \mathcal{B}_{Lp\ell mn}(t)\frac{\sqrt{\ell(\ell + 1)}}{k_{L\ell n}}\frac{\partial j_{\ell}(k_{L\ell n}a)}{\partial a}\right)+ e\eta\sum_{p\ell mn}4\pi(1 - \delta_{p1}\delta_{m0})\frac{1}{\sqrt{2\ell + 1}}\mathcal{B}_{Lp\ell mn}(t)\\
&\qquad\qquad\times\left(\nabla f_{p\ell m}^>(\mathbf{r})\Theta(a - r)r^{\ell + 2}j_{\ell + 1}(k_{L\ell n}r) + \nabla f_{p\ell m}^<(\mathbf{r})\Theta(a - r)\frac{j_{\ell - 1}(k_{L\ell n}r)}{r^{\ell - 1}}\right.\\
&\qquad\qquad + \nabla f_{p\ell m}^>(\mathbf{r})\Theta(r - a)a^{\ell + 2}j_{\ell + 1}(k_{L\ell n}a) - \nabla f_{p\ell m}^<(\mathbf{r})\Theta(a - r)\frac{j_{\ell - 1}(k_{L\ell n}a)}{a^{\ell - 1}}\\
&\qquad\qquad + f_{p\ell m}^>(\mathbf{r})\Theta(a - r)(\ell + 2)r^{\ell + 1}j_{\ell + 1}(k_{L\ell n}r)\hat{\mathbf{r}} + (-\ell + 1)f_{p\ell m}^<(\mathbf{r})\Theta(a - r)\frac{j_{\ell - 1}(k_{L\ell n}r)}{r^\ell}\hat{\mathbf{r}}\\
&\qquad\qquad\left. + f_{p\ell m}^>(\mathbf{r})\Theta(a - r)r^{\ell + 2}\frac{\partial j_{\ell + 1}(k_{L\ell n}r)}{\partial r}\hat{\mathbf{r}} + f_{p\ell m}^<(\mathbf{r})\Theta(a - r)\frac{1}{r^{\ell - 1}}\frac{\partial j_{\ell - 1}(k_{L\ell n}r)}{\partial r}\hat{\mathbf{r}}\right).
\end{split}
\end{equation}
Therefore, noting that
\begin{equation}
\begin{split}
\hat{\mathbf{r}}\times\nabla f_{p\ell m}^<(\mathbf{r}) &= \sqrt{(2 - \delta_{m0})\frac{(\ell - m)!}{(\ell + m)!}}r^{\ell - 1}\left(\frac{\partial P_{\ell m}(\cos\theta)}{\partial \theta}S_p(m\phi)\hat{\bm{\phi}} + \frac{(-1)^pm}{\sin\theta}P_{\ell m}(\cos\theta)S_{p+1}(m\phi)\hat{\bm{\theta}}\right),\\
\hat{\mathbf{r}}\times\nabla f_{p\ell m}^>(\mathbf{r}) &= \sqrt{(2 - \delta_{m0})\frac{(\ell - m)!}{(\ell + m)!}}\frac{1}{r^{\ell + 2}}\left(\frac{\partial P_{\ell m}(\cos\theta)}{\partial \theta}S_p(m\phi)\hat{\bm{\phi}} + \frac{(-1)^pm}{\sin\theta}P_{\ell m}(\cos\theta)S_{p+1}(m\phi)\hat{\bm{\theta}}\right),\\
\hat{\mathbf{r}}\cdot\nabla f_{p\ell m}^<(\mathbf{r}) &= \sqrt{(2 - \delta_{m0})\frac{(\ell - m)!}{(\ell + m)!}}\ell r^{\ell - 1}P_{\ell m}(\cos\theta)S_p(m\phi),\\
\hat{\mathbf{r}}\cdot\nabla f_{p\ell m}^>(\mathbf{r}) &= \sqrt{(2 - \delta_{m0})\frac{(\ell - m)!}{(\ell + m)!}}\frac{(-\ell - 1)}{r^{\ell + 2}}P_{\ell m}(\cos\theta)S_p(m\phi),
\end{split}
\end{equation}
we can say
\begin{equation}
\begin{split}
-\hat{\mathbf{r}}\times\nabla\Phi_m(\mathbf{r},t) &= -e\eta\sum_{p\ell mn}\frac{4\pi(1 - \delta_{p1}\delta_{m0})}{\sqrt{\ell(\ell + 1)}\sqrt{2\ell + 1}}\sqrt{(2 - \delta_{m0})\frac{(\ell - m)!}{(\ell + m)!}}\left(\frac{a^{\ell + 2}}{r^{\ell + 2}}\Theta(r - a) + \frac{r^{\ell - 1}}{a^{\ell - 1}}\Theta(a - r)\right)\\
&\qquad\times\left(\frac{\partial P_{\ell m}(\cos\theta)}{\partial \theta}S_p(m\phi)\hat{\bm{\phi}} + \frac{(-1)^pm}{\sin\theta}P_{\ell m}(\cos\theta)S_{p+1}(m\phi)\hat{\bm{\theta}}\right)\\
&\qquad\times\left(\mathcal{B}_{Ep\ell mn}(t)\frac{\ell(\ell + 1)}{k_{Ep\ell mn}a}j_\ell(k_{E\ell n}a) + \mathcal{B}_{Lp\ell mn}(t)\frac{\sqrt{\ell(\ell + 1)}}{k_{L\ell n}}\frac{\partial j_{\ell}(k_{L\ell n}a)}{\partial a}\right)\\
&\qquad + e\eta\sum_{p\ell mn}4\pi(1 - \delta_{p1}\delta_{m0})\frac{1}{\sqrt{2\ell + 1}}\mathcal{B}_{Lp\ell mn}(t)\sqrt{(2 - \delta_{m0})\frac{(\ell - m)!}{(\ell + m)!}}\\
&\qquad\qquad\times\left(\frac{\partial P_{\ell m}(\cos\theta)}{\partial \theta}S_p(m\phi)\hat{\bm{\phi}} + \frac{(-1)^pm}{\sin\theta}P_{\ell m}(\cos\theta)S_{p+1}(m\phi)\hat{\bm{\theta}}\right)\\
&\qquad\qquad\times\left[j_{\ell + 1}(k_{L\ell n}r)\Theta(a - r) + j_{\ell - 1}(k_{L\ell n}r)\Theta(a - r) + \frac{a^{\ell + 2}}{r^{\ell + 2}}j_{\ell + 1}(k_{L\ell n}a)\Theta(r - a)\right.\\
&\qquad\qquad\qquad\left. - \frac{r^{\ell - 1}}{a^{\ell - 1}}j_{\ell - 1}(k_{L\ell n}a)\Theta(a - r)\right].
\end{split}
\end{equation}
Due to the symmetric structure of the $r$-dependent factor of each term of the first sum above, the first sum approaches the same value as $r\to a$ from inside or outside the boundary. The $r$-dependent factors of the second sum are, after some cancellation, similarly identical on either side of the boundary as $r\to a$. Therefore,
\begin{equation}
\hat{\mathbf{r}}\times\left(-\lim_{r\to a^+}\nabla\Phi_m(\mathbf{r},t) + \lim_{r\to a^-}\nabla\Phi_m(\mathbf{r},t)\right) = 0
\end{equation}
such that the surface-tangential components of the longitudinal electric field obey Maxwell's boundary conditions.

The component of $-\nabla\Phi_m(\mathbf{r},t)$ normal to the surface is
\begin{equation}
\begin{split}
-\hat{\mathbf{r}}\cdot\nabla\Phi_m(\mathbf{r},t) &= -en\sum_{p\ell mn}\frac{4\pi(1 - \delta_{p1}\delta_{m0})}{\sqrt{\ell(\ell + 1)}\sqrt{2\ell + 1}}\sqrt{(2 - \delta_{m0})\frac{(\ell - m)!}{(\ell + m)!}}P_{\ell m}(\cos\theta)S_p(m\phi)\\
&\qquad\times\left(-(\ell + 1)\frac{a^{\ell + 2}}{r^{\ell + 2}}\Theta(r - a) + \ell\frac{r^{\ell - 1}}{a^{\ell - 1}}\Theta(a - r)\right)\\
&\qquad\times\left(\mathcal{B}_{Ep\ell mn}(t)\frac{\ell(\ell + 1)}{k_{Ep\ell mn}a}j_\ell(k_{E\ell n}a) + \mathcal{B}_{Lp\ell mn}(t)\frac{\sqrt{\ell(\ell + 1)}}{k_{L\ell n}}\frac{\partial j_{\ell}(k_{L\ell n}a)}{\partial a}\right)\\
&\qquad + en\sum_{p\ell mn}4\pi(1 - \delta_{p1}\delta_{m0})\frac{1}{\sqrt{2\ell + 1}}\mathcal{B}_{Lp\ell mn}(t)\sqrt{(2 - \delta_{m0})\frac{(\ell - m)!}{(\ell + m)!}}P_{\ell m}(\cos\theta)S_p(m\phi)\\
&\qquad\qquad\times\left[ \vphantom{\frac{\partial}{\partial r}} -(\ell + 1)j_{\ell + 1}(k_{L\ell n}r)\Theta(a - r) + \ell j_{\ell - 1}(k_{L\ell n}r)\Theta(a - r)\right.\\
&\qquad\qquad - (\ell + 1)\frac{a^{\ell + 2}}{r^{\ell + 2}}j_{\ell + 1}(k_{L\ell n}a)\Theta(r - a) - \ell\frac{r^{\ell - 1}}{a^{\ell - 1}}j_{\ell - 1}(k_{L\ell n}a)\Theta(a - r)\\
&\qquad\qquad + (\ell + 2)j_{\ell + 1}(k_{L\ell n}r)\Theta(a - r) + (-\ell + 1)j_{\ell - 1}(k_{L\ell n}r)\Theta(a - r)\\
&\qquad\qquad\left. + r\frac{\partial j_{\ell + 1}(k_{L\ell n}r)}{\partial r}\Theta(a - r) + r\frac{\partial j_{\ell - 1}(k_{L\ell n}r)}{\partial r}\Theta(a - r)\right].
\end{split}
\end{equation}
Analyzing this term takes a little more work. Taking limits on either side of the boundary, one finds
\begin{equation}
\begin{split}
-\lim_{r\to a^+}\hat{\mathbf{r}}\cdot\nabla\Phi_m(\mathbf{r},t) &= e\eta\sum_{p\ell mn}\frac{4\pi(1 - \delta_{p1}\delta_{m0})}{\sqrt{\ell(\ell + 1)}\sqrt{2\ell + 1}}\sqrt{(2 - \delta_{m0})\frac{(\ell - m)!}{(\ell + m)!}}P_{\ell m}(\cos\theta)S_p(m\phi)(\ell + 1)\\
&\qquad\times\left(\mathcal{B}_{Ep\ell mn}(t)\frac{\ell(\ell + 1)}{k_{Ep\ell mn}a}j_\ell(k_{E\ell n}a) + \mathcal{B}_{Lp\ell mn}(t)\frac{\sqrt{\ell(\ell + 1)}}{k_{L\ell n}}\frac{\partial j_{\ell}(k_{L\ell n}a)}{\partial a}\right)\\
&\qquad - e\eta\sum_{p\ell mn}4\pi(1 - \delta_{p1}\delta_{m0})\frac{1}{\sqrt{2\ell + 1}}\mathcal{B}_{Lp\ell mn}(t)\sqrt{(2 - \delta_{m0})\frac{(\ell - m)!}{(\ell + m)!}}\\
&\qquad\qquad\times P_{\ell m}(\cos\theta)S_p(m\phi)(\ell + 1)j_{\ell + 1}(k_{L\ell n}a)
\end{split}
\end{equation}
and
\begin{equation}
\begin{split}
-\lim_{r\to a^-}\hat{\mathbf{r}}\cdot\nabla\Phi_m(\mathbf{r},t) 
% &= -e\eta\sum_{p\ell mn}\frac{4\pi(1 - \delta_{p1}\delta_{m0})}{\sqrt{\ell(\ell + 1)}\sqrt{2\ell + 1}}\sqrt{(2 - \delta_{m0})\frac{(\ell - m)!}{(\ell + m)!}}P_{\ell m}(\cos\theta)S_p(m\phi)\ell\\
% &\qquad\times\left(\mathcal{B}_{Ep\ell mn}(t)\frac{\ell(\ell + 1)}{k_{Ep\ell mn}a}j_\ell(k_{E\ell n}a) + \mathcal{B}_{Lp\ell mn}(t)\frac{\sqrt{\ell(\ell + 1)}}{k_{L\ell n}}\frac{\partial j_{\ell}(k_{L\ell n}a)}{\partial a}\right)\\
% &\qquad + e\eta\sum_{p\ell mn}4\pi(1 - \delta_{p1}\delta_{m0})\frac{1}{\sqrt{2\ell + 1}}\mathcal{B}_{Lp\ell mn}(t)\sqrt{(2 - \delta_{m0})\frac{(\ell - m)!}{(\ell + m)!}}\\
% &\qquad\qquad\times P_{\ell m}(\cos\theta)S_p(m\phi)\left[ \vphantom{\frac{\partial}{\partial r}}  j_{\ell + 1}(k_{L\ell n}a) + (-\ell + 1)j_{\ell - 1}(k_{L\ell n}a)\right.\\
% &\qquad\qquad\left. + a\frac{\partial}{\partial a}\left\{j_{\ell + 1}(k_{L\ell n}a) + j_{\ell - 1}(k_{L\ell n}a)\right\}\right]\\
% &= -e\eta\sum_{p\ell mn}\frac{4\pi(1 - \delta_{p1}\delta_{m0})}{\sqrt{\ell(\ell + 1)}\sqrt{2\ell + 1}}\sqrt{(2 - \delta_{m0})\frac{(\ell - m)!}{(\ell + m)!}}P_{\ell m}(\cos\theta)S_p(m\phi)\ell\\
% &\qquad\times\left(\mathcal{B}_{Ep\ell mn}(t)\frac{\ell(\ell + 1)}{k_{Ep\ell mn}a}j_\ell(k_{E\ell n}a) + \mathcal{B}_{Lp\ell mn}(t)\frac{\sqrt{\ell(\ell + 1)}}{k_{L\ell n}}\frac{\partial j_{\ell}(k_{L\ell n}a)}{\partial a}\right)\\
% &\qquad + e\eta\sum_{p\ell mn}4\pi(1 - \delta_{p1}\delta_{m0})\frac{1}{\sqrt{2\ell + 1}}\mathcal{B}_{Lp\ell mn}(t)\sqrt{(2 - \delta_{m0})\frac{(\ell - m)!}{(\ell + m)!}}\\
% &\qquad\qquad\times P_{\ell m}(\cos\theta)S_p(m\phi)\left[ \vphantom{\frac{\partial}{\partial r}} j_{\ell + 1}(k_{L\ell n}a) + (-\ell + 1)j_{\ell - 1}(k_{L\ell n}a)\right.\\
% &\qquad\qquad\left. + k_{L\ell n}a\left(j_{\ell}(k_{L\ell n}a) - \frac{\ell + 2}{k_{L\ell n}a}j_{\ell + 1}(k_{L\ell n}a) - j_\ell(k_{L\ell n}a) + \frac{\ell - 1}{k_{L\ell n}a}j_{\ell - 1}(k_{L\ell n}a)\right)\right]\\
&= -e\eta\sum_{p\ell mn}\frac{4\pi(1 - \delta_{p1}\delta_{m0})}{\sqrt{\ell(\ell + 1)}\sqrt{2\ell + 1}}\sqrt{(2 - \delta_{m0})\frac{(\ell - m)!}{(\ell + m)!}}P_{\ell m}(\cos\theta)S_p(m\phi)\ell\\
&\qquad\times\left(\mathcal{B}_{Ep\ell mn}(t)\frac{\ell(\ell + 1)}{k_{Ep\ell mn}a}j_\ell(k_{E\ell n}a) + \mathcal{B}_{Lp\ell mn}(t)\frac{\sqrt{\ell(\ell + 1)}}{k_{L\ell n}}\frac{\partial j_{\ell}(k_{L\ell n}a)}{\partial a}\right)\\
&\qquad - e\eta\sum_{p\ell mn}4\pi(1 - \delta_{p1}\delta_{m0})\frac{1}{\sqrt{2\ell + 1}}\mathcal{B}_{Lp\ell mn}(t)\sqrt{(2 - \delta_{m0})\frac{(\ell - m)!}{(\ell + m)!}}\\
&\qquad\qquad\times P_{\ell m}(\cos\theta)S_p(m\phi)(\ell + 1)j_{\ell + 1}(k_{L\ell n}a),
\end{split}
\end{equation}
wherein we have used the recurrence relations
\begin{equation}
\begin{split}
\frac{\partial j_{\ell + 1}(x)}{\partial x} &= j_\ell(x) - \frac{\ell + 2}{x}j_{\ell + 1}(x),\\
\frac{\partial j_{\ell - 1}(x)}{\partial x} &= -j_\ell(x) + \frac{\ell - 1}{x}j_{\ell - 1}(x).
\end{split}
\end{equation}
Therefore, using the identities
\begin{equation}
\begin{split}
\hat{\mathbf{r}}\cdot\mathbf{N}_{p\ell m}(\mathbf{r},k) &= \sqrt{(2 - \delta_{m0})\frac{2\ell + 1}{\ell(\ell + 1)}\frac{(\ell - m)!}{(\ell + m)!}}\frac{\ell(\ell + 1)}{kr}j_{\ell}(kr)P_{\ell m}(\cos\theta)S_p(m\phi),\\
\hat{\mathbf{r}}\cdot\mathbf{L}_{p\ell m}(\mathbf{r},k) &= \sqrt{(2 - \delta_{m0})\frac{2\ell + 1}{\ell(\ell + 1)}\frac{(\ell - m)!}{(\ell + m)!}}\sqrt{\ell(\ell + 1)}\frac{1}{k}\frac{\partial j_{\ell}(kr)}{\partial r}P_{\ell m}(\cos\theta)S_p(m\phi),
\end{split}
\end{equation}
and $(1 - \delta_{p1}\delta_{m0})S_p(m\phi) = S_p(m\phi)$, we can see that
\begin{equation}
\begin{split}
\hat{\mathbf{r}}&\cdot\left(-\lim_{r\to a^+}\nabla\Phi_m(\mathbf{r},t) + \lim_{r\to a^-}\nabla\Phi_m(\mathbf{r},t)\right)
% = en\sum_{p\ell mn}\frac{4\pi}{\sqrt{\ell(\ell + 1)}\sqrt{2\ell + 1}}\sqrt{(2 - \delta_{m0})\frac{(\ell - m)!}{(\ell + m)!}}(2\ell + 1)\\
% &\qquad\times P_{\ell m}(\cos\theta)S_p(m\phi)\left(\mathcal{B}_{Ep\ell mn}(t)\frac{\ell(\ell + 1)}{k_{Ep\ell mn}a}j_\ell(k_{E\ell n}a) + \mathcal{B}_{Lp\ell mn}(t)\frac{\sqrt{\ell(\ell + 1)}}{k_{L\ell n}}\frac{\partial j_{\ell}(k_{L\ell n}a)}{\partial a}\right)\\
% &= 4\pi \hat{\mathbf{r}}\cdot en\sum_{p\ell mn}\left[\mathcal{B}_{Ep\ell mn}(t)\mathbf{N}_{p\ell m}(a,\theta,\phi;k_{E\ell n}) + \mathcal{B}_{Lp\ell mn}(t)\mathbf{L}_{p\ell m}(a,\theta,\phi;k_{L\ell n})\right]\\
% &
= 4\pi\hat{\mathbf{r}}\cdot\left.\mathbf{P}(\mathbf{r},t)\right|_{r\to a}\\
\end{split}
\end{equation}
The surface normal components of the longitudinal electric field therefore also obey Maxwell's boundary conditions. Therefore, there are no restrictions on the wavenumbers $k_{\bm{\beta}n}$ other than those set by the requirements of Sturm-Liouville expansions.











\subsection{Expansion of the Lagrangian}

For simplicity, then, we will choose the wavenumbers $k_{\bm{\beta}n} = (\delta_{TM} + \delta_{TE})z_{\ell n}/a + \delta_{TL}w_{\ell n}/a$, where $z_{\ell n}$ are the roots of the spherical Bessel function $j_\ell(x)$ and $w_{\ell n}$ are the roots of the derivative $\partial j_\ell(x)/\partial x$, with $n = 1,2,\ldots$ identifying the first, second, etc. root above zero. These wavenumbers guarantee that the radial components of each vector harmonic are zero at the boundary, and they simplify many of our calculations while providing a complete basis for expansion.

Explicitly, we can define the system Lagrangian as $L = \int\mathcal{L}\,\mathrm{d}^3\mathbf{r}$. Using the identities of Appendix \ref{app:harmonicAlgebra}, we can say
\begin{equation}
\begin{split}
L_m &= \frac{\mu}{2}\int_{r<a}\left(\dot{\mathbf{Q}}^2(\mathbf{r},t) - \omega_0^2\mathbf{Q}^2(\mathbf{r},t)\right)\mathrm{d}^3\mathbf{r}\\
% &= \frac{\mu}{2}\left(\;\int_{r<a}\sum_{\bm{\beta}n}\dot{\mathcal{B}}_{\bm{\beta}n}(t)\mathbf{X}_{\bm{\beta}}(\mathbf{r},k_{\bm{\beta} n})\cdot\sum_{\bm{\beta}'n'}\dot{\mathcal{B}}_{\bm{\beta}'n'}(t)\mathbf{X}_{\bm{\beta}'}(\mathbf{r},k_{\bm{\beta}'n'})\;\mathrm{d}^3\mathbf{r}\right.\\
% &\qquad\left. - \int_{r<a}\omega_0^2\sum_{\bm{\beta}n}\mathcal{B}_{\bm{\beta}n}(t)\mathbf{X}_{\bm{\beta}}(\mathbf{r},k_{\bm{\beta} n})\cdot\sum_{\bm{\beta}'n'}\mathcal{B}_{\bm{\beta}'n'}(t)\mathbf{X}_{\bm{\beta}'}(\mathbf{r},k_{\bm{\beta}'n'})\;\mathrm{d}^3\mathbf{r}\right)\\
% &= \frac{\mu}{2}\left(\sum_{\bm{\beta}\bm{\beta}'nn'}\left[\dot{\mathcal{B}}_{\bm{\beta}n}(t)\dot{\mathcal{B}}_{\bm{\beta}'n'}(t) - \omega_0^2\mathcal{B}_{\bm{\beta}n}(t)\mathcal{B}_{\bm{\beta}'n'}(t)\right]2\pi a^3(1 - \delta_{p1}\delta_{m0})\delta_{\bm{\beta}\bm{\beta}'}\delta_{nn'}\right.\\
% &\qquad\left.\times\left[(\delta_{TM} + \delta_{TE})j_{\ell+1}^2(k_{\bm{\beta}n}a) + \delta_{TL}\left(1 - \frac{\ell(\ell + 1)}{(k_{\bm{\beta}n}a)^2}\right)j_\ell^2(k_{\bm{\beta}n}a)\right]\right)\\
% &= \pi a^3\mu\sum_{\bm{\beta}n}(1 - \delta_{p1}\delta_{m0})\left[ \vphantom{\frac{\ell(\ell + 1)}{(k_{\bm{\beta}n}a)^2}} (\delta_{TM} + \delta_{TE})j_{\ell + 1}^2(z_{\ell n})\left(\dot{\mathcal{B}}_{\bm{\beta}n}^2(t) - \omega_0^2\mathcal{B}_{\bm{\beta}n}^2(t)\right)\right.\\
% &\qquad\left. + \delta_{TL}j_\ell^2(w_{\ell n})\left(1 - \frac{\ell(\ell + 1)}{w_{\ell n}^2}\right)\left(\dot{\mathcal{B}}_{\bm{\beta}n}^2(t) - \omega_0^2\mathcal{B}_{\bm{\beta}n}^2(t)\right)\right]\\
&= \sum_{\bm{\beta}n}\frac{m_{\bm{\beta}n}}{2}\left(\dot{\mathcal{B}}_{\bm{\beta}n}^2(t) - \omega_0^2\mathcal{B}_{\bm{\beta}n}^2(t)\right)
\end{split}
\end{equation}
and
\begin{equation}
\begin{split}
L_r &= \int_{r<a}\int_0^\infty\frac{\mu}{2}\left(\dot{\mathbf{Q}}_\nu^2(\mathbf{r},t) - \nu^2\mathbf{Q}_\nu^2(\mathbf{r},t)\right)\mathrm{d}\nu\,\mathrm{d}^3\mathbf{r}\\
% &= \int_0^\infty\frac{\mu_{\nu}}{2}\left(\;\int_{r<a}\sum_{\bm{\gamma}n}\dot{\mathcal{C}}_{\bm{\gamma}n}(\nu,t)\mathbf{X}_{\bm{\gamma}}(\mathbf{r},k_{\bm{\gamma} n})\cdot\sum_{\bm{\gamma}'n'}\dot{\mathcal{C}}_{\bm{\gamma}'n'}(\nu,t)\mathbf{X}_{\bm{\gamma}'}(\mathbf{r},k_{\bm{\gamma}'n'})\;\mathrm{d}^3\mathbf{r}\right.\\
% &\qquad\left. - \int_{r<a}\nu^2\sum_{\bm{\gamma}n}\mathcal{C}_{\bm{\gamma}n}(\nu,t)\mathbf{X}_{\bm{\gamma}}(\mathbf{r},k_{\bm{\gamma} n})\cdot\sum_{\bm{\gamma}'n'}\mathcal{C}_{\bm{\gamma}'n'}(\nu,t)\mathbf{X}_{\bm{\gamma}'}(\mathbf{r},k_{\bm{\gamma}'n'})\;\mathrm{d}^3\mathbf{r}\right)\\
% &= \int_0^\infty\frac{\mu_{\nu}}{2}\left(\sum_{\bm{\gamma}\bm{\gamma}'nn'}\left[\dot{\mathcal{C}}_{\bm{\gamma}n}(\nu,t)\dot{\mathcal{C}}_{\bm{\gamma}'n'}(\nu,t) - \nu^2\mathcal{C}_{\bm{\gamma}n}(\nu,t)\mathcal{C}_{\bm{\gamma}'n'}(\nu,t)\right]2\pi a^3(1 - \delta_{p1}\delta_{m0})\delta_{\bm{\gamma}\bm{\gamma}'}\delta_{nn'}\right.\\
% &\qquad\left.\times \left[(\delta_{TM} + \delta_{TE})j_{\ell + 1}^2(k_{\bm{\gamma}}a) + \delta_{TL}\left(1 - \frac{\ell(\ell + 1)}{(k_{\bm{\gamma}}a)^2}\right)j_\ell^2(k_{\bm{\gamma}n}a)\right] \vphantom{\sum_{\bm{\gamma'}}} \right)\\
% &= \int_0^\infty\pi a^3\mu\sum_{\bm{\gamma}n}(1 - \delta_{p1}\delta_{m0})\left[ \vphantom{\left[\nu^2 - \frac{D_\nu w_{\ell n}^2}{m_\nu a^2}\right]} (\delta_{TM} + \delta_{TE})j_{\ell + 1}^2(z_{\ell n})\left(\dot{\mathcal{C}}_{\bm{\gamma}n}^2(\nu,t) - \nu^2\mathcal{C}_{\bm{\gamma}n}^2(\nu,t)\right)\right.\\
% &\qquad\left. + \delta_{TL}j_\ell^2(w_{\ell n})\left(1 - \frac{\ell(\ell + 1)}{w_{\ell n}^2}\right)\left(\dot{\mathcal{C}}_{\bm{\gamma}n}^2(\nu,t) - \nu^2\mathcal{C}_{\bm{\gamma}n}^2(\nu,t)\right)\right]\mathrm{d}\nu\\
&= \sum_{\bm{\gamma}n}\int_0^\infty\frac{m_{\bm{\gamma}n}}{2}\left(\dot{\mathcal{C}}_{\bm{\gamma}n}^2(\nu,t) - \nu^2\mathcal{C}_{\bm{\gamma}n}^2(\nu,t)\right)\mathrm{d}\nu
\end{split}
\end{equation}
where
\begin{equation}
m_{\bm{\beta}n} = 2\pi a^3\mu\left[(\delta_{TM} + \delta_{TE})j_{\ell + 1}^2(z_{\ell n}) + \delta_{TL}\left(1 - \frac{\ell(\ell + 1)}{w_{\ell n}^2}\right)j_\ell^2(w_{\ell n})\right]
\end{equation}
is the effective mass of the $(\bm{\beta},n)^\mathrm{th}$ mode.

Further, after inserting our specified values of the roots $k_{\bm{\beta}n}$ into the matter field such that it simplifies to
\begin{equation}
\begin{split}
-\nabla\Phi_m(\mathbf{r},t) 
% &= e\eta\sum_{p\ell mn}\frac{4\pi}{\sqrt{2\ell + 1}}\mathcal{B}_{Lp\ell mn}(t)\sqrt{(2 - \delta_{m0})\frac{(\ell - m)!}{(\ell + m)!}}\\
% &\qquad\times\left(\frac{1}{r^{\ell + 2}}\left[-(\ell + 1)P_{\ell m}(\cos\theta)S_p(m\phi)\hat{\mathbf{r}} + \frac{\partial P_{\ell m}(\cos\theta)}{\partial \theta}S_p(m\phi)\hat{\bm{\theta}}\right.\right.\\
% &\qquad\qquad\left. + (-1)^{p+1}m\frac{P_{\ell m}(\cos\theta)}{\sin\theta}S_{p+1}(m\phi)\hat{\bm{\phi}}\right]r^{\ell + 2}j_{\ell + 1}\left(\frac{w_{\ell n}r}{a}\right)\Theta(a - r)\\
% &\qquad+ r^{\ell - 1}\left[\ell P_{\ell m}(\cos\theta)S_p(m\phi)\hat{\mathbf{r}}+ \frac{\partial P_{\ell m}(\cos\theta)}{\partial \theta}S_p(m\phi)\hat{\bm{\theta}}\right.\\
% &\qquad\qquad\left. + (-1)^{p+1}m\frac{P_{\ell m}(\cos\theta)}{\sin\theta}S_{p+1}(m\phi)\hat{\bm{\phi}}\right]\frac{1}{r^{\ell - 1}}j_{\ell - 1}\left(\frac{w_{\ell n}r}{a}\right)\Theta(a - r)\\
% &\qquad + \frac{1}{r^{\ell + 2}}\left[-(\ell + 1)P_{\ell m}(\cos\theta)S_p(m\phi)\hat{\mathbf{r}} + \frac{\partial P_{\ell m}(\cos\theta)}{\partial \theta}S_p(m\phi)\hat{\bm{\theta}}\right.\\
% &\qquad\qquad\left. + (-1)^{p+1}m\frac{P_{\ell m}(\cos\theta)}{\sin\theta}S_{p+1}(m\phi)\hat{\bm{\phi}}\right]a^{\ell + 2}j_{\ell + 1}(w_{\ell n})\Theta(r - a)\\
% &\qquad - r^{\ell - 1}\left[\ell P_{\ell m}(\cos\theta)S_p(m\phi)\hat{\mathbf{r}}+ \frac{\partial P_{\ell m}(\cos\theta)}{\partial \theta}S_p(m\phi)\hat{\bm{\theta}}\right.\\
% &\qquad\qquad\left. + (-1)^{p+1}m\frac{P_{\ell m}(\cos\theta)}{\sin\theta}S_{p+1}(m\phi)\hat{\bm{\phi}}\right]\frac{1}{a^{\ell - 1}}j_{\ell - 1}(w_{\ell n})\Theta(a - r)\\
% &\qquad + (\ell + 2)j_{\ell + 1}\left(\frac{w_{\ell n}r}{a}\right)P_{\ell m}(\cos\theta)S_p(m\phi)\Theta(a - r)\hat{\mathbf{r}}\\
% &\qquad + (-\ell + 1)j_{\ell - 1}\left(\frac{w_{\ell n}r}{a}\right)P_{\ell m}(\cos\theta)S_p(m\phi)\Theta(a - r)\hat{\mathbf{r}}\\
% &\qquad + \frac{w_{\ell n}r}{a}\left[j_{\ell}\left(\frac{w_{\ell n}r}{a}\right) - \frac{(\ell + 2)a}{w_{\ell n}r}j_{\ell + 1}\left(\frac{w_{\ell n}r}{a}\right)\right]P_{\ell m}(\cos\theta)S_p(m\phi)\Theta(a - r)\hat{\mathbf{r}}\\
% &\qquad\left. + \frac{w_{\ell n}r}{a}\left[j_{\ell}\left(\frac{w_{\ell n}r}{a}\right) + \frac{(\ell - 1)a}{w_{\ell n}r}j_{\ell - 1}\left(\frac{w_{\ell n}r}{a}\right)\right]P_{\ell m}(\cos\theta)S_p(m\phi)\Theta(a - r)\hat{\mathbf{r}}\right]\\
% &= e\eta\sum_{p\ell mn}\frac{4\pi}{\sqrt{2\ell + 1}}\mathcal{B}_{Lp\ell mn}(t)\sqrt{(2 - \delta_{m0})\frac{(\ell - m)!}{(\ell + m)!}}\left( \vphantom{\frac{a^\ell}{r^\ell}} P_{\ell m}(\cos\theta)S_p(m\phi)\right.\\
% &\qquad \times\left[-(\ell + 1)j_{\ell + 1}\left(\frac{w_{\ell n}r}{a}\right)\Theta(a - r) + \ell j_{\ell - 1}\left(\frac{w_{\ell n}r}{a}\right)\Theta(a - r)\right.\\
% &\qquad\qquad\left. - (\ell + 1)\frac{a^{\ell + 2}}{r^{\ell + 2}}j_{\ell + 1}(w_{\ell n})\Theta(r - a) - \ell\frac{r^{\ell - 1}}{a^{\ell - 1}}j_{\ell - 1}(w_{\ell n})\Theta(a - r)\right]\hat{\mathbf{r}}\\
% &\qquad + \frac{\partial P_{\ell m}(\cos\theta)}{\partial \theta}S_p(m\phi)\left[j_{\ell + 1}\left(\frac{w_{\ell n}r}{a}\right)\Theta(a - r) + j_{\ell - 1}\left(\frac{w_{\ell n}r}{a}\right)\Theta(a - r)\right.\\
% &\qquad\qquad\left. +\frac{a^{\ell + 2}}{r^{\ell + 2}}j_{\ell + 1}(w_{\ell n})\Theta(r - a) - \frac{r^{\ell - 1}}{a^{\ell - 1}}j_{\ell - 1}(w_{\ell n})\Theta(a - r)\right]\hat{\bm{\theta}}\\
% &\qquad + (-1)^{p+1}m\frac{P_{\ell m}(\cos\theta)}{\sin\theta}S_{p+1}(m\phi)\left[j_{\ell + 1}\left(\frac{w_{\ell n}r}{a}\right)\Theta(a - r) + j_{\ell - 1}\left(\frac{w_{\ell n}r}{a}\right)\Theta(a - r)\right.\\
% &\qquad\qquad\left. +\frac{a^{\ell + 2}}{r^{\ell + 2}}j_{\ell + 1}(w_{\ell n})\Theta(r - a) - \frac{r^{\ell - 1}}{a^{\ell - 1}}j_{\ell - 1}(w_{\ell n})\Theta(a - r)\right]\hat{\bm{\phi}}\\
% &= e\eta\sum_{p\ell mn}\frac{4\pi}{\sqrt{2\ell + 1}}\mathcal{B}_{Lp\ell mn}(t)\sqrt{(2 - \delta_{m0})\frac{(\ell - m)!}{(\ell + m)!}}\\
% &\qquad\times\left(\left[(2\ell + 1)\frac{\partial j_{\ell}\left(\frac{w_{\ell n}r}{a}\right)}{\partial \left(\frac{w_{\ell n}r}{a}\right)}P_{\ell m}(\cos\theta)S_p(m\phi)\hat{\mathbf{r}} + \frac{(2\ell + 1)a}{w_{\ell n}r}j_\ell\left(\frac{w_{\ell n}r}{a}\right)\frac{\partial P_{\ell m}(\cos\theta)}{\partial\theta}S_p(m\phi)\hat{\bm{\theta}}\right.\right.\\
% &\qquad\qquad\left. + \frac{(2\ell + 1)a}{w_{\ell n}r}j_\ell\left(\frac{w_{\ell n}r}{a}\right)(-1)^{p + 1}m\frac{P_{\ell m}(\cos\theta)}{\sin\theta}S_{p+1}(m\phi)\hat{\bm{\phi}} \vphantom{\frac{\partial j_{\ell}\left(\frac{w_{\ell n}r}{a}\right)}{\partial \left(\frac{w_{\ell n}r}{a}\right)}} \right]\Theta(a - r)\\
% &\qquad + \frac{a^{\ell + 2}}{r^{\ell + 2}}j_{\ell + 1}(w_{\ell n})\left[-(\ell + 1)P_{\ell m}(\cos\theta)S_p(m\phi)\hat{\mathbf{r}} + \frac{\partial P_{\ell m}(\cos\theta)}{\partial \theta}S_p(m\phi)\hat{\bm{\theta}}\right.\\
% &\qquad\qquad\left. + (-1)^{p + 1}m\frac{P_{\ell m}(\cos\theta)}{\sin\theta}S_{p + 1}(m\phi)\hat{\bm{\phi}}\right]\Theta(r - a)\\
% &\qquad - \frac{r^{\ell - 1}}{a^{\ell - 1}}j_{\ell - 1}(w_{\ell n})\left[\ell P_{\ell m}(\cos\theta)S_p(m\phi)\hat{\mathbf{r}} + \frac{\partial P_{\ell m}(\cos\theta)}{\partial \theta}S_p(m\phi)\hat{\bm{\theta}}\right.\\
% &\qquad\qquad\left.\left. + (-1)^{p + 1}m\frac{P_{\ell m}(\cos\theta)}{\sin\theta}S_{p + 1}(m\phi)\hat{\bm{\phi}}\right]\Theta(a - r)  \vphantom{\frac{\partial j_{\ell}\left(\frac{w_{\ell n}r}{a}\right)}{\partial \left(\frac{w_{\ell n}r}{a}\right)}} \right)\\
&= e\eta\sum_{p\ell mn}\frac{4\pi}{\sqrt{2\ell + 1}}\mathcal{B}_{Lp\ell mn}(t)\left[\sqrt{2\ell + 1}\mathbf{L}_{p\ell m}\left(\mathbf{r},\frac{w_{\ell n}}{a}\right)\Theta(a - r) \vphantom{\frac{j_{\ell + 1}(w_{\ell n})}{a^{\ell - 1}}} \right.\\
&\qquad\left. + a^{\ell + 2}j_{\ell + 1}(w_{\ell n})\nabla f_{p\ell m}^>(\mathbf{r})\Theta(r - a) - \frac{j_{\ell - 1}(w_{\ell n})}{a^{\ell - 1}}\nabla f_{p\ell m}^<(\mathbf{r})\Theta(a - r)\right],
\end{split}
\end{equation}
we can see that
\begin{equation}
\begin{split}
\int&\nabla\Phi_m(\mathbf{r},t)\cdot\nabla\Phi_m(\mathbf{r},t)\;\mathrm{d}^3\mathbf{r}
% \\
% &= e^2\eta^2\sum_{p\ell nm}\sum_{p'\ell'm'n'}16\pi^2\mathcal{B}_{Lp\ell m}(t)\mathcal{B}_{Lp'\ell'm'}(t)\\
% &\qquad\times\left[\;\int_{r<a}\mathbf{L}_{p\ell m}(\mathbf{r},k_{L\ell n})\cdot\mathbf{L}_{p'\ell'm'}(\mathbf{r},k_{L\ell'n'})\;\mathrm{d}^3\mathbf{r}\right.\\
% &\qquad - \frac{1}{\sqrt{2\ell' + 1}}\frac{j_{\ell'+1}(w_{\ell'n'})}{a^{\ell' - 1}}\int_{r<a}\mathbf{L}_{p\ell m}(\mathbf{r},k_{L\ell n})\cdot\nabla f_{p'\ell'm'}^<(\mathbf{r})\;\mathrm{d}^3\mathbf{r}\\
% &\qquad - \frac{1}{\sqrt{2\ell + 1}}\frac{j_{\ell+1}(w_{\ell n})}{a^{\ell - 1}}\int_{r<a}\mathbf{L}_{p'\ell' m'}(\mathbf{r},k_{L\ell' n'})\cdot\nabla f_{p\ell m}^<(\mathbf{r})\;\mathrm{d}^3\mathbf{r}\\
% &\qquad + \frac{1}{\sqrt{(2\ell' + 1)(2\ell + 1)}}\frac{j_{\ell - 1}(w_{\ell n})j_{\ell' - 1}(w_{\ell'n'})}{a^{\ell - 1}a^{\ell' - 1}}\int_{r<a}\nabla f_{p\ell m}^<(\mathbf{r})\cdot\nabla f_{p'\ell'm'}^<(\mathbf{r})\;\mathrm{d}^3\mathbf{r}\\
% &\qquad\left. + \frac{1}{\sqrt{(2\ell' + 1)(2\ell + 1)}}a^{\ell+2}a^{\ell' + 2}j_{\ell + 1}(w_{\ell n})j_{\ell' + 1}(w_{\ell'n'})\int_{r>a}\nabla f_{p\ell m}^>(\mathbf{r})\cdot\nabla f_{p'\ell'm'}^>(\mathbf{r})\;\mathrm{d}^3\mathbf{r}\right]\\
% &= e^2\eta^2\sum_{p\ell m n}\sum_{p'\ell'm'n'}16\pi^2\mathcal{B}_{Lp\ell m}(t)\mathcal{B}_{Lp'\ell'm'}(t)\\
% &\qquad\times \left[2\pi a^3j_\ell^2(w_{\ell n})\left(1 - \frac{\ell(\ell + 1)}{w_{\ell n}^2}\right)\delta_{pp'}\delta_{\ell\ell'}\delta_{mm'}\delta_{nn'} - 8\pi a^3\frac{j_{\ell + 1}(w_{\ell n})j_{\ell + 1}(w_{\ell n'})}{2\ell + 1}\delta_{pp'}\delta_{\ell\ell'}\delta_{mm'}\right.\\
% &\qquad\left. + 4\pi a^3\frac{\ell + 1}{(2\ell + 1)^2}j_{\ell + 1}(w_{\ell n})j_{\ell + 1}(w_{\ell n'})\delta_{pp'}\delta_{\ell\ell'}\delta_{mm'} + 4\pi a^3\frac{\ell}{(2\ell + 1)^2}j_{\ell - 1}(w_{\ell n})j_{\ell - 1}(w_{\ell n'})\delta_{pp'}\delta_{\ell\ell'}\delta_{mm'}\right]\\
%&
% = 16\pi^2e^2\eta^2\sum_{\bm{\beta}}\sum_{nn'}\delta_{TL}\mathcal{B}_{\bm{\beta}n}(t)\mathcal{B}_{\bm{\beta}n'}(t)\left[2\pi a^3j_\ell^2(w_{\ell n})\left(1 - \frac{\ell(\ell + 1)}{w_{\ell n}^2}\right)\delta_{nn'}\right.\\
% &\qquad\left. - 4\pi a^3\frac{3\ell + 1}{(2\ell + 1)^2}j_{\ell + 1}(w_{\ell n})j_{\ell + 1}(w_{\ell n'}) + 4\pi a^3\frac{\ell}{(2\ell + 1)^2}j_{\ell - 1}(w_{\ell n})j_{\ell - 1}(w_{\ell n'})\right]\\
% &
= -8\pi\sum_{\bm{\beta}}\sum_{nn'}g_{\bm{\beta}nn'}^{(1)}\mathcal{B}_{\bm{\beta}n}(t)\mathcal{B}_{\bm{\beta}n'}(t),
\end{split}
\end{equation}
where
\begin{equation}
\begin{split}
g_{\bm{\beta}nn'}^{(1)} 
% &= -\delta_{TL}2\pi e^2\eta^2\left[2\pi a^3j_\ell^2(w_{\ell n})\left(1 - \frac{\ell(\ell + 1)}{w_{\ell n}^2}\right)\delta_{nn'}\right.\\
% &\qquad\left.- 4\pi a^3\frac{3\ell + 1}{(2\ell + 1)^2}j_{\ell + 1}(w_{\ell n})j_{\ell + 1}(w_{\ell n'}) + 4\pi a^3\frac{\ell}{(2\ell + 1)^2}j_{\ell - 1}(w_{\ell n})j_{\ell - 1}(w_{\ell n'})\right]\\
&= -\delta_{TL}\delta_{nn'}4\pi^2e^2a^3\eta^2j_\ell^2(w_{\ell n})\left(1 - \frac{\ell(\ell + 1)}{w_{\ell n}^2}\right)
\end{split}
\end{equation}
is a characteristic coupling strength between the mobile charge modes arising from the energy bound in their electric fields. The superscript $(1)$ denotes that field-overlap coupling is considered a coupling strength of the first kind in this model. Further, we can define
\begin{equation}
\begin{split}
\int&\nabla\Phi_m(\mathbf{r},t)\cdot\nabla\Phi_f(\mathbf{r},t)\;\mathrm{d}^3\mathbf{r} = 8\pi\sum_{\bm{\beta}n}F_{\bm{\beta}n}^{(1)}[\{\mathbf{x}_{fi}(t)\}]\mathcal{B}_{\bm{\beta}n}(t)
\end{split}
\end{equation}
as the energy of the driving force on the polarization field by the external charges arising from their field overlaps. Together, these allow us to say
\begin{equation}
\begin{split}
L_\mathrm{EM} &= \frac{1}{8\pi}\int\left( \vphantom{\frac{1}{c^2}} \nabla\Phi_m(\mathbf{r},t)\cdot\nabla\Phi_m(\mathbf{r},t) + \nabla\Phi_f(\mathbf{r},t)\cdot\nabla\Phi_f(\mathbf{r},t) + 2\nabla\Phi_m(\mathbf{r},t)\cdot\nabla\Phi_f(\mathbf{r},t)\right.\\
&\qquad\left. + \frac{1}{c^2}\dot{\mathbf{A}}^2(\mathbf{r},t) - \left[\nabla\times\mathbf{A}(\mathbf{r},t)\right]^2\right)\mathrm{d}^3\mathbf{r}\\
% &= \frac{1}{8\pi}\int\left[\nabla\Phi_f(\mathbf{r},t)\right]^2\mathrm{d}^3\mathbf{r} - \sum_{\bm{\beta}}\sum_{nn'}g_{\bm{\beta}nn'}^{(1)}\mathcal{B}_{\bm{\beta}n}(t)\mathcal{B}_{\bm{\beta}n'}(t) + 2\sum_{\bm{\beta}n}F_{\bm{\beta}n}^{(1)}[\{\mathbf{x}_{fi}(t)\}]\mathcal{B}_{\bm{\beta}n}(t)\\
% &\qquad + \frac{1}{c^2}\sum_{\bm{\alpha}\bm{\alpha'}}\int_0^\infty\int_0^\infty\frac{e^2\omega^\frac{3}{2}\omega^{\prime\frac{3}{2}}}{c^4}\dot{\mathcal{A}}_{\bm{\alpha}}(\omega,t)\dot{\mathcal{A}}_{\bm{\alpha}'}(\omega',t)\int\mathbf{X}_{\bm{\alpha}}(\mathbf{r},k)\cdot\mathbf{X}_{\bm{\alpha'}}(k',t)\;\mathrm{d}^3\mathbf{r}\,\mathrm{d}\omega\,\mathrm{d}\omega'\\
% &\qquad - \sum_{\bm{\alpha}\bm{\alpha'}}\int_0^\infty\int_0^\infty \frac{e^2\omega^\frac{3}{2}\omega^{\prime\frac{3}{2}}}{c^4}kk'\mathcal{A}_{\bm{\alpha}}(\omega,t)\mathcal{A}_{\bm{\alpha}'}(\omega',t)\int\mathbf{Y}_{\bm{\alpha}}(\mathbf{r},k)\cdot\mathbf{Y}_{\bm{\alpha'}}(k',t)\;\mathrm{d}^3\mathbf{r}\,\mathrm{d}\omega\,\mathrm{d}\omega'\\
% &= \frac{1}{8\pi}\int\left[\nabla\Phi_f(\mathbf{r},t)\right]^2\mathrm{d}^3\mathbf{r} - \sum_{\bm{\beta}}\sum_{nn'}g_{\bm{\beta}nn'}^{(1)}\mathcal{B}_{\bm{\beta}n}(t)\mathcal{B}_{\bm{\beta}n'}(t) + 2\sum_{\bm{\beta}n}F_{\bm{\beta}n}^{(1)}[\{\mathbf{x}_{fi}(t)\}]\mathcal{B}_{\bm{\beta}n}(t)\\
% &\qquad + \frac{1}{8\pi c^2}\sum_{\bm{\alpha}\bm{\alpha}'}\int_0^\infty\int_0^\infty\frac{e^2\omega^\frac{3}{2}\omega^{\prime\frac{3}{2}}}{c^4}\frac{2\pi^2}{k^2}\delta(k - k')(1 - \delta_{p1}\delta_{m0})\delta_{\bm{\alpha}\bm{\alpha}'}\dot{\mathcal{A}}_{\bm{\alpha}}(\omega,t)\dot{\mathcal{A}}_{\bm{\alpha}'}(\omega',t)\;\mathrm{d}\omega\,\mathrm{d}\omega'\\
% &\qquad - \frac{1}{8\pi}\sum_{\bm{\alpha}\bm{\alpha}'}\int_0^\infty\int_0^\infty \frac{e^2\omega^\frac{3}{2}\omega^{\prime\frac{3}{2}}}{c^4}kk'\frac{2\pi^2}{k^2}\delta(k - k')(1 - \delta_{p1}\delta_{m0})\delta_{\bm{\alpha}\bm{\alpha}'}\mathcal{A}_{\bm{\alpha}}(\omega,t)\mathcal{A}_{\bm{\alpha}'}(\omega',t)\;\mathrm{d}\omega\,\mathrm{d}\omega'\\
% &= \frac{1}{8\pi}\int\left[\nabla\Phi_f(\mathbf{r},t)\right]^2\mathrm{d}^3\mathbf{r} - \sum_{\bm{\beta}}\sum_{nn'}g_{\bm{\beta}nn'}^{(1)}\mathcal{B}_{\bm{\beta}n}(t)\mathcal{B}_{\bm{\beta}n'}(t) + 2\sum_{\bm{\beta}n}F_{\bm{\beta}n}^{(1)}[\{\mathbf{x}_{fi}(t)\}]\mathcal{B}_{\bm{\beta}n}(t)\\
% &\qquad + \frac{1}{8\pi}\sum_{\bm{\alpha}}\int_0^\infty\frac{e^2\omega^3}{c^4}c\left[\frac{2\pi^2}{k^2c^2}\dot{\mathcal{A}}_{\bm{\alpha}}^2(\omega,t) - 2\pi^2\mathcal{A}_{\bm{\alpha}}^2(\omega,t)\right]\;\mathrm{d}\omega\\
% &= \frac{1}{8\pi}\int\left[\nabla\Phi_f(\mathbf{r},t)\right]^2\mathrm{d}^3\mathbf{r} - \sum_{\bm{\beta}}\sum_{nn'}g_{\bm{\beta}nn'}^{(1)}\mathcal{B}_{\bm{\beta}n}(t)\mathcal{B}_{\bm{\beta}n'}(t) + 2\sum_{\bm{\beta}n}F_{\bm{\beta}n}^{(1)}[\{\mathbf{x}_{fi}(t)\}]\mathcal{B}_{\bm{\beta}n}(t)\\
% &\qquad + \sum_{\bm{\alpha}}\int_0^\infty\frac{\pi}{4}\frac{e^2\omega^3}{c^3}\left[\frac{1}{k^2c^2}\dot{\mathcal{A}}_{\bm{\alpha}}^2(\omega,t) - \mathcal{A}_{\bm{\alpha}}^2(\omega,t)\right]\;\mathrm{d}\omega\\
&= \frac{1}{8\pi}\int\left[\nabla\Phi_f(\mathbf{r},t)\right]^2\mathrm{d}^3\mathbf{r} - \sum_{\bm{\beta}}\sum_{nn'}g_{\bm{\beta}nn'}^{(1)}\mathcal{B}_{\bm{\beta}n}(t)\mathcal{B}_{\bm{\beta}n'}(t) + 2\sum_{\bm{\beta}n}F_{\bm{\beta}n}^{(1)}[\{\mathbf{x}_{fi}(t)\}]\mathcal{B}_{\bm{\beta}n}(t)\\
&\qquad + \sum_{\bm{\alpha}}\int_0^\infty\frac{\pi}{4}\frac{e^2\omega}{c^3}\left[\dot{\mathcal{A}}_{\bm{\alpha}}^2(\omega,t) - \omega^2\mathcal{A}_{\bm{\alpha}}^2(\omega,t)\right]\mathrm{d}\omega
\end{split}
\end{equation}

The interaction Lagrangian can be handled similarly. We can use integration by parts and separate the scalar potential into its constituent parts to say
\begin{equation}
\begin{split}
L_\mathrm{int} &=
% -\int\rho_f(\mathbf{r},t)\Phi(\mathbf{r},t)\;\mathrm{d}^3\mathbf{r} - e\eta\int\Theta(a - r)\mathbf{Q}(\mathbf{r},t)\cdot\nabla\Phi(\mathbf{r},t)\;\mathrm{d}^3\mathbf{r}\\
% &\qquad + \frac{e\eta}{c}\int\Theta(a - r)\dot{\mathbf{Q}}(\mathbf{r},t)\cdot\mathbf{A}(\mathbf{r},t)\;\mathrm{d}^3\mathbf{r} + \frac{1}{c}\int\mathbf{J}_f(\mathbf{r},t)\cdot\mathbf{A}(\mathbf{r},t)\;\mathrm{d}^3\mathbf{r}\\
% &\qquad- \int\int_0^\infty v(\nu)\mathbf{Q}(\mathbf{r},t)\cdot\dot{\mathbf{Q}}_\nu(\mathbf{r},t)\Theta(a - r)\;\mathrm{d}\nu\,\mathrm{d}^3\mathbf{r}\\
% &= \frac{1}{4\pi}\int\nabla^2\Phi_f(\mathbf{r},t)\Phi_f(\mathbf{r},t)\;\mathrm{d}^3\mathbf{r} + \frac{1}{4\pi}\int\nabla^2\Phi_f(\mathbf{r},t)\Phi_m(\mathbf{r},t)\;\mathrm{d}^3\mathbf{r} - e\eta\int_{r<a}\mathbf{Q}(\mathbf{r},t)\cdot\nabla\Phi_f(\mathbf{r},t)\;\mathrm{d}^3\mathbf{r}\\
% &\qquad - e\eta\int_{r<a}\mathbf{Q}(\mathbf{r},t)\cdot\nabla\Phi_m(\mathbf{r},t)\;\mathrm{d}^3\mathbf{r} + \frac{e\eta}{c}\int_{r<a}\dot{\mathbf{Q}}(\mathbf{r},t)\cdot\mathbf{A}(\mathbf{r},t)\;\mathrm{d}^3\mathbf{r} + \frac{1}{c}\int\mathbf{J}_f(\mathbf{r},t)\cdot\mathbf{A}(\mathbf{r},t)\;\mathrm{d}^3\mathbf{r}\\
% &\qquad - \int_0^\infty\int_{r < a} v(\nu)\mathbf{Q}(\mathbf{r},t)\cdot\dot{\mathbf{Q}}_\nu(\mathbf{r},t)\;\mathrm{d}^3\mathbf{r}\,\mathrm{d}\nu\\
% &= 
-\frac{1}{4\pi}\int\nabla\Phi_f(\mathbf{r},t)\cdot\nabla\Phi_f(\mathbf{r},t)\;\mathrm{d}^3\mathbf{r} - \frac{1}{4\pi}\int\nabla\Phi_f(\mathbf{r},t)\cdot\nabla\Phi_m(\mathbf{r},t)\;\mathrm{d}^3\mathbf{r} - e\eta\int_{r<a}\mathbf{Q}(\mathbf{r},t)\cdot\nabla\Phi_f(\mathbf{r},t)\;\mathrm{d}^3\mathbf{r}\\
&\qquad + \frac{1}{c}\int\mathbf{J}_f(\mathbf{r},t)\cdot\mathbf{A}(\mathbf{r},t)\;\mathrm{d}^3\mathbf{r} - e\eta\int_{r<a}\mathbf{Q}(\mathbf{r},t)\cdot\nabla\Phi_m(\mathbf{r},t)\;\mathrm{d}^3\mathbf{r} + \frac{e\eta}{c}\int_{r<a}\dot{\mathbf{Q}}(\mathbf{r},t)\cdot\mathbf{A}(\mathbf{r},t)\;\mathrm{d}^3\mathbf{r}\\
&\qquad - \eta\int_0^\infty\int_{r < a} v(\nu)\mathbf{Q}(\mathbf{r},t)\cdot\dot{\mathbf{Q}}_\nu(\mathbf{r},t)\;\mathrm{d}^3\mathbf{r}\,\mathrm{d}\nu.\\
\end{split}
\end{equation}
We can then simplify the terms detailing the interaction energies between the free charges and the matter and electromagnetic fields as
\begin{equation}
\begin{split}
-\int\nabla\Phi_f(\mathbf{r},t)\cdot\nabla\Phi_m(\mathbf{r},t)\;\mathrm{d}^3\mathbf{r} &= -8\pi\sum_{\bm{\beta}n}F_{\bm{\beta}n}^{(1)}[\{\mathbf{x}_{fi}(t)\}]\mathcal{B}_{\bm{\beta}n}(t),\\
-en\int_{r<a}\mathbf{Q}(\mathbf{r},t)\cdot\nabla\Phi_f(\mathbf{r},t)\;\mathrm{d}^3\mathbf{r} &= -\sum_{\bm{\beta}n}F_{\bm{\beta}n}^{(2)}[\{\mathbf{x}_{fi}(t)\}]\mathcal{B}_{\bm{\beta}n}(t),\\
\frac{1}{c}\int\mathbf{J}_f(\mathbf{r},t)\cdot\mathbf{A}(\mathbf{r},t)\;\mathrm{d}^3\mathbf{r} &= -\sum_{\bm{\alpha}}\int_0^\infty G_{\bm{\alpha}}^{(2)}[\{x_{fi}(t)\},\omega]\mathcal{A}_{\bm{\alpha}}(\omega,t)\;\mathrm{d}\omega,
\end{split}
\end{equation}
where the superscripts $(2)$ denote matter-field type coupling and $G$ is used in place of $F$ to label interaction energies between the free charges and vector potential. The remaining terms describe the interactions between the mobile charge, transverse electromagnetic field, and reservoir modes. Analyzing these terms in order of appearance, we find that
\begin{equation}
\begin{split}
-e\eta&\int_{r<a}\mathbf{Q}(\mathbf{r},t)\cdot\nabla\Phi_m(\mathbf{r},t)\;\mathrm{d}^3\mathbf{r}
% = -e\eta\int_{r<a}\sum_{\bm{\beta}n}\mathcal{B}_{\bm{\beta}n}(t)\mathbf{X}_{\bm{\beta}}(\mathbf{r},k_{\bm{\beta}n})\cdot(-1)e\eta\sum_{p'\ell'm'n'}\frac{4\pi}{\sqrt{2\ell' + 1}}\mathcal{B}_{Lp'\ell'm'n'}(t)\\
% &\qquad\times\left[\sqrt{2\ell' + 1}\mathbf{L}_{p'\ell'm'}\left(\mathbf{r},\frac{w_{\ell' n'}}{a}\right) - \frac{j_{\ell' - 1}(w_{\ell' n'})}{a^{\ell' - 1}}\nabla f_{p'\ell' m'}(\mathbf{r})\right]\mathrm{d}^3\mathbf{r}\\
% &= e^2n^2\sum_{\bm{\beta}n}\sum_{p'\ell'm'n'}4\pi\mathcal{B}_{\bm{\beta}n}(t)\mathcal{B}_{Lp'\ell'm'n'}(t)\left[\;\int_{r<a}\mathbf{X}_{\bm{\beta}}(\mathbf{r},k_{\bm{\beta}n})\cdot\mathbf{L}_{p'\ell'm'}\left(\mathbf{r},\frac{w_{\ell'n'}}{a}\right)\mathrm{d}^3\mathbf{r}\right.\\
% &\left.\qquad - \frac{j_{\ell ' - 1}(w_{\ell'n'})}{a^{\ell' - 1}\sqrt{2\ell' + 1}}\int_{r < a}\mathbf{X}_{\bm{\beta}}(\mathbf{r},k_{\bm{\beta}n})\cdot\nabla f_{p'\ell'm'}^<(\mathbf{r})\;\mathrm{d}^3\mathbf{r}\right]\\
% &= e^2n^2\sum_{\bm{\beta}n}\sum_{p'\ell'm'n'}4\pi\mathcal{B}_{\bm{\beta}n}(t)\mathcal{B}_{Lp'\ell'm'n'}(t)\left[4\pi(1 - \delta_{p1}\delta_{m0})\delta_{TL}\delta_{pp'}\delta_{\ell\ell'}\delta_{mm'}R_{T\ell}^\ll\left(\frac{w_{\ell n}}{a},\frac{w_{\ell n'}}{a};0,a\right)\right.\\
% &\qquad\qquad + 4\pi(1 - \delta_{p1}\delta_{m0})\sqrt{\ell(\ell + 1)}\delta_{TE}\delta_{pp'}\delta_{\ell\ell'}\delta_{mm'}\frac{a^2}{w_{\ell n}z_{\ell n'}}aj_\ell(w_{\ell n})j_\ell(z_{\ell n'})\\
% &\qquad\qquad\left. - \frac{j_{\ell ' - 1}(w_{\ell'n'})}{a^{\ell' - 1}\sqrt{2\ell' + 1}}4\pi\delta_{TL}(1 - \delta_{p1}\delta_{m0})\delta_{pp'}\delta_{\ell\ell'}\delta_{mm'}\frac{a^{\ell + 2}}{\sqrt{2\ell + 1}}j_{\ell + 1}(w_{\ell n})\right]\\
% &= 16\pi^2 e^2n^2\sum_{p\ell m}\sum_{nn'}\mathcal{B}_{Lp\ell mn}(t)\mathcal{B}_{Lp\ell mn'}(t)\left[R_{T\ell}^\ll\left(\frac{w_{\ell n}}{a},\frac{w_{\ell n'}}{a};0,a\right) - \frac{a^3}{2\ell + 1}j_{\ell + 1}(w_{\ell n})j_{\ell - 1}(w_{\ell n'})\right]\\
% &
= -\sum_{\bm{\beta}}\sum_{nn'}g_{\bm{\beta}nn'}^{(2)}\mathcal{B}_{\bm{\beta}n}(t)\mathcal{B}_{\bm{\beta}n'}(t),
\end{split}
\end{equation}
where
\begin{equation}
g_{\bm{\beta}nn'}^{(2)} = -\delta_{TL}16\pi^2e^2\eta^2\left[R_{T\ell}^\ll\left(\frac{w_{\ell n}}{a},\frac{w_{\ell n'}}{a};0,a\right) - \frac{a^3}{2\ell + 1}j_{\ell + 1}(w_{\ell n})j_{\ell - 1}(w_{\ell n'})\right].
\end{equation}
Next,
\begin{equation}
\begin{split}
\frac{e\eta}{c}\int_{r<a}&\dot{\mathbf{Q}}(\mathbf{r},t)\cdot\mathbf{A}(\mathbf{r},t)\;\mathrm{d}^3\mathbf{r}
% = \frac{e\eta}{c}\int_{r<a}\sum_{\bm{\beta}n}\dot{\mathcal{B}}_{\bm{\beta}n}(t)\mathbf{X}_{\bm{\beta}}(\mathbf{r},k_{\bm{\beta}n})\cdot\sum_{\bm{\alpha}}\int_0^\infty\frac{ek\sqrt{\omega}}{c}\mathcal{A}_{\bm{\alpha}}(\omega,t)\mathbf{X}_{\bm{\alpha}}(\mathbf{r},k)\;\mathrm{d}\omega\,\mathrm{d}^3\mathbf{r}\\
% &= \frac{e^2\eta}{c^3}\sum_{\bm{\alpha}\bm{\beta}}\sum_n\int_0^\infty\omega^\frac{3}{2}\dot{\mathcal{B}}_{\bm{\beta}n}(t)\mathcal{A}_{\bm{\alpha}}(\omega,t)\left[ \vphantom{\frac{a^2}{w_{\ell n}k}} 4\pi(1 - \delta_{p1}\delta_{m0})\delta_{TT'}\delta_{pp'}\delta_{\ell\ell'}\delta_{mm'}R_{T\ell}^\ll\left(k_{\bm{\beta}n},k;0,a\right)\right.\\
% &\qquad\left. + 4\pi(1 - \delta_{p1}\delta_{m0})\sqrt{\ell(\ell + 1)}\delta_{TL}\delta_{T'E}\delta_{pp'}\delta_{\ell\ell'}\delta_{mm'}\frac{a^2}{w_{\ell n}k}j_\ell(w_{\ell n})j_\ell(ka)\right]\mathrm{d}\omega\\
% &= 4\pi\frac{e^2\eta}{c^3}\sum_{T = M,E,L}\;\sum_{T' = M,E}\sum_{p\ell m}\sum_{n}\int_0^\infty\dot{\mathcal{B}}_{Tp\ell mn}(t)\mathcal{A}_{T'p\ell m}(\omega,t)\omega^\frac{3}{2}\left[ \vphantom{\frac{a^2}{w_{\ell n}k}} \delta_{TT'}R_{T\ell}^\ll\left(\frac{z_{\ell n}}{a},\frac{\omega}{c};0,a\right)\right.\\
% &\qquad\left. + \delta_{TL}\delta_{T'E}\frac{a^2c}{w_{\ell n}\omega}j_\ell(w_{\ell n})j_\ell\left(\frac{\omega a}{c}\right)\right]\mathrm{d}\omega\\
% &
= \sum_{\bm{\alpha}\bm{\beta}}\sum_n\int_0^\infty\lambda_{\bm{\alpha}\bm{\beta}n}(\omega)\mathcal{A}_{\bm{\alpha}}(\omega,t)\dot{\mathcal{B}}_{\bm{\beta}n}(t)\;\mathrm{d}\omega
\end{split}
\end{equation}
with
\begin{equation}
\lambda_{\bm{\alpha}\bm{\beta}n}(\omega) = \frac{4\pi e^2\omega^\frac{3}{2}\eta}{c^3}\left[\delta_{TT'}R_{T\ell}^\ll\left(\frac{z_{\ell n}}{a},\frac{\omega}{c};0,a\right) + \delta_{TL}\delta_{T'E}\frac{a^2c}{w_{\ell n}\omega}j_\ell(w_{\ell n})j_{\ell}\left(\frac{\omega a}{c}\right)\right]\delta_{pp'}\delta_{\ell\ell'}\delta_{mm'}
\end{equation}
and, finally,
\begin{equation}
\begin{split}
-\eta\int_0^\infty&\int_{r<a} v(\nu)\mathbf{Q}(\mathbf{r},t)\cdot\dot{\mathbf{Q}}_\nu(\mathbf{r},t)\;\mathrm{d}^3\mathbf{r}\,\mathrm{d}\nu
\\
&= -\eta\int_0^\infty\int_{r<a} v(\nu)\sum_{\bm{\beta}n}\mathcal{B}_{\bm{\beta}n}(t)\mathbf{X}_{\bm{\beta}}(\mathbf{r},k_{\bm{\beta}n})\cdot\sum_{\bm{\gamma}n'}\dot{\mathcal{C}}_{\bm{\gamma}n'}(\nu,t)\mathbf{X}_{\bm{\gamma}}(\mathbf{r},k_{\bm{\gamma}n})\;\mathrm{d}^3\mathbf{r}\,\mathrm{d}\nu\\
% &= -\eta\sum_{\bm{\beta}n}\sum_{\bm{\gamma}n'}\int_0^\infty v(\nu)\mathcal{B}_{\bm{\beta}n}(t)\dot{\mathcal{C}}_{\bm{\gamma}n'}(\nu,t)\int_{r<a}\mathbf{X}_{\bm{\beta}}(\mathbf{r},k_{\bm{\beta}n})\cdot\mathbf{X}_{\bm{\gamma}}(\mathbf{r},k_{\bm{\gamma}n'})\;\mathrm{d}^3\mathbf{r}\,\mathrm{d}\nu\\
% &= \eta\sum_{\bm{\beta}\bm{\gamma}}\sum_{nn'}\int_0^\infty v(\nu)\mathcal{B}_{\bm{\beta}n}(t)\dot{\mathcal{C}}_{\bm{\gamma}n'}(\nu,t)4\pi(1 - \delta_{p1}\delta_{m0})\delta_{TT'}\delta_{pp'}\delta_{\ell\ell'}\delta_{mm'}R_{T\ell}^\ll\left(k_{\bm{\beta}n},k_{\bm{\beta}n'};0,a\right)\mathrm{d}\nu\\
% &= \eta\sum_{\bm{\beta}\bm{\gamma}}\sum_{nn'}\int_0^\infty v(\nu)\mathcal{B}_{\bm{\beta}n}(t)\dot{\mathcal{C}}_{\bm{\gamma}n'}(\nu,t)4\pi(1 - \delta_{p1}\delta_{m0})\delta_{TT'}\delta_{pp'}\delta_{\ell\ell'}\delta_{mm'}\\
% &\qquad\times\delta_{nn'}\left[(\delta_{TM} + \delta_{TE})\frac{a^3}{2}j_{\ell + 1}^2(z_{\ell n}) + \delta_{TL}\frac{a^3}{2}\left(1 - \frac{\ell(\ell + 1)}{w_{\ell n}^2}\right)j_\ell^2(w_{\ell n})\right]\mathrm{d}\nu\\
% &= \eta\sum_{\bm{\beta}}\sum_n\int_0^\infty2\pi a^3 v(\nu)\left[(\delta_{TM} + \delta_{TE})j_{\ell + 1}^2(z_{\ell n}) + \delta_{TL}\left(1 - \frac{\ell(\ell + 1)}{w_{\ell n}^2}\right)j_\ell^2(w_{\ell n})\right]\mathcal{B}_{\bm{\beta}n}(t)\dot{\mathcal{C}}_{\bm{\beta}n}(\nu,t)\;\mathrm{d}\nu\\
&
= -\sum_{\bm{\beta}}\sum_{n}\int_0^\infty v_{\bm{\beta}n}(\nu)\mathcal{B}_{\bm{\beta}n}(t)\dot{\mathcal{C}}_{\bm{\beta}n}(\nu,t),
\end{split}
\end{equation}
with
\begin{equation}
v_{\bm{\beta}n}(\nu) = -2\pi a^3\eta v(\nu)\left[(\delta_{TM} + \delta_{TE})j_{\ell + 1}^2(z_{\ell n}) + \delta_{TL}\left(1 - \frac{\ell(\ell + 1)}{w_{\ell n}^2}\right)j_\ell^2(w_{\ell n})\right].
\end{equation}
Altogether, then, we can say
\begin{equation}
\begin{split}
L_\mathrm{int} &= -\frac{1}{4\pi}\int\nabla\Phi_f(\mathbf{r},t)\cdot\nabla\Phi_f(\mathbf{r},t)\;\mathrm{d}^3\mathbf{r} - 2\sum_{\bm{\beta}n}F_{\bm{\beta}n}^{(1)}[\{\mathbf{x}_{fi}(t)\}]\mathcal{B}_{\bm{\beta}n}(t) - \sum_{\bm{\beta}}\sum_{n}F_{\bm{\beta}n}^{(2)}[\{\mathbf{x}_{fi}(t)\}]\mathcal{B}_{\bm{\beta}n}(t)\\
&\qquad - \sum_{\bm{\alpha}}\int_0^\infty G_{\bm{\alpha}}^{(2)}[\{\mathbf{x}_{fi}(t)\},\omega]\mathcal{A}_{\bm{\alpha}}(\omega,t)\;\mathrm{d}\omega - \sum_{\bm{\beta}nn'}g_{\bm{\beta}nn'}^{(2)}\mathcal{B}_{\bm{\beta}n}(t)\mathcal{B}_{\bm{\beta}n'}(t)\\
&\qquad + \sum_{\bm{\alpha}\bm{\beta}n}\int_0^\infty\lambda_{\bm{\alpha}\bm{\beta}n}(\omega)\mathcal{A}_{\bm{\alpha}}(\omega,t)\dot{\mathcal{B}}_{\bm{\beta}n}(t)\;\mathrm{d}\omega - \sum_{\bm{\beta}}\sum_{n}\int_0^\infty v_{\bm{\beta}n}(\nu)\mathcal{B}_{\bm{\beta}n}\dot{\mathcal{C}}_{\bm{\beta}n}(\nu,t)\;\mathrm{d}\nu
\end{split}
\end{equation}
such that
\begin{equation}
\begin{split}
L &= 
% L_f + L_\mathrm{bind} + \sum_{\bm{\beta}n}\frac{m_{\bm{\beta}n}}{2}\left(\dot{\mathcal{B}}_{\bm{\beta}n}^2(t) - \omega_0^2\mathcal{B}_{\bm{\beta}n}^2(t)\right) + \sum_{\bm{\gamma}n}\int_0^\infty\frac{m_{\bm{\gamma}n}}{2}\left(\dot{\mathcal{C}}_{\bm{\gamma}n}^2(\nu,t) - \nu^2\mathcal{C}_{\bm{\gamma}n}^2(t)\right)\mathrm{d}\nu\\
% &\qquad + \sum_{\bm{\alpha}}\int_0^\infty\frac{\pi e^2}{4c^3}\left[\dot{\mathcal{A}}_{\bm{\alpha}}^2(\omega,t) - \omega^2\mathcal{A}_{\bm{\alpha}}^2(\omega,t)\right]\mathrm{d}\omega + \frac{1}{8\pi}\int\left[\nabla\Phi_f(\mathbf{r},t)\right]^2\;\mathrm{d}^3\mathbf{r} - \sum_{\bm{\beta}}\sum_{nn'}g_{\bm{\beta}nn'}^{(1)}\mathcal{B}_{\bm{\beta}n}(t)\mathcal{B}_{\bm{\beta}n'}(t)\\
% &\qquad + 2\sum_{\bm{\beta}n}F_{\bm{\beta}n}^{(1)}[\{\mathbf{x}_{fi}(t)\}]\mathcal{B}_{\bm{\beta}n}(t) - \frac{1}{4\pi}\int\left[\nabla\Phi_f(\mathbf{r},t)\right]^2\;\mathrm{d}^3\mathbf{r} - 2\sum_{\bm{\beta}n}F_{\bm{\beta}n}^{(1)}[\{\mathbf{x}_{fi}(t)\}]\mathcal{B}_{\bm{\beta}n}(t)\\
% &\qquad + \sum_{\bm{\alpha}}\int_0^\infty G_{\bm{\alpha}}^{(2)}[\{\mathbf{x}_{fi}(t)\},\omega]\mathcal{A}_{\bm{\alpha}}(\omega,t)\;\mathrm{d}\omega - \sum_{\bm{\beta}nn'}g_{\bm{\beta}nn'}^{(2)}\mathcal{B}_{\bm{\beta}n}(t)\mathcal{B}_{\bm{\beta}n'}(t)\\
% &\qquad + \sum_{\bm{\alpha}\bm{\beta}n}\int_0^\infty\lambda_{\bm{\alpha}\bm{\beta}n}(\omega)\mathcal{A}_{\bm{\alpha}}(\omega,t)\mathcal{B}_{\bm{\beta}n}(t)\;\mathrm{d}\omega + \sum_{\bm{\beta}}\sum_{n}\int_0^\infty v_{\bm{\beta}n}(\nu)\mathcal{B}_{\bm{\beta}n}\dot{\mathcal{C}}_{\bm{\beta}n}(\nu,t)\;\mathrm{d}\nu\\
% &= 
L_f + L_\mathrm{bind} + \sum_{\bm{\beta}n}\frac{m_{\bm{\beta}n}}{2}\left(\dot{\mathcal{B}}_{\bm{\beta}n}^2(t) - \omega_0^2\mathcal{B}_{\bm{\beta}n}^2(t)\right) - \sum_{\bm{\beta}}\sum_{nn'}g_{\bm{\beta}nn'}\mathcal{B}_{\bm{\beta}n}(t)\mathcal{B}_{\bm{\beta}n'}(t)\\
&\qquad+ \sum_{\bm{\gamma}n}\int_0^\infty\frac{m_{\bm{\gamma}n}}{2}\left(\dot{\mathcal{C}}_{\bm{\gamma}n}^2(\nu,t) - \nu^2\mathcal{C}_{\bm{\gamma}n}^2(t)\right)\mathrm{d}\nu - \sum_{\bm{\beta}n}\int_0^\infty v_{\bm{\beta}n}(\nu)\mathcal{B}_{\bm{\beta}n}(t)\dot{\mathcal{C}}_{\bm{\beta}n}(\nu,t)\;\mathrm{d}\nu\\
&\qquad + \sum_{\bm{\alpha}}\int_0^\infty\frac{\pi e^2\omega}{4c^3}\left[\dot{\mathcal{A}}_{\bm{\alpha}}^2(\omega,t) - \omega^2\mathcal{A}_{\bm{\alpha}}^2(\omega,t)\right]\mathrm{d}\omega + \sum_{\bm{\alpha}\bm{\beta}n}\int_0^\infty\lambda_{\bm{\alpha}\bm{\beta}n}(\omega)\mathcal{A}_{\bm{\alpha}}(\omega,t)\dot{\mathcal{B}}_{\bm{\beta}n}(t)\;\mathrm{d}\omega\\
&\qquad - \frac{1}{4\pi}\int\left[\nabla\Phi_f(\mathbf{r},t)\right]^2\;\mathrm{d}^3\mathbf{r} - \sum_{\bm{\alpha}}\int_0^\infty G_{\bm{\alpha}}[\{\mathbf{x}_{fi}(t)\},\omega]\mathcal{A}_{\bm{\alpha}}(\omega,t)\;\mathrm{d}\omega
\end{split}
\end{equation}
where
\begin{equation}
\begin{split}
g_{\bm{\beta}nn'} &= g_{\bm{\beta}nn'}^{(1)} + g_{\bm{\beta}nn'}^{(2)},\\
G_{\bm{\alpha}}[\{\mathbf{x}_{fi}\},\omega] &= G_{\bm{\alpha}}^{(2)}[\{\mathbf{x}_{fi}\},\omega].
\end{split}
\end{equation}















\subsection{Quantization with Classical Free Charges}

In order to quantize our model, we need to define the system's Hamiltonian. It will be advantageous to first define a new Lagrangian
\begin{equation}
\begin{split}
L' &= L - \frac{\mathrm{d}}{\mathrm{d}t}\left\{\sum_{\bm{\alpha}\bm{\beta}n}\int_0^\infty\lambda_{\bm{\alpha}\bm{\beta}n}(\omega)\mathcal{A}_{\bm{\alpha}}(\omega,t)\mathcal{B}_{\bm{\beta}n}(t)\;\mathrm{d}\omega\right\}\\
&= L_f + L_\mathrm{bind} + \sum_{\bm{\beta}n}\frac{m_{\bm{\beta}n}}{2}\left(\dot{\mathcal{B}}_{\bm{\beta}n}^2(t) - \omega_0^2\mathcal{B}_{\bm{\beta}n}^2(t)\right) - \sum_{\bm{\beta}}\sum_{nn'}g_{\bm{\beta}nn'}\mathcal{B}_{\bm{\beta}n}(t)\mathcal{B}_{\bm{\beta}n'}(t)\\
&\qquad+ \sum_{\bm{\gamma}n}\int_0^\infty\frac{m_{\bm{\gamma}n}}{2}\left(\dot{\mathcal{C}}_{\bm{\gamma}n}^2(\nu,t) - \nu^2\mathcal{C}_{\bm{\gamma}n}^2(t)\right)\mathrm{d}\nu - \sum_{\bm{\beta}n}\int_0^\infty v_{\bm{\beta}n}(\nu)\mathcal{B}_{\bm{\beta}n}(t)\dot{\mathcal{C}}_{\bm{\beta}n}(\nu,t)\;\mathrm{d}\nu\\
&\qquad + \sum_{\bm{\alpha}}\int_0^\infty\frac{\pi e^2\omega}{4c^3}\left[\dot{\mathcal{A}}_{\bm{\alpha}}^2(\omega,t) - \omega^2\mathcal{A}_{\bm{\alpha}}^2(\omega,t)\right]\mathrm{d}\omega - \sum_{\bm{\alpha}\bm{\beta}n}\int_0^\infty\lambda_{\bm{\alpha}\bm{\beta}n}(\omega)\dot{\mathcal{A}}_{\bm{\alpha}}(\omega,t)\mathcal{B}_{\bm{\beta}n}(t)\;\mathrm{d}\omega\\
&\qquad - \frac{1}{4\pi}\int\left[\nabla\Phi_f(\mathbf{r},t)\right]^2\;\mathrm{d}^3\mathbf{r} - \sum_{\bm{\alpha}}\int_0^\infty G_{\bm{\alpha}}[\{\mathbf{x}_{fi}(t)\},\omega]\mathcal{A}_{\bm{\alpha}}(\omega,t)\;\mathrm{d}\omega
\end{split}
\end{equation}
that describes the same motion as $L$. We can then define the canonical momenta of our coordinates as
\begin{equation}
\begin{split}
\mathcal{P}_{\bm{\beta}n}(t) &= \frac{\partial L'}{\partial \dot{\mathcal{B}}_{\bm{\beta}n}(t)} = m_{\bm{\beta}n}\dot{\mathcal{B}}_{\bm{\beta}n}(t),\\
\mathcal{Q}_{\bm{\gamma}n}(\nu,t) &= \frac{\partial L'}{\partial \dot{\mathcal{C}}_{\bm{\gamma}n}(\nu,t)} = m_{\bm{\gamma}n}\dot{\mathcal{C}}_{\bm{\gamma}n}(\nu,t) - v_{\bm{\gamma}n}(\nu)\mathcal{B}_{\bm{\gamma}n}(t),\\
\mathit{\Pi}_{\bm{\alpha}}(\omega,t) &= \frac{\partial L'}{\partial \dot{\mathcal{A}}_{\bm{\alpha}}(\omega,t)} = \frac{\pi e^2\omega}{2c^3}\dot{\mathcal{A}}_{\bm{\alpha}}(\omega,t) - \sum_{\bm{\beta}n}\lambda_{\bm{\alpha}\bm{\beta}n}(\omega)\mathcal{B}_{\bm{\beta}n}(t).
\end{split}
\end{equation}
Calculation of the Hamiltonian is then straightforward:
\begin{equation}
\begin{split}
H &= \sum_{\bm{\alpha}}\int_0^\infty\dot{\mathcal{A}}_{\bm{\alpha}}(\omega,t)\mathit{\Pi}_{\bm{\alpha}}(\omega,t)\;\mathrm{d}\omega + \sum_{\bm{\beta}n}\dot{\mathcal{B}}_{\bm{\beta}n}(t)\mathcal{P}_{\bm{\beta}n}(t) + \sum_{\bm{\gamma}n}\int_0^\infty\dot{\mathcal{C}}_{\bm{\gamma}n}(\nu,t)\mathcal{Q}_{\bm{\gamma}n}(\nu,t)\;\mathrm{d}\nu - L'\\
% &= \sum_{\bm{\alpha}}\int_0^\infty\frac{2c^3}{\pi e^2\omega}\left(\mathit{\Pi}_{\bm{\alpha}}(\omega,t) + \sum_{\bm{\beta}n}\lambda_{\bm{\alpha}\bm{\beta}n}(\omega)\mathcal{B}_{\bm{\beta}n}(t)\right)\mathit{\Pi}_{\bm{\alpha}}(\omega,t)\;\mathrm{d}\omega
% + \sum_{\bm{\beta}n}\frac{\mathcal{P}_{\bm{\beta}n}^2(t)}{m_{\bm{\beta}n}}\\
% &\qquad + \sum_{\bm{\gamma}n}\int_0^\infty\frac{1}{m_{\bm{\gamma}n}}\left(\mathcal{Q}_{\bm{\gamma}n}(\nu,t) + v_{\bm{\gamma}n}(\nu)\mathcal{B}_{\bm{\gamma}n}(t)\right)\mathcal{Q}_{\bm{\gamma}n}(\nu,t)\;\mathrm{d}\nu - L_f - L_\mathrm{bind}\\
% &\qquad - \sum_{\bm{\alpha}}\int_0^\infty\frac{\pi e^2\omega}{4c^3}\left(\left[\frac{2c^3}{\pi e^2\omega}\left(\mathit{\Pi}_{\bm{\alpha}}(\omega,t) + \sum_{\bm{\beta}n}\lambda_{\bm{\alpha}\bm{\beta}n}(\omega)\mathcal{B}_{\bm{\beta}n}(t)\right)\right]^2 - \omega^2\mathcal{A}_{\bm{\alpha}}^2(\omega,t)\right)\mathrm{d}\omega\\
% &\qquad - \sum_{\bm{\gamma}n}\int_0^\infty\frac{m_{\bm{\gamma}n}}{2}\left(\left[\frac{1}{m_{\bm{\gamma}n}}\left(\mathcal{Q}_{\bm{\gamma}n}(\nu,t) + v_{\bm{\gamma}n}(\nu)\mathcal{B}_{\bm{\gamma}n}(t)\right)\right]^2 - \nu^2\mathcal{C}_{\bm{\gamma}n}^2(\nu,t)\right)\mathrm{d}\nu\\
% &\qquad - \sum_{\bm{\beta}n}\frac{m_{\bm{\beta}n}}{2}\left(\left[\frac{\mathcal{P}_{\bm{\beta}n}(t)}{m_{\bm{\beta}n}}\right]^2 - \omega_0^2\mathcal{B}_{\bm{\beta}n}^2(t)\right) + \sum_{\bm{\beta}nn'}g_{\bm{\beta}nn'}\mathcal{B}_{\bm{\beta}n}(t)\mathcal{B}_{\bm{\beta}n'}(t)\\
% &\qquad + \sum_{\bm{\beta}n}\int_0^\infty v_{\bm{\beta}n}(\nu)\mathcal{B}_{\bm{\beta}n}(t)\left[\frac{1}{m_{\bm{\beta}n}}\left(\mathcal{Q}_{\bm{\beta}n}(\nu,t) + v_{\bm{\beta}n}(\nu)\mathcal{B}_{\bm{\beta}n}(t)\right)\right]\mathrm{d}\nu\\
% &\qquad + \sum_{\bm{\alpha}\bm{\beta}n}\int_0^\infty\lambda_{\bm{\alpha}\bm{\beta}n}(\omega)\frac{2c^3}{\pi e^2\omega}\left(\mathit{\Pi}_{\bm{\alpha}}(\omega,t) + \sum_{\bm{\beta}'n'}\lambda_{\bm{\alpha}\bm{\beta}'n'}(\omega)\mathcal{B}_{\bm{\beta}'n'}(t)\right)\mathcal{B}_{\bm{\beta}n}(t)\mathrm{d}\omega\\
% &\qquad + \frac{1}{4\pi}\int\left[\nabla\Phi_f(\mathbf{r},t)\right]^2\;\mathrm{d}^3\mathbf{r} + \sum_{\bm{\alpha}}\int_0^\infty G_{\bm{\alpha}}[\{\mathbf{x}_{fi}(t)\},\omega]\mathcal{A}_{\bm{\alpha}}(\omega,t)\;\mathrm{d}\omega\\
% &= \sum_{\bm{\alpha}}\int_0^\infty\frac{2c^3}{\pi e^2\omega}\mathit{\Pi}_{\bm{\alpha}}^2(\omega,t)\;\mathrm{d}\omega + \sum_{\bm{\alpha}\bm{\beta}n}\int_0^\infty\frac{2c^3}{\pi e^2\omega}\lambda_{\bm{\alpha}\bm{\beta}n}(\omega)\mathit{\Pi}_{\bm{\alpha}}(\omega,t)\mathcal{B}_{\bm{\beta}n}(t)\;\mathrm{d}\omega\\
% &\qquad +\sum_{\bm{\alpha}}\int_0^\infty\left(-\frac{c^3}{\pi e^2\omega}\mathit{\Pi}_{\bm{\alpha}}^2(\omega,t) + \frac{\pi e^2\omega}{4c^3}\omega^2\mathcal{A}_{\bm{\alpha}}^2(\omega,t)\right)\mathrm{d}\omega - 2\sum_{\bm{\alpha}\bm{\beta}n}\int_0^\infty\frac{c^3}{\pi e^2\omega}\lambda_{\bm{\alpha}\bm{\beta}n}(\omega)\mathit{\Pi}_{\bm{\alpha}}(\omega,t)\mathcal{B}_{\bm{\beta}n}(t)\;\mathrm{d}\omega\\
% &\qquad - \sum_{\alpha}\sum_{\bm{\beta}\bm{\beta}'}\sum_{nn'}\int_0^\infty\frac{c^3}{\pi e^2\omega}\lambda_{\bm{\alpha}\bm{\beta}n}(\omega)\lambda_{\bm{\alpha}\bm{\beta}'n'}(\omega)\mathcal{B}_{\bm{\beta}n}(t)\mathcal{B}_{\bm{\beta}'n'}(t)\;\mathrm{d}\omega + \sum_{\bm{\beta}n}\frac{\mathcal{P}_{\bm{\beta}n}^2(t)}{m_{\bm{\beta}n}}\\
% &\qquad - \sum_{\bm{\beta}n}\left(\frac{\mathcal{P}_{\bm{\beta}n}^2(t)}{2m_{\bm{\beta}n}} - \frac{1}{2}m_{\bm{\beta}n}\omega_0^2\mathcal{B}_{\bm{\beta}n}^2(t)\right) + \sum_{\bm{\beta}nn'}g_{\bm{\beta}nn'}\mathcal{B}_{\bm{\beta}n}(t)\mathcal{B}_{\bm{\beta}n'}(t)\\
% &\qquad + \sum_{\bm{\gamma}n}\int_0^\infty\frac{\mathcal{Q}_{\bm{\gamma}n}^2(\nu,t)}{m_{\bm{\gamma}n}}\mathrm{d}\nu + \sum_{\bm{\beta}n}\int_0^\infty\frac{v_{\bm{\beta}n}(\nu)}{m_{\bm{\beta}n}}\mathcal{B}_{\bm{\beta}n}(t)\mathcal{Q}_{\bm{\beta}n}(\nu,t)\;\mathrm{d}\nu\\
% &\qquad - \sum_{\bm{\gamma}n}\int_0^\infty\left(\frac{\mathcal{Q}_{\bm{\gamma}n}^2(\nu,t)}{2m_{\bm{\gamma}n}} - \frac{1}{2}m_{\bm{\gamma}n}\nu^2\mathcal{C}_{\bm{\gamma}n}(\nu,t)\right)\;\mathrm{d}\nu - \sum_{\bm{\beta}n}\int_0^\infty\frac{v_{\bm{\beta}n}(\nu)}{m_{\bm{\beta}n}}\mathcal{B}_{\bm{\beta}n}(t)\mathcal{Q}_{\bm{\beta}n}(\nu,t)\;\mathrm{d}\nu\\
% &\qquad + \sum_{\bm{\beta}n}\int_0^\infty\frac{v_{\bm{\beta}n}(\nu)}{m_{\bm{\beta}n}}\mathcal{B}_{\bm{\beta}n}(t)\mathcal{Q}_{\bm{\beta}n}(\nu,t)\;\mathrm{d}\nu + \sum_{\bm{\beta}n}\int_0^\infty\frac{v_{\bm{\beta}n}^2(\nu)}{m_{\bm{\beta}n}}\mathcal{B}_{\bm{\beta}n}^2(t)\;\mathrm{d}\nu\\
% &\qquad + \sum_{\bm{\alpha}\bm{\beta}n}\int_0^\infty\frac{2c^3}{\pi e^2\omega}\lambda_{\bm{\alpha}\bm{\beta}n}(\omega)\mathit{\Pi}_{\bm{\alpha}}(\omega,t)\mathcal{B}_{\bm{\beta}n}(t)\;\mathrm{d}\omega + \sum_{\bm{\alpha}}\sum_{\bm{\beta}\bm{\beta}'}\sum_{nn'}\int_0^\infty\frac{2c^3}{\pi e^2\omega}\lambda_{\bm{\alpha}\bm{\beta}n}(\omega)\lambda_{\bm{\alpha}\bm{\beta}'n'}(\omega)\mathcal{B}_{\bm{\beta}n}(t)\mathcal{B}_{\bm{\beta}'n'}(t)\;\mathrm{d}\omega\\
% &\qquad + \frac{1}{4\pi}\int\left[\nabla\Phi_f(\mathbf{r},t)\right]^2\;\mathrm{d}^3\mathbf{r} + \sum_{\bm{\alpha}}\int_0^\infty G_{\bm{\alpha}}[\{\mathbf{x}_{fi}(t)\},\omega]\mathcal{A}_{\bm{\alpha}}(\omega,t)\;\mathrm{d}\omega\\
% &= \sum_{\bm{\alpha}}\int_0^\infty\left(\frac{c^3}{\pi e^2\omega}\mathit{\Pi}_{\bm{\alpha}}^2(\omega,t) + \frac{\pi e^2\omega}{4c^3}\omega^2\mathcal{A}_{\bm{\alpha}}^2(\omega,t)\right)\mathrm{d}\omega + \sum_{\bm{\beta}n}\left(\frac{\mathcal{P}_{\bm{\beta}}^2(t)}{2m_{\bm{\beta}n}} + \frac{1}{2}m_{\bm{\beta}n}\left[\omega_0^2 + \int_0^\infty\frac{2v_{\bm{\beta}n}^2(\nu)}{m_{\bm{\beta}n}^2}\;\mathrm{d}\nu\right]\mathcal{B}_{\bm{\beta}n}^2(t)\right)\\
% &\qquad + \sum_{\bm{\gamma}n}\int_0^\infty\left(\frac{\mathcal{Q}_{\bm{\gamma}n}^2(\nu,t)}{2m_{\bm{\gamma}n}} + \frac{1}{2}m_{\bm{\gamma}n}\nu^2\mathcal{C}_{\bm{\gamma}n}^2(\nu,t)\right)\mathrm{d}\nu + \sum_{\bm{\beta}nn'}g_{\bm{\beta}nn'}\mathcal{B}_{\bm{\beta}n}(t)\mathcal{B}_{\bm{\beta}n'}(t)\\
% &\qquad + \sum_{\bm{\beta}\bm{\beta}'}\sum_{nn'}\mathcal{B}_{\bm{\beta}n}(t)\mathcal{B}_{\bm{\beta}'n'}(t)\sum_{\bm{\alpha}}\int_0^\infty\frac{c^3}{\pi e^2\omega}\lambda_{\bm{\alpha}\bm{\beta}n}(\omega)\lambda_{\bm{\alpha}\bm{\beta}'n'}(\omega)\;\mathrm{d}\omega\\
% &\qquad + \sum_{\bm{\alpha}\bm{\beta}n}\int_0^\infty\frac{2c^3}{\pi e^2\omega}\lambda_{\bm{\alpha}\bm{\beta}n}(\omega)\mathit{\Pi}_{\bm{\alpha}}(\omega,t)\mathcal{B}_{\bm{\beta}n}(t)\;\mathrm{d}\omega + \sum_{\bm{\beta}n}\int_0^\infty\frac{v_{\bm{\beta}n}(\nu)}{m_{\bm{\beta}n}}\mathcal{B}_{\bm{\beta}n}(t)\mathcal{Q}_{\bm{\beta}n}(\nu,t)\;\mathrm{d}\nu\\
% &\qquad + \frac{1}{4\pi}\int\left[\nabla\Phi_f(\mathbf{r},t)\right]^2\;\mathrm{d}^3\mathbf{r} + \sum_{\bm{\alpha}}\int_0^\infty G_{\bm{\alpha}}[\{\mathbf{x}_{fi}(t)\},\omega]\mathcal{A}_{\bm{\alpha}}(\omega,t)\;\mathrm{d}\omega\\
&= \sum_{\bm{\alpha}}\int_0^\infty\left(\frac{c^3}{\pi e^2\omega}\mathit{\Pi}_{\bm{\alpha}}^2(\omega,t) + \frac{\pi e^2\omega}{4c^3}\omega^2\mathcal{A}_{\bm{\alpha}}^2(\omega,t)\right)\mathrm{d}\omega + \sum_{\bm{\beta}n}\left(\frac{\mathcal{P}_{\bm{\beta}}^2(t)}{2m_{\bm{\beta}n}} + \frac{1}{2}m_{\bm{\beta}n}\tilde{\omega}_{\bm{\beta}n}^2\mathcal{B}_{\bm{\beta}n}^2(t)\right)\\
&\qquad + \sum_{\bm{\gamma}n}\int_0^\infty\left(\frac{\mathcal{Q}_{\bm{\gamma}n}^2(\nu,t)}{2m_{\bm{\gamma}n}} + \frac{1}{2}m_{\bm{\gamma}n}\nu^2\mathcal{C}_{\bm{\gamma}n}^2(\nu,t)\right)\mathrm{d}\nu + \sum_{\bm{\beta}\bm{\beta}'}\sum_{nn'}\tilde{g}_{\bm{\beta}\bm{\beta}'nn'}\mathcal{B}_{\bm{\beta}n}(t)\mathcal{B}_{\bm{\beta}'n'}(t)\\
&\qquad + \sum_{\bm{\alpha}\bm{\beta}n}\int_0^\infty\frac{2c^3}{\pi e^2\omega}\lambda_{\bm{\alpha}\bm{\beta}n}(\omega)\mathit{\Pi}_{\bm{\alpha}}(\omega,t)\mathcal{B}_{\bm{\beta}n}(t)\;\mathrm{d}\omega + \sum_{\bm{\beta}n}\int_0^\infty\frac{v_{\bm{\beta}n}(\nu)}{m_{\bm{\beta}n}}\mathcal{B}_{\bm{\beta}n}(t)\mathcal{Q}_{\bm{\beta}n}(\nu,t)\;\mathrm{d}\nu\\
&\qquad + \frac{1}{4\pi}\int\left[\nabla\Phi_f(\mathbf{r},t)\right]^2\;\mathrm{d}^3\mathbf{r} + \sum_{\bm{\alpha}}\int_0^\infty G_{\bm{\alpha}}[\{\mathbf{x}_{fi}(t)\},\omega]\mathcal{A}_{\bm{\alpha}}(\omega,t)\;\mathrm{d}\omega
\end{split}
\end{equation}
where
\begin{equation}
\begin{split}
\tilde{\omega}_{\bm{\beta}n} &= \left[\omega_0^2 + \int_0^\infty\frac{2v_{\bm{\beta}n}^2(\nu)}{m_{\bm{\beta}n}^2}\;\mathrm{d}\nu\right]^\frac{1}{2},\\
\tilde{g}_{\bm{\beta}\bm{\beta}'nn'} &= g_{\bm{\beta}nn'}\delta_{\bm{\beta}\bm{\beta}'} + \sum_{\bm{\alpha}}\int_0^\infty\frac{c^3}{\pi e^2\omega}\lambda_{\bm{\alpha}\bm{\beta}n}(\omega)\lambda_{\bm{\alpha}\bm{\beta}'n'}(\omega)\;\mathrm{d}\omega.
\end{split}
\end{equation}
Quantization is trivial here, with each coordinate and momentum replaced by an operator. Assuming each coordinate represents the excitation of a bosonic field, we can define
\begin{equation}
\begin{split}
\left[\hat{\mathcal{A}}_{\bm{\alpha}}(\omega,t),\hat{\mathit{\Pi}}_{\bm{\alpha}'}(\omega',t)\right] &= \mathrm{i}\hbar\delta_{\bm{\alpha}\bm{\alpha}'}\delta(\omega - \omega'),\\
\left[\hat{\mathcal{B}}_{\bm{\beta}n}(t),\hat{\mathcal{P}}_{\bm{\beta}'n'}(t)\right] &= \mathrm{i}\hbar\delta_{\bm{\beta}\bm{\beta}'}\delta_{nn'},\\
\left[\hat{\mathcal{C}}_{\bm{\gamma}n}(\nu,t),\hat{\mathcal{Q}}_{\bm{\gamma'}n'}(\nu',t)\right] &= \mathrm{i}\hbar\delta_{\bm{\gamma}\bm{\gamma}'}\delta_{nn'}\delta(\nu - \nu').
\end{split}
\end{equation}
Convenient definitions of the annihilation operators associated with each coordinate are then
\begin{equation}
\begin{split}
\hat{a}_{\bm{\alpha}}(\omega,t) &= \sqrt{\frac{\pi e^2}{4\hbar c^3}}\left[-\mathrm{i}\omega\hat{\mathcal{A}}_{\bm{\alpha}}(\omega,t) + \frac{2c^3}{\pi e^2\omega}\hat{\mathit{\Pi}}_{\bm{\alpha}}(\omega,t)\right],\\
\hat{b}_{\bm{\beta}n}(t) &= \sqrt{\frac{m_{\bm{\beta}n}}{2\hbar\tilde{\omega}_{\bm{\beta}n}}}\left[\tilde{\omega}_{\bm{\beta}n}\hat{\mathcal{B}}_{\bm{\beta}n}(t) + \frac{\mathrm{i}}{m_{\bm{\beta}n}}\hat{\mathcal{P}}_{\bm{\beta}n}(t)\right],\\
\hat{c}_{\bm{\gamma}n}(\nu,t) &= \sqrt{\frac{m_{\bm{\gamma}n}}{2\hbar\nu}}\left[-\mathrm{i}\nu\hat{\mathcal{C}}_{\bm{\gamma}n}(\nu,t) + \frac{1}{m_{\bm{\gamma}n}}\hat{\mathcal{Q}}_{\bm{\gamma}n}(\nu,t)\right],
\end{split}
\end{equation}
such that
\begin{equation}
\begin{split}
\left[\hat{a}_{\bm{\alpha}}(\omega,t),\hat{a}_{\bm{\alpha}'}^\dagger(\omega',t)\right] &= \delta_{\bm{\alpha}\bm{\alpha}'}\delta(\omega - \omega'),\\
\left[\hat{b}_{\bm{\beta}n}(t),\hat{b}_{\bm{\beta}'n'}^\dagger(t)\right] &= \delta_{\bm{\beta}\bm{\beta}'}\delta_{nn'},\\
\left[\hat{c}_{\bm{\gamma}n}(\nu,t),\hat{c}_{\bm{\gamma}'n'}^\dagger(\nu',t)\right] &= \delta_{\bm{\gamma}\bm{\gamma}'}\delta_{nn'}\delta(\nu - \nu'),
\end{split}
\end{equation}
and
\begin{equation}
\begin{split}
\left[\hat{a}_{\bm{\alpha}}(\omega,t),\hat{a}_{\bm{\alpha}'}(\omega',t)\right] &= \left[\hat{a}_{\bm{\alpha}}^\dagger(\omega,t),\hat{a}_{\bm{\alpha}'}^\dagger(\omega',t)\right] = 0,\\
\left[\hat{b}_{\bm{\beta}n}(t),\hat{b}_{\bm{\beta}'n'}(t)\right] &= \left[\hat{b}_{\bm{\beta}n}^\dagger(t),\hat{b}_{\bm{\beta}'n'}^\dagger(t)\right] = 0,\\
\left[\hat{c}_{\bm{\gamma}n}(\nu,t),\hat{c}_{\bm{\gamma}'n'}(\nu',t)\right] &= \left[\hat{c}_{\bm{\gamma}n}^\dagger(\nu,t),\hat{c}_{\bm{\gamma}'n'}^\dagger(\nu',t)\right] = 0.
\end{split}
\end{equation}
The coordinate and momentum operators can then be reconstructed such that
\begin{equation}
\begin{split}
\hat{\mathcal{A}}_{\bm{\alpha}}(\omega,t) &= \mathrm{i}\sqrt{\frac{\hbar c^3}{\pi e^2\omega^2}}\left[\hat{a}_{\bm{\alpha}}(\omega,t) - \hat{a}_{\bm{\alpha}}^\dagger(\omega,t)\right],\\
\hat{\mathit{\Pi}}_{\bm{\alpha}}(\omega,t) &= \sqrt{\frac{\pi e^2\hbar\omega^2}{4 c^3}}\left[\hat{a}_{\bm{\alpha}}(\omega,t) + \hat{a}_{\bm{\alpha}}^\dagger(\omega,t)\right],\\[0.5em]
\hat{\mathcal{B}}_{\bm{\beta}n}(t) &= \sqrt{\frac{\hbar}{2\tilde{\omega}_{\bm{\beta}n}m_{\bm{\beta}n}}}\left[\hat{b}_{\bm{\beta}n}(t) + \hat{b}_{\bm{\beta}n}^\dagger(t)\right],\\
\hat{\mathcal{P}}_{\bm{\beta}n}(t) &= -\mathrm{i}\sqrt{\frac{m_{\bm{\beta}n}\hbar\tilde{\omega}_{\bm{\beta}n}}{2}}\left[\hat{b}_{\bm{\beta}n}(t) - \hat{b}_{\bm{\beta}n}^\dagger(t)\right],\\[0.5em]
\hat{\mathcal{C}}_{\bm{\gamma}n}(\nu,t) &= \mathrm{i}\sqrt{\frac{\hbar}{2\nu m_{\bm{\gamma}n}}}\left[\hat{c}_{\bm{\gamma}n}(\nu,t) - \hat{c}_{\bm{\gamma}n}^\dagger(\nu,t)\right],\\
\hat{\mathcal{Q}}_{\bm{\gamma}n}(\nu,t) &= \sqrt{\frac{m_{\bm{\gamma}n}\hbar\nu}{2}}\left[\hat{c}_{\bm{\gamma}n}(\nu,t) + \hat{c}_{\bm{\gamma}n}^\dagger(\nu,t)\right].
\end{split}
\end{equation}
This allows us to write down the Hamiltonian operator as
\begin{equation}
\begin{split}
\hat{H} 
% &= \sum_{\bm{\alpha}}\int_0^\infty\left(\frac{c^3}{\pi e^2\omega}\frac{\pi e^2\hbar\omega^2}{4c^3}\left[\hat{a}_{\bm{\alpha}}(\omega,t) + \hat{a}_{\bm{\alpha}}^\dagger(\omega,t)\right]^2 + \frac{\pi e^2\omega}{4c^3}\omega^2(-1)\frac{\hbar c^3}{\pi e^2\omega^2}\left[\hat{a}_{\bm{\alpha}}(\omega,t) - \hat{a}_{\bm{\alpha}}^\dagger(\omega,t)\right]^2\right)\mathrm{d}\omega\\
% &\qquad + \sum_{\bm{\beta}n}\left(\frac{1}{2m_{\bm{\beta}n}}(-1)\frac{m_{\bm{\beta}n}\hbar\tilde{\omega}_{\bm{\beta}n}}{2}\left[\hat{b}_{\bm{\beta}n}(t) - \hat{b}_{\bm{\beta}n}^\dagger(t)\right]^2 + \frac{m_{\bm{\beta}n}\tilde{\omega}_{\bm{\beta}n}^2}{2}\frac{\hbar}{2\tilde{\omega}_{\bm{\beta}n}m_{\bm{\beta}n}}\left[\hat{b}_{\bm{\beta}n}(t) + \hat{b}_{\bm{\beta}n}(t)\right]^2\right)\\
% &\qquad + \sum_{\bm{\gamma}n}\int_0^\infty\left(\frac{1}{2m_{\bm{\gamma}n}}\frac{m_{\bm{\gamma}n}\hbar\nu}{2}\left[\hat{c}_{\bm{\gamma}n}(\nu,t) + \hat{c}_{\bm{\gamma}n}^\dagger(\nu,t)\right]^2 + \frac{m_{\bm{\gamma}n}\nu^2}{2}(-1)\frac{\hbar}{2\nu m_{\bm{\gamma}n}}\left[\hat{c}_{\bm{\gamma}n}(\nu,t) - \hat{c}_{\bm{\gamma}n}^\dagger(\nu,t)\right]^2\right)\mathrm{d}\nu\\
% &\qquad + \sum_{\bm{\beta}\bm{\beta}'}\sum_{nn'}\tilde{g}_{\bm{\beta}\bm{\beta}'nn'}\frac{\hbar}{2\tilde{\omega}_{\bm{\beta}n}\sqrt{m_{\bm{\beta}n}m_{\bm{\beta}'n'}}}\left[\hat{b}_{\bm{\beta}n}(t) + \hat{b}_{\bm{\beta}n}^\dagger(t)\right]\left[\hat{b}_{\bm{\beta}'n'}(t) + \hat{b}_{\bm{\beta}'n'}^\dagger(t)\right]\\
% &\qquad + \sum_{\bm{\alpha}\bm{\beta}n}\int_0^\infty\frac{2c^3}{\pi e^2\omega}\lambda_{\bm{\alpha}\bm{\beta}n}(\omega)\sqrt{\frac{\pi e^2\hbar\omega^2}{4c^3}}\left[\hat{a}_{\bm{\alpha}}(\omega,t) + \hat{a}_{\bm{\alpha}}^\dagger(\omega,t)\right]\sqrt{\frac{\hbar}{2\tilde{\omega}_{\bm{\beta}n}m_{\bm{\beta}n}}}\left[\hat{b}_{\bm{\beta}n}(t) + \hat{b}_{\bm{\beta}n}^\dagger(t)\right]\mathrm{d}\omega\\
% &\qquad + \sum_{\bm{\beta}n}\int_0^\infty\frac{v_{\bm{\beta}n}(\nu)}{m_{\bm{\beta}n}}\sqrt{\frac{\hbar}{2\tilde{\omega}_{\bm{\beta}n}m_{\bm{\beta}n}}}\left[\hat{b}_{\bm{\beta}n}(t) + \hat{b}_{\bm{\beta}n}^\dagger(t)\right]\sqrt{\frac{m_{\bm{\beta}n}\hbar\nu}{2}}\left[\hat{c}_{\bm{\beta}n}(\nu,t) + \hat{c}_{\bm{\beta}n}^\dagger(\nu,t)\right]\mathrm{d}\nu\\
% &= \sum_{\bm{\alpha}}\int_0^\infty\left(\frac{\hbar\omega}{4}\left[\hat{a}_{\bm{\alpha}}(\omega,t) + \hat{a}_{\bm{\alpha}}^\dagger(\omega,t)\right]^2 - \frac{\hbar\omega}{4}\left[\hat{a}_{\bm{\alpha}}(\omega,t) - \hat{a}_{\bm{\alpha}}^\dagger(\omega,t)\right]^2\right)\mathrm{d}\omega\\
% &\qquad + \sum_{\bm{\beta}n}\left(-\frac{\hbar\tilde{\omega}_{\bm{\beta}n}}{4}\left[\hat{b}_{\bm{\beta}n}(t) - \hat{b}_{\bm{\beta}n}^\dagger(t)\right]^2 + \frac{\hbar\tilde{\omega}_{\bm{\beta}n}}{4}\left[\hat{b}_{\bm{\beta}n}(t) + \hat{b}_{\bm{\beta}n}^\dagger(t)\right]^2\right)\\
% &\qquad + \sum_{\bm{\gamma}n}\int_0^\infty\left(\frac{\hbar\nu}{4}\left[\hat{c}_{\bm{\gamma}n}(\nu,t) + \hat{c}_{\bm{\gamma}n}^\dagger(\nu,t)\right]^2 - \frac{\hbar\nu}{4}\left[\hat{c}_{\bm{\gamma}n}(\nu,t) - \hat{c}_{\bm{\gamma}n}^\dagger(\nu,t)\right]^2\right)\mathrm{d}\nu\\
% &\qquad + \sum_{\bm{\beta}\bm{\beta}'}\sum_{nn'}\tilde{g}_{\bm{\beta}\bm{\beta}'nn'}\frac{\hbar}{2\tilde{\omega}_{\bm{\beta}n}\sqrt{m_{\bm{\beta}n}m_{\bm{\beta}'n'}}}\left[\hat{b}_{\bm{\beta}n}(t) + \hat{b}_{\bm{\beta}n}^\dagger(t)\right]\left[\hat{b}_{\bm{\beta}'n'}(t) + \hat{b}_{\bm{\beta}'n'}^\dagger(t)\right]\\
% &\qquad + \sum_{\bm{\alpha}\bm{\beta}n}\int_0^\infty\sqrt{\frac{c^3\hbar^2}{2\pi e^2\tilde{\omega}_{\bm{\beta}n}m_{\bm{\beta}n}}}\lambda_{\bm{\alpha}\bm{\beta}n}(\omega)\left[\hat{a}_{\bm{\alpha}}(\omega,t) + \hat{a}_{\bm{\alpha}}^\dagger(\omega,t)\right]\left[\hat{b}_{\bm{\beta}n}(t) + \hat{b}_{\bm{\beta}n}^\dagger(t)\right]\mathrm{d}\omega\\
% &\qquad + \sum_{\bm{\beta}n}\int_0^\infty \sqrt{\frac{\hbar^2\nu}{4\tilde{\omega}_{\bm{\beta}n}m_{\bm{\beta}n}^2}}v_{\bm{\beta}n}(\nu)\left[\hat{b}_{\bm{\beta}n}(t) + \hat{b}_{\bm{\beta}n}^\dagger(t)\right]\left[\hat{c}_{\bm{\beta}n}(\nu,t) + \hat{c}_{\bm{\beta}n}^\dagger(\nu,t)\right]\mathrm{d}\nu\\
% &= \sum_{\bm{\alpha}}\int_0^\infty\frac{\hbar\omega}{4}\left[2\hat{a}_{\bm{\alpha}}^\dagger(\omega,t)\hat{a}_{\bm{\alpha}}(\omega,t) + 2\hat{a}_{\bm{\alpha}}(\omega,t)\hat{a}_{\bm{\alpha}}^\dagger(\omega,t)\right]\mathrm{d}\omega\\
% &\qquad + \sum_{\bm{\beta}n}\frac{\hbar\tilde{\omega}_{\bm{\beta}n}}{4}\left[2\hat{b}_{\bm{\beta}n}^\dagger(t)\hat{b}_{\bm{\beta}n}(t) + 2\hat{b}_{\bm{\beta}n}(t)\hat{b}_{\bm{\beta}n}^\dagger(t)\right]\\
% &\qquad + \sum_{\bm{\gamma}n}\int_0^\infty\frac{\hbar\nu}{4}\left[2\hat{c}_{\bm{\gamma}n}^\dagger(\nu,t)\hat{c}_{\bm{\gamma}n}(\nu,t) + \hat{c}_{\bm{\gamma}n}(\nu,t)\hat{c}_{\bm{\gamma}n}^\dagger(\nu,t)\right]\mathrm{d}\nu\\
% &\qquad + \sum_{\bm{\beta}\bm{\beta}'}\sum_{nn'}\hbar\sigma_{\bm{\beta}\bm{\beta}'nn'}\left[\hat{b}_{\bm{\beta}n}(t) + \hat{b}_{\bm{\beta}n}^\dagger(t)\right]\left[\hat{b}_{\bm{\beta}'n'}(t) + \hat{b}_{\bm{\beta}'n'}^\dagger(t)\right]\\
% &\qquad + \sum_{\bm{\alpha}\bm{\beta}n}\int_0^\infty\hbar\Lambda_{\bm{\alpha}\bm{\beta}n}(\omega)\left[\hat{a}_{\bm{\alpha}}(\omega,t) + \hat{a}_{\bm{\alpha}}^\dagger(\omega,t)\right]\left[\hat{b}_{\bm{\beta}n}(t) + \hat{b}_{\bm{\beta}n}^\dagger(t)\right]\mathrm{d}\omega\\
% &\qquad + \sum_{\bm{\beta}n}\int_0^\infty\hbar V_{\bm{\beta}n}(\nu)\left[\hat{b}_{\bm{\beta}n}(t) + \hat{b}_{\bm{\beta}n}^\dagger(t)\right]\left[\hat{c}_{\bm{\beta}n}(\nu,t) + \hat{c}_{\bm{\beta}n}^\dagger(\nu,t)\right]\mathrm{d}\nu\\
&= \sum_{\bm{\alpha}}\int_0^\infty\hbar\omega\hat{a}_{\bm{\alpha}}^\dagger(\omega,t)\hat{a}_{\bm{\alpha}}(\omega,t)\;\mathrm{d}\omega + \sum_{\bm{\beta}n}\hbar\tilde{\omega}_{\bm{\beta}n}\hat{b}_{\bm{\beta}n}^\dagger(t)\hat{b}_{\bm{\beta}n}(t) + \sum_{\bm{\gamma}n}\int_0^\infty\hbar\nu\hat{c}_{\bm{\gamma}n}^\dagger(\nu,t)\hat{c}_{\bm{\gamma}n}(\nu,t)\;\mathrm{d}\nu\\
&\qquad + \sum_{\bm{\beta}\bm{\beta}'}\sum_{nn'}\hbar\sigma_{\bm{\beta}\bm{\beta}'nn'}\left[\hat{b}_{\bm{\beta}n}(t) + \hat{b}_{\bm{\beta}n}^\dagger(t)\right]\left[\hat{b}_{\bm{\beta}'n'}(t) + \hat{b}_{\bm{\beta}'n'}^\dagger(t)\right]\\
&\qquad + \sum_{\bm{\alpha}\bm{\beta}n}\int_0^\infty\hbar\Lambda_{\bm{\alpha}\bm{\beta}n}(\omega)\left[\hat{a}_{\bm{\alpha}}(\omega,t) + \hat{a}_{\bm{\alpha}}^\dagger(\omega,t)\right]\left[\hat{b}_{\bm{\beta}n}(t) + \hat{b}_{\bm{\beta}n}^\dagger(t)\right]\mathrm{d}\omega\\
&\qquad + \sum_{\bm{\beta}n}\int_0^\infty\hbar V_{\bm{\beta}n}(\nu)\left[\hat{b}_{\bm{\beta}n}(t) + \hat{b}_{\bm{\beta}n}^\dagger(t)\right]\left[\hat{c}_{\bm{\beta}n}(\nu,t) + \hat{c}_{\bm{\beta}n}^\dagger(\nu,t)\right]\mathrm{d}\nu
\end{split}
\end{equation}
where
\begin{equation}
\begin{split}
\sigma_{\bm{\beta}\bm{\beta}'nn'} &= \frac{\tilde{g}_{\bm{\beta}\bm{\beta}'nn'}}{2\sqrt{\tilde{\omega}_{\bm{\beta}n}\tilde{\omega}_{\bm{\beta}'n'}}\sqrt{m_{\bm{\beta}n}m_{\bm{\beta}'n'}}},\\
\Lambda_{\bm{\alpha}\bm{\beta}n}(\omega) &= \sqrt{\frac{c^3}{2\pi e^2\tilde{\omega}_{\bm{\beta}n}m_{\bm{\beta}n}}}\lambda_{\bm{\alpha}\bm{\beta}n}(\omega),\\
 V_{\bm{\beta}n}(\nu) &= \sqrt{\frac{\nu}{4\tilde{\omega}_{\bm{\beta}n}m^2_{\bm{\beta}n}}}v_{\bm{\beta}n}(\nu)
\end{split}
\end{equation}
and we have neglected any constant terms, as they do not modify the motion of any of the system's modes.

Diagonalization from this point is done by first identifying that the coupling between the different modes of the system is strongly limited by symmetry. All coupling occurs between modes of common indices $p$, $\ell$, and $m$. Further, modes of magnetic type ($T = M$) only couple to other magnetic modes. We can then rewrite $\hat{H}$ such that
\begin{equation}
\hat{H} = \sum_{p\ell m}\left(\hat{H}_{p\ell m}^M + \hat{H}_{p\ell m}^{EL}\right),
\end{equation}
where terms of superscript $M$ only include magnetic modes and terms of superscript $EL$ include the coupled electric- and longitudinal-type modes. To make each term in the sum above more readable, we can suppress some of our repeated indices such that our operators simplify to
\begin{equation}
\begin{split}
\hat{H}_{p\ell m}^M&\to\hat{H}_M,\\
\hat{H}_{p\ell m}^{EL}&\to\hat{H}_{EL},\\
\hat{a}_{Mp\ell m}(t)(\omega,t)&\to\hat{a}(\omega,t),\\
\hat{a}_{Sp\ell m}(t)(\omega,t)&\to\hat{a}_{S}(\omega,t),\\
\hat{b}_{Mp\ell mn}(t)&\to\hat{b}_n(t)\\
\hat{b}_{Sp\ell mn}(t)&\to\hat{b}_{Sn}(t),\\
\hat{c}_{Mp\ell mn}(\nu,t)&\to\hat{c}_n(\nu,t),\\
\hat{c}_{Sp\ell mn}(\nu,r)&\to\hat{c}_{Sn}(\nu,t),
\end{split}
\end{equation}
where $S$ here can take the value $E$ or $L$. Our coupling constants and mode masses and natural frequencies transform similarly, such that
\begin{equation}
\begin{split}
m_{Mp\ell m}&\to m,\\
m_{Sp\ell m}&\to m_S,\\
\tilde{\omega}_{Mp\ell mn}&\to\tilde{\omega}_n,\\
\tilde{\omega}_{Sp\ell mn}&\to\tilde{\omega}_{Sn},\\
\sigma_{Mp\ell m;Mp\ell m;n;n'}&\to \sigma_{nn'},\\
\sigma_{Sp\ell m;S'p\ell m;n;n'}&\to\sigma_{SS'nn'},\\
\Lambda_{Mp\ell m;Mp\ell m;n}(\omega)&\to\lambda_n(\omega),\\
\Lambda_{Sp\ell m;S'p\ell m;n}(\omega)&\to \lambda_{SS'n}(\omega),\\
 V_{Mp\ell mn}(\nu)&\to V_n(\nu),\\
 V_{Sp\ell mn}(\nu)&\to V_{Sn}(\nu).
\end{split}
\end{equation}
While we now need to be careful to not ignore any suppressed indices, we can write the cleaner expressions
\begin{equation}
\begin{split}
\hat{H}_M &= \int_0^\infty\hbar\omega\hat{a}^\dagger(\omega,t)\hat{a}(\omega,t)\;\mathrm{d}\omega + \sum_n\hbar\tilde{\omega}_n\hat{b}_n^\dagger(t)\hat{b}_n(t) + \sum_n\int_0^\infty\hbar\nu\hat{c}_n^\dagger(\nu,t)\hat{c}_n(\nu,t)\;\mathrm{d}\nu\\
&\qquad + \sum_{nn'}\hbar\sigma_{nn'}\left[\hat{b}_n(t) + \hat{b}_n^\dagger(t)\right]\left[\hat{b}_{n'}(t) + \hat{b}_{n'}^\dagger(t)\right] \\
&\qquad + \sum_n\int_0^\infty\hbar\Lambda_n(\omega)\left[\hat{a}(\omega,t) + \hat{a}^\dagger(\omega,t)\right]\left[\hat{b}_n(t) + \hat{b}_n^\dagger(t)\right]\mathrm{d}\omega,\\
&\qquad + \sum_n\int_0^\infty\hbar V_n(\nu)\left[\hat{b}_n(t) + \hat{b}_n^\dagger(t)\right]\left[\hat{c}_n(\nu,t) + \hat{c}_n^\dagger(\nu,t)\right]\mathrm{d}\nu
\end{split}
\end{equation}
and
\begin{equation}
\begin{split}
\hat{H}_{EL} &= \int_0^\infty\hbar\omega\hat{a}_E^\dagger(\omega,t)\hat{a}_E(\omega,t)\;\mathrm{d}\omega + \sum_{Sn}\hbar\tilde{\omega}_{Sn}\hat{b}_{Sn}^\dagger(t)\hat{b}_{Sn}(t) + \sum_{Sn}\int_0^\infty\hbar\nu\hat{c}_{Sn}^\dagger(\nu,t)\hat{c}_{Sn}(\nu,t)\;\mathrm{d}\nu\\
&\qquad + \sum_{SS'}\sum_{nn'}\hbar\sigma_{SS'nn'}\left[\hat{b}_{Sn}(t) + \hat{b}_{Sn}^\dagger(t)\right]\left[\hat{b}_{S'n'}(t) + \hat{b}_{S'n'}^\dagger(t)\right]\\
&\qquad + \sum_{S}\sum_n\hbar\Lambda_{ESn}(\omega)\left[\hat{a}_{E}(\omega,t) + \hat{a}_{E}^\dagger(\omega,t)\right]\left[\hat{b}_{Sn}(t) + \hat{b}_{Sn}^\dagger(t)\right]\mathrm{d}\omega\\
&\qquad + \sum_{Sn}\hbar V_{Sn}(\nu)\left[\hat{b}_{Sn}(t) + \hat{b}_{Sn}^\dagger(t)\right]\left[\hat{c}_{Sn}(\nu,t) + \hat{c}_{Sn}^\dagger(\nu,t)\right]\mathrm{d}\nu,
\end{split}
\end{equation}
wherein we have remembered that the photon operators $\hat{a}_E(\omega,t)$ do not have longitudinal counterparts. Each term $\hat{H}_M$ and $\hat{H}_{EL}$ can be diagonalized separately in two steps. Looking first at the magnetic Hamiltonian terms, we can let $\hat{H}_M = \hat{H}_M' + \hat{H}_M^\mathrm{mat}$, where
\begin{equation}
\hat{H}_M^\mathrm{mat} = \sum_{n}\hbar\tilde{\omega}_n\hat{b}_n^\dagger(t)\hat{b}_n(t) + \sum_{nn'}\hbar\sigma_{nn'}\left[\hat{b}_n(t) + \hat{b}_n^\dagger(t)\right]\left[\hat{b}_{n'}(t) + \hat{b}_{n'}^\dagger(t)\right].
\end{equation}
In matrix-vector form, $\hat{H}_M^\mathrm{mat}$ can be represented as
\begin{equation}
\begin{split}
\hat{H}_M^\mathrm{mat} &\overset{.}{=}
\begin{pmatrix}
\hat{b}_1^\dagger(t) & \hat{b}_1(t) & \hat{b}_2^\dagger(t) & \hat{b}_2(t) & \cdots
\end{pmatrix}
\begin{pmatrix}
\hbar\tilde{\omega}_1 + \hbar\sigma_{11} & \hbar\sigma_{11} & \hbar\sigma_{12} & \hbar\sigma_{12} & \cdots\\
\hbar\sigma_{11} & \hbar\sigma_{11} & \hbar\sigma_{12} & \hbar\sigma_{12} &\\
\hbar\sigma_{21} & \hbar\sigma_{21} & \hbar\tilde{\omega}_2 + \hbar\sigma_{22} & \hbar\sigma_{22} & \\
\hbar\sigma_{21} & \hbar\sigma_{21} & \hbar\sigma_{22} & \hbar\sigma_{22} & \\
\vdots & & & & \ddots
\end{pmatrix}
\begin{pmatrix}
\hat{b}_1(t)\\
\hat{b}_1^\dagger(t)\\
\hat{b}_2(t)\\
\hat{b}_2^\dagger(t)\\
\vdots
\end{pmatrix}\\
&= \hat{\mathbf{b}}^\dagger(t)\mathbf{M}\hat{\mathbf{b}}(t).
\end{split}
\end{equation}
The normalized (real) eigenvectors $\mathbf{u}_i$ of the characteristic matrix $\mathbf{M}$ can be calculated such that
\begin{equation}
\mathbf{M}\mathbf{u}_i = \hbar\Omega_i\mathbf{u}_i
\end{equation}
and appended as column vectors into the (orthogonal) matrix $\mathbf{U}$ such that
\begin{equation}
\mathbf{U} = 
\begin{pmatrix}
\mathbf{u}_1 & \mathbf{u}_2 & \cdots
\end{pmatrix}.
\end{equation}
Therefore, we can see that
\begin{equation}
\begin{split}
\mathbf{M} = \mathbf{U}\bm{\Omega}\mathbf{U}^\top,
\end{split}
\end{equation}
where
\begin{equation}
\bm{\Omega} = 
\begin{pmatrix}
\hbar\Omega_1 & 0 & \cdots\\
0 & \hbar\Omega_2 &\\
\vdots & & \ddots
\end{pmatrix}
\end{equation}
is the diagonal matrix of eigenvalues of $\mathbf{M}$. Thus, 
\begin{equation}
\begin{split}
\hat{H}_M^\mathrm{mat} &= \hat{\mathbf{b}}^\dagger(t)\mathbf{U}\bm{\Omega}\mathbf{U}^\top\hat{\mathbf{b}}(t)\\
&= \hat{\mathbf{B}}^\dagger(t)\bm{\Omega}\hat{\mathbf{B}}(t)
\end{split}
\end{equation}
where
\begin{equation}
\begin{split}
\hat{\mathbf{B}}(t) &= \mathbf{U}^\top\hat{\mathbf{b}}(t)\\
&= 
\begin{pmatrix}
\hat{B}_1(t) & \hat{B}_2(t) & \cdots
\end{pmatrix}^\top
\end{split}
\end{equation}
is the vector of hybrid operators
\begin{equation}
\hat{B}_i(t) = \sum_n\left[u_{i,2n-1}\hat{b}_n(t) + u_{i,2n}\hat{b}_n^\dagger(t)\right]
\end{equation}
with expansion coefficients extracted from the eigenvectors $\mathbf{u}_i$. The inverse transformation is equally simple, with
\begin{equation}
\hat{\mathbf{b}}_n(t) = \mathbf{U}\hat{\mathbf{B}}(t)
\end{equation}
such that
\begin{equation}
\hat{b}_n(t) = \sum_iu_{in}\hat{B}_i(t).
\end{equation}
Transitioning back from matrix-vector notation, we can see that
\begin{equation}
\begin{split}
\hat{H}_M^\mathrm{mat} &= \sum_i\hbar\Omega_i\hat{B}_i^\dagger(t)\hat{B}_i(t)
\end{split}
\end{equation}
and
\begin{equation}
\begin{split}
\hat{H}_M 
% &= \int_0^\infty\hbar\omega\hat{a}^\dagger(\omega,t)\hat{a}(\omega,t)\;\mathrm{d}\omega + \sum_i\hbar\Omega_i\hat{B}_i^\dagger(t)\hat{B}_i(t) + \sum_n\int_0^\infty\hbar\nu\hat{c}_n^\dagger(\nu,t)\hat{c}_n(\nu,t)\;\mathrm{d}\nu\\
% &\qquad + \sum_i\int_0^\infty\hbar\left(\sum_n\Lambda_n(\omega)u_{in}\right)\left[\hat{a}(\omega,t) + \hat{a}^\dagger(\omega,t)\right]\left[\hat{B}_i(t) + \hat{B}_i^\dagger(t)\right]\mathrm{d}\omega\\
% &\qquad + \sum_{in}\int_0^\infty\hbar V_n(\nu)u_{in}\left[\hat{c}_n(\nu,t) + \hat{c}_n^\dagger(\nu,t)\right]\left[\hat{B}_i(t) + \hat{B}_i^\dagger(t)\right]\mathrm{d}\nu\\
&= \int_0^\infty\hbar\omega\hat{a}^\dagger(\omega,t)\hat{a}(\omega,t)\;\mathrm{d}\omega + \sum_i\hbar\Omega_i\hat{B}_i^\dagger(t)\hat{B}_i(t) + \sum_n\int_0^\infty\hbar\nu\hat{c}_n^\dagger(\nu,t)\hat{c}_n(\nu,t)\;\mathrm{d}\nu\\
&\qquad + \sum_i\int_0^\infty\hbar\tilde{\Lambda}_i(\omega)\left[\hat{a}(\omega,t) + \hat{a}^\dagger(\omega,t)\right]\left[\hat{B}_i(t) + \hat{B}_i^\dagger(t)\right]\mathrm{d}\omega\\
&\qquad + \sum_{in}\int_0^\infty\hbar\tilde{V}_{in}(\nu)\left[\hat{c}_n(\nu,t) + \hat{c}_n^\dagger(\nu,t)\right]\left[\hat{B}_i(t) + \hat{B}_i^\dagger(t)\right]\mathrm{d}\nu,
\end{split}
\end{equation}
where
\begin{equation}
\begin{split}
\tilde{\Lambda}_i(\omega) &= \sum_n\Lambda_n(\omega)u_{in},\\
\tilde{V}_{in}(\nu) &= V_n(\nu)u_{in}.
\end{split}
\end{equation}

Repeating this process for $\hat{H}_{EL} = \hat{H}_{EL}' + \hat{H}_{EL}^\mathrm{mat}$, we can define
\begin{equation}
\hat{H}_{EL}^\mathrm{mat} = \sum_{Sn}\hbar\tilde{\omega}_{Sn}\hat{b}_{Sn}^\dagger(t)\hat{b}_{Sn}(t) + \sum_{SS'}\sum_{nn'}\hbar\sigma_{SS'nn'}\left[\hat{b}_{Sn}(t) + \hat{b}_{Sn}^\dagger(t)\right]\left[\hat{b}_{S'n'}(t) + \hat{b}_{S'n'}^\dagger(t)\right]
\end{equation}
and diagonalize it. Explicitly, we can expand in matrix-vector notation as
\begin{equation}
\begin{split}
\hat{H}_{EL}^\mathrm{mat} &\overset{.}{=} 
\begin{pmatrix}
\hat{b}_{E1}^\dagger(t) & \hat{b}_{E1}(t) & \hat{b}_{E2}^\dagger(t) & \hat{b}_{E2}(t) & \cdots & \hat{b}_{L1}^\dagger(t) & \hat{b}_{L1}(t) & \hat{b}_{L2}^\dagger(t) & \hat{b}_{L2}(t) & \cdots
\end{pmatrix}\\
&\times\left(
\begin{matrix}
\hbar\tilde{\omega}_{E1} + \hbar\sigma_{EE11} & \hbar\sigma_{EE11} & \hbar\sigma_{EE12} & \hbar\sigma_{EE12}\\ 
\hbar\sigma_{EE11} & \hbar\sigma_{EE11} & \hbar\sigma_{EE12} & \hbar\sigma_{EE12}\\
\hbar\sigma_{EE21} & \hbar\sigma_{EE21} & \hbar\tilde{\omega}_{E2} + \hbar\sigma_{EE22} & \hbar\sigma_{EE22}\\
\hbar\sigma_{EE21} & \hbar\sigma_{EE21} & \hbar\sigma_{EE22} & \hbar\sigma_{EE22}\\
\vdots & & & \\
\hbar\sigma_{EL11} & \hbar\sigma_{EL11} & \hbar\sigma_{EL21} & \hbar\sigma_{EL21}\\
\hbar\sigma_{EL11} & \hbar\sigma_{EL11} & \hbar\sigma_{EL21} & \hbar\sigma_{EL21}\\
\hbar\sigma_{EL12} & \hbar\sigma_{EL12} & \hbar\sigma_{EL22} & \hbar\sigma_{EL22}\\
\hbar\sigma_{EL12} & \hbar\sigma_{EL12} & \hbar\sigma_{EL22} & \hbar\sigma_{EL22}\\
\vdots & & &
\end{matrix}
\right.\\
&\qquad\qquad\qquad\qquad\left.
\begin{matrix}
\cdots & \hbar\sigma_{EL11} & \hbar\sigma_{EL11} & \hbar\sigma_{EL12} & \hbar\sigma_{EL12} & \cdots\\
\cdots & \hbar\sigma_{EL11} & \hbar\sigma_{EL11} & \hbar\sigma_{EL12} & \hbar\sigma_{EL12} &\\
\cdots & \hbar\sigma_{EL21} & \hbar\sigma_{EL21} & \hbar\sigma_{EL22} & \hbar\sigma_{EL22} &\\
\cdots & \hbar\sigma_{EL21} & \hbar\sigma_{EL21} & \hbar\sigma_{EL22} & \hbar\sigma_{EL22} &\\
\cdots & \hbar\sigma_{EL21} & \hbar\sigma_{EL21} & \hbar\sigma_{EL22} & \hbar\sigma_{EL22} &\\
& & & & \vdots &\\
\cdots & \hbar\tilde{\omega}_{L1} + \hbar\sigma_{LL11} & \hbar\sigma_{LL11} & \hbar\sigma_{LL12} & \hbar\sigma_{LL12} &\\
\cdots & \hbar\sigma_{LL11} & \hbar\sigma_{LL11} & \hbar\sigma_{LL12} & \hbar\sigma_{LL12} &\\
\cdots & \hbar\sigma_{LL12} & \hbar\sigma_{LL12} & \hbar\tilde{\omega}_{L2} + \hbar\sigma_{LL22} & \hbar\sigma_{LL22} &\\
\cdots & \hbar\sigma_{LL12} & \hbar\sigma_{LL12} & \hbar\sigma_{LL22} & \hbar\sigma_{LL22} &\\
& & & & & \ddots
\end{matrix}
\right)
\begin{pmatrix}
\hat{b}_{E1}(t)\\
\hat{b}_{E1}^\dagger(t)\\
\hat{b}_{E2}(t)\\
\hat{b}_{E2}^\dagger(t)\\
\vdots\\
\hat{b}_{L1}(t)\\
\hat{b}_{L1}^\dagger(t)\\
\hat{b}_{L2}(t)\\
\hat{b}_{L2}^\dagger(t)\\
\vdots
\end{pmatrix}\\
&= \hat{\mathbf{b}}^{\prime\dagger}(t)\mathbf{M}'\hat{\mathbf{b}}'(t).
\end{split}
\end{equation}
We can find the normalized eigenvectors $\mathbf{u}'_i$ of our characteristic matrix numerically, such that
\begin{equation}
\mathbf{M}'\mathbf{u}'_i = \hbar\tilde{\Omega}_i\mathbf{u}'_i
\end{equation}
and
\begin{equation}
\mathbf{M}' = \mathbf{U}'\bm{\Omega}'\mathbf{U}'^\top,
\end{equation}
with
\begin{equation}
\mathbf{U}' = 
\begin{pmatrix}
\mathbf{u}'_1 & \mathbf{u}'_2 & \cdots
\end{pmatrix}
\end{equation}
and
\begin{equation}
\bm{\Omega}' = 
\begin{pmatrix}
\hbar\Omega_1' & 0 & \cdots\\
0 & \hbar\Omega_2' & \\
\vdots & & \ddots
\end{pmatrix}.
\end{equation}
This allows us to write
\begin{equation}
\begin{split}
\hat{H}_{EL}^\mathrm{mat} &= \hat{\mathbf{b}}^{\prime\dagger}(t)\mathbf{U}'\bm{\Omega}'\mathbf{U}'^\top\hat{\mathbf{b}}'(t)\\
&= \hat{\mathbf{B}}^{\prime\dagger}(t)\bm{\Omega}'\hat{\mathbf{B}}'(t),
\end{split}
\end{equation}
where
\begin{equation}
\begin{split}
\hat{\mathbf{B}}'(t) &= \mathbf{U}'^\top\hat{\mathbf{b}}'(t)\\
&= 
\begin{pmatrix}
\hat{B}'_1(t) & \hat{B}'_2(t) & \cdots
\end{pmatrix}^\top
\end{split}
\end{equation}
is the vector of hybrid operators
\begin{equation}
\hat{B}'_i = \sum_n\left[u'_{i,2n - 1}\hat{b}_{En}(t) + u'_{i,2n}\hat{b}_{En}^\dagger(t) + u'_{i,2n - 1 + N}\hat{b}_{Ln}(t) + u'_{i,2n + N}\hat{b}_{Ln}^\dagger(t)\right].
\end{equation}
Here we have assumed that $n = 1,2,3,\ldots,N$, where $N$ is the (large) number of spherical Bessel radial indices in the model. We can formally take $N$ to infinity as we wish. Again, inversion is straightforward, with
\begin{equation}
\mathbf{U}'\hat{\mathbf{B}}'(t) = \hat{\mathbf{b}}'(t)
\end{equation}
implying
\begin{equation}
\begin{split}
\hat{b}_{En}(t) &= \Theta_{N - n}\sum_iu'_{in}\hat{B}'_i(t),\\
\hat{b}_{Ln}(t) &= \Theta_{n - N - 1}\sum_iu'_{in}\hat{B}'_i(t),
\end{split}
\end{equation}
where
\begin{equation}
\Theta_k = 
\begin{cases}
1, & k \geq 0,\\
0, & k < 0
\end{cases}
\end{equation}
is the discrete Heaviside function. This tells us that
\begin{equation}
\begin{split}
\hat{H}_{EL}^\mathrm{mat} = \sum_i\hbar\Omega_i'\hat{B}_i^{\prime\dagger}(t)\hat{B}'_i(t)
\end{split}
\end{equation}
and
\begin{equation}
\begin{split}
\hat{H}_{EL} 
% &= \int_0^\infty\hbar\omega\hat{a}_E^\dagger(\omega,t)\hat{a}_E(\omega,t)\;\mathrm{d}\omega + \sum_i\hbar\Omega_i'\hat{B}_i^{\prime\dagger}(t)\hat{B}'_i(t) + \sum_{S}\sum_{n = 1}^N\int_0^\infty\hbar\nu\hat{c}_{Sn}^\dagger(\nu,t)\hat{c}_{Sn}(\nu,t)\;\mathrm{d}\nu\\
% &\qquad + \sum_{n = 1}^{2N}\int_0^\infty\hbar\Lambda_{EEn}(\omega)\left[\hat{a}_E(\omega,t) + \hat{a}_E^\dagger(\omega,t)\right]\sum_iu'_{in}\Theta_{N - n}\left[\hat{B}'_i(t) + \hat{B}_i^{\prime\dagger}(t)\right]\;\mathrm{d}\omega\\
% &\qquad + \sum_{n = 1}^{2N}\int_0^\infty\hbar\Lambda_{ELn}(\omega)\left[\hat{a}_E(\omega,t) + \hat{a}_E^\dagger(\omega,t)\right]\sum_iu'_{in}\Theta_{n - N - 1}\left[\hat{B}'_i(t) + \hat{B}_i^{\prime\dagger}(t)\right]\mathrm{d}\omega\\
% &\qquad + \sum_{n = 1}^{2N}\int_0^\infty\hbar V_{En}(\nu)\sum_iu'_{in}\Theta_{N - n}\left[\hat{B}'_i(t) + \hat{B}_i^{\prime\dagger}(t)\right]\left[\hat{c}_{En}(\nu,t) + \hat{c}_{En}^\dagger(\nu,t)\right]\mathrm{d}\nu\\
% &\qquad + \sum_{n = 1}^{2N}\int_0^\infty\hbar V_{L,n - N}(\nu)\sum_iu'_{in}\Theta_{n - N - 1}\left[\hat{B}'_i(t) + \hat{B}_i^{\prime\dagger}(t)\right]\left[\hat{c}_{L,n - N}(\nu,t) + \hat{c}_{L,n - N}^\dagger(\nu,t)\right]\mathrm{d}\nu\\
% &= \int_0^\infty\hbar\omega\hat{a}_E^\dagger(\omega,t)\hat{a}_E(\omega,t)\;\mathrm{d}\omega + \sum_i\hbar\Omega_i'\hat{B}_i^{\prime\dagger}(t)\hat{B}'_i(t) + \sum_{S}\sum_{n = 1}^N\int_0^\infty\hbar\nu\hat{c}_{Sn}^\dagger(\nu,t)\hat{c}_{Sn}(\nu,t)\;\mathrm{d}\nu\\
% &\qquad + \sum_i\int_0^\infty\hbar\sum_{n = 1}^{2N}u'_{in}\left(\Lambda_{EEn}(\omega)\Theta_{N - n} + \Lambda_{ELn}(\omega)\Theta_{n - N - 1}\right)\left[\hat{a}_E(\omega,t) + \hat{a}_E^\dagger(\omega,t)\right]\left[\hat{B}'_i(t) + \hat{B}_i^{\prime\dagger}(t)\right]\mathrm{d}\omega\\
% &\qquad + \sum_{i}\sum_{n = 1}^{2N}\int_0^\infty\hbar V_{En}(\nu)u'_{in}\Theta_{N - n}\left[\hat{c}_{En}(\nu,t) + \hat{c}_{En}^\dagger(\nu,t)\right]\left[\hat{B}'_i(t) + \hat{B}_i^{\prime\dagger}(t)\right]\mathrm{d}\nu\\
% &\qquad + \sum_i\sum_{n = 1}^{2N}\int_0^\infty\hbar V_{L,n - N}(\nu)\sum_iu'_{in}\Theta_{n - N - 1}\left[\hat{c}_{L,n - N}(\nu,t) + \hat{c}_{L,n - N}^\dagger(\nu,t)\right]\left[\hat{B}'_i(t) + \hat{B}_i^{\prime\dagger}(t)\right]\mathrm{d}\nu\\
% &= \int_0^\infty\hbar\omega\hat{a}_E^\dagger(\omega,t)\hat{a}_E(\omega,t)\;\mathrm{d}\omega + \sum_i\hbar\Omega_i'\hat{B}_i^{\prime\dagger}(t)\hat{B}'_i(t) + \sum_{S}\sum_{n = 1}^N\int_0^\infty\hbar\nu\hat{c}_{Sn}^\dagger(\nu,t)\hat{c}_{Sn}(\nu,t)\;\mathrm{d}\nu\\
% &\qquad + \sum_i\int_0^\infty\hbar\tilde{\Lambda}_{Ei}(\omega)\left[\hat{a}_E(\omega,t) + \hat{a}_E^\dagger(\omega,t)\right]\left[\hat{B}'_i(t) + \hat{B}_i^{\prime\dagger}(t)\right]\mathrm{d}\omega\\
% &\qquad + \sum_i\sum_{n = 1}^{N}\int_0^\infty\hbar\tilde{V}_{Ein}(\nu)\left[\hat{c}_{En}(\nu,t) + \hat{c}_{En}^\dagger(\nu,t)\right]\left[\hat{B}'_i(t) + \hat{B}_i^{\prime\dagger}(t)\right]\mathrm{d}\nu\\
% &\qquad + \sum_i\sum_{n = 1}^{N}\int_0^\infty\hbar\tilde{V}_{Lin}(\nu)\left[\hat{c}_{Ln}(\nu,t) + \hat{c}_{Ln}^\dagger(\nu,t)\right]\left[\hat{B}'_i(t) + \hat{B}_i^{\prime\dagger}(t)\right]\mathrm{d}\nu\\
&= \int_0^\infty\hbar\omega\hat{a}_E^\dagger(\omega,t)\hat{a}_E(\omega,t)\;\mathrm{d}\omega + \sum_i\hbar\Omega_i'\hat{B}_i^{\prime\dagger}(t)\hat{B}'_i(t) + \sum_{S}\sum_{n = 1}^N\int_0^\infty\hbar\nu\hat{c}_{Sn}^\dagger(\nu,t)\hat{c}_{Sn}(\nu,t)\;\mathrm{d}\nu\\
&\qquad + \sum_i\int_0^\infty\hbar\tilde{\Lambda}_{Ei}(\omega)\left[\hat{a}_E(\omega,t) + \hat{a}_E^\dagger(\omega,t)\right]\left[\hat{B}'_i(t) + \hat{B}_i^{\prime\dagger}(t)\right]\mathrm{d}\omega\\
&\qquad + \sum_{iS}\sum_{n = 1}^{N}\int_0^\infty\hbar\tilde{V}_{Sin}(\nu)\left[\hat{c}_{Sn}(\nu,t) + \hat{c}_{Sn}^\dagger(\nu,t)\right]\left[\hat{B}'_i(t) + \hat{B}_i^{\prime\dagger}(t)\right]\mathrm{d}\nu\\
\end{split}
\end{equation}
where
\begin{equation}
\begin{split}
\tilde{\Lambda}_{Ei}(\omega) 
% &= \sum_{n = 1}^{2N}u'_{in}\left(\Lambda_{EEn}(\omega)\Theta_{N - n} + \Lambda_{ELn}(\omega)\Theta_{n - N - 1}\right)\\
&= \sum_{n = 1}^N\left(\Lambda_{EEn}(\omega)u'_{in} + \Lambda_{ELn}(\omega)u'_{i,n+N}\right),\\
\tilde{V}_{Ein}(\nu) &= V_{En}(\nu)u'_{in},\\
\tilde{V}_{Lin}(\nu) &= V_{Ln}(\nu)u'_{i,n + N}.
\end{split}
\end{equation}

We are now in a position to define our polariton operators. Using the logic of Appendix \ref{app:fanoDiagonalizationManyModes}, we can define polariton operators
\begin{equation}
\begin{split}
\hat{C}_\kappa(\omega,t) &= \sum_i\left[\frac{\omega + \Omega_i}{2\Omega_i}p_{\kappa i}(\omega)\hat{B}_i(\omega) + q_{\kappa i}(\omega)\hat{B}_i^\dagger(\omega)\right] + \int_0^\infty\left[r_{\kappa 0}(\omega,\nu)\hat{a}(\nu,t) + s_{\kappa 0}(\omega,\nu)\hat{a}^\dagger(\nu,t)\right]\mathrm{d}\nu\\
&\qquad + \sum_n\int_0^\infty\left[s_{\kappa n}(\omega,\nu)\hat{c}_n(\nu,t) + w_{\kappa n}(\omega,\nu)\hat{c}_n^\dagger(\nu,t)\right]\mathrm{d}\nu,\\
\hat{C}'_\kappa(\omega,t) &= \sum_i\left[\frac{\omega + \Omega_i'}{2\Omega_i'}x_{\kappa i}(\omega)\hat{B}'_i(\omega) + y_{\kappa i}(\omega)\hat{B}_i^{\prime\dagger}(\omega)\right] + \int_0^\infty\left[z_{\kappa 0}^E(\omega,\nu)\hat{a}_E(\nu,t) + w_{\kappa 0}^E(\omega,\nu)\hat{a}_E^\dagger(\nu,t)\right]\mathrm{d}\nu\\
&\qquad + \sum_{Sn}\int_0^\infty\left[z_{\kappa n}^S(\omega,\nu)\hat{c}_{Sn}(\nu,t) + w_{\kappa n}^S(\omega,\nu)\hat{c}_{Sn}^\dagger(\nu,t)\right]\mathrm{d}\nu,
\end{split}
\end{equation}
such that
\begin{equation}
\begin{split}
\hat{H}_M &= \sum_\kappa\int_0^\infty\hat{C}_\kappa^\dagger(\omega,t)\hat{C}_\kappa(\omega,t)\;\mathrm{d}\omega,\\
\hat{H}_{EL} &= \sum_\kappa\int_0^\infty\hat{C}_\kappa^{\prime\dagger}(\omega,t)\hat{C}'_\kappa(\omega,t)\;\mathrm{d}\omega.\\
\end{split}
\end{equation}
The expansion coefficients of the polariton operators are derived from the values of $p_{\kappa i}(\omega)$ and $x_{\kappa i}(\omega)$, which are themselves defined as the elements of eigenvectors
\begin{equation}
\begin{split}
\mathbf{p}_\kappa(\omega) &= 
\begin{pmatrix}
p_{\kappa 1}(\omega) & p_{\kappa 2}(\omega) & \cdots
\end{pmatrix}^\top,\\
\mathbf{x}_\kappa(\omega) &= 
\begin{pmatrix}
x_{\kappa 1}(\omega) & x_{\kappa 2}(\omega) & \cdots
\end{pmatrix}^\top.
\end{split}
\end{equation}
The first of these eigenvectors is defined such that (see Appendix \ref{app:fanoDiagonalizationManyModes} for details)
\begin{equation}
\begin{split}
&\left[\pi^2
\begin{pmatrix}
\tilde{\Lambda}_1^2(\omega) & \tilde{\Lambda}_2(\omega)\tilde{\Lambda}_1(\omega) & \cdots\\
\tilde{\Lambda}_1(\omega)\tilde{\Lambda}_2(\omega) & \tilde{\Lambda}_2^2(\omega) & \\
\vdots & & \ddots
\end{pmatrix}
+ \pi^2\sum_n
\begin{pmatrix}
\tilde{V}_{1n}^2(\omega) & \tilde{V}_{2n}(\omega)\tilde{V}_{1n}(\omega) & \cdots\\
\tilde{V}_{1n}(\omega)\tilde{V}_{2n}(\omega) & \tilde{V}_{2n}^2(\omega) & \\
\vdots & & \ddots
\end{pmatrix}
\right.\\
&\qquad + 
\begin{pmatrix}
|\psi_1(\omega)|^2\tilde{\Lambda}_1^2(\omega) & \psi_2^*(\omega)\psi_1(\omega)\tilde{\Lambda}_2(\omega)\tilde{\Lambda}_1(\omega) & \cdots\\
\psi_1^*(\omega)\psi_2(\omega)\tilde{\Lambda}_1(\omega)\tilde{\Lambda}_2(\omega) & |\psi_2(\omega)|^2\tilde{\Lambda}_2^2(\omega) & \\
\vdots & & \ddots
\end{pmatrix}
\\
&\left.\qquad + 
\begin{pmatrix}
|\psi_1(\omega)|^2\tilde{V}_{1n}^2(\omega) & \psi_2^*(\omega)\psi_1(\omega)\tilde{V}_{2n}(\omega)\tilde{V}_{1n}(\omega) & \cdots\\
\psi_1^*(\omega)\psi_2(\omega)\tilde{V}_{1n}(\omega)\tilde{V}_{2n}(\omega) & |\psi_2(\omega)|^2\tilde{V}_{2n}^2(\omega) & \\
\vdots & & \ddots
\end{pmatrix}\right]\mathbf{p}_\kappa(\omega)\\
% &\overset{.}{=} \sum_{ij}\left[\pi^2\left(\tilde{\Lambda}_i(\omega)\tilde{\Lambda}_j(\omega) + \sum_n\tilde{V}_{in}(\omega)\tilde{V}_{jn}(\omega)\right)\right.\\
% &\qquad\left. + \psi_i(\omega)\psi_j^*(\omega)\left(\tilde{\Lambda}_i(\omega)\tilde{\Lambda}_j(\omega) + \sum_n\tilde{V}_{in}(\omega)\tilde{V}_{jn}(\omega)\right)\right]\hat{\mathbf{e}}_i\hat{\mathbf{e}}_j\cdot\mathbf{p}_\kappa(\omega)\\
&\overset{.}{=} \sum_{ij}\left[\pi^2 + \psi_i(\omega)\psi_j^*(\omega)\right]\left(\tilde{\Lambda}_i(\omega)\tilde{\Lambda}_j(\omega) + \sum_n\tilde{V}_{in}(\omega)\tilde{V}_{jn}(\omega)\right)\hat{\mathbf{e}}_i\hat{\mathbf{e}}_j\cdot\mathbf{p}_\kappa(\omega)\\
&= \alpha_\kappa(\omega)\mathbf{p}_\kappa(\omega),
\end{split}
\end{equation}
wherein $\alpha_\kappa(\omega)$ is a real eigenvalue that serves only as a normalization factor and
\begin{equation}
\psi_i(\omega) = \mathrm{i}\pi + \frac{\omega^2 - \Omega_i^2Z_i(\omega)}{2\Omega_i\Sigma_i^2(\omega)}
\end{equation}
is a complex regularization function defined by
\begin{equation}
\begin{split}
Z_i(\omega) &= 1 + \lim_{\epsilon\to0}\frac{2}{\Omega_i}\int_{-\infty}^\infty\frac{\Sigma_i^2(\nu')}{\omega - \nu' - \mathrm{i}\epsilon}\;\mathrm{d}\nu',\\
\Sigma_i(\omega) &= \sqrt{\tilde{\Lambda}_i^2(\omega) + \sum_n\tilde{V}_{in}^2(\omega)}.
\end{split}
\end{equation}
The second is defined similarly, with
\begin{equation}
\begin{split}
\sum_{ij}\left[\pi^2 + \psi_i'(\omega)\psi_j^{\prime*}(\omega)\right]\left(\tilde{\Lambda}_{Ei}(\omega)\tilde{\Lambda}_{Ej}(\omega) + \sum_{Sn}\tilde{V}_{Sin}(\omega)\tilde{V}_{Sjn}(\omega)\right)\hat{\mathbf{e}}_i\hat{\mathbf{e}}_j\cdot\mathbf{x}_\kappa(\omega) &= \alpha'_\kappa(\omega)\mathbf{x}_\kappa(\omega)
\end{split}
\end{equation}
and
\begin{equation}
\begin{split}
\psi_i'(\omega) &= \mathrm{i}\pi + \frac{\omega^2 - \Omega_i'^2Z_i'(\omega)}{2\Omega_i'\Sigma_i^{\prime2}(\omega)},\\
Z_i'(\omega) &= 1 + \lim_{\epsilon\to0}\frac{2}{\Omega_i'}\int_{-\infty}^\infty\frac{\Sigma_i^{\prime2}(\nu')}{\omega - \nu' - \mathrm{i}\epsilon}\;\mathrm{d}\nu',\\
\Sigma_i'(\omega) &= \sqrt{\tilde{\Lambda}_{Ei}^2(\omega) + \sum_{Sn}\tilde{V}_{Sin}^2(\omega)}.
\end{split}
\end{equation}
The rest of the expansion coefficients follow straightforwardly:
\begin{equation}
\begin{split}
q_{\kappa i}(\omega) &= \frac{\omega - \Omega_i}{2\Omega_i}p_{\kappa i}(\omega),\\
r_{\kappa n}(\omega,\nu) &= \sum_ip_{\kappa i}(\omega)\left[PV\left\{\frac{1}{\omega - \nu}\right\} + \psi_i(\omega)\delta(\omega - \nu)\right]
\left(\tilde{\Lambda}_i(\omega)\delta_{n0} + \tilde{V}_{in}(\omega)(1 - \delta_{n0})\right),\\
s_{\kappa n}(\omega,\nu) &= \sum_i\frac{p_{\kappa i}(\omega)}{\omega + \nu}\left(\tilde{\Lambda}_i(\omega)\delta_{n0} + \tilde{V}_{in}(\omega)(1 - \delta_{n0})\right),
\end{split}
\end{equation}
and
\begin{equation}
\begin{split}
y_{\kappa i}(\omega) &= \frac{\omega - \Omega_i'}{2\Omega_i'}x_{\kappa i}(\omega),\\
z_{\kappa i}^S(\omega,\nu) &= \sum_{i}x_{\kappa i}(\omega)\left[PV\left\{\frac{1}{\omega - \nu}\right\} + \psi_i'(\omega)\delta(\omega - \nu)\right]\left(\tilde{\Lambda}_{Ei}(\omega)\delta_{n0}\delta_{SE} + \tilde{V}_{Sin}(\omega)(1 - \delta_{n0})\right),\\
w_{\kappa i}^S(\omega,\nu) &= \sum_i\frac{x_{\kappa i}(\omega)}{\omega + \nu}\left(\tilde{\Lambda}_{Ei}(\omega)\delta_{n0}\delta_{SE} + \tilde{V}_{Sin}(\omega)(1 - \delta_{n0})\right).
\end{split}
\end{equation}





 






















\newpage
% 


























%%%%%%%%%%%%%%%%%%%%%%%%%%%%%%%% Appendix %%%%%%%%%%%%%%%%%%%%%%%%%%%%%%%%%%

\newpage
\appendix
\numberwithin{equation}{section}

\section{Unit Conventions}\label{app:unitConventions}

The chosen dimensions and units for each important quantity are shown below.

\begin{center}
\def\arraystretch{1.5}
\begin{longtable}{c|c|c}
    Symbol Name & Dimensions & Units (cgs Gaussian) \\
    \hline
    $\mathcal{L}$ & $\frac{[U]}{[L]^3} = \frac{[M]}{[L][T^2]}$ & $\frac{\mathrm{erg}}{\mathrm{cm}^3}$\\
    $L$, $H$ & $[U]$ & erg\\
    $\mathcal{A}_{\bm{\alpha}}(\omega,t)$ & $[L][T]^\frac{1}{2}$ & $\mathrm{cm}\cdot\mathrm{s}^\frac{1}{2}$\\
    $\mathcal{B}_{\bm{\beta}n}(t)$ & $[L]$ & cm\\
    $\mathcal{C}_{\bm{\gamma}n}(\nu,t)$ & $[L][T]^\frac{1}{2}$ & $\mathrm{cm}\cdot\mathrm{s}^\frac{1}{2}$\\
    $\mathit{\Pi}_{\bm{\alpha}}(\omega,t)$ & $\frac{[L][M]}{[T]^\frac{1}{2}}$ & $\frac{\mathrm{cm}\cdot\mathrm{g}}{\mathrm{s}^\frac{1}{2}}$\\
    $\mathcal{P}_{\bm{\beta}n}(t)$ & $\frac{[M][L]}{[T]}$ & $\frac{\mathrm{cm}\cdot\mathrm{cm}}{\mathrm{s}}$\\
    $\mathcal{Q}_{\bm{\gamma}n}(\nu,t)$ & $\frac{[L][M]}{[T]^\frac{1}{2}}$ & $\frac{\mathrm{cm}\cdot\mathrm{g}}{\mathrm{s}^\frac{1}{2}}$\\
    $g_{\bm{\beta}nn}$, $\tilde{g}_{\bm{\beta}\bm{\beta}'nn'}$ & $\frac{[M]}{[T]^2}$ & $\frac{\mathrm{g}}{\mathrm{s}^2}$\\
    $\lambda_{\bm{\alpha}\bm{\beta}n}(\omega)$ & $\frac{[M]}{[T]^\frac{1}{2}}$ & $\frac{\mathrm{g}}{\mathrm{s}^\frac{1}{2}}$\\
    $v(\nu)$, $v_{\bm{\beta}\nu}(\nu)$ & $\frac{[M]}{[T]^\frac{1}{2}}$ & $\frac{\mathrm{g}}{\mathrm{s}^\frac{1}{2}}$ \\
    $\hat{a}_{\bm{\alpha}}(\omega,t)$ & $[T]^\frac{1}{2}$ & s$^\frac{1}{2}$ \\
    $\hat{b}_{\bm{\beta}n}(t)$ & 1 & 1\\
    $\hat{c}_{\bm{\gamma}n}(\nu,t)$ & $[T]^\frac{1}{2}$ & s$^\frac{1}{2}$\\
    $\sigma_{\bm{\beta}\bm{\beta}nn'}$ & $\frac{1}{[T]}$ & s$^{-1}$\\
    $\Lambda_{\bm{\alpha}\bm{\beta}n}(\omega)$, $\tilde{\Lambda}_{\bm{\alpha}\bm{\beta}n}(\omega)$ & $\frac{1}{[T]^\frac{1}{2}}$ & s$^{-\frac{1}{2}}$\\
    $V_{\bm{\beta}n}(\nu)$, $\tilde{V}_{\bm{\beta}n}(\nu)$ & $\frac{1}{[T]^\frac{1}{2}}$ & s$^{-\frac{1}{2}}$\\
\end{longtable}
\end{center}



%%%%%%%%%%%%%%%%%%%%%% Import Standalone Appendices %%%%%%%%%%%%%%%%%%%%%%%%


%%%%%%%%%%%%%%%%%%%%%%%%%%%%%%%%%%%%%%%%%%%%%%%%%%%%%%%%%%%%%%%%%%%%%%%%%%%%
% Vector Spherical Harmonic Algebra
%%%%%%%%%%%%%%%%%%%%%%%%%%%%%%%%%%%%%%%%%%%%%%%%%%%%%%%%%%%%%%%%%%%%%%%%%%%%
%!TEX root = finiteMQED.tex

\section{Harmonic Definitions}\label{app:harmonicAlgebra}

The dyadic Green's function of a sphere is composed of transverse vector spherical harmonics
\begin{equation}
\begin{split}
\mathbf{M}_{p\ell m}(\mathbf{r},k) &= \sqrt{K_{\ell m}}\left[\frac{(-1)^{p + 1}m}{\sin\theta}j_\ell(kr)P_{\ell m}(\cos\theta)S_{p+1}(m\phi)\hat{\bm{\theta}} - j_\ell(kr)\frac{\partial P_{\ell m}(\cos\theta)}{\partial\theta}S_p(m\phi)\hat{\bm{\phi}}\right],\\[0.5em]
\mathbf{N}_{p\ell m}(\mathbf{r},k) &= \sqrt{K_{\ell m}}\left[\frac{\ell(\ell + 1)}{kr}j_\ell(kr)P_{\ell m}(\cos\theta)S_p(m\phi)\hat{\mathbf{r}}\right.\\
&+ \left.\frac{1}{kr}\frac{\partial\{rj_\ell(kr)\}}{\partial r}\left(\frac{\partial P_{\ell m}(\cos\theta)}{\partial\theta}S_p(m\phi)\hat{\bm{\theta}} + \frac{(-1)^{p+1}m}{\sin\theta}P_{\ell m}(\cos\theta)S_{p + 1}(m\phi)\hat{\bm{\phi}}\right)\right],
\end{split}
\end{equation}
which each have zero divergence, i.e. $\nabla\cdot\mathbf{M}_{p\ell m}(\mathbf{r},k) = \nabla\cdot\mathbf{N}_{p\ell m}(\mathbf{r},k) = 0$, and longitudinal vector spherical harmonics
\begin{equation}
\begin{split}
\mathbf{L}_{p\ell m}(\mathbf{r},k) &= \frac{1}{k}\nabla\left\{\sqrt{K_{\ell m}}\sqrt{\ell(\ell + 1)}j_\ell(kr)P_{\ell m}(\cos\theta)S_p(m\phi)\right\}\\
&= \sqrt{K_{\ell m}}\sqrt{\ell(\ell + 1)}\left[\frac{1}{k}\frac{\partial\{j_\ell(kr)\}}{\partial r}P_{\ell m}(\cos\theta)S_p(m\phi)\hat{\mathbf{r}} + \frac{1}{kr}j_\ell(kr)\frac{\partial P_{\ell m}(\cos\theta)}{\partial\theta}S_p(m\phi)\hat{\bm{\theta}}\right.\\
&\left. + \frac{(-1)^{p+1}m}{kr\sin\theta}j_\ell(kr)P_{\ell m}(\cos\theta)S_{p+1}(m\phi)\hat{\bm{\phi}}\right]
\end{split}
\end{equation}
which obey $\nabla\times\mathbf{L}_{p\ell m}(\mathbf{r},k) = 0$. Here $j_\ell(x)$ are spherical Bessel functions with $\ell = 1,2,3,\ldots,\infty$; $P_{\ell m}(x)$ are associated Legendre polynomials with the same range in $\ell$; $m = 0,1,2,\ldots,\ell$; and 
\begin{equation}
S_p(mx) = 
\begin{cases}
\cos(mx), & p\;\;\mathrm{even},\\
\sin(mx), & p \;\;\mathrm{odd}
\end{cases}
\end{equation}
and the same range in $m$. Due to the binary nature of $p$, we will in general need to use special notation to assert that $S_p(x) = S_{p + 2}(x) = S_{p+4}(x) = \cdots$. However, to save space, we will use the convention that any nonzero even value of $p$ is automatically replaced by 0 and any non-one odd value of $p$ is automatically replaced by 1 in what follows. Further, the prefactors $K_{\ell m}$, which are set to $1$ in most texts, are set to
\begin{equation}
K_{\ell m} = (2 - \delta_{m0})\frac{2\ell + 1}{\ell(\ell + 1)}\frac{(\ell - m)!}{(\ell + m)!}
\end{equation}
in an effort to regularize the transverse harmonics, i.e. ensure that their maximum magnitudes in $\mathbf{r}$ for fixed $k$ are similar for all index combinations $T,p,\ell,m$. The prefactors of the Laplacian harmonics are defined similarly, but with an added factor of $\delta_{TE}$. Here, $T$ is a fourth ``type'' index, similar to the parity ($p$), order ($\ell$), and degree ($m$) indices, that takes values of $M$ (magnetic type) or $E$ (electric type) to describe whether the characteristic field profiles of each mode resemble those of magnetic or electric multipoles. Only the electric longitudinal modes $\mathbf{L}_{p\ell m}(\mathbf{r})$ are nonzero such that we won't bother to invent a second symbol for magnetic longitudinal modes, but the transverse harmonics clearly come in two different types: magnetic modes $\mathbf{M}_{p\ell m}(\mathbf{r},k)$ ($T = M$) and electric modes $\mathbf{N}_{p\ell m}(\mathbf{r},k)$ ($T = E$).

Both the transverse and Laplacian modes also come with a fifth modifier, the ``region'' index, taking values of $<$ for \textit{interior} harmonics and $>$ for \textit{exterior} harmonics, although we have neglected append our definitions with yet another symbol. The interior harmonics, listed above as $\mathbf{L}$, $\mathbf{M}$, and $\mathbf{N}$, are generally used for $\mathbf{r}$ confined within some closed spherical surface such that $r<\infty$ always and $r = 0$ somwhere within the region. To describe fields in regions that do not include the origin, one can find the second set of solutions to the second-order wave equation PDE, which are the so-called \textit{exterior} vector spherical harmonics $\bm{\mathcal{L}}_{p\ell m}(\mathbf{r})$, $\bm{\mathcal{M}}_{p\ell m}(\mathbf{r},k)$, and $\bm{\mathcal{N}}_{p\ell m}(\mathbf{r},k)$. These harmonics are defined with the simple substitutions $j_\ell(kr)\to h_\ell(kr)$ and are explicitly given by
\begin{equation}
\begin{split}
\bm{\mathcal{L}}_{p\ell m}(\mathbf{r},k) &= \frac{1}{k}\nabla\left\{\sqrt{K_{\ell m}}\sqrt{\ell(\ell + 1)}h_\ell(kr)P_{\ell m}(\cos\theta)S_p(m\phi)\right\},\\
&= \sqrt{K_{\ell m}}\sqrt{\ell(\ell + 1)}\left[\frac{1}{k}\frac{\partial\{h_\ell(kr)\}}{\partial r}P_{\ell m}(\cos\theta)S_p(m\phi)\hat{\mathbf{r}} + \frac{1}{kr}h_\ell(kr)\frac{\partial P_{\ell m}(\cos\theta)}{\partial\theta}S_p(m\phi)\hat{\bm{\theta}}\right.\\
&\left. + \frac{(-1)^{p+1}m}{kr\sin\theta}h_\ell(kr)P_{\ell m}(\cos\theta)S_{p+1}(m\phi)\hat{\bm{\phi}}\right],\\[1.0em]
\bm{\mathcal{M}}_{p\ell m}(\mathbf{r},k) &= \sqrt{K_{\ell m}}\left[\frac{(-1)^{p + 1}m}{\sin\theta}h_\ell(kr)P_{\ell m}(\cos\theta)S_{p+1}(m\phi)\hat{\bm{\theta}} - h_\ell(kr)\frac{\partial P_{\ell m}(\cos\theta)}{\partial\theta}S_p(m\phi)\hat{\bm{\phi}}\right],\\[0.5em]
\bm{\mathcal{N}}_{p\ell m}(\mathbf{r},k) &= \sqrt{K_{\ell m}}\left[\frac{\ell(\ell + 1)}{kr}h_\ell(kr)P_{\ell m}(\cos\theta)S_p(m\phi)\hat{\mathbf{r}}\right.\\
&+ \left.\frac{1}{kr}\frac{\partial\{rh_\ell(kr)\}}{\partial r}\left(\frac{\partial P_{\ell m}(\cos\theta)}{\partial\theta}S_p(m\phi)\hat{\bm{\theta}} + \frac{(-1)^{p+1}m}{\sin\theta}P_{\ell m}(\cos\theta)S_{p + 1}(m\phi)\hat{\bm{\phi}}\right)\right].
\end{split}
\end{equation}
The functions $h_\ell(kr) = j_\ell(kr) + \mathrm{i}y_\ell(kr)$ are the spherical Hankel functions of the first kind and involve the spherical Bessel functions of both the first kind ($j_\ell[kr]$) and the second kind, $y_\ell(kr)$, the latter of which are irregular at $r\to0$, justifying their exclusion from use in all regions that include the point $r = 0$.

We would like the notation for the harmonics to be as succinct as possible. Therefore, for shorthand, we can define
\begin{equation}
\begin{split}
\mathbf{X}_{Tp\ell m}(\mathbf{r},k) &= 
\begin{cases}
\mathbf{L}_{p\ell m}(\mathbf{r},k), & T=L,\\
\mathbf{M}_{p\ell m}(\mathbf{r},k), & T=M,\\
\mathbf{N}_{p\ell m}(\mathbf{r},k), & T=E;
\end{cases}\\[1.0em]
\bm{\mathcal{X}}_{Tp\ell m}(\mathbf{r},k) &=
\begin{cases}
\bm{\mathcal{L}}_{p\ell m}(\mathbf{r},k), & T=L,\\
\bm{\mathcal{M}}_{p\ell m}(\mathbf{r},k), & T=M,\\
\bm{\mathcal{N}}_{p\ell m}(\mathbf{r},k), & T=E.
\end{cases}
\end{split}
\end{equation}
Here, we have finally introduced the fourth ``type'' index to the harmonic notation, with $L$, $M$, and $E$ denoting the $L$ongitudinal, transverse $M$agnetic, and transverse $E$lectric harmonics, respectively. We can condense the notation further by writing the string of four indices as $\bm{\alpha} = (T,p,\ell,m)$, such that our harmonics can now be written as $\mathbf{X}_{\bm{\alpha}}(\mathbf{r},k)$ and $\bm{\mathcal{X}}_{\bm{\alpha}}(\mathbf{r},k)$.

The symmetries of the vector spherical harmonics are such that
\begin{equation}
\begin{split}
\mathbf{L}_{p\ell m}(\mathbf{r},k) &= (-1)^{\ell+1}\mathbf{L}_{p\ell m}(\mathbf{r},-k),\\
\mathbf{M}_{p\ell m}^*(\mathbf{r},k) &= (-1)^\ell\mathbf{M}_{p\ell m}(\mathbf{r},-k),\\
\mathbf{N}_{p\ell m}^*(\mathbf{r},k) &= (-1)^{\ell + 1}\mathbf{N}_{p\ell m}(\mathbf{r},-k),
\end{split}
\end{equation}
with identical relations for the respective exterior harmonics. Further, the transverse harmonics obey the duality
\begin{equation}
\begin{split}
\nabla\times\mathbf{M}_{p\ell m}(\mathbf{r},k) &= k\mathbf{N}_{p\ell m}(\mathbf{r},k),\\
\nabla\times\mathbf{N}_{p\ell m}(\mathbf{r},k) &= k\mathbf{M}_{p\ell m}(\mathbf{r},k),
\end{split}
\end{equation}
and similar for the exterior transverse harmonics, such that a second set of harmonics can be defined as 
\begin{equation}
\begin{split}
\mathbf{Y}_{Lp\ell m}(\mathbf{r},t) &= 0,\\
\mathbf{Y}_{Mp\ell m}(\mathbf{r},k) &= \mathbf{N}_{p\ell m}(\mathbf{r},t),\\
\mathbf{Y}_{Ep\ell m}(\mathbf{r},k) &= \mathbf{M}_{p\ell m}(\mathbf{r},t).
\end{split}
\end{equation}
More succinctly,
\begin{equation}
\begin{split}
\mathbf{Y}_{\bm{\alpha}}(\mathbf{r},k) &= \frac{1}{k}\nabla\times\mathbf{X}_{\bm{\alpha}}(\mathbf{r},k),\\
\bm{\mathcal{Y}}_{\bm{\alpha}}(\mathbf{r},k) &= \frac{1}{k}\nabla\times\bm{\mathcal{X}}_{\bm{\alpha}}(\mathbf{r},k),
\end{split}
\end{equation}
where $\bm{\mathcal{Y}}(\mathbf{r},k)$ are the corresponding exterior harmonics to $\mathbf{Y}_{\bm{\alpha}}(\mathbf{r},k)$.

Due to their regularity at all points $\mathbf{r}$, the interior harmonics obey the orthogonality condition
\begin{equation}\label{eq:vectorSphericalHarmonicOrthogonality}
\int \mathbf{X}_{\bm{\alpha}}(\mathbf{r},k)\cdot\mathbf{X}_{\bm{\alpha}'}(\mathbf{r},k')\;\mathrm{d}^3\mathbf{r} = \frac{2\pi^2}{k^2}\delta(k - k')(1 - \delta_{p1}\delta_{m0})\delta_{TT'}\delta_{pp'}\delta_{\ell\ell'}\delta_{mm'}\\
\end{equation}
for $k,k' > 0$ and integration across the entire universe. The transverse magnetic harmonics also satisfy convenient orthogonality relations when integrated over finite regions in $r$, with
\begin{equation}\label{eq:vectorSphericalHarmonicOrthogonalityConditionFinite}
\begin{split}
\int_0^{2\pi}\int_0^\pi\int_a^b\mathbf{L}_{p\ell m}(\mathbf{r},k)\cdot\mathbf{M}_{p'\ell'm'}(\mathbf{r},k')r^2\sin\theta\;\mathrm{d}r\,\mathrm{d}\theta\,\mathrm{d}\phi &= 0,\\
\int_0^{2\pi}\int_0^\pi\int_a^b\mathbf{N}_{p\ell m}(\mathbf{r},k)\cdot\mathbf{M}_{p'\ell'm'}(\mathbf{r},k')r^2\sin\theta\;\mathrm{d}r\,\mathrm{d}\theta\,\mathrm{d}\phi &= 0.
\end{split}
\end{equation}
However, finite spherical integrals of the longitudinal and transverse electric harmonics produce
\begin{equation}\label{eq:vectorSphericalHarmonicOrthogonalityConditionFiniteMixed}
\begin{split}
\int_0^{2\pi}\int_0^{\pi}\int_a^b\mathbf{L}_{p\ell m}(\mathbf{r},k)\cdot\mathbf{N}_{p'\ell'm'}(\mathbf{r},k')r^2\sin\theta\;\mathrm{d}r\,\mathrm{d}\theta\,\mathrm{d}\phi &= 4\pi(1 - \delta_{p1}\delta_{m0})\sqrt{\ell(\ell + 1)}\delta_{pp'}\delta_{\ell\ell'}\delta_{mm'}\\[-0.5em]
&\qquad\times\frac{1}{kk'}\left[bj_\ell(kb)j_\ell(k'b) - aj_\ell(ka)j_\ell(k'a)\right],
\end{split}
\end{equation}
which can be seen to produce zero in the limits $a\to0$, $b\to\infty$ as $\lim_{r\to0}rj_\ell(kr)j_\ell(k'r) = \lim_{r\to\infty}rj_\ell(kr)j_\ell(k'r) = 0$ for all real, positive $k$ and $k'$ if $\ell$ is an integer greater than 0. Finally, the harmonics of the same type are normalized such that integrals over finite spherical regions produce
\begin{equation}
\int_0^{2\pi}\int_0^\pi\int_a^b\mathbf{X}_{Tp\ell m}(\mathbf{r},k)\cdot\mathbf{X}_{Tp'\ell'm'}(\mathbf{r},k')r^2\sin\theta\;\mathrm{d}r\,\mathrm{d}\theta\,\mathrm{d}\phi = 4\pi(1 - \delta_{p1}\delta_{m0})\delta_{pp'}\delta_{\ell\ell'}\delta_{mm'}R_{T\ell}^{\ll}(k,k';a,b),
\end{equation}
where $R_{T\ell}^\ll(k,k';a,b)$ are radial integral functions defined by
\begin{equation}\label{eq:finiteRadialIntegrals}
\begin{split}
R_{M\ell}^\ll(k,k';a,b) &= \int_a^bj_\ell(kr)j_\ell(k'r)r^2\;\mathrm{d}r\\
&= \left.\frac{r^2}{k^2 - k'^2}\left[k'j_{\ell - 1}(k'r)j_\ell(kr) - kj_{\ell - 1}(kr)j_\ell(k'r)\right]\right|_a^b,\\[0.5em]
R_{E\ell}^\ll(k,k';a,b) &= \frac{\ell + 1}{2\ell + 1}R_{M,\ell - 1}^\ll(k,k';a,b) + \frac{\ell}{2\ell + 1}R_{M,\ell + 1}^\ll(k,k';a,b),\\
R_{L\ell}^\ll(k,k';a,b) &= \frac{\ell + 1}{2\ell + 1}R_{M,\ell + 1}^\ll(k,k';a,b) + \frac{\ell}{2\ell + 1}R_{M,\ell - 1}^\ll(k,k';a,b).
\end{split}
\end{equation}
The superscript $\ll$ indicates that the integral involves two spherical Bessel functions of the first kind and thus arises from the integration of two interior harmonics. 

An important case of this integral occurs when $b\to a$ and $a\to0$, i.e. during integration over a finite sphere. In this case, the characteristic wavenumbers are often defined such that $k = z_{\ell n}/a$ and $k' = z_{\ell n'}/a$, where $z_{\ell n}$ is the $n^\mathrm{th}$ unitless root of $j_\ell(x)$ with $n = 1,2,\ldots$, or as $k = w_{\ell n}/a$ and $k' = w_{\ell n'}/a$, where $w_{\ell n}$ is the $n^\mathrm{th}$ unitless root of $\partial j_\ell(x)/\partial x$ with $n = 1,2,\ldots$. With these values of $k$ and $k'$ fixed, we can use the identity $j_{\ell - 1}(w_{\ell n}) = (\ell + 1)j_\ell(w_{\ell n})/w_{\ell n}$ to see that $R_{M\ell}^\ll(z_{\ell n}/a,z_{\ell n'}/a;0,a) = R_{M\ell}^\ll(w_{\ell n}/a,w_{\ell n'}/a;0,a) = 0$ for $n\neq n'$. Further, with
\begin{equation}
\lim_{k'\to k}R_{M\ell}^\ll(k,k';0,a) = \frac{a^3}{2}\left[j_\ell^2(ka) - j_{\ell - 1}(ka)j_{\ell + 1}(ka)\right]
\end{equation}
by L'Hopital's rule, we can use the identity $j_{\ell - 1}(z_{n\ell}) = -j_{\ell + 1}(z_{n\ell})$ to say
\begin{equation}
R_{M\ell}^\ll\left(\frac{z_{\ell n}}{a},\frac{z_{\ell n'}}{a};0,a\right) = R_{M,\ell-1}^\ll\left(\frac{z_{\ell n}}{a},\frac{z_{\ell n'}}{a};0,a\right) = R_{M,\ell+1}^\ll\left(\frac{z_{\ell n}}{a},\frac{z_{\ell n'}}{a};0,a\right) = \delta_{nn'}\frac{a^3}{2}j_{\ell + 1}^2(z_{\ell n}).
\end{equation}
Further, we can use the identities $j_{\ell + 1}(w_{\ell n}) = \ell j_\ell(w_{\ell n})/w_{\ell n}$, $j_{\ell - 2}(x) = (2\ell - 1)j_{\ell - 1}(x)/x - j_\ell(x)$, and $j_{\ell + 2}(x) = (2\ell + 3)j_{\ell + 1}(x)/x - j_\ell(x)$ to say
\begin{equation}
\begin{split}
R_{M\ell}^\ll\left(\frac{w_{\ell n}}{a},\frac{w_{\ell n'}}{a};0,a\right) &= \delta_{nn'}\frac{a^3}{2}j_\ell^2(w_{\ell n})\left(1 - \frac{\ell(\ell + 1)}{w_{\ell n}^2}\right),\\
R_{M,\ell - 1}^\ll\left(\frac{w_{\ell n}}{a},\frac{w_{\ell n'}}{a};0,a\right) &= \delta_{nn'}\frac{a^3}{2}j_{\ell}^2(w_{\ell n})\left(1 - \frac{\ell^2 - \ell-2}{w_{\ell n}^2}\right),\\
R_{M,\ell - 1}^\ll\left(\frac{w_{\ell n}}{a},\frac{w_{\ell n'}}{a};0,a\right) &= \delta_{nn'}\frac{a^3}{2}j_{\ell}^2(w_{\ell n})\left(1 - \frac{\ell(\ell + 3)}{w_{\ell n}^2}\right).
\end{split}
\end{equation}
These expressions can be used to construct the useful identities
\begin{equation}
\begin{split}
R_{E\ell}^\ll\left(\frac{z_{\ell n}}{a},\frac{z_{\ell n'}}{a};0,a\right) &= \delta_{nn'}\frac{a^3}{2}j_{\ell + 1}^2(z_{\ell n}),\\[0.5em]
R_{L\ell}^\ll\left(\frac{w_{\ell n}}{a},\frac{w_{\ell n'}}{a};0,a\right) &= \delta_{nn'}\frac{a^3}{2}\left(1 - \frac{\ell(\ell + 1)}{w_{\ell n}^2}\right)j_\ell^2(w_{\ell n}).
\end{split}
\end{equation}
Therefore, letting $k_{Mp\ell mn} = k_{Ep\ell mn} = z_{\ell n}/a$ and $k_{Lp\ell m n} = w_{\ell n}/a$, we have
\begin{equation}
\begin{split}
\int_0^{2\pi}\int_0^\pi\int_0^a\mathbf{X}_{\bm{\alpha}}(\mathbf{r},k_{\bm{\alpha}n})&\cdot\mathbf{X}_{\bm{\alpha}'}(\mathbf{r},k_{\bm{\alpha}'n'})r^2\sin\theta\;\mathrm{d}r\;\mathrm{d}\theta\;\mathrm{d}\phi = 2\pi a^3(1 - \delta_{p1}\delta_{m0})\delta_{\bm{\alpha}\bm{\alpha}'}\delta_{nn'}\\
&\qquad\times\left[(\delta_{TM} + \delta_{TE})j_{\ell + 1}^2(z_{\ell n}) + \delta_{TL}\left(1 - \frac{\ell(\ell + 1)}{w_{\ell n}^2}\right)j_{\ell}^2(w_{\ell n})\right].
\end{split}
\end{equation}
In related fashion, using the identity
\begin{equation}
\nabla\cdot\mathbf{L}_{p\ell m}(\mathbf{r},k) = -k\sqrt{K_{\ell m}}\sqrt{\ell(\ell + 1)}j_\ell(kr)P_{\ell m}(\cos\theta)S_p(m\phi),
\end{equation}
we have
\begin{equation}
\begin{split}
\int_0^{2\pi}\int_0^\pi\int_0^a&\left[\nabla\cdot\mathbf{L}_{p\ell m}(\mathbf{r},k_{L\ell n})\right]\left[\nabla\cdot\mathbf{L}_{p'\ell'm'}(\mathbf{r},k_{L\ell'n'})\right]r^2\sin\theta\;\mathrm{d}r\;\mathrm{d}\theta\;\mathrm{d}\phi\\
&= 2\pi a^3(1 - \delta_{p1}\delta_{m0})\delta_{pp'}\delta_{\ell\ell'}\delta_{mm'}\delta_{nn'}k_{L\ell n}^2j_\ell^2(k_{L\ell n}a)\left(1 - \frac{\ell(\ell + 1)}{(k_{L\ell n}a)^2}\right).
\end{split}
\end{equation}

In addition to our vector spherical harmonics, the expansion of the Coulomb Green's function
\begin{equation}
1/|\mathbf{r} - \mathbf{r}'| = \sum_{p\ell m}[f_{p\ell m}^>(\mathbf{r})f_{p\ell m}^<(\mathbf{r}')\Theta(r - r') + f_{p\ell m}^<(\mathbf{r})f_{p\ell m}^>(\mathbf{r}')\Theta(r' - r)]
\end{equation}
into \textit{scalar} spherical harmonics
\begin{equation}\label{eq:scalarHarmonics}
f_{p\ell m}^i(\mathbf{r}) = \sqrt{K_{\ell m}}\sqrt{\frac{\ell(\ell + 1)}{2\ell + 1}}P_{\ell m}(\cos\theta)S_p(m\phi)\times
\begin{cases}
r^\ell, & i = \, <,\\
\dfrac{1}{r^{\ell + 1}}, & i = \,>
\end{cases}
\end{equation}
motivates the calculation of a further set of integral identities. In particular, the radial integrals
\begin{equation}
\begin{split}
\int_a^b j_\ell(kr)r^{\ell+2}\;\mathrm{d}r &= \frac{b^{\ell + 2}}{k}j_{\ell + 1}(kb) - \frac{a^{\ell + 2}}{k}j_{\ell + 1}(ka),\\
\int_a^b j_\ell(kr)r^{-\ell + 1}\;\mathrm{d}r &=  \frac{1}{ka^{\ell - 1}}j_{\ell - 1}(ka) - \frac{1}{kb^{\ell -1}}j_{\ell - 1}(kb).\\
\end{split}
\end{equation}
are useful to define, as are the volume integrals
\begin{equation}
\int_{r<a}\mathbf{X}_{\bm{\beta}}(\mathbf{r},k_{\bm{\beta}n})\cdot\nabla f_{p'\ell'm'}^<(\mathbf{r})\;\mathrm{d}^3\mathbf{r} = 4\pi\delta_{TL}(1 - \delta_{p1}\delta_{m0})\delta_{pp'}\delta_{\ell\ell'}\delta_{mm'}\frac{a^{\ell + 2}}{\sqrt{2\ell + 1}}j_{\ell + 1}(w_{\ell n}).
\end{equation}
and
\begin{equation}\label{eq:laplacianHarmonicOrthogonality}
\begin{split}
\int_0^{2\pi}\int_0^\pi\int_a^b\nabla f_{p\ell m}^<(\mathbf{r})\cdot\nabla f_{p'\ell'm'}^<(\mathbf{r})r^2\sin\theta\;\mathrm{d}r\,\mathrm{d}\theta\,\mathrm{d}\phi &= 4\pi\frac{\ell}{2\ell + 1}\left(b^{2\ell + 1} - a^{2\ell + 1}\right)\delta_{pp'}\delta_{\ell\ell'}\delta_{mm'},\\
\int_0^{2\pi}\int_0^\pi\int_a^b\nabla f_{p\ell m}^>(\mathbf{r})\cdot\nabla f_{p'\ell'm'}^>(\mathbf{r})r^2\sin\theta\;\mathrm{d}r\,\mathrm{d}\theta\,\mathrm{d}\phi &= -4\pi\frac{\ell + 1}{2\ell + 1}\left(b^{-2\ell - 1} - a^{-2\ell - 1}\right)\delta_{pp'}\delta_{\ell\ell'}\delta_{mm'},\\
\int_0^{2\pi}\int_0^\pi\int_a^b\nabla f_{p\ell m}^<(\mathbf{r})\cdot\nabla f_{p'\ell'm'}^>(\mathbf{r})r^2\sin\theta\;\mathrm{d}r\,\mathrm{d}\theta\,\mathrm{d}\phi &= 0.
\end{split}
\end{equation}










% \noindent\rule{\textwidth}{0.5pt}

% \begin{equation}
% \begin{split}
% \mathbf{Y}_{\bm{\alpha}}(\mathbf{r},k) &= \frac{1}{k}\nabla\times\mathbf{X}_{\bm{\alpha}}(\mathbf{r},k),\\
% \bm{\mathcal{Y}}_{\bm{\alpha}}(\mathbf{r},k) &= \frac{1}{k}\nabla\times\bm{\mathcal{X}}_{\bm{\alpha}}(\mathbf{r},k).
% \end{split}
% \end{equation}
% These obey the same conjugation symmetry as $\mathbf{X}_{\bm{\alpha}}$ and $\bm{\mathcal{X}}_{\bm{\alpha}}$ but with an extra factor of $-1$ such that
% \begin{equation}
% \begin{split}
% \bm{\mathcal{X}}_{\bm{\alpha}}^*(\mathbf{r},k) &= 
% \begin{cases}
% (-1)^{\ell}\bm{\mathcal{X}}_{\bm{\alpha}}(\mathbf{r},-k), & T = M,\\
% (-1)^{\ell + 1}\bm{\mathcal{X}}_{\bm{\alpha}}(\mathbf{r},-k), & T = E;\\
% \end{cases}\\
% \bm{\mathcal{Y}}_{\bm{\alpha}}^*(\mathbf{r},k) &= 
% \begin{cases}
% (-1)^{\ell + 1}\bm{\mathcal{Y}}_{\bm{\alpha}}(\mathbf{r},-k), & T = M,\\
% (-1)^{\ell}\bm{\mathcal{Y}}_{\bm{\alpha}}(\mathbf{r},-k), & T = E;\\
% \end{cases}
% \end{split}
% \end{equation}
% and similar for the interior transverse harmonics. The longitudinal vector spherical harmonics obey the symmetries
% \begin{equation}
% \begin{split}
% \mathbf{L}_{p\ell m}(\mathbf{r},k) &= (-1)^{\ell+1}\mathbf{L}_{p\ell m}(\mathbf{r},-k),\\
% \bm{\mathcal{L}}_{p\ell m}^*(\mathbf{r},k) &= (-1)^{\ell+1}\bm{\mathcal{L}}_{p\ell m}(\mathbf{r},-k).
% \end{split}
% \end{equation}

% Due to their regularity at all points $\mathbf{r}$, the transverse interior harmonics obey the orthogonality condition
% \begin{equation}\label{eq:vectorSphericalHarmonicOrthogonality}
% \int \mathbf{X}_{\bm{\alpha}}(\mathbf{r},k)\cdot\mathbf{X}_{\bm{\alpha}'}(\mathbf{r},k')\;\mathrm{d}^3\mathbf{r} = \frac{2\pi^2}{k^2}\delta(k - k')(1 - \delta_{p1}\delta_{m0})\delta_{TT'}\delta_{pp'}\delta_{\ell\ell'}\delta_{mm'}\\
% \end{equation}
% for $k,k' > 0$ and integration across the entire universe. The simpler orthogonality relations that contribute to this simple result are given in Appendix \ref{sec:simpleOrthogonality}. The longitudinal interior harmonics obey the orthogonality condition
% \begin{equation}\label{eq:longitudinalHarmonicOrthogonality}
% \int\mathbf{L}_{\bm{\alpha}}(\mathbf{r},k)\cdot\mathbf{L}_{\bm{\alpha}'}(\mathbf{r},k')\;\mathrm{d}^3\mathbf{r} = \frac{2\pi^2}{k^2}\delta(k - k')(1 - \delta_{p1}\delta_{m0})\delta_{TE}\delta_{TT'}\delta_{pp'}\delta_{\ell\ell'}\delta_{mm'}.
% \end{equation}
% Accordingly, the transverse and longitudinal harmonics also satisfy convenient orthogonality relations when integrated over finite regions in $r$. Explicitly,
% \begin{equation}\label{eq:vectorSphericalHarmonicOrthogonalityConditionFinite}
% \begin{split}
% \int_0^{2\pi}\int_0^\pi\int_a^b\mathbf{X}_{\bm{\alpha}}(\mathbf{r},k)\cdot\mathbf{X}_{\bm{\alpha}'}(\mathbf{r},k')r^2\sin\theta\;\mathrm{d}r\,\mathrm{d}\theta\,\mathrm{d}\phi &= 4\pi(1 - \delta_{p1}\delta_{m0})\delta_{\bm{\alpha}\bm{\alpha}'}R_{T\ell}^{\ll}(k,k';a,b),\\
% \int_0^{2\pi}\int_0^\pi\int_a^b\mathbf{L}_{\bm{\alpha}}(\mathbf{r},k)\cdot\mathbf{L}_{\bm{\alpha}'}(\mathbf{r},k')r^2\sin\theta\;\mathrm{d}r\,\mathrm{d}\theta\,\mathrm{d}\phi &= 4\pi(1 - \delta_{p1}\delta_{m0})\delta_{\bm{\alpha}\bm{\alpha}'}R_{L\ell}^{\ll}(k,k';a,b)
% \end{split}
% \end{equation}
% where $R_{T\ell}^\ll(k,k';a,b)$ and $R_{L\ell}^\ll(k,k';a,b)$ are radial integral functions defined by
% \begin{equation}\label{eq:finiteRadialIntegrals}
% \begin{split}
% R_{M\ell}^\ll(k,k';a,b) &= \int_a^bj_\ell(kr)j_\ell(k'r)r^2\;\mathrm{d}r\\
% &= \left.\frac{r^2}{k^2 - k'^2}\left[k'j_{\ell - 1}(k'r)j_\ell(kr) - kj_{\ell - 1}(kr)j_\ell(k'r)\right]\right|_a^b,\\[0.5em]
% R_{E\ell}^\ll(k,k';a,b) &= \frac{\ell + 1}{2\ell + 1}R_{M,\ell - 1}^\ll(k,k';a,b) + \frac{\ell}{2\ell + 1}R_{M,\ell + 1}^\ll(k,k';a,b),\\
% R_{L\ell}^\ll(k,k';a,b) &= \frac{\ell + 1}{2\ell + 1}R_{M,\ell + 1}^\ll(k,k';a,b) + \frac{\ell}{2\ell + 1}R_{M,\ell - 1}^\ll(k,k';a,b).
% \end{split}
% \end{equation}
% The superscript $\ll$ indicates that the integral involves two spherical Bessel functions of the first kind and thus arises from the integration of two interior harmonics. 

% From their opposing Helmholtz symmetries, we can conclude that integrals of the inner product of $\mathbf{M}$ or $\mathbf{N}$ and $\mathbf{L}$ over all space are null. However, the same is not true when integrating over finite regions of space. Again choosing the region of integration to be finite in $r$, we can see that
% \begin{equation}\label{eq:vectorSphericalHarmonicOrthogonalityConditionFiniteMixed}
% \begin{split}
% \int_0^{2\pi}\int_0^{\pi}\int_a^b\mathbf{L}_{\bm{\alpha}}(\mathbf{r},k)\cdot\mathbf{X}_{\bm{\alpha}'}(\mathbf{r},k')r^2\sin\theta\;\mathrm{d}r\,\mathrm{d}\theta\,\mathrm{d}\phi &= 4\pi(1 - \delta_{p1}\delta_{m0})\sqrt{\ell(\ell + 1)}\delta_{TE}\delta_{T'E}\delta_{pp'}\delta_{\ell\ell'}\delta_{mm'}\\[-0.5em]
% &\qquad\times\frac{1}{kk'}\left[bj_\ell(kb)j_\ell(k'b) - aj_\ell(ka)j_\ell(k'a)\right].
% \end{split}
% \end{equation}
% Note that integration of the longitudinal and transverse magnetic harmonics always produces zero, even when carried out over finite regions in $r$. Integration of the longitudinal and transverse electric harmonics can be seen to produce zero in the limit where the integration region is allowed to become infinite. Explicitly, with $a\to0$ and $b\to\infty$, one can see that $\lim_{r\to0}rj_\ell(kr)j_\ell(k'r) = \lim_{r\to\infty}rj_\ell(kr)j_\ell(k'r) = 0$ for all real, positive $k$ and $k'$ if $\ell$ is an integer greater than 0.

% An important case of this integral occurs when $b\to a$ and $a\to0$, i.e. during integration over a finite sphere. In this case, the characteristic wavenumbers are often defined such that $k = z_{\ell n}/a$ and $k' = z_{\ell n'}/a$, where $z_{\ell n}$ is the $n^\mathrm{th}$ unitless root of $j_\ell(x)$ with $n = 1,2,\ldots$. With these values of $k$ and $k'$ fixed, we can easily see that $R_{M\ell}^\ll(z_{\ell n}/a,z_{\ell n'}/a;0,a) = 0$ for $n\neq n'$. Further, with
% \begin{equation}
% \lim_{k'\to k}R_{M\ell}^\ll(k,k';0,a) = \frac{a^3}{2}\left[j_\ell^2(ka) - j_{\ell - 1}(ka)j_{\ell + 1}(ka)\right]
% \end{equation}
% by L'Hopital's rule, we can see that
% \begin{equation}
% R_{M\ell}^\ll\left(\frac{z_{\ell n}}{a},\frac{z_{\ell n'}}{a};0,a\right) = -\delta_{nn'}\frac{a^3}{2}j_{\ell - 1}(z_{\ell n})j_{\ell + 1}(z_{\ell n}).
% \end{equation}
% Finally, using the recursion identity $j_\ell(x) = [x/(2\ell + 1)][j_{\ell - 1}(x) + j_{\ell 1}(x)]$, we can see that $j_{\ell - 1}(z_{n\ell}) = -j_{\ell + 1}(z_{n\ell})$. Therefore, letting $k_{\bm{\alpha}n} = z_{\ell n}/a$ and $k_{\bm{\alpha}'n'} = z_{\ell'n'}/a$, we have
% \begin{equation}
% \begin{split}
% \int_0^{2\pi}\int_0^\pi\int_0^a\mathbf{X}_{\bm{\alpha}}(\mathbf{r},k_{\bm{\alpha}n})\cdot\mathbf{X}_{\bm{\alpha}'}(\mathbf{r},k_{\bm{\alpha}'n'})r^2\sin\theta\;\mathrm{d}r\;\mathrm{d}\theta\;\mathrm{d}\phi &= 2\pi a^3(1 - \delta_{p1}\delta_{m0})\delta_{\bm{\alpha}\bm{\alpha}'}\delta_{nn'}j_{\ell + 1}^2(z_{\ell n}),\\
% \int_0^{2\pi}\int_0^\pi\int_0^a\mathbf{L}_{\bm{\alpha}}(\mathbf{r},k_{\bm{\alpha}n})\cdot\mathbf{L}_{\bm{\alpha}'}(\mathbf{r},k_{\bm{\alpha}'n'})r^2\sin\theta\;\mathrm{d}r\;\mathrm{d}\theta\;\mathrm{d}\phi &= 2\pi a^3(1 - \delta_{p1}\delta_{m0})\delta_{\bm{\alpha}\bm{\alpha}'}\delta_{nn'}j_{\ell + 1}^2(z_{\ell n}).
% \end{split}
% \end{equation}

% \noindent\rule{\textwidth}{0.5pt}\\

% For calculations involving one or more \textit{exterior} harmonics, one simply replaces one or both of the spherical Bessel functions with spherical Hankel functions. More explicitly, if we define 
% \begin{equation}
% z_\ell^i(x) = 
% \begin{cases}
% j_\ell(x), & i = \,<,\\
% h_\ell(x), & i = \,>,
% \end{cases}
% \end{equation}
% then we can generalize Eq. \eqref{eq:finiteRadialIntegrals} with
% \begin{equation}
% \begin{split}
% R_{M\ell}^{ij}(k,k';a,b) &= \int_a^bz_\ell^i(kr)z_\ell^j(k'r)r^2\;\mathrm{d}r\\
% &= \lim_{\kappa\to k'}\frac{r^2}{k^2 - \kappa^2}\left[\kappa z_{\ell-1}^j(\kappa r)z_{\ell}^i(kr) - kz_{\ell-1}^i(kr)z_{\ell}^j(\kappa r)\right]_a^b
% \end{split}
% \end{equation}
% and $R_{E\ell}^{ij}(k,k';a,b) = (\ell+1)R_{M,\ell - 1}^{ij}(k,k';a,b)/(2\ell + 1) + \ell R_{M,\ell + 1}^{ij}(k,k';a,b)/(2\ell + 1)$. For $k'\neq k$, the limits within both the magnetic and electric radial integrals can be evaluated with simple substitution, but for $k'=k$ it is necessary to evaluate the limit carefully.




% and Laplacian vector spherical harmonics
% \begin{equation}
% \begin{split}
% \mathbf{F}_{Tp\ell m}(\mathbf{r}) &= \nabla\left\{\delta_{TE}\sqrt{(2 - \delta_{m0})\frac{(\ell - m)!}{(\ell + m)!}}r^\ell P_{\ell m}(\cos\theta)S_p(m\phi)\right\},\\
% % \nabla f_{Tp\ell m}^{>}(\mathbf{r}) &= \nabla\left\{\delta_{TE}\sqrt{(2 - \delta_{m0})\frac{(\ell - m)!}{(\ell + m)!}}\frac{1}{r^{\ell + 1}}P_{\ell m}(\cos\theta)S_p(m\phi)\right\},
% &= \delta_{TE}\sqrt{(2 - \delta_{m0})\frac{(\ell - m)!}{(\ell + m)!}}\left[ \vphantom{\frac{(-1)^{p+1}}{\sin\theta}} \ell r^{\ell - 1}P_{\ell m}(\cos\theta)S_p(m\phi)\hat{\mathbf{r}}\right.\\
% &\qquad+ \left. r^{\ell - 1}\frac{\partial P_{\ell m}(\cos\theta)}{\partial \theta}S_p(m\phi)\hat{\bm{\theta}} + \frac{(-1)^{p+1}m}{\sin\theta}r^{\ell - 1}P_{\ell m}(\cos\theta)S_{p+1}(m\phi)\hat{\bm{\phi}} \right]
% \end{split}
% \end{equation}
% which have neither curl ($\nabla\times\mathbf{F}_{Tp\ell m}(\mathbf{r}) = 0$) or divergence ($\nabla\cdot\mathbf{F}_{Tp\ell m}(\mathbf{r}) = 0$). 

% \begin{equation}
% \begin{split}
% \mathbf{M}_{p\ell m}(\mathbf{r},k) &= \sqrt{K_{\ell m}}\left[\frac{(-1)^{p + 1}m}{\sin\theta}j_\ell(kr)P_{\ell m}(\cos\theta)S_{p+1}(m\phi)\hat{\bm{\theta}} - j_\ell(kr)\frac{\partial P_{\ell m}(\cos\theta)}{\partial\theta}S_p(m\phi)\hat{\bm{\phi}}\right],\\[0.5em]
% \mathbf{N}_{p\ell m}(\mathbf{r},k) &= \sqrt{K_{\ell m}}\left[\frac{\ell(\ell + 1)}{kr}j_\ell(kr)P_{\ell m}(\cos\theta)S_p(m\phi)\hat{\mathbf{r}}\right.\\
% &+ \left.\frac{1}{kr}\frac{\partial\{rj_\ell(kr)\}}{\partial r}\left(\frac{\partial P_{\ell m}(\cos\theta)}{\partial\theta}S_p(m\phi)\hat{\bm{\theta}} + \frac{(-1)^{p+1}m}{\sin\theta}P_{\ell m}(\cos\theta)S_{p + 1}(m\phi)\hat{\bm{\phi}}\right)\right],
% \end{split}
% \end{equation}


% \begin{equation}
% \begin{split}
% \bm{\mathcal{F}}_{Tp\ell m}(\mathbf{r},k) &= \nabla\left\{\delta_{TE}\sqrt{(2 - \delta_{m0})\frac{(\ell - m)!}{(\ell + m)!}}\frac{1}{r^{\ell + 1}} P_{\ell m}(\cos\theta)S_p(m\phi)\right\},\\
% &= \delta_{TE}\sqrt{(2 - \delta_{m0})\frac{(\ell - m)!}{(\ell + m)!}}\left[(-\ell - 1)\frac{1}{r^{\ell + 2}}P_{\ell m}(\cos\theta)S_p(m\phi)\hat{\mathbf{r}}\right.\\
% &\qquad+ \left.\frac{1}{r^{\ell + 2}}\frac{\partial P_{\ell m}(\cos\theta)}{\partial \theta}S_p(m\phi)\hat{\bm{\theta}} + \frac{(-1)^{p+1}m}{\sin\theta}\frac{1}{r^{\ell + 2}}P_{\ell m}(\cos\theta)S_{p+1}(m\phi)\hat{\bm{\phi}} \right]
% \end{split}
% \end{equation}
 





%  The Laplacian harmonics are either irregular at infinity ($r^\ell$) or at the origin ($1/r^{\ell + 1}$) for $\ell > 0$, such that their orthogonality relations can only be defined across finite regions in $r$:
% \begin{equation}\label{eq:laplacianHarmonicOrthogonality}
% \begin{split}
% \int_0^{2\pi}\int_0^\pi\int_a^b\mathbf{F}_{\bm{\alpha}}(\mathbf{r})\cdot\mathbf{F}_{\bm{\alpha}'}(\mathbf{r})r^2\sin\theta\;\mathrm{d}r\,\mathrm{d}\theta\,\mathrm{d}\phi &= 4\pi\frac{\ell}{2\ell + 1}\left(b^{2\ell + 1} - a^{2\ell + 1}\right)\delta_{T\!E}\delta_{TT'}\delta_{pp'}\delta_{\ell\ell'}\delta_{mm'},\\
% \int_0^{2\pi}\int_0^\pi\int_a^b\bm{\mathcal{F}}_{\bm{\alpha}}(\mathbf{r})\cdot\bm{\mathcal{F}}_{\bm{\alpha}'}(\mathbf{r})r^2\sin\theta\;\mathrm{d}r\,\mathrm{d}\theta\,\mathrm{d}\phi &= -4\pi\frac{\ell + 1}{2\ell + 1}\left(b^{-2\ell - 1} - a^{-2\ell - 1}\right)\delta_{TE}\delta_{TT'}\delta_{pp'}\delta_{\ell\ell'}\delta_{mm'},\\
% \int_0^{2\pi}\int_0^\pi\int_a^b\mathbf{F}_{\bm{\alpha}}(\mathbf{r})\cdot\bm{\mathcal{F}}_{\bm{\alpha}'}(\mathbf{r})r^2\sin\theta\;\mathrm{d}r\,\mathrm{d}\theta\,\mathrm{d}\phi &= 0.
% \end{split}
% \end{equation}

% A useful mixed Laplacian-transverse orthogonality conditions also exists:
% \begin{equation}\label{eq:angularMixedOrthogonalityNf}
% \begin{split}
% \int_0^{2\pi}\int_0^\pi&\left[\hat{\mathbf{r}}\cdot\mathbf{N}_{p\ell m}(r,\theta,\phi;k)\right]f_{p'\ell'm'}^i(r,\theta,\phi)\sin\theta\;\mathrm{d}\theta\,\mathrm{d}\phi\\
% &= 4\pi\delta_{pp'}\delta_{\ell\ell'}\delta_{mm'}(1 - \delta_{p1}\delta_{m0})\sqrt{\frac{\ell(\ell + 1)}{2\ell + 1}}\frac{j_\ell(kr)}{kr}\left[r^\ell\delta_{i,<} + \frac{\delta_{i,>}}{r^{\ell + 1}}\right],
% \end{split}
% \end{equation}
% as does the mixed Laplacian-longitudinal orthogonality condition
% \begin{equation}\label{eq:angularMixedOrthogonalityLf}
% \begin{split}
% \int_0^{2\pi}\int_0^\pi&\left[\hat{\mathbf{r}}\cdot\mathbf{L}_{p\ell m}(r,\theta,\phi;k)\right]f_{p'\ell'm'}^i(r,\theta,\phi)\sin\theta\;\mathrm{d}\theta\,\mathrm{d}\phi\\
% &= 4\pi\delta_{pp'}\delta_{\ell\ell'}\delta_{mm'}(1 - \delta_{p1}\delta_{m0})\frac{1}{\sqrt{2\ell + 1}}\frac{1}{k}\frac{\partial j_\ell(kr)}{\partial r}\left[r^\ell\delta_{i,<} + \frac{\delta_{i,>}}{r^{\ell + 1}}\right].
% \end{split}
% \end{equation}

% \begin{equation}
% \begin{split}
% \int_{r<a}\mathbf{F}_{\bm{\alpha}}(\mathbf{r})\cdot\mathbf{X}_{\bm{\beta}}(\mathbf{r},k)\;\mathrm{d}^3\mathbf{r} &= \int_{r < a}\mathbf{X}_{\bm{\beta}}(\mathbf{r},k)\cdot\nabla f_{\bm{\alpha}}^<(\mathbf{r})\;\mathrm{d}^3\mathbf{r}\\
% &= \int_{r < a}\nabla\cdot\left\{\mathbf{X}_{\bm{\beta}}(\mathbf{r},k) f_{\bm{\alpha}}^<(\mathbf{r})\right\}\;\mathrm{d}^3\mathbf{r} - \int_{r < a}\nabla\cdot\mathbf{X}_{\bm{\beta}}(\mathbf{r},k)f_{\bm{\alpha}}^<(\mathbf{r})\;\mathrm{d}^3\mathbf{r}\\
% &= \int_0^{2\pi}\int_0^\pi\mathbf{X}_{\bm{\beta}}(a,\theta,\phi;k)f_{\bm{\alpha}}^<(a,\theta,\phi)\cdot \hat{\mathbf{r}}a^2\sin\theta\;\mathrm{d}\theta\,\mathrm{d}\phi - 0\\
% &= \delta_{TE}\delta_{T'E}\int_0^{2\pi}\int_0^\pi\left[\mathbf{N}_{p\ell m}(a,\theta,\phi;k)\cdot\hat{\mathbf{r}}\right]f_{p'\ell' m'}^<(a,\theta,\phi)a^2\sin\theta\;\mathrm{d}\theta\,\mathrm{d}\phi\\
% &= 4\pi\delta_{TE}\delta_{T'E}\delta_{pp'}\delta_{\ell\ell'}\delta_{mm'}(1 - \delta_{p1}\delta_{m0})\sqrt{\frac{\ell(\ell + 1)}{2\ell + 1}}\frac{j_\ell(ka)}{ka}a^{\ell+2}
% \end{split}
% \end{equation}

% \begin{equation}
% \begin{split}
% \int_{a<r<b}\mathbf{F}_{\bm{\alpha}}(\mathbf{r})\cdot\mathbf{L}_{\bm{\beta}}(\mathbf{r},k)\;\mathrm{d}^3\mathbf{r} &= 4\pi\delta_{TE}\delta_{T'E}\delta_{pp'}\delta_{\ell\ell'}\delta_{mm'}(1 - \delta_{p1}\delta_{m0})\frac{\ell}{\sqrt{2\ell+1}}\left(\frac{j_\ell(kb)}{kb}b^{\ell+2} - \frac{j_\ell(ka)}{ka}a^{\ell+2}\right),\\
% \int_{a<r<b}\bm{\mathcal{F}}_{\bm{\alpha}}(\mathbf{r})\cdot\mathbf{L}_{\bm{\beta}}(\mathbf{r},k)\;\mathrm{d}^3\mathbf{r} &= 4\pi\delta_{TE}\delta_{T'E}\delta_{pp'}\delta_{\ell\ell'}\delta_{mm'}(1 - \delta_{p1}\delta_{m0})\frac{\ell+1}{\sqrt{2\ell+1}}\frac{1}{k}\left(\frac{1}{a^\ell}j_\ell(ka) - \frac{1}{b^\ell}j_\ell(kb)\right).
% \end{split}
% \end{equation}

% The associated completeness relations to these orthogonality conditions can be derived from the Helmholtz expansion of $\bm{1}_2\delta(\mathbf{r} - \mathbf{r}')$, which states that 
% \begin{equation}\label{eq:vectorSphericalHarmonicCompleteness}
% \begin{split}
% \bm{1}_2\delta(\mathbf{r} - \mathbf{r}') &= \frac{1}{2\pi^2}\int_0^\infty\sum_{\bm{\alpha}}k^2\mathbf{X}_{\bm{\alpha}}(\mathbf{r},k)\mathbf{X}_{\bm{\alpha}}(\mathbf{r}',k)\;\mathrm{d}k\\
% &+ \frac{1}{4\pi}\sum_{\bm{\alpha}}\nabla\nabla'\left\{ f_{\bm{\alpha}}^>(\mathbf{r})f_{\bm{\alpha}}^<(\mathbf{r}')\Theta(r - r') +  f_{\bm{\alpha}}^<(\mathbf{r})f_{\bm{\alpha}}^>(\mathbf{r}')\Theta(r' - r)\right\}.
% \end{split}
% \end{equation}
% Here, the scalar harmonics 
% \begin{equation}\label{eq:scalarHarmonics}
% f_{\bm{\alpha}}^i(\mathbf{r}) = \delta_{TE}\sqrt{(2 - \delta_{m0})\frac{(\ell - m)!}{(\ell + m)!}}P_{\ell m}(\cos\theta)S_p(m\phi)\times
% \begin{cases}
% r^\ell, & i = \, <,\\
% \dfrac{1}{r^{\ell + 1}}, & i = \,>
% \end{cases}
% \end{equation}
% are defined such that $\mathbf{F}_{\bm{\alpha}}(\mathbf{r}) = \nabla f_{\bm{\alpha}}^<(\mathbf{r})$ and $\bm{\mathcal{F}}_{\bm{\alpha}}(\mathbf{r}) = \nabla f_{\bm{\alpha}}^>(\mathbf{r})$.
% Note that, as is clear form the derivation of Section \ref{app:helmholtzDelta}, the second term is curl-free in both the $\mathbf{r}$ and $\mathbf{r}'$ and the first term is likewise transverse in both coordinate systems.

% The transverse exterior harmonics also satisfy the convenient relation at large radii
% \begin{equation}
% \begin{split}
% \lim_{r\to\infty}\int_0^{2\pi}\int_0^\pi \bm{\mathcal{X}}_{\bm{\alpha}}(\mathbf{r},k)&\times\bm{\mathcal{Y}}_{\bm{\alpha}'}(\mathbf{r},k)\cdot\hat{\mathbf{r}}r^2\sin\theta\;\mathrm{d}\theta\,\mathrm{d}\phi = \\
% &(1 + \delta_{p0}\delta_{m0} - \delta_{p1}\delta_{m0})2\pi\frac{(-1)^{\ell + 1}\mathrm{i}^{2\ell + 1}}{k^2}\frac{\ell(\ell + 1)}{2\ell + 1}\frac{(\ell + m)!}{(\ell - m)!}\delta_{\bm{\alpha}\bm{\alpha}'}.
% \end{split}
% \end{equation}

% It is also useful from time to time to use the unitless regularized longitudinal harmonics
% \begin{equation}
% \begin{split}
% \mathbf{Z}_{\bm{\alpha}}(\mathbf{r};s) &= s^{-\ell + 1}\nabla f_{\bm{\alpha}}^<(\mathbf{r})\Theta(s - r) + s^{\ell + 2}\nabla f_{\bm{\alpha}}^>(\mathbf{r})\Theta(r - s)\\
% &= \nabla\left\{s^{-\ell + 1} f_{\bm{\alpha}}^<(\mathbf{r})\Theta(s - r) + s^{\ell + 2} f_{\bm{\alpha}}^>(\mathbf{r})\Theta(r - s)\right\}\\
% &= s^{-\ell + 1}\mathbf{F}_{\bm{\alpha}}(\mathbf{r})\Theta(s - r) + s^{\ell + 2}\bm{\mathcal{F}}_{\bm{\alpha}}(\mathbf{r})\Theta(r - s)
% \end{split}
% \end{equation}
% which obey the orthogonality conditions
% \begin{equation}
% \begin{split}
% \int\mathbf{Z}_{\bm{\alpha}}(\mathbf{r};a)\cdot\mathbf{Z}_{\bm{\beta}}(\mathbf{r};b)\;\mathrm{d}^3\mathbf{r} &= 4\pi\delta_{TE}\delta_{\bm{\alpha}\bm{\beta}}(1 - \delta_{p1}\delta_{m0})\left[\Theta(a - b)\frac{b^{\ell + 2}}{a^{\ell - 1}} + \Theta(b - a)\frac{a^{\ell + 2}}{b^{\ell - 1}}\right],\\
% \int\mathbf{Z}_{\bm{\alpha}}(\mathbf{r};s)\cdot\mathbf{Z}_{\bm{\beta}}(\mathbf{r};s)\;\mathrm{d}^3\mathbf{r} &= 4\pi s^3\delta_{TE}\delta_{\bm{\alpha}\bm{\beta}}(1 - \delta_{p1}\delta_{m0}).
% \end{split}
% \end{equation}








\subsection{Useful Orthogonality Relations}\label{sec:simpleOrthogonality}

A list of the useful and simple orthogonality relations useful to our calculations begins with the angular integral
\begin{equation}
\int_0^{2\pi}S_p(m\phi)S_{p'}(m'\phi)\;\mathrm{d}\phi = \pi(1 + \delta_{p0}\delta_{m0} - \delta_{p1}\delta_{m0})\delta_{pp'}\delta_{mm'},
\end{equation}
wherein the convention is used that the replacements $p + 2n\to 0$ for even $p$ and $p + 2n\to1$ for odd $p$ are taken automatically (here $n = 0,1,2,\ldots$ is a nonnegative integer). Similarly, $p + 2n + 1\to 1$ for even $p$ and $p + 2n + 1\to0$ for odd $p$ are implied. 

Two more angular integrals, this time in $\theta$, are of use:
\begin{equation}
\int_0^\pi P_{\ell m}(\cos\theta)P_{\ell'm}(\cos\theta)\sin\theta\;\mathrm{d}\theta = \frac{2}{(2\ell + 1)}\frac{(\ell + m)!}{(\ell - m)!}\delta_{\ell\ell'}\\
\end{equation}
and
\begin{equation}
\begin{split}
\int_0^\pi\left(\frac{\partial P_{\ell m}(\cos\theta)}{\partial\theta}\frac{\partial P_{\ell'm}(\cos\theta)}{\partial\theta}\right. &+ \left.m^2\frac{P_{\ell m}(\cos\theta)P_{\ell'm}(\cos\theta)}{\sin^2\theta}\right)\sin\theta\;\mathrm{d}\theta\\
&= \int_0^\pi\ell(\ell + 1)P_{\ell m}(\cos\theta)P_{\ell'm}(\cos\theta)\sin\theta\;\mathrm{d}\theta\\
&= \frac{2\ell(\ell + 1)}{(2\ell + 1)}\frac{(\ell + m)!}{(\ell - m)!}\delta_{\ell\ell'},
\end{split}
\end{equation}
for $\ell > 0$, $0 \leq m \leq \ell$. The second identity can be derived using the identity
\begin{equation}
\begin{split}
&\frac{\partial}{\partial\theta}\left\{\sin\theta\frac{\partial P_{\ell'm}(\cos\theta)}{\partial\theta}P_{\ell m}(\cos\theta) + \sin\theta\frac{\partial P_{\ell m}(\cos\theta)}{\partial \theta}P_{\ell' m}(\cos\theta)\right\}\\
&= 2\sin\theta\left(\frac{\partial P_{\ell m}(\cos\theta)}{\partial\theta}\frac{\partial P_{\ell'm}(\cos\theta)}{\partial\theta} + m^2\frac{P_{\ell m}(\cos\theta)P_{\ell'm}(\cos\theta)}{\sin^2\theta}\right) - 2\ell(\ell + 1)\sin\theta P_{\ell m}(\cos\theta)P_{\ell' m}(\cos\theta),
\end{split}
\end{equation}
which can in turn be derived from the generalized Legendre equation defining the functions $P_{\ell m}(\cos\theta)$,
\begin{equation}
\frac{\partial^2}{\partial\theta^2}P_{\ell m}(\cos\theta) + \frac{\cos\theta}{\sin\theta}\frac{\partial}{\partial\theta}P_{\ell m}(\cos\theta) + \left(\ell(\ell + 1) - \frac{m^2}{\sin^2\theta}\right)P_{\ell m}(\cos\theta) = 0.
\end{equation}

Further,
\begin{equation}
\begin{split}
\int_0^\pi\left[\frac{P_{\ell m}(\cos\theta)}{\sin\theta}\frac{\partial P_{\ell'm}(\cos\theta)}{\partial\theta} + \frac{P_{\ell'm}(\cos\theta)}{\sin\theta}\frac{\partial P_{\ell m}(\cos\theta)}{\partial\theta}\right]\sin\theta\,\mathrm{d}\theta &= \int_0^\pi\frac{\partial}{\partial \theta}\left\{P_{\ell m}(\cos\theta)P_{\ell' m}(\cos\theta)\right\}\mathrm{d}\theta\\
&= \left.P_{\ell m}(\cos\theta)P_{\ell' m}(\cos\theta)\right|_0^\pi\\
&= 2\delta_{m0}\delta_{\mathrm{par}(\ell),1},
\end{split}
\end{equation}
where
\begin{equation}
\mathrm{par}(n) = 
\begin{cases}
0, & |n|\;\mathrm{even},\\
1, & |n|\;\mathrm{odd}
\end{cases}
\end{equation}
is the parity function that returns 0 for even integers and 1 for odd integers.








\subsection{Useful recursion relations}
It is sometimes useful to know that
\begin{equation}
\begin{split}
z_\ell(x) &= \frac{x}{2\ell + 1}\left[z_{\ell - 1}(x) + z_{\ell + 1}(x)\right],\\
\frac{\partial z_\ell(x)}{\partial x} &= z_{\ell - 1}(x) - \frac{\ell + 1}{x}z_\ell(x),\\
\frac{\partial z_\ell(x)}{\partial x} &= -z_{\ell + 1}(x) + \frac{\ell}{x}z_\ell(x),
\end{split}
\end{equation}
and
\begin{equation}
\frac{\partial\{xz_\ell(x)\}}{\partial x} = \frac{x}{2\ell + 1}\left[(\ell + 1)z_{\ell - 1}(x) - \ell z_{\ell + 1}(x)\right]
\end{equation}
where $z_\ell(x)$ is any spherical Bessel or Hankel function. These identities can be combined to give
\begin{equation}
\begin{split}
\ell(\ell + 1)\frac{w_\ell(x_1)z_\ell(x_2)}{x_1x_2} &+ \frac{1}{x_1x_2}\frac{\partial\{x_1w_\ell(x_1)\}}{\partial x_1}\frac{\partial\{x_2z_\ell(x_2)\}}{\partial x_2}\\
&= \frac{1}{2\ell + 1}\left[(\ell + 1)w_{\ell - 1}(x_1)z_{\ell - 1}(x_2) + \ell w_{\ell + 1}(x_1)z_{\ell + 1}(x_2)\right]
\end{split}
\end{equation}
and
\begin{equation}
\frac{\partial z_\ell(x_1)}{\partial x_1}\frac{\partial w_\ell(x_2)}{\partial x_2} = \frac{\ell + 1}{2\ell + 1}z_{\ell + 1}(x_1)w_{\ell + 1}(x_2) + \frac{\ell}{2\ell + 1}z_{\ell - 1}(x_1)w_{\ell - 1}(x_2) - \frac{\ell(\ell + 1)}{x_1x_2}z_\ell(x_1)w_\ell(x_2)
\end{equation}
where $w_\ell$ and $z_\ell$ are any two spherical Bessel or Hankel functions.






%%%%%%%%%%%%%%%%%%%%%%%%%%%%%%%%%%%%%%%%%%%%%%%%%%%%%%%%%%%%%%%%%%%%%%%%%%%%
% Derivation of Hybridized Operators
%%%%%%%%%%%%%%%%%%%%%%%%%%%%%%%%%%%%%%%%%%%%%%%%%%%%%%%%%%%%%%%%%%%%%%%%%%%%
%!TEX root = finiteMQED.tex

\section{Fano Diagonalization of an Oscillator Coupled to a Bath}\label{app:fanoDiagonalization}

Beginning with a Hamiltonian
\begin{equation}
\hat{H} = \hat{H}_0 + \hat{H}_\mathrm{int}
\end{equation}
where 
\begin{equation}
\begin{split}
\hat{H}_0 &= \sum_{\bm{\alpha}}\int_0^\infty\frac{1}{2}\hbar\Omega\left[\hat{a}_{\bm{\alpha}}^\dagger(k,t)\hat{a}_{\bm{\alpha}}(k,t) + \hat{a}_{\bm{\alpha}}(k,t)\hat{a}_{\bm{\alpha}}^\dagger(k,t)\right]\mathrm{d}k\\
&\qquad + \sum_{\bm{\beta}}\int_0^\infty\int_0^\infty\frac{1}{2}\hbar\nu\left[\hat{b}_{\bm{\beta}}^\dagger(k,\nu;t)\hat{b}_{\bm{\beta}}(k,\nu;t) + \hat{b}_{\bm{\beta}}(k,\nu;t)\hat{b}_{\bm{\beta}}^\dagger(k,\nu;t)\right]\mathrm{d}\nu\,\mathrm{d}k
\end{split}
\end{equation}
and
\begin{equation}
\hat{H}_\mathrm{int} = \sum_{\bm{\beta}}\int_0^\infty\int_0^\infty \hbar g(\nu)\left[\hat{a}_{\bm{\beta}}(k,t) + \hat{a}_{\bm{\beta}}^\dagger(k,t)\right]\left[\hat{b}_{\bm{\beta}}(k,\nu;t) + \hat{b}_{\bm{\beta}}^\dagger(k,\nu;t)\right]\mathrm{d}\nu\,\mathrm{d}k,
\end{equation}
we can define the boson operators $\hat{a}_{\bm{\alpha}}(k,t)$ and $\hat{b}_{\bm{\beta}}(k,\nu;t)$ by their commutation relations
\begin{equation}
\begin{split}
\left[\hat{a}_{\bm{\alpha}}(k,t),\hat{a}_{\bm{\alpha}'}^\dagger(k',t)\right] &= \delta_{\bm{\alpha}\bm{\alpha}'}\delta(k - k'),\\
\left[\hat{b}_{\bm{\beta}}(k,\nu;t),\hat{b}_{\bm{\beta}'}^\dagger(k',\nu';t)\right] &= \delta_{\bm{\beta}\bm{\beta}'}\delta(k - k')\delta(\nu - \nu')
\end{split}
\end{equation}
and
\begin{equation}
\begin{split}
\left[\hat{a}_{\bm{\alpha}}(k,t),\hat{a}_{\bm{\alpha}'}(k',t)\right] &= 0,\\
\left[\hat{b}_{\bm{\beta}}(k,\nu;t),\hat{b}_{\bm{\beta}'}(k',\nu';t)\right] &= 0,\\
\left[\hat{a}_{\bm{\alpha}}(k,t),\hat{b}_{\bm{\alpha}'}(k',\nu;t)\right] &= \left[\hat{a}_{\bm{\alpha}}(k,t),\hat{b}_{\bm{\alpha}'}^\dagger(k',\nu;t)\right] = 0.
\end{split}
\end{equation}
Note that commutation relations between the system's creation operators follow from the above via the Hermitian conjugate identity $[\hat{A},\hat{B}]^\dagger = -[\hat{A}^\dagger,\hat{B}^\dagger]$. Diagonalization of this system can be performed by defining new boson operators
\begin{equation}
\hat{B}_{\bm{\beta}}(k,\nu;t) = q(\nu)\hat{a}_{\bm{\beta}}(k,t) + s(\nu)\hat{a}_{\bm{\beta}}^\dagger(k,t) + \int_0^\infty\left[u(\nu,\nu')\hat{b}_{\bm{\beta}}(k,\nu';t) + v(\nu,\nu')\hat{b}_{\bm{\beta}}^\dagger(k,\nu';t)\right]\mathrm{d}\nu'
\end{equation}
that obey commutation relations
\begin{equation}
\begin{split}
\left[\hat{B}_{\bm{\beta}}(k,\nu;t),\hat{B}_{\bm{\beta}'}^\dagger(k',\nu';t)\right] &= \delta_{\bm{\beta}\bm{\beta}'}\delta(k - k')\delta(\nu - \nu'),\\
\left[\hat{B}_{\bm{\beta}}(k,\nu;t),\hat{B}_{\bm{\beta}'}(k',\nu';t)\right] &= 0
\end{split}
\end{equation}
and provide a new Hamiltonian
\begin{equation}
\hat{H} = \sum_{\bm{\beta}}\int_0^\infty\int_0^\infty\hbar\nu\hat{B}_{\bm{\beta}}^\dagger(k,\nu;t)\hat{B}_{\bm{\beta}}(k,\nu;t)\;\mathrm{d}\nu\,\mathrm{d}k.
\end{equation}

The diagonalization process begins with the identification of the commutation relations
\begin{equation}
\begin{split}
\left[\hat{a}_{\bm{\alpha}}(k,t),\hat{H}\right] &= \sum_{\bm{\alpha}'}\int_0^\infty\frac{1}{2}\hbar\Omega\left(\left[\hat{a}_{\bm{\alpha}}(k,t),\hat{a}_{\bm{\alpha}'}^\dagger(k',t)\hat{a}_{\bm{\alpha}'}(k',t)\right] + \left[\hat{a}_{\bm{\alpha}}(k,t),\hat{a}_{\bm{\alpha}'}(k',t)\hat{a}_{\bm{\alpha}'}^\dagger(k',t)\right]\right)\mathrm{d}k'\\
&\qquad + \sum_{\bm{\beta}}\int_0^\infty\int_0^\infty \hbar g(\nu)\left(\left[\hat{a}_{\bm{\alpha}}(k,t),\hat{a}_{\bm{\beta}}(k',t)\right] + \left[\hat{a}_{\bm{\alpha}}(k,t),\hat{a}_{\bm{\beta}}^\dagger(k',t)\right]\right)\\
&\qquad\qquad\times\left[\hat{b}_{\bm{\beta}}(k',\nu;t) + \hat{b}_{\bm{\beta}}^\dagger(k',\nu;t)\right]\mathrm{d}\nu\,\mathrm{d}k'\\[0.5em]
% &= \sum_{\bm{\alpha}'}\int_0^\infty\frac{1}{2}\hbar\Omega\left(\hat{a}_{\bm{\alpha}'}^\dagger(k',t)\left[\hat{a}_{\bm{\alpha}}(k,t),\hat{a}_{\bm{\alpha}'}(k',t)\right] + \hat{a}_{\bm{\alpha}'}(k',t)\left[\hat{a}_{\bm{\alpha}}(k,t),\hat{a}_{\bm{\alpha}'}^\dagger(k',t)\right]\right.\\
% &\qquad \left. + \hat{a}_{\bm{\alpha}'}(k',t)\left[\hat{a}_{\bm{\alpha}}(k,t),\hat{a}_{\bm{\alpha}'}^\dagger(k',t)\right] + \hat{a}_{\bm{\alpha}'}^\dagger(k',t)\left[\hat{a}_{\bm{\alpha}}(k,t),\hat{a}_{\bm{\alpha}'}(k',t)\right]\right)\mathrm{d}k'\\
% &\qquad + \sum_{\bm{\beta}}\int_0^\infty\int_0^\infty \hbar g(\nu)\delta_{\bm{\alpha}\bm{\beta}}\delta(k - k')\left[\hat{b}_{\bm{\beta}}(k',\nu;t) + \hat{b}_{\bm{\beta}}^\dagger(k',\nu;t)\right]\mathrm{d}\nu\,\mathrm{d}k'\\
% &= \sum_{\bm{\alpha}'}\int_0^\infty\frac{1}{2}\hbar\Omega\left[2\hat{a}_{\bm{\alpha}'}(k',t)\delta_{\bm{\alpha}\bm{\alpha}'}\delta(k - k')\right] + \int_0^\infty \hbar g(\nu)\left[\hat{b}_{\bm{\alpha}}(k,\nu;t) + \hat{b}_{\bm{\alpha}}^\dagger(k,\nu;t)\right]\mathrm{d}\nu\\
&= \hbar\Omega\hat{a}_{\bm{\alpha}}(k,t) + \int_0^\infty \hbar g(\nu)\left[\hat{b}_{\bm{\alpha}}(k,\nu;t) + \hat{b}_{\bm{\alpha}}^\dagger(k,\nu;t)\right]\mathrm{d}\nu
\end{split}
\end{equation}
and
\begin{equation}
\begin{split}
\left[\hat{b}_{\bm{\beta}}(k,\nu;t),\hat{H}\right] &= \sum_{\bm{\beta}'}\int_0^\infty\int_0^\infty\frac{1}{2}\hbar\nu'\left(\left[\hat{b}_{\bm{\beta}}(k,\nu;t),\hat{b}_{\bm{\beta}'}^\dagger(k',\nu';t)\hat{b}_{\bm{\beta}'}(k',\nu';t)\right]\right.\\
&\qquad\qquad\left. + \left[\hat{b}_{\bm{\beta}}(k,\nu;t),\hat{b}_{\bm{\beta}'}(k',\nu';t)\hat{b}_{\bm{\beta}'}^\dagger(k',\nu';t)\right]\right)\mathrm{d}\nu'\,\mathrm{d}k'\\
&\qquad + \sum_{\bm{\beta}'}\int_0^\infty\int_0^\infty \hbar g(\nu')\left[\hat{a}_{\bm{\beta}'}(k',t) + \hat{a}_{\bm{\beta}'}^\dagger(k',t)\right]\\
&\qquad\qquad\times\left(\left[\hat{b}_{\bm{\beta}}(k,\nu;t),\hat{b}_{\bm{\beta}'}(k',\nu';t)\right] + \left[\hat{b}_{\bm{\beta}}(k,\nu;t),\hat{b}_{\bm{\beta}'}^\dagger(k',\nu';t)\right]\right)\mathrm{d}\nu'\,\mathrm{d}k'\\[1.0em]
% &= \sum_{\bm{\beta}'}\int_0^\infty\int_0^\infty\frac{1}{2}\hbar\nu'\left(\hat{b}_{\bm{\beta}'}^\dagger(k',\nu';t)\left[\hat{b}_{\bm{\beta}}(k,\nu;t),\hat{b}_{\bm{\beta}'}(k',\nu';t)\right]\right.\\
% &\qquad\qquad + \hat{b}_{\bm{\beta}'}(k',\nu';t)\left[\hat{b}_{\bm{\beta}}(k,\nu;t),\hat{b}_{\bm{\beta}'}^\dagger(k',\nu';t)\right]\\
% &\qquad\qquad + \hat{b}_{\bm{\beta}'}(k',\nu';t)\left[\hat{b}_{\bm{\beta}}(k,\nu;t),\hat{b}_{\bm{\beta}'}^\dagger(k',\nu';t)\right]\\
% &\qquad\qquad \left. + \hat{b}_{\bm{\beta}'}^\dagger(k',\nu';t)\left[\hat{b}_{\bm{\beta}}(k,\nu;t),\hat{b}_{\bm{\beta}'}(k',\nu';t)\right]\right)\mathrm{d}\nu'\,\mathrm{d}k'\\
% &\qquad + \sum_{\bm{\beta}'}\int_0^\infty\int_0^\infty \hbar g(\nu')\left[\hat{a}_{\bm{\beta}'}(k',t) + \hat{a}_{\bm{\beta}'}^\dagger(k',t)\right]\delta_{\bm{\beta}\bm{\beta}'}\delta(k - k')\delta(\nu - \nu')\mathrm{d}\nu'\,\mathrm{d}k'\\
% &= \sum_{\bm{\beta}'}\int_0^\infty\int_0^\infty\frac{1}{2}\hbar\nu'2\hat{b}_{\bm{\beta}'}(k',\nu';t)\delta_{\bm{\beta}\bm{\beta}'}\delta(k - k')\delta(\nu - \nu')\;\mathrm{d}\nu'\,\mathrm{d}k' + \hbar g(\nu)\left[\hat{a}_{\bm{\beta}}(k,t) + \hat{a}_{\bm{\beta}}^\dagger(k,t)\right]\\
&= \hbar\nu\hat{b}_{\bm{\beta}}(k,\nu;t) + \hbar g(\nu)\left[\hat{a}_{\bm{\beta}}(k,t) + \hat{a}_{\bm{\beta}}^\dagger(k,t)\right].
\end{split}
\end{equation}
With these and the identity $[\hat{A},\hat{B}]^\dagger = -[\hat{A}^\dagger,\hat{B}^\dagger]$, we can calculate the commutator of our hybridized annihilation operators with both the hybridized and unhybridized forms of the Hamtilonian. In detail,
\begin{equation}\label{eq:BcommH1}
\begin{split}
\left[\hat{B}_{\bm{\beta}}(k,\nu;t),\hat{H}\right] &= \sum_{\bm{\beta}'}\int_0^\infty\int_0^\infty\hbar\nu'\left[\hat{B}_{\bm{\beta}}(k,\nu;t),\hat{B}_{\bm{\beta}'}^\dagger(k',\nu';t)\hat{B}_{\bm{\beta}'}(k',\nu';t)\right]\mathrm{d}\nu'\,\mathrm{d}k'\\
% &= \sum_{\bm{\beta}'}\int_0^\infty\int_0^\infty\hbar\nu'\left(\hat{B}_{\bm{\beta}'}^\dagger(k',\nu';t)\left[\hat{B}_{\bm{\beta}}(k,\nu;t),\hat{B}_{\bm{\beta}'}(k',\nu';t)\right]\right.\\
% &\qquad\left. + \hat{B}_{\bm{\beta}'}(k',\nu';t)\left[\hat{B}_{\bm{\beta}}(k,\nu;t),\hat{B}_{\bm{\beta}'}^\dagger(k',\nu';t)\right]\right)\mathrm{d}\nu'\,\mathrm{d}k'\\
% &= \sum_{\bm{\beta}}\int_0^\infty\int_0^\infty\hbar\nu'\hat{B}_{\bm{\beta}'}(k',\nu';t)\delta_{\bm{\beta}'\bm{\beta}}\delta(k - k')\delta(\nu - \nu')\;\mathrm{d}\nu'\,\mathrm{d}k'\\
% &= \hbar\nu\hat{B}_{\bm{\beta}}(k,\nu;t)\\
&= \hbar\nu\left(q(\nu)\hat{a}_{\bm{\beta}}(k,t) + s(\nu)\hat{a}_{\bm{\beta}}^\dagger(k,t) + \int_0^\infty\left[u(\nu,\nu')\hat{b}_{\bm{\beta}}(k,\nu';t) + v(\nu,\nu')\hat{b}_{\bm{\beta}}^\dagger(k,\nu';t)\right]\mathrm{d}\nu'\right).
\end{split}
\end{equation}
and
\begin{equation}\label{eq:BcommH2}
\begin{split}
\left[\hat{B}_{\bm{\beta}}(k,\nu;t),\hat{H}\right] &= \left[q(\nu)\hat{a}_{\bm{\beta}}(k,t) + s(\nu)\hat{a}_{\bm{\beta}}^\dagger(k,t) + \int_0^\infty\left[u(\nu,\nu')\hat{b}_{\bm{\beta}}(k,\nu';t) + v(\nu,\nu')\hat{b}_{\bm{\beta}}^\dagger(k,\nu';t)\right]\mathrm{d}\nu',\hat{H}\right]\\
&= q(\nu)\left(\hbar\Omega\hat{a}_{\bm{\alpha}}(k,t) + \int_0^\infty \hbar g(\nu')\left[\hat{b}_{\bm{\alpha}}(k,\nu';t) + \hat{b}_{\bm{\alpha}}^\dagger(k,\nu';t)\right]\mathrm{d}\nu'\right)\\
&\qquad - s(\nu)\left(\hbar\Omega\hat{a}_{\bm{\alpha}}^\dagger(k,t) + \int_0^\infty \hbar g(\nu')\left[\hat{b}_{\bm{\alpha}}(k,\nu';t) + \hat{b}_{\bm{\alpha}}^\dagger(k,\nu';t)\right]\mathrm{d}\nu'\right)\\
&\qquad + \int_0^\infty u(\nu,\nu')\left(\hbar\nu'\hat{b}_{\bm{\beta}}(k,\nu';t) + \hbar g(\nu')\left[\hat{a}_{\bm{\beta}}(k,t) + \hat{a}_{\bm{\beta}}^\dagger(k,t)\right]\right)\mathrm{d}\nu'\\
&\qquad - \int_0^\infty v(\nu,\nu')\left(\hbar\nu'\hat{b}_{\bm{\beta}}^\dagger(k,\nu';t) + \hbar g(\nu')\left[\hat{a}_{\bm{\beta}}(k,t) + \hat{a}_{\bm{\beta}}^\dagger(k,t)\right]\right)\mathrm{d}\nu'.
\end{split}
\end{equation}
Noting that each operator is linearly independent from the others, we can see that Eqs. \eqref{eq:BcommH1} and \eqref{eq:BcommH2} form a system of four equations,
\begin{equation}\label{eq:hybridizationCoefficientsSystem}
\begin{split}
\hbar\nu q(\nu) &= \hbar\Omega q(\nu) + \int_0^\infty\left[u(\nu,\nu') - v(\nu,\nu')\right]\hbar g(\nu')\mathrm{d}\nu',\\
\hbar\nu s(\nu) &= -\hbar\Omega s(\nu) + \int_0^\infty\left[u(\nu,\nu') - v(\nu,\nu')\right]\hbar g(\nu')\mathrm{d}\nu',\\
\hbar\nu u(\nu,\nu') &= \hbar g(\nu')[q(\nu) - s(\nu)] + \hbar\nu'u(\nu,\nu'),\\
\hbar\nu v(\nu,\nu') &= \hbar g(\nu')[q(\nu) - s(\nu)] - \hbar\nu'v(\nu,\nu').
\end{split}
\end{equation}
From the first two equations of Eq. \eqref{eq:hybridizationCoefficientsSystem}, we can see that
\begin{equation}
s(\nu) = \frac{\nu - \Omega}{\nu + \Omega}q(\nu).
\end{equation}
This result can be plugged into the latter two lines to produce
\begin{equation}
\begin{split}
(\nu - \nu')u(\nu,\nu') &= \frac{2\Omega}{\nu + \Omega} g(\nu')q(\nu),\\
(\nu + \nu')v(\nu,\nu') &= \frac{2\Omega}{\nu + \Omega} g(\nu')q(\nu).\\
\end{split}
\end{equation}
It will be useful later in our calculations to extend the definitions of our coefficients to negative bath frequencies, such that either equation above needs to be carefully handled where $\nu = \pm\nu'$, respectively. In these cases, the equations give no information about $u(\nu,\nu')$ or $v(\nu,\nu')$, such that we must use the formal solution $f(x) = PV\left\{1/x\right\} + \mathcal{C}\delta(x)$ to the equation $xf(x) = 1$ in order to proceed. With $PV$ indicating the Cauchy principal value and $\mathcal{C}$ a complex nonzero constant with respect to $x$, we can map this solution onto our current problem by letting $\nu$ be fixed such that 
\begin{equation}
\begin{split}
u(\nu,\nu') &= \left[PV\left\{\frac{1}{\nu - \nu'}\right\} + [y(\nu) + r]\delta(\nu - \nu')\right]\frac{2\Omega}{\nu + \Omega}g(\nu')q(\nu),\\
v(\nu,\nu') &= \left[PV\left\{\frac{1}{\nu + \nu'}\right\} + [x(\nu) + p]\delta(\nu + \nu')\right]\frac{2\Omega}{\nu + \Omega}g(\nu')q(\nu).
\end{split}
\end{equation}
Here, $x(\nu)$ and $y(\nu)$ are complex functions of $\nu$ and $r$ and $p$ are complex constants. Plugging these coefficients back into either of the first two lines of Eq. \eqref{eq:hybridizationCoefficientsSystem} produces the condition
\begin{equation}
y(\nu) + r = \frac{\nu^2 - \Omega^2}{2\Omega g^2(\nu)} - \frac{1}{g^2(\nu)}\int_0^\infty\left[PV\left\{\frac{1}{\nu - \nu'}\right\} - PV\left\{\frac{1}{\nu + \nu'}\right\}\right]g^2(\nu')\;\mathrm{d}\nu'
\end{equation}
under the assumption that $g^2(-\nu) = -g^2(\nu)$. The simplest combination of constants that satisfies this condition is
\begin{equation}
\begin{split}
p &= 0,\\
x(\nu) &= 0,\\
y(\nu) + r &= \frac{\nu^2 - \Omega^2}{2\Omega g^2(\nu)} - \frac{1}{g^2(\nu)}\int_0^\infty\left[PV\left\{\frac{1}{\nu - \nu'}\right\} - PV\left\{\frac{1}{\nu + \nu'}\right\}\right]g^2(\nu')\;\mathrm{d}\nu'\\
% &= \frac{\nu^2 - \Omega^2}{2\Omega g^2(\nu)} - \frac{1}{g^2(\nu)}\int_0^\infty PV\left\{\frac{1}{\nu - \nu'}\right\}g^2(\nu')\;\mathrm{d}\nu' + \int_{-\nu' = 0}^{-\nu' = \infty}PV\left\{\frac{1}{\nu - \nu'}\right\}g^2(-\nu')\;\mathrm{d}(-\nu')\\
% &= \frac{\nu^2 - \Omega^2}{2\Omega g^2(\nu)} - \frac{1}{g^2(\nu)}\int_0^\infty PV\left\{\frac{1}{\nu - \nu'}\right\}g^2(\nu')\;\mathrm{d}\nu' - \int_{\nu' = -\infty}^{\nu' = 0}PV\left\{\frac{1}{\nu - \nu'}\right\}g^2(\nu')\;\mathrm{d}\nu'\\
&= \frac{\nu^2 - \Omega^2}{2\Omega g^2(\nu)} - \frac{1}{g^2(\nu)}\int_{-\infty}^\infty PV\left\{\frac{1}{\nu - \nu'}\right\}g^2(\nu')\;\mathrm{d}\nu',
\end{split}
\end{equation}
such that we can now represent $u(\nu,\nu')$ and $v(\nu,\nu')$ as products of known quantities times $q(\nu)$.

We are now left with the task of calculating $q(\nu)$. To do so, we can use the commutation relation
\begin{equation}
\begin{split}
\left[\hat{B}_{\bm{\beta}}(k,\nu;t),\hat{B}_{\bm{\beta}'}^\dagger(k',\nu';t)\right] &= \left[\left(q(\nu)\hat{a}_{\bm{\beta}}(k,t) + s(\nu)\hat{a}_{\bm{\beta}}^\dagger(k,t) + \int_0^\infty u(\nu,\omega)\hat{b}_{\bm{\beta}}(k,\omega;t)\;\mathrm{d}\omega\right.\right.\\
&\qquad + \left. \int_0^\infty v(\nu,\omega)\hat{b}_{\bm{\beta}}^\dagger(k,\omega;t)\;\mathrm{d}\omega \right),\left(q^*(\nu')\hat{a}_{\bm{\beta}'}^\dagger(k',\nu') + s^*(\nu')\hat{a}_{\bm{\beta}'}(k',\nu')  \vphantom{\int_0^\infty}\right.\\
&\qquad + \left.\left.\int_0^\infty\left[u^*(\nu',\omega')\hat{b}_{\bm{\beta}'}^\dagger(k',\omega';t) + v^*(\nu',\omega')\hat{b}_{\bm{\beta}'}(k',\omega';t)\right]\;\mathrm{d}\omega'\right)\right]\\
% &= [q(\nu)q^*(\nu') - s(\nu)s^*(\nu')]\delta(k - k')\delta_{\bm{\beta}\bm{\beta}'}\\
% &\qquad + \int_0^\infty\int_0^\infty [u(\nu,\omega)u^*(\nu',\omega') - v(\nu,\omega)v^*(\nu',\omega')]\delta(k - k')\delta(\omega - \omega')\delta_{\bm{\beta}\bm{\beta}'}\;\mathrm{d}\omega\,\mathrm{d}\omega'\\
&= [q(\nu)q^*(\nu') - s(\nu)s^*(\nu')]\delta(k - k')\delta_{\bm{\beta}\bm{\beta}'}\\
&\qquad + \delta(k - k')\delta_{\bm{\beta}\bm{\beta}'}\int_0^\infty\left[u(\nu,\omega)u^*(\nu',\omega) - v(\nu,\omega)v^*(\nu',\omega)\right]\mathrm{d}\omega
\end{split}
\end{equation}
Since we have demanded that
\begin{equation}
\left[\hat{B}_{\bm{\beta}}(k,\nu;t),\hat{B}_{\bm{\beta}'}^\dagger(k',\nu';t)\right] = \delta(k - k')\delta(\nu - \nu')\delta_{\bm{\beta}\bm{\beta}'},
\end{equation}
we can say
\begin{equation}\label{eq:prefactorAlgebra1}
\begin{split}
\delta(\nu - \nu') &= q(\nu)q^*(\nu') - s(\nu)s^*(\nu') + \int_0^\infty\left[u(\nu,\omega)u
^*(\nu',\omega) - v(\nu,\omega)v^*(\nu',\omega)\right]\mathrm{d}\omega\\
% &= q(\nu)q(\nu')\left[1 - \frac{(\nu - \Omega)(\nu' - \Omega)}{(\nu + \Omega)(\nu' + \Omega)}\right] + \int_0^\infty\left[PV\left\{\frac{1}{\nu - \omega}\right\} + [y(\nu) + r]\delta(\nu - \omega)\right]\frac{2\Omega}{\nu + \Omega}g(\omega)q(\nu)\\
% &\qquad\qquad\times \left[PV\left\{\frac{1}{\nu' - \omega}\right\} + [y(\nu') + r]^*\delta(\nu' - \omega)\right]\frac{2\Omega}{\nu' + \Omega}g(\omega)q(\nu')\;\mathrm{d}\omega\\
% &\qquad - \int_0^\infty PV\left\{\frac{1}{\nu + \omega}\right\}\frac{2\Omega}{\nu + \Omega}g(\omega)q(\nu)PV\left\{\frac{1}{\nu' + \omega}\right\}\frac{2\Omega}{\nu' + \Omega}g(\omega)q(\nu')\;\mathrm{d}\omega\\
% &= q(\nu)q^*(\nu')\left(\frac{2\Omega(\nu + \nu')}{(\nu + \Omega)(\nu' + \Omega)} + \int_0^\infty PV\left\{\frac{1}{(\nu - \omega)(\nu' - \omega)}\right\}\frac{4\Omega^2}{(\nu + \Omega)(\nu' + \Omega)}g^2(\omega)\;\mathrm{d}\omega\right.\\
% &\qquad - \int_0^\infty PV\left\{\frac{1}{(\nu + \omega)(\nu' + \omega)}\right\}\frac{4\Omega^2}{(\nu + \Omega)(\nu' + \Omega)}g^2(\omega)\;\mathrm{d}\omega\\
% &\qquad +  PV\left\{\frac{1}{\nu - \nu'}\right\}\frac{2\Omega}{\nu + \Omega}g(\nu')[y(\nu') + r]^*\frac{2\Omega}{\nu' + \Omega}g(\nu')\\
% &\qquad + PV\left\{\frac{1}{\nu' - \nu}\right\}\frac{2\Omega}{\nu' + \Omega}g(\nu)[y(\nu) + r]\frac{2\Omega}{\nu + \Omega}g(\nu)\\
% &\qquad\left. + \frac{4\Omega^2}{(\nu + \Omega)(\nu' + \Omega)}[y(\nu) + r][y(\nu') + r]^*g^2(\nu)\delta(\nu' - \nu) \vphantom{\int_0^\infty} \right)\\
% &= q(\nu)q^*(\nu')\left(\frac{2\Omega(\nu + \nu')}{(\nu + \Omega)(\nu' + \Omega)} + \frac{4\Omega^2g^2(\nu)}{(\nu + \Omega)(\nu' + \Omega)}[y(\nu) + r][y(\nu') + r]^*\delta(\nu - \nu') \vphantom{\int_0^\infty} \right.\\
% &\qquad + \int_0^\infty \left(PV\left\{\frac{1}{(\nu - \omega)(\nu' - \omega)}\right\} - PV\left\{\frac{1}{(\nu + \omega)(\nu' + \omega)}\right\}\right)\frac{4\Omega^2g^2(\omega)}{(\nu + \Omega)(\nu' + \Omega)}\;\mathrm{d}\omega\\
% &\qquad\left. + \frac{4\Omega^2}{(\nu + \Omega)(\nu' + \Omega)}\left[PV\left\{\frac{1}{\nu' - \nu}\right\}g^2(\nu)[y(\nu) + r] + PV\left\{\frac{1}{\nu - \nu'}\right\}g^2(\nu')[y(\nu') + r]^*\right] \vphantom{\int_0^\infty} \right)\\
&= q(\nu)q^*(\nu')\frac{4\Omega^2}{(\nu + \Omega)(\nu' + \Omega)}\left(\frac{\nu + \nu'}{2\Omega} + g^2(\nu)[y(\nu) + r][y(\nu') + r]^*\delta(\nu - \nu') \vphantom{\int_0^\infty} \right.\\
&\qquad +  \int_{-\infty}^\infty PV\left\{\frac{1}{(\nu - \omega)(\nu' - \omega)}\right\}g^2(\omega)\;\mathrm{d}\omega\\
&\qquad\left. + \left[PV\left\{\frac{1}{\nu' - \nu}\right\}g^2(\nu)[y(\nu) + r] + PV\left\{\frac{1}{\nu - \nu'}\right\}g^2(\nu')[y(\nu') + r]^*\right] \vphantom{\int_0^\infty} \right)
\end{split}
\end{equation}
To simplify, we can note that
\begin{equation}
PV\left\{\frac{1}{x}\right\} = \lim_{\epsilon\to0}\frac{x}{x^2 + \epsilon^2}
\end{equation}
such that $PV\{-1/x\} = -PV\{1/x\}$. Further, noting that
\begin{equation}
\begin{split}
\lim_{\epsilon\to0}\frac{\epsilon}{x^2 + \epsilon^2} &= \pi\delta(x),\\
\lim_{\epsilon\to0}\frac{x^2}{x^2 + \epsilon^2} &= 1,
\end{split}
\end{equation}
we can see that
\begin{equation}
\begin{split}
&\left[PV\left\{\frac{1}{\nu' - \nu}\right\}g^2(\nu)[y(\nu) + r] + PV\left\{\frac{1}{\nu - \nu'}\right\}g^2(\nu')[y(\nu') + r]^*\right]\\
% &= PV\left\{\frac{1}{\nu' - \nu}\right\}\left(\frac{\nu^2 - \Omega^2}{2\Omega} - \int_{-\infty}^\infty PV\left\{\frac{1}{\nu - \omega}\right\}g^2(\omega)\;\mathrm{d}\omega\right)\\
% &\qquad + PV\left\{\frac{1}{\nu - \nu'}\right\}\left(\frac{\nu'^2 - \Omega^2}{2\Omega} - \int_{-\infty}^\infty PV\left\{\frac{1}{\nu' - \omega}\right\}g^2(\omega)\;\mathrm{d}\omega\right)\\
% &= PV\left\{\frac{1}{\nu' - \nu}\right\}\frac{\nu^2 - \nu'^2}{2\Omega} - PV\left\{\frac{1}{\nu' - \nu}\right\}\int_{-\infty}^\infty\left(PV\left\{\frac{1}{\nu - \omega}\right\} - PV\left\{\frac{1}{\nu' - \omega}\right\}\right)g^2(\omega)\;\mathrm{d}\omega\\
% &= -\lim_{\epsilon\to0}\frac{\nu' - \nu}{(\nu' - \nu)^2 + \epsilon^2}\frac{(\nu' - \nu)(\nu' + \nu)}{2\Omega}\\
% &\qquad - \lim_{\epsilon\to0}\frac{\nu' - \nu}{(\nu' - \nu)^2 + \epsilon^2}\int_{-\infty}^\infty\lim_{\epsilon'\to0}\left(\frac{\nu - \omega}{(\nu - \omega)^2 + \epsilon'^2} - \frac{\nu' - \omega}{(\nu' - \omega)^2 + \epsilon'^2}\right)g^2(\omega)\;\mathrm{d}\omega\\
% &= -\frac{\nu + \nu'}{2\Omega} - \lim_{\epsilon\to0}\frac{\nu' - \nu}{(\nu' - \nu)^2 + \epsilon^2}\int_{-\infty}^\infty\lim_{\epsilon'\to0}\left(\frac{(\nu - \omega)[(\nu' - \omega)^2 + \epsilon'^2]}{[(\nu - \omega)^2 + \epsilon'^2][(\nu' - \omega)^2 + \epsilon'^2]}\right.\\
% &\qquad\left.- \frac{(\nu' - \omega)[(\nu - \omega)^2 + \epsilon'^2]}{[(\nu - \omega)^2 + \epsilon'^2][(\nu' - \omega)^2 + \epsilon'^2}\right)g^2(\omega)\;\mathrm{d}\omega\\
% &= -\frac{\nu + \nu'}{2\Omega} - \lim_{\epsilon\to 0}\frac{\nu' - \nu}{(\nu' - \nu)^2 + \epsilon^2}\int_{-\infty}^\infty\lim_{\epsilon'\to0}\left(\frac{(\nu - \omega)(\nu' - \omega)(\nu' - \nu) - \epsilon'^2(\nu' - \nu)}{[(\nu - \omega)^2 + \epsilon'^2][(\nu' - \omega)^2 + \epsilon'^2]}\right)g^2(\omega)\;\mathrm{d}\omega\\
% &= -\frac{\nu + \nu'}{2\Omega} - \int_{-\infty}^\infty PV\left\{\frac{1}{(\nu - \omega)(\nu' - \omega)}\right\}g^2(\omega)\;\mathrm{d}\omega + \int_{-\infty}^\infty\pi^2\delta(\nu - \omega)\delta(\nu' - \omega)g^2(\omega)\;\mathrm{d}\omega\\
&= -\frac{\nu + \nu'}{2\Omega} - \int_{-\infty}^\infty PV\left\{\frac{1}{(\nu - \omega)(\nu' - \omega)}\right\}g^2(\omega)\;\mathrm{d}\omega + \pi^2\delta(\nu - \nu')g^2(\nu).
\end{split}
\end{equation}
Plugging this result back into Eq. \eqref{eq:prefactorAlgebra1}, we find that
\begin{equation}
\delta(\nu - \nu') = q(\nu)q^*(\nu')\frac{4\Omega^2}{(\nu + \Omega)(\nu' + \Omega)}\left[g^2(\nu')[y(\nu') + r][y(\nu) + r]^*\delta(\nu - \nu') + \pi^2\delta(\nu - \nu')g^2(\nu) \vphantom{\sum} \right].
\end{equation}
This implies that
\begin{equation}
\begin{split}
1 = |q(\nu)|^2(|y(\nu) + r|^2 + \pi^2)\frac{4\Omega^2g^2(\nu)}{(\nu + \Omega)^2}.
\end{split}
\end{equation}
The most advantageous value of $r$ will turn out to be $r = 0$ such that
\begin{equation}
|y(\nu) + r|^2 + \pi^2 = |y(\nu) - \mathrm{i}\pi|^2
\end{equation}
and
\begin{equation}
q(\nu) = \frac{\nu + \Omega}{2\Omega g(\nu)[y(\nu) - \mathrm{i}\pi]}.
\end{equation}
Further, with
\begin{equation}
PV\left\{\frac{1}{x}\right\} = \lim_{\epsilon\to0}\frac{1}{x - \mathrm{i}\epsilon} - \mathrm{i}\pi\delta(x),
\end{equation}
we can see that
\begin{equation}
\begin{split}
y(\nu) - \mathrm{i}\pi &= \frac{\nu^2 - \Omega^2}{2\Omega g^2(\nu)} - \frac{1}{g^2(\nu)}\int_{-\infty}^\infty\left[\lim_{\epsilon\to0}\frac{1}{\nu - \nu' - \mathrm{i}\epsilon} - \mathrm{i}\pi\delta(\nu - \nu')\right]g^2(\nu')\;\mathrm{d}\nu' - \mathrm{i}\pi\\
% &= \frac{\nu^2 - \Omega^2}{2\Omega g^2(\nu)} - \frac{1}{g^2(\nu)}\int_{-\infty}^\infty\lim_{\epsilon\to0}\frac{1}{\nu - \nu' - \mathrm{i}\epsilon}g^2(\nu')\;\mathrm{d}\nu' - \mathrm{i}\pi + \mathrm{i}\pi\\
&= \frac{\nu^2 - \Omega^2}{2\Omega g^2(\nu)} - \frac{1}{g^2(\nu)}\int_{-\infty}^\infty\lim_{\epsilon\to0}\frac{1}{\nu - \nu' - \mathrm{i}\epsilon}g^2(\nu')\;\mathrm{d}\nu'.
\end{split}
\end{equation}
We can then define a new function
\begin{equation}
z(\nu) = 1 + \frac{2}{\Omega}\int_{-\infty}^\infty \frac{g^2(\omega)}{\nu - \omega - \mathrm{i}\epsilon}\mathrm{d}\omega
\end{equation}
such that
\begin{equation}
y(\nu) - \mathrm{i}\pi = \frac{\nu^2 - \Omega^2z(\nu)}{2\Omega g^2(\nu)},
\end{equation}
which allows us to finally define our expansions coefficients:
\begin{equation}
\begin{split}
q(\nu) &= \frac{(\nu + \Omega)g(\nu)}{\nu^2 - \Omega^2 z(\nu)},\\
s(\nu) &= \frac{(\nu - \Omega)g(\nu)}{\nu^2 - \Omega^2 z(\nu)},\\
u(\nu,\nu') &= \delta(\nu - \nu') + \frac{2\Omega g(\nu)}{\nu^2 - \Omega^2z(\nu)}\frac{g(\nu')}{\nu - \nu' - \mathrm{i}\epsilon},\\
v(\nu,\nu') &= \frac{2\Omega g(\nu)}{\nu^2 - \Omega^2z(\nu)}PV\left\{\frac{g(\nu')}{\nu + \nu'}\right\}.
\end{split}
\end{equation}

Finally, we also need to be able to represent our unhybridized coordinates in terms of our hybridized ones. To do this, we can see that
\begin{equation}
\begin{split}
\int_0^\infty&\left[q^*(\nu)\hat{B}_{\bm{\beta}}(k,\nu;t) - s(\nu)\hat{B}_{\bm{\beta}}^\dagger(k,\nu;t)\right]\mathrm{d}\nu = \hat{a}_{\bm{\beta}}(k,t)\int_0^\infty\left(|q(\nu)|^2 - |s(\nu)|^2\right)\mathrm{d}\nu\\
& + \hat{a}^\dagger_{\bm{\beta}}(k,t)\int_0^\infty\left[q^*(\nu)s(\nu) - s(\nu)q^*(\nu)\right]\mathrm{d}\nu + \int_0^\infty\int_0^\infty\left[q^*(\nu)u(\nu,\nu') - s(\nu)v^*(\nu,\nu')\right]\hat{b}_{\bm{\beta}}(k,\nu';t)\;\mathrm{d}\nu'\,\mathrm{d}\nu\\
& + \int_{0}^\infty\int_0^\infty\left[q^*(\nu)v(\nu,\nu') - s(\nu)u^*(\nu,\nu')\right]\hat{b}_{\bm{\beta}}^\dagger(k,\nu';t)\;\mathrm{d}\nu'\,\mathrm{d}\nu,
\end{split}
\end{equation}
wherein the terms under integration must go to zero independently for the linear combination of hybridized coordinates to return an expression solely dependent on the operator $\hat{a}_{\bm{\beta}}(k,t)$. We can first immediately see that $q^*(\nu)s(\nu) - s(\nu)q^*(\nu) = 0$. Second,
\begin{equation}\label{eq:consistencyCheckIntegral1}
\begin{split}
\int_0^\infty\left(|q(\nu)|^2 - |s(\nu)|^2\right)\mathrm{d}\nu &= \int_0^\infty\left[(\nu + \Omega)^2 - (\nu - \Omega)^2\right]\frac{g^2(\nu)}{[\nu^2 - \Omega^2z(\nu)]^2}\\
&= \int_0^\infty\frac{4\nu\Omega g^2(\nu)}{|\nu^2 - \Omega^2z(\nu)|^2}\;\mathrm{d}\nu.
\end{split}
\end{equation}
From here, we can note that integrals of the type in  Eq. \eqref{eq:consistencyCheckIntegral1} are often done via contour integration using a closed contour comprised of the real line and a semicircle of infinite radius that lies in the complex plane. Before we can construct such a contour, however, it is useful to first analyze the pole structure of the function $\nu^2 - \Omega^2z(\nu)$. Letting $\nu = a + \mathrm{i}b$, we can see that
\begin{equation}\label{eq:complexRootCheck}
\begin{split}
(a + \mathrm{i}b)^2 - \Omega^2z(a + \mathrm{i}b) &= a^2 + 2\mathrm{i}ab - b^2 - \Omega^2 - 2\Omega\int_{-\infty}^\infty \frac{g^2(\omega)}{a + \mathrm{i}b - \omega - \mathrm{i}\epsilon}\;\mathrm{d}\omega\\
% &= a^2 - b^2 - \Omega^2 + 2\mathrm{i}ab - 2\Omega\int_{-\infty}^\infty \frac{g^2(\omega)([a - \omega] - \mathrm{i}[b-\epsilon])}{(a - \omega)^2 + (b - \epsilon)^2}\;\mathrm{d}\omega\\
&= a^2 - b^2 - \Omega^2 - 2\Omega\int_{-\infty}^\infty \frac{g^2(\omega)(a - \omega)}{(a - \omega)^2 + (b - \epsilon)^2}\;\mathrm{d}\omega\\
&\qquad + \mathrm{i}(b - \epsilon)\left(2a + 2\Omega\int_{-\infty}^\infty \frac{g^2(\omega)}{(a - \omega)^2 + (b - \epsilon)^2}\;\mathrm{d}\omega\right).
\end{split}
\end{equation}
There are two cases to consider here:
\begin{itemize}
    \item{
    First, we can see that the imaginary part of $\nu^2 - \Omega^2z(\nu)$ is zero when $b = \epsilon$. In this case, the function has a zero if there exists a value $a$ such that
    \begin{equation}
    a^2 - \Omega^2 - \epsilon^2 - 2\Omega\int_{-\infty}^\infty\frac{g^2(\omega)}{a - \omega}\;\mathrm{d}\omega = 0.
    \end{equation}
    For functions $g^2(\omega)$ and frequencies $a$ defined such that $g^2(a)\neq0$, the integral above is divergent and $\nu^2 - \Omega^2z(\nu)$ has no zeros with $b = \epsilon$. Otherwise, there can exist any number of zeros, depending on the functional form of $g^2(\omega)$. An important case considered by Huttner and Barnett\cite{huttner1992quantization} is the case in which $g^2(\omega)$ is only zero at $\omega = 0$. In this case, with $g^2(a) = 0$ only at $a = 0$, and only one zero of $\nu^2 - \Omega^2z(\nu)$ exists at $a + \mathrm{i}b = \mathrm{i}\epsilon$.
    }
    \item{
    Second, in the case that $b\neq\epsilon$, we can see that the imaginary part of the RHS of Eq. \eqref{eq:complexRootCheck} can be zero if the quantity inside the parentheses is zero. We will assume $b\neq\epsilon$ in the following logic. Because $g(\omega)\sim\omega\tilde{v}^2(\omega)$ and $\tilde{v}^2(\omega)$ is a nonnegative even function, the integrand of this term is always positive for positive $\omega$ and negative for negative $\omega$. Moreover, because the area under $1/[(a - \omega)^2 + (b - \epsilon)^2]$ is larger on the same side of the origin as the sign of $a$, the integral is always positive for positive $a$ and negative for negative $a$. The first term in the parentheses, $2a$, has the same signs in the same regions of the real line. Therefore, the imaginary part of $\nu^2 - \Omega^2z(\nu)$ can only be zero when $a = \mathrm{Re}\{\nu\} = 0$, such that $\nu^2 - \Omega^2z(\nu)$ can only have zeros along the imaginary axis. Further noting that $\int_{-\infty}^\infty g^2(\omega)/[\omega^2 + (b - \epsilon)^2]\;\mathrm{d}\omega \to 0$ as $\epsilon\to0$ due to the odd and even parities of the numerator and denominator, respectively, we can see that, in the limit $a\to0$, our function becomes
    \begin{equation}
    (\mathrm{i}b)^2 - \Omega^2z(\mathrm{i}b) = -(b^2 + \Omega^2) + 2\Omega\int_{-\infty}^\infty \frac{\omega g^2(\omega)}{\omega^2 + (b - \epsilon)^2}\;\mathrm{d}\omega.
    \end{equation}
    Because the first term on the RHS ($-[b^2 + \Omega^2]$) is strictly negative and the second is strictly positive, our function always has zeros at $b = \pm b_0$ except in the case where the two terms on the RHS never have equal magnitudes at any $b$. Because $b^2$ is strictly increasing with $|b|$ and $1/(\omega^2 + [b - \epsilon]^2)$ is strictly decreasing, the second term on the RHS can never be greater than the first for all $b$. Therefore, the only condition that guarantees the nonexistence of zeros is the condition that the sceond term on the RHS is always \textit{less} than the first, i.e.
    \begin{equation}
    2\Omega\int_{-\infty}^\infty\frac{\omega g^2(\omega)}{\omega^2 + (b - \epsilon)^2}\;\mathrm{d}\omega < b^2 + \Omega^2
    \end{equation}
    for all $b$. We can guarantee this condition by guaranteeing it is satisfied when the LHS is maximized and the RHS is minimized, which occurs at $b = 0$. Therefore, we can guarantee that $\nu^2 - \Omega^2z(\nu)$ has no zeros in the complex plane (other than $b = \epsilon$) only when 
    \begin{equation}\label{eq:consistencyCondition1}
    \begin{split}
    2\int_{-\infty}^\infty \frac{\omega g^2(\omega)}{\omega^2 + \epsilon^2}\;\mathrm{d}\omega &= 4\int_0^\infty PV\left\{\frac{g^2(\omega)}{\omega}\right\}\mathrm{d}\omega < \Omega.
    \end{split}
    \end{equation}
    }
\end{itemize}

Assuming that the second condition on $g(\nu)$ is satisfied, using contour integration to simplify Eq. \eqref{eq:consistencyCheckIntegral1} becomes much simpler. In more detail, since a contour comprised of the real line and a great semicircle in the lower complex half-plane contains no poles, the residue theorem tells us that
\begin{equation}
\int_{-\infty}^\infty\frac{4\Omega\nu g^2(\nu)}{|\nu^2 - \Omega ^2z(\nu)|^2}\;\mathrm{d}\nu = -\int_{C_R^-}\frac{4\Omega\nu g^2(\nu)}{|\nu^2 - \Omega ^2z(\nu)|^2}\;\mathrm{d}\nu
\end{equation}
with $C_R^-$ symbolizing the great circle portion of the contour. For our hybridization scheme to be consistent, we require the integral of Eq. \eqref{eq:consistencyCheckIntegral1} to be equal to one. Conveniently, we can see that the method of partial fractions gives
\begin{equation}
\frac{\mathrm{i}}{[x - (a + \mathrm{i}b)][x - (a - \mathrm{i}b)]} = -\frac{1}{2b}\frac{1}{x - (a + \mathrm{i}b)} + \frac{\mathrm{i}}{2b}\frac{1}{x - (a - \mathrm{i}b)}
\end{equation}
such that
\begin{equation}
\begin{split}
\frac{1}{|\nu^2 - \Omega^2z(\nu)|^2} &= \frac{1}{[\nu^2 - \Omega^2z(\nu)][\nu^2 - \Omega^2z^*(\nu)]}\\
% & = -\frac{\mathrm{i}}{2\Omega^2\mathrm{Im}\{z(\nu)\}}\frac{1}{\nu^2 - \Omega^2z(\nu)} + \frac{\mathrm{i}}{2\Omega^2\mathrm{Im}\{z(\nu)\}}\frac{1}{\nu^2 - \Omega^2z^*(\nu)}\\
% &= -\frac{\mathrm{i}}{2\Omega^2}\left(\frac{2\pi}{\Omega}g^2(\nu)\right)^{-1}\frac{1}{\nu^2 - \Omega^2z(\nu)} + -\frac{\mathrm{i}}{2\Omega^2}\left(\frac{2\pi}{\Omega}g^2(\nu)\right)^{-1}\frac{1}{\nu^2 - \Omega^2z^*(\nu)}\\
& = -\frac{\mathrm{i}}{4\pi\Omega g^2(\nu)}\frac{1}{\nu^2 - \Omega^2z(\nu)} + \frac{\mathrm{i}}{4\pi\Omega g^2(\nu)}\frac{1}{\nu^2 - \Omega^2z^*(\nu)}.
\end{split}
\end{equation}
Therefore, using the fact that $z(-\nu) = z^*(\nu)$, as is clear from the expansion
\begin{equation}
\begin{split}
z(\nu) &= 1 + \frac{2}{\Omega}\int_{-\infty}^\infty \frac{g^2(\omega)}{\nu - \omega - \mathrm{i}\epsilon}\mathrm{d}\omega\\
% &= 1 + \frac{2}{\Omega}\left[\int_0^\infty \frac{g^2(\omega)}{\nu - \omega - \mathrm{i}\epsilon}\;\mathrm{d}\omega + \int_{-\infty}^0 \frac{g^2(\omega)}{\nu - \omega - \mathrm{i}\epsilon}\;\mathrm{d}\omega\right]\\
% &= 1 + \frac{2}{\Omega}\left[\int_0^\infty \frac{g^2(\omega)}{\nu - \omega - \mathrm{i}\epsilon}\;\mathrm{d}\omega + \int_{-\omega = -\infty}^{-\omega = 0} \frac{g^2(-\omega)}{\nu + \omega - \mathrm{i}\epsilon}\;\mathrm{d}(-\omega)\right]\\
% &= 1 + \frac{2}{\Omega}\left[\int_0^\infty \frac{g^2(\omega)}{\nu - \omega - \mathrm{i}\epsilon}\;\mathrm{d}\omega - \int^{\omega = \infty}_{\omega = 0} \frac{g^2(\omega)}{\nu + \omega - \mathrm{i}\epsilon}\;\mathrm{d}\omega\right]\\
&= 1 + \frac{2}{\Omega}\left[\int_0^\infty \frac{g^2(\omega)}{\nu - \omega - \mathrm{i}\epsilon}\;\mathrm{d}\omega + \int^{\infty}_{0} \frac{g^2(\omega)}{-\nu - \omega + \mathrm{i}\epsilon}\;\mathrm{d}\omega\right],
\end{split}
\end{equation}
we can say
\begin{equation}
\begin{split}
\int_{0}^\infty\frac{4\Omega\nu g^2(\nu)}{|\nu^2 - \Omega^2z(\nu)|^2}\;\mathrm{d}\nu &= -\frac{\mathrm{i}}{\pi}\int_0^\infty\left(\frac{\nu}{\nu^2 - \Omega^2z(\nu)} - \frac{\nu}{\nu^2 - \Omega^2z^*(\nu)}\right)\mathrm{d}\nu\\
% &= -\frac{\mathrm{i}}{\pi}\int_0^\infty\frac{\nu}{\nu^2 - \Omega^2z(\nu)}\;\mathrm{d}\nu + \frac{\mathrm{i}}{\pi}\int_{-\nu = 0}^{-\nu = \infty}\frac{(-\nu)}{(-\nu)^2 - \Omega^2z^*(-\nu)}\;\mathrm{d}(-\nu)\\
% &= -\frac{\mathrm{i}}{\pi}\int_0^\infty\frac{\nu}{\nu^2 - \Omega^2z(\nu)}\;\mathrm{d}\nu - \frac{\mathrm{i}}{\pi}\int_{-\infty}^0\frac{\nu}{\nu^2 - \Omega^2z(\nu)}\;\mathrm{d}\nu\\
% &= -\frac{\mathrm{i}}{\pi}\int_{-\infty}^\infty\frac{\nu}{\nu^2 - \Omega^2z(\nu)}\;\mathrm{d}\nu\\
% &= \frac{\mathrm{i}}{\pi}\int_{C_R^-}\frac{\nu}{\nu^2 - \Omega^2z(\nu)}\;\mathrm{d}\nu\\
&= \lim_{R\to\infty}\frac{\mathrm{i}}{\pi}\int_{-\pi}^0\frac{R\mathrm{e}^{\mathrm{i}\theta}}{R^2\mathrm{e}^{2\mathrm{i}\theta} - \Omega^2z(R\mathrm{e}^{\mathrm{i}\theta})}R\mathrm{i}\mathrm{e}^{\mathrm{i}\theta}\;\mathrm{d}\theta\\
% &= -\frac{1}{\pi}(-\pi)\\
&= 1,
\end{split}
\end{equation}
wherein we have noticed that $|z[R\exp(\mathrm{i}\theta)]|\ll R^2$ for large $R$. Therefore, as long as Eq. \eqref{eq:consistencyCondition1} is satisfied, we have
\begin{equation}
\int_{0}^\infty\left(|q_\perp(\nu)|^2 - |s_\perp(\nu)|^2\right)\mathrm{d}\nu = 1.
\end{equation}

Next, we can see that
\begin{equation}
\begin{split}
\int_0^\infty&\left[q^*(\nu)u(\nu,\nu') - s(\nu)v^*(\nu,\nu')\right]\mathrm{d}\nu = \int_0^\infty\left(\left[\delta(\nu - \nu') + \frac{2\Omega g(\nu)}{\nu^2 - \Omega^2z(\nu)}\frac{g(\nu')}{\nu - \nu' - \mathrm{i}\epsilon}\right]\frac{(\nu + \Omega)g(\nu)}{\nu^2 - \Omega^2z^*(\nu)}\right.\\
&\qquad\left. - \frac{(\nu - \Omega)g(\nu)}{\nu^2 - \Omega^2z(\nu)}\frac{2\Omega g(\nu)}{\nu^2 - \Omega^2z^*(\nu)}PV\left\{\frac{g(\nu')}{\nu + \nu'}\right\}\right)\mathrm{d}\nu\\[0.5em]
&= \\
&= 0
\end{split}
\end{equation}
and
\begin{equation}
\begin{split}
\int_0^\infty&\left[q^*(\nu)v(\nu,\nu') - s(\nu)u^*(\nu,\nu')\right]\mathrm{d}\nu = \int_0^\infty\left(\frac{(\nu + \Omega)g(\nu)}{\nu^2 - \Omega^2z^*(\nu)}\frac{2\Omega g(\nu)}{\nu^2 - \Omega^2z(\nu)}PV\left\{\frac{g(\nu')}{\nu + \nu'}\right\}\right.\\
&\qquad\left. - \frac{(\nu - \Omega)g(\nu)}{\nu^2 - \Omega^2z(\nu)}\left[\delta(\nu - \nu') + \frac{2\Omega g(\nu)}{\nu^2 - \Omega^2z^*(\nu)}\frac{g(\nu')}{\nu - \nu' + \mathrm{i}\epsilon}\right]\right)\mathrm{d}\nu\\[0.5em]
&= \\
&= 0
\end{split}
\end{equation}
such that
\begin{equation}
\int_0^\infty\left[q^*(\nu)\hat{B}_{\bm{\beta}}(k,\nu;t) - s(\nu)\hat{B}_{\bm{\beta}}^\dagger(k,\nu;t)\right]\mathrm{d}\nu = \hat{a}_{\bm{\beta}}(k,t).
\end{equation}

We can repeat this process to find the representation of the bath operator $\hat{b}_{\bm{\beta}}(k,\nu;t)$ in terms of the hybrid operators. Beginning with
\begin{equation}
\begin{split}
\int_0^\infty&\left[u^*(\nu',\nu)\hat{B}_{\bm{\beta}}(k,\nu';t) - v(\nu',\nu)\hat{B}_{\bm{\beta}}^\dagger(k,\nu';t)\right]\mathrm{d}\nu' = \hat{a}_{\bm{\beta}}(k,t)\int_0^\infty\left[q(\nu')u^*(\nu',\nu) - s^*(\nu')v(\nu',\nu)\right]\mathrm{d}\nu'\\
&\qquad + \hat{a}_{\bm{\beta}}^\dagger(k,t)\int_0^\infty\left[s(\nu')u^*(\nu',\nu) - q^*(\nu')v(\nu',\nu)\right]\mathrm{d}\nu'\\
&\qquad + \int_0^\infty\int_0^\infty\left[u^*(\nu',\nu)u(\nu',\omega) - v(\nu',\nu)v^*(\nu',\omega)\right]\hat{b}_{\bm{\beta}}(k,\omega;t)\;\mathrm{d}\nu'\,\mathrm{d}\omega\\
&\qquad + \int_0^\infty\int_0^\infty\left[u^*(\nu',\nu)v(\nu',\omega) - v(\nu',\nu)u^*(\nu',\omega)\right]\hat{b}_{\bm{\beta}}^\dagger(k,\omega;t)\;\mathrm{d}\nu'\,\mathrm{d}\omega,
\end{split}
\end{equation}
we can see that the integral prefactors of the first two terms on the RHS above are complex conjugates or reverses of integrals we already know to be zero and are thus themselves zero. The fourth term on the RHS is also zero, as is shown by
\begin{equation}
\begin{split}
\int_0^\infty&\left[u^*(\nu',\nu)v(\nu',\omega) - v(\nu',\nu)u^*(\nu',\omega)\right]\mathrm{d}\nu' = \\
&= \\
&= 0.
\end{split}
\end{equation}
Finally, the integral prefactor of the third term gives
\begin{equation}
\begin{split}
\int_0^\infty&\left[u^*(\nu',\omega)u(\nu',\omega) - v(\nu',\nu)v^*(\nu',\omega)\right]\mathrm{d}\nu' = \\
&= \\
&= \delta(\nu - \omega)
\end{split}
\end{equation}
such that
\begin{equation}
\begin{split}
\int_0^\infty\left[u^*(\nu',\nu)\hat{B}_{\bm{\beta}}(k,\nu';t) - v(\nu',\nu)\hat{B}_{\bm{\beta}}^\dagger(k,\nu';t)\right]\mathrm{d}\nu' &= \int_0^\infty\delta(\nu - \omega)\hat{b}_{\bm{\beta}}(k,\omega;t)\;\mathrm{d}\omega\\
&= \hat{b}_{\bm{\beta}}(k,\nu;t).
\end{split}
\end{equation}








\subsection{Hybridized Operator Expansion Coefficients}

The first hybridization step from the main text produces expansion coefficients
\begin{equation}
\begin{split}
q_{\perp,\parallel}(\nu) &= \frac{\nu + \Omega_{\perp,\parallel}}{2}\frac{V_{\perp,\parallel}(\nu)}{\nu^2 - \Omega_{\perp,\parallel}^2z_{\perp,\parallel}(\nu)},\\
s_{\perp,\parallel}(\nu) &= \frac{\nu - \Omega_{\perp,\parallel}}{2}\frac{V_{\perp,\parallel}(\nu)}{\nu^2 - \Omega_{\perp,\parallel}^2z_{\perp,\parallel}(\nu)},\\
u_{\perp,\parallel}(\nu,\nu') &= \delta(\nu - \nu') + \frac{\Omega_{\perp,\parallel}}{2}\frac{V_{\perp,\parallel}(\nu)}{\nu^2 - \Omega_{\perp,\parallel}^2z_{\perp,\parallel}(\nu)}\frac{V_{\perp,\parallel}(\nu')}{\nu - \nu' - \mathrm{i}\epsilon},\\
v_{\perp,\parallel}(\nu,\nu') &= \frac{\Omega_{\perp,\parallel}}{2}\frac{V_{\perp,\parallel}(\nu)}{\nu^2 - \Omega^2_{\perp,\parallel}z_{\perp,\parallel}(\nu)}PV\left\{\frac{V_{\perp,\parallel}(\nu)}{\nu + \nu'}\right\},
\end{split}
\end{equation}
with the substitutions $g(\nu)\to V_{\perp,\parallel}(\nu)/2$, $\Omega\to\Omega_{\perp,\parallel}$, and
\begin{equation}
z(\nu) \to z_{\perp,\parallel}(\nu) = 1 + \frac{1}{2\Omega_{\perp,\parallel}}\int_{-\infty}^\infty\frac{V_{\perp,\parallel}^2(\omega)}{\nu - \omega - \mathrm{i}\epsilon}\;\mathrm{d}\omega.
\end{equation}
The second step uses the substitutions $\Omega\to\tilde{k}c$,
\begin{equation}
\begin{split}
g(\nu)&\to\frac{\mathrm{i}}{2}\Lambda(k)\left[q_\perp(\nu) + s_\perp(\nu)\right]\\
&= \frac{\mathrm{i}}{2}\Lambda(k)\left(\frac{\Omega_\perp V_\perp(\nu)}{\nu^2 - \Omega_\perp^2z_\perp(\nu)}\right),
\end{split}
\end{equation}
and
\begin{equation}
\begin{split}
z(\nu)\to\tilde{z}(\nu) &= 1 + \frac{2}{\tilde{k}c}\int_{-\infty}^\infty\frac{\left(\frac{\mathrm{i}}{2}\Lambda(k)[q_\perp(\omega) + s_\perp(\omega)]\right)^2}{\nu - \omega - \mathrm{i}\epsilon}\;\mathrm{d}\omega\\
&= 1 - \frac{\Lambda^2(k)\Omega_\perp^2}{2\tilde{k}c}\int_{-\infty}^\infty\frac{V_\perp^2(\omega)}{[\omega^2 - \Omega_\perp^2z_\perp(\omega)]^2}\frac{1}{\nu - \omega - \mathrm{i}\epsilon}\;\mathrm{d}\omega
\end{split}
\end{equation}
to produce
\begin{equation}
\begin{split}
\tilde{q}(k,\nu) &= \mathrm{i}\frac{(\nu + \tilde{k}c)\Lambda(k)}{2}\frac{\Omega_\perp V_\perp(\nu)}{\nu^2 - \Omega_\perp^2z_\perp(\nu)}\frac{1}{\nu^2 - \tilde{k}^2c^2\tilde{z}(\nu)},\\
\tilde{s}(k,\nu) &= \mathrm{i}\frac{(\nu - \tilde{k}c)\Lambda(k)}{2}\frac{\Omega_\perp V_\perp(\nu)}{\nu^2 - \Omega_\perp^2z_\perp(\nu)}\frac{1}{\nu^2 - \tilde{k}^2c^2\tilde{z}(\nu)},\\[0.5em]
\tilde{u}(k;\nu,\nu') &= \delta(\nu - \nu') + \frac{2\tilde{k}c}{\nu^2 - \tilde{k}c\tilde{z}(\nu)}\frac{\mathrm{i}}{2}\Lambda(k)\frac{\Omega_\perp V_\perp(\nu)}{\nu^2 - \Omega_\perp^2z_\perp(\nu)}\frac{\mathrm{i}}{2}\Lambda(k)\frac{\Omega_\perp V_\perp(\nu')}{\nu'^2 - \Omega_\perp^2z_\perp(\nu')}\frac{1}{\nu - \nu' - \mathrm{i}\epsilon}\\
&= \delta(\nu - \nu') - \frac{\tilde{k}c\Lambda^2(k)\Omega_\perp^2}{2}\frac{V_\perp(\nu)}{\left[\nu^2 - \tilde{k}^2c^2\tilde{z}(\nu)\right][\nu^2 - \Omega_\perp^2z_\perp(\nu)][\nu'^2 - \Omega_\perp^2z_\perp(\nu')]}\frac{V_\perp(\nu')}{\nu - \nu' - \mathrm{i}\epsilon},\\[0.5em]
\tilde{v}(k;\nu,\nu') &= - \frac{\tilde{k}c\Lambda^2(k)\Omega_\perp^2}{2}\frac{V_\perp(\nu)}{\left[\nu^2 - \tilde{k}^2c^2\tilde{z}(\nu)\right][\nu^2 - \Omega_\perp^2z_\perp(\nu)][\nu'^2 - \Omega_\perp^2z_\perp(\nu')]}PV\left\{\frac{V_\perp(\nu')}{\nu + \nu'}\right\}.
\end{split}
\end{equation}
%!TEX root = finiteMQED.tex

\section{Fano Diagonalization of a Set of Oscillators Coupled to a Set of Baths}\label{app:fanoDiagonalizationManyModes}

We begin with a Hamiltonian
\begin{equation}\label{eq:HmanyModes}
\hat{H} = \hat{H}_0 + \hat{H}_\mathrm{int}
\end{equation}
where 
\begin{equation}
\hat{H}_0 = \sum_{i = 1}^{M}\hbar\Omega_i\hat{a}_{i}^\dagger(t)\hat{a}_{i}(t) + \sum_{\sigma = 1}^N\int_0^\infty\hbar\tilde{\nu}_\sigma\hat{b}_\sigma^\dagger(\nu,t)\hat{b}_\sigma(\nu,t)\;\mathrm{d}\nu
\end{equation}
and
\begin{equation}
\hat{H}_\mathrm{int} = \sum_{i,\sigma}\int_0^\infty\hbar g_{i \sigma}(\nu)\left[\hat{a}_{i}(t) + \hat{a}_{i}^\dagger(t)\right]\left[\hat{b}_\sigma(\nu,t) + \hat{b}_\sigma^\dagger(\nu,t)\right]\mathrm{d}\nu.
\end{equation}
The $M$ discrete boson operators $\hat{a}_i(t)$ and $N$ sets of bath operators $\hat{b}_\sigma(\nu,t)$ have natural frequencies $\Omega_i$ and $\tilde{\nu}_\sigma(\nu)$, respectively, and commutation relations
\begin{equation}
\begin{split}
\left[\hat{a}_i(t),\hat{a}_{j}^\dagger(t)\right] &= \delta_{ij},\\
\left[\hat{b}_\sigma(\nu,t),\hat{b}_{\sigma'}^\dagger(\nu',t)\right] &= \delta_{\sigma\sigma'}\delta(\nu - \nu'),
\end{split}
\end{equation}
and
\begin{equation}
\begin{split}
\left[\hat{a}_i(t),\hat{a}_{j}(t)\right] &= 0,\\
\left[\hat{b}_\sigma(\nu,t),\hat{b}_\sigma(\nu',t)\right] &= 0,\\
\left[\hat{a}_i(t),\hat{b}_\sigma(\nu,t)\right] &= \left[\hat{a}_i(t),\hat{b}_\sigma^\dagger(\nu,t)\right] = 0.
\end{split}
\end{equation}
The coupling strength between the $i^\mathrm{th}$ discrete state and the $(\sigma,\nu)^\mathrm{th}$ bath operator is assumed to be real and is given by $\hbar g_{i\sigma}(\nu)$. As was shown in a simpler setup in Appendix \ref{app:fanoDiagonalization}, diagonalization of this system can be performed by defining new hybridized boson operators
\begin{equation}\label{eq:hybridOperatorDefinition}
\hat{C}_n(\omega,t) = \sum_i\left[\frac{\omega + \Omega_i}{2\Omega_i}q_{ni}(\omega)\hat{a}_i(t) + s_{ni}(\omega)\hat{a}^\dagger_i(t)\right] + \sum_\sigma\int_0^\infty\left[u_{n\sigma}(\omega,\nu)\hat{b}_\sigma(\nu,t) + v_{n\sigma}(\omega,\nu)\hat{b}_\sigma^\dagger(\nu,t)\right]\mathrm{d}\nu
\end{equation}
that provide a simple form for the Hamiltonian,
\begin{equation}\label{eq:HdiagManyModes}
\hat{H} = \sum_n\int_0^\infty\hbar\omega\hat{C}_n^\dagger(\omega,t)\hat{C}_n(\omega,t)\;\mathrm{d}\omega,
\end{equation}
and satisfy the commutation relations
\begin{equation}\label{eq:hybridOperatorCommutatorManyModes}
\begin{split}
\left[\hat{C}_n(\omega,t),\hat{C}_{n'}^\dagger(\omega',t)\right] &= \delta_{nn'}\delta(\omega - \omega'),\\
\left[\hat{C}_n(\omega,t),\hat{C}_{n'}(\omega',t)\right] &= 0.\\
\end{split}
\end{equation}
Further, using the algebra of the previous section, we can employ the commutators
\begin{equation}
\begin{split}
\left[\hat{a}_i(t),\hat{H}\right] &= \hbar\Omega_i\hat{a}_i(t) + \sum_\sigma\int_0^\infty\hbar g_{i\sigma}(\nu)\left[\hat{b}_\sigma(\nu,t) + \hat{b}_\sigma^\dagger(\nu,t)\right]\mathrm{d}\nu,\\
\left[\hat{b}_\sigma(\nu,t),\hat{H}\right] &= \hbar\tilde{\nu}_\sigma\hat{b}_\sigma(\nu,t) + \sum_i\hbar g_{i\sigma}(\nu)\left[\hat{a}_i(t) + \hat{a}_i^\dagger(t)\right],
\end{split}
\end{equation}
and
\begin{equation}
\begin{split}
\left[\hat{a}_i^\dagger(t),\hat{H}\right] &= -\hbar\Omega_i\hat{a}_i^\dagger(t) - \sum_\sigma\int_0^\infty\hbar g_{i\sigma}(\nu)\left[\hat{b}_\sigma(\nu,t) + \hat{b}_\sigma^\dagger(\nu,t)\right]\mathrm{d}\nu,\\
\left[\hat{b}_\sigma^\dagger(\nu,t),\hat{H}\right] &= -\hbar\tilde{\nu}_\sigma\hat{b}_\sigma^\dagger(\nu,t) - \sum_i\hbar g_{i\sigma}(\nu)\left[\hat{a}_i(t) + \hat{a}_i^\dagger(t)\right],
\end{split}
\end{equation}
via Eq. \eqref{eq:HmanyModes} as well as
\begin{equation}
\left[\hat{C}_n(\omega,t),\hat{H}\right] = \hbar\omega\hat{C}_n(\omega,t)
\end{equation}
via Eq. \eqref{eq:HdiagManyModes}. Alongside the definition of the hybridized operators (Eq. [\ref{eq:hybridOperatorDefinition}]), these identities can be to say
\begin{equation}
\begin{split}
&\hbar\omega\left(\sum_i\left[\frac{\omega + \Omega_i}{2\Omega_i}q_{ni}(\omega)\hat{a}_i(t) + s_{ni}(\omega)\hat{a}^\dagger_i(t)\right] + \sum_\sigma\int_0^\infty\left[u_{n\sigma}(\omega,\nu)\hat{b}_\sigma(\nu,t) + v_{n\sigma}(\omega,\nu)\hat{b}_\sigma^\dagger(\nu,t)\right]\mathrm{d}\nu\right)\\
&= \sum_i\left(\frac{\omega + \Omega_i}{2\Omega_i}q_{ni}(\omega)\left[\hat{a}_i(t),\hat{H}\right] + s_{ni}(\omega)\left[\hat{a}_i^\dagger(t),\hat{H}\right]\right) + \sum_\sigma\int_0^\infty\left(u_{n\sigma}(\omega,\nu)\left[\hat{b}_\sigma(\nu,t),\hat{H}\right]\right.\\
&\qquad\left. + v_{n\sigma}(\omega,\nu)\left[\hat{b}_\sigma^\dagger(\nu,t),\hat{H}\right]\right)\mathrm{d}\nu\\
&= \sum_i\left(\frac{\omega + \Omega_i}{2\Omega_i}q_{ni}(\omega)\left[\hbar\Omega_i\hat{a}_i(t) + \sum_\sigma\int_0^\infty\hbar g_{i\sigma}(\nu)\left(\hat{b}_\sigma(\nu,t) + \hat{b}_\sigma^\dagger(\nu,t)\right)\mathrm{d}\nu\right]\right.\\
&\qquad\left. + s_{ni}(\omega)\left[-\hbar\Omega_i\hat{a}_i^\dagger(t) - \sum_\sigma\int_0^\infty\hbar g_{i\sigma}(\nu)\left(\hat{b}_\sigma(\nu,t) + \hat{b}_\sigma^\dagger(\nu,t)\right)\mathrm{d}\nu\right]\right)\\
&\qquad + \sum_\sigma\int_0^\infty\left(u_{n\sigma}(\omega,\nu)\left[\hbar\tilde{\nu}_\sigma\hat{b}_\sigma(\nu,t) + \sum_i\hbar g_{i\sigma}(\nu)\left(\hat{a}_i(t) + \hat{a}_i^\dagger(t)\right)\right]\right.\\
&\qquad\left. + v_{n\sigma}(\omega,\nu)\left[-\hbar\tilde{\nu}_\sigma\hat{b}_\sigma^\dagger(\nu,t) - \sum_i\hbar g_{i\sigma}(\nu)\left(\hat{a}_i(t) + \hat{a}_i^\dagger(t)\right)\right] \right)\mathrm{d}\nu.
\end{split}
\end{equation}
This provides a set of equations, one for each independent operator, that serve to fix the values of each expansion coefficient $q_{ni}(\omega)$, $s_{ni}(\omega)$, $u_{n\sigma}(\omega,\nu)$, and $v_{n\sigma}(\omega,\nu)$. Explicitly,
\begin{equation}\label{eq:hybridizationCoefficientsSystemManyModes}
\begin{split}
(\omega - \Omega_i)\frac{\omega + \Omega_i}{2\Omega_i}q_{ni}(\omega) &= \sum_\sigma\int_0^\infty\left[u_{n\sigma}(\omega,\nu) - v_{n\sigma}(\omega,\nu)\right]g_{i\sigma}(\nu)\;\mathrm{d}\nu,\\
(\omega + \Omega_i)s_{ni}(\omega) &= \sum_\sigma\int_0^\infty\left[u_{n\sigma}(\omega,\nu) - v_{n\sigma}(\omega,\nu)\right]g_{i\sigma}(\nu)\;\mathrm{d}\nu,\\
(\omega - \tilde{\nu}_\sigma)u_{n\sigma}(\omega,\nu) &= \sum_i\left[\frac{\omega + \Omega_i}{2\Omega_i}q_{ni}(\omega) - s_{ni}(\omega)\right]g_{i\sigma}(\nu),\\
(\omega + \tilde{\nu}_\sigma)v_{n\sigma}(\omega,\nu) &= \sum_i\left[\frac{\omega + \Omega_i}{2\Omega_i}q_{ni}(\omega) - s_{ni}(\omega)\right]g_{i\sigma}(\nu).
\end{split}
\end{equation}

The latter three equations have solutions in terms of $q_{ni}(\omega)$. First noting that
\begin{equation}
\begin{split}
s_{ni}(\omega) &= \frac{\omega - \Omega_i}{\omega + \Omega_i}\frac{\omega + \Omega_i}{2\Omega_i}q_{ni}(\omega)\\
&= \frac{\omega - \Omega_i}{2\Omega_i}q_{ni}(\omega),
\end{split}
\end{equation}
we can see that the most general solutions for $u_{n\sigma}(\omega,\nu)$ and $v_{n\sigma}(\omega,\nu)$ are
\begin{equation}
\begin{split}
u_{n\sigma}(\omega,\nu) &= PV\left\{\frac{1}{\omega - \tilde{\nu}_\sigma}\right\}\sum_ig_{i\sigma}(\nu)q_{ni}(\omega) + \sum_ig_{i\sigma}(\nu)q_{ni}(\omega)x_{ni\sigma}(\omega)\delta(\omega - \tilde{\nu}_\sigma),\\
v_{n\sigma}(\omega,\nu) &= \frac{1}{\omega + \tilde{\nu}_\sigma}\sum_ig_{i\sigma}(\nu)q_{ni}(\omega),
\end{split}
\end{equation}
wherein the Cauchy principal value operator $PV$ has been introduced to handle the singularity in $u_{n\sigma}(\omega,\nu)$ at $\omega = \nu$. The functions $x_{ni\sigma}(\omega)$ are smooth, complex, and nonzero and have forms determined through physical constraints, as will be shown below. 

In practice, common forms of the bath frequencies are $\tilde{\nu}_\sigma(\nu) = \nu$ and $\tilde{\nu}_\sigma(\nu) = \sqrt{\nu^2 + \xi_\sigma^2}$. We can then separate our $N$ reservoirs into $N_1$ baths of the first kind and $N_2 = N - N_1$ baths of the second kind such that, using the identity $\delta(g[x]) = \sum_i\delta(x - x_i)/\abs{(\partial g(y)/\partial y)_{y\to x_i}}$ with $x_i$ the roots of $g(x)$, we find
\begin{equation}
\delta(\omega - \tilde{\nu}_\sigma) =
\begin{cases}
\delta(\nu - \omega), &\sigma\leq N_1,\\
\frac{\omega}{\sqrt{\omega^2 - \xi_\sigma^2}}\left[\delta(\nu + \sqrt{\omega^2 - \xi_\sigma^2}) + \delta(\nu - \sqrt{\omega^2 - \xi_\sigma^2})\right], &\sigma > N_1.
\end{cases}
\end{equation}
Plugging these solutions back into the first equation of Eq. \eqref{eq:hybridizationCoefficientsSystemManyModes} and defining $\omega_\sigma = \sqrt{\omega^2 - \xi_\sigma^2}$, one finds
\begin{equation}
\begin{split}
\frac{\omega^2 - \Omega_i^2}{2\Omega_i}q_{ni}(\omega) &= \sum_\sigma\sum_jq_{nj}(\omega)\int_0^\infty\left(PV\left\{\frac{1}{\omega - \tilde{\nu}_\sigma}\right\} - \frac{1}{\omega + \tilde{\nu}_\sigma}\right)g_{i\sigma}(\nu)g_{j\sigma}(\nu)\;\mathrm{d}\nu\\
&\qquad + \sum_\sigma\sum_j x_{nj\sigma}(\omega)q_{nj}(\omega)\frac{\omega}{\omega_\sigma}g_{i\sigma}(\omega_\sigma)g_{j\sigma}(\omega_\sigma).
\end{split}
\end{equation}
Note that the thermal bath terms in the above sum are identical to the photonic bath therms under the transformation $\xi_\sigma\to0$.

Summation over $i$ allows for the separation of this relation into matrix-vector notation. Using the unit vectors $\hat{\mathbf{e}}_i(\omega)$ to represent the orthogonal axes of the Hilbert space formed by the expansion coefficients $q_{ni}(\omega)$, we can say
\begin{equation}
\begin{split}
\sum_i&\frac{\omega^2 - \Omega_i^2}{2\Omega_i}\hat{\mathbf{e}}_i\hat{\mathbf{e}}_i\cdot\sum_kq_{nk}(\omega)\hat{\mathbf{e}}_k\\
&= \sum_\sigma\sum_{ij}\int_0^\infty\left(PV\left\{\frac{1}{\omega - \tilde{\nu}_\sigma}\right\} - \frac{1}{\omega + \tilde{\nu}_\sigma}\right)g_{i\sigma}(\nu)g_{j\sigma}(\nu)\;\mathrm{d}\nu\;\hat{\mathbf{e}}_i\hat{\mathbf{e}}_j\cdot\sum_kq_{nk}(\omega)\hat{\mathbf{e}}_k\\
&\qquad + \sum_{\sigma}\sum_{ij}\frac{\omega}{\omega_\sigma}x_{nj\sigma}(\omega)g_{i\sigma}(\omega_\sigma)g_{j\sigma}(\omega_\sigma)\hat{\mathbf{e}}_i\hat{\mathbf{e}}_j\cdot\sum_kq_{nk
}(\omega)\hat{\mathbf{e}}_k.
\end{split}
\end{equation}
Therefore, we can define the matrices 
\begin{equation}
\begin{split}
\mathbf{A}(\omega) &= \sum_i\frac{(\omega^2 - \Omega_i^2)}{2\Omega_i}\hat{\mathbf{e}}_i\hat{\mathbf{e}}_i,\\
\mathbf{B}_\sigma(\omega) &= \sum_{ij}\int_0^\infty\left(PV\left\{\frac{1}{\omega - \tilde{\nu}_\sigma}\right\} - \frac{1}{\omega + \tilde{\nu}_\sigma}\right)g_{i\sigma}(\nu)g_{j\sigma}(\nu)\;\mathrm{d}\nu\,\hat{\mathbf{e}}_i\hat{\mathbf{e}}_j,\\
\mathbf{C}_{n\sigma}(\omega) &= \sum_{ij} x_{nj\sigma}(\omega)g_{i\sigma}(\omega_\sigma)g_{j\sigma}(\omega_\sigma)\hat{\mathbf{e}}_i\hat{\mathbf{e}}_j,\\
\mathbf{D}_\sigma(\omega) &= \sum_{ij}g_{i\sigma}(\omega_\sigma)g_{j\sigma}(\omega_\sigma)\hat{\mathbf{e}}_i\hat{\mathbf{e}}_j,
\end{split}
\end{equation}
and the vector
\begin{equation}
\mathbf{q}_n(\omega) = \sum_iq_{ni}(\omega)\hat{\mathbf{e}}_i,
\end{equation}
such that the first line of Eq. \eqref{eq:hybridizationCoefficientsSystemManyModes} can be rewritten as the vector equation
\begin{equation}\label{eq:ABCmatrixIdentity}
\mathbf{A}(\omega)\cdot\mathbf{q}_n(\omega) = \sum_\sigma\mathbf{B}_\sigma(\omega)\cdot\mathbf{q}_n(\omega) + \sum_\sigma \frac{\omega}{\omega_\sigma}\mathbf{C}_\sigma(\omega)\cdot\mathbf{q}_n(\omega).
\end{equation}

This process can calso be used to expand our final physical constraint, the commutator of Eq. \eqref{eq:hybridOperatorCommutatorManyModes}, as a matrix-vector equation. In more detail,
\begin{equation}
\begin{split}
\left[\hat{C}_n(\omega,t),\hat{C}_{n'}^\dagger(\omega',t)\right] &= \sum_{ij}\left(\frac{(\omega + \Omega_i)(\omega' + \Omega_j)}{4\Omega_i\Omega_j}q_{ni}(\omega)q_{n'j}^*(\omega')\left[\hat{a}_i(t),\hat{a}_j^\dagger(t)\right] + s_{ni}(\omega)s_{n'j}^*(\omega')\left[\hat{a}_i^\dagger(t),\hat{a}_j(t)\right]\right)\\
&\qquad + \sum_{\sigma\sigma'}\int_0^\infty\int_0^\infty\left(u_{n\sigma}(\omega,\nu)u_{n'\sigma'}^*(\omega',\nu')\left[\hat{b}_{\sigma}(\nu,t),\hat{b}_{\sigma'}^\dagger(\nu',t)\right]\right.\\
&\qquad\left. + v_{n\sigma}(\omega,\nu)v_{n'\sigma'}^*(\omega',\nu')\left[\hat{b}_\sigma^\dagger(\nu,t),\hat{b}_{\sigma'}(\nu',t)\right]\right)\mathrm{d}\nu\,\mathrm{d}\nu'\\
&= \sum_i\left[\frac{(\omega + \Omega_i)(\omega' + \Omega_i)}{4\Omega_i^2}q_{ni}(\omega)q_{n'i}^*(\omega') - s_{ni}(\omega)s_{n'i}^*(\omega')\right]\\
&\qquad+ \sum_\sigma\int_0^\infty \left[u_{n\sigma}(\omega,\nu)u_{n'\sigma}^*(\omega',\nu) - v_{n\sigma}(\omega,\nu)v_{n'\sigma}^*(\omega',\nu)\right]\mathrm{d}\nu\\
&= \delta_{nn'}\delta(\omega - \omega')
\end{split}
\end{equation}
leads, with the substitution of the explicit form of each expansion coefficient, to
\begin{equation}
\begin{split}
\delta_{nn'}\delta(\omega - \omega') 
&= \sum_i\left(\frac{(\omega + \Omega_i)(\omega' + \Omega_i)}{4\Omega_i^2} - \frac{(\omega - \Omega_i)(\omega' - \Omega_i)}{4\Omega_i^2}\right)q_{ni}(\omega)q_{n'i}^*(\omega')\\
&\qquad + \sum_\sigma\sum_{ij}\int_0^\infty g_{i\sigma}(\nu)g_{j\sigma}(\nu)q_{ni}(\omega)q_{n'j}^*(\omega')\\
&\qquad\times \left(\left[PV\left\{\frac{1}{\omega - \tilde{\nu}_\sigma}\right\} + x_{ni\sigma}(\omega)\delta(\omega - \tilde{\nu}_\sigma)\right]\left[PV\left\{\frac{1}{\omega - \tilde{\nu}_\sigma}\right\} + x_{n'j\sigma}^*(\omega')\delta(\omega' - \tilde{\nu}_\sigma)\right]\right.\\
&\qquad\left. - \frac{1}{\omega + \tilde{\nu}_\sigma}\frac{1}{\omega' + \tilde{\nu}_\sigma} \right)\mathrm{d}\nu\\
&= \sum_i\frac{\omega + \omega'}{2\Omega_i}q_{ni}(\omega)q_{n'i}^*(\omega') + \sum_\sigma\sum_{ij}\int_0^\infty g_{i\sigma}(\nu)g_{j\sigma}(\nu)q_{ni}(\omega)q_{n'j}^*(\omega')\\
&\qquad\times\left(\left[PV\left\{\frac{1}{\omega - \tilde{\nu}_\sigma}\right\} + x_{ni\sigma}(\omega)\delta(\omega - \tilde{\nu}_\sigma)\right]\left[PV\left\{\frac{1}{\omega' - \tilde{\nu}_\sigma}\right\} + x_{n'j\sigma}^*(\omega')\delta(\omega' - \tilde{\nu}_\sigma)\right]\right.\\
&\qquad\left. - \frac{1}{\omega + \tilde{\nu}_\sigma}\frac{1}{\omega' + \tilde{\nu}_\sigma} \right)\mathrm{d}\nu.
\end{split}
\end{equation}
Using the identities
\begin{equation}
\begin{split}
PV\left\{\frac{1}{x}\right\} &= \lim_{\epsilon\to0}\frac{x}{x^2 + \epsilon^2},\\
\delta(x) &= \lim_{\epsilon\to0}\frac{\epsilon}{x^2 + \epsilon^2},
\end{split}
\end{equation}
and the method of partial fractions, we can expand Cauchy principal value products as
\begin{equation}
\begin{split}
&PV\left\{\frac{1}{\omega - \nu}\right\}PV\left\{\frac{1}{\omega' - \nu}\right\} = \lim_{\epsilon\to0}\frac{\omega - \nu}{(\omega - \nu)^2 + \epsilon^2}\frac{\omega' - \nu}{(\omega' - \nu)^2 + \epsilon^2}\\
% &= \lim_{\epsilon\to0}\left(\frac{\nu}{\omega - \omega'}\frac{\epsilon}{(\omega - \nu)^2 + \epsilon^2}\frac{2\epsilon}{(\omega - \omega')^2 + (2\epsilon)^2} - \frac{\omega + \omega'}{2(\omega - \omega')}\frac{\epsilon}{(\omega - \nu)^2 + \epsilon^2}\frac{2\epsilon}{(\omega - \omega')^2 + (2\epsilon)^2}\right.\\
% &\qquad + \frac{\nu - \omega}{(\omega - \nu)^2 + \epsilon^2}\frac{\omega - \omega'}{(\omega - \omega')^2 + (2\epsilon)^2} + \frac{\nu - \omega'}{(\omega' - \nu)^2 + \epsilon^2}\frac{(\omega' - \omega)}{(\omega - \omega')^2 + (2\epsilon)^2}\\
% &\qquad\left. - \frac{\nu}{\omega - \omega'}\frac{\epsilon}{(\nu - \omega')^2 + \epsilon^2}\frac{2\epsilon}{(\omega - \omega')^2 + (2\epsilon)^2} + \frac{\omega + \omega'}{2(\omega - \omega')}\frac{\epsilon}{(\omega' - \nu)^2 + \epsilon^2}\frac{2\epsilon}{(\omega - \omega')^2 + (2\epsilon)^2}\right)\\
% &= \pi^2\frac{(\nu - \omega) + (\nu - \omega')}{2(\omega - \omega')}\delta(\omega - \omega')\left[\delta(\nu - \omega) - \delta(\nu - \omega')\right] + PV\left\{\frac{1}{\omega - \omega'}\right\}\left(PV\left\{\frac{1}{\omega' - \nu}\right\} - PV\left\{\frac{1}{\omega - \nu}\right\}\right)\\
&= \pi^2\delta(\nu - \omega')\delta(\nu - \omega) + PV\left\{\frac{1}{\omega - \omega'}\right\}\left(PV\left\{\frac{1}{\omega' - \nu}\right\} - PV\left\{\frac{1}{\omega - \nu}\right\}\right)
\end{split}
\end{equation}
and note that, for $\omega$, $\omega'$, and $\nu$ all nonnegative,
\begin{equation}
\begin{split}
\frac{1}{\omega + \nu}\frac{1}{\omega' + \nu} &= PV\left\{\frac{1}{\omega + \nu}\right\}PV\left\{\frac{1}{\omega' + \nu}\right\}\\
% &= \pi^2\delta(\nu + \omega')\delta(\nu + \omega) + PV\left\{\frac{1}{\omega - \omega'}\right\}\left(PV\left\{\frac{1}{\omega' + \nu}\right\} - PV\left\{\frac{1}{\omega + \nu}\right\}\right)\\
&= PV\left\{\frac{1}{\omega - \omega'}\right\}\left(\frac{1}{\omega' + \nu} - \frac{1}{\omega + \nu}\right).
\end{split}
\end{equation} 
Therefore,
\begin{equation}
\begin{split}
\delta_{nn'}\delta(\omega - \omega') &= \sum_i\frac{(\omega + \omega')}{2\Omega_i}q_{ni}(\omega)q_{n'i}^*(\omega') + \sum_\sigma\sum_{ij}\int_0^\infty g_{i\sigma}(\nu)g_{j\sigma}(\nu)q_{ni}(\omega)q_{n'j}^*(\omega')\\
&\qquad\times\left(\pi^2\delta(\omega' - \tilde{\nu}_\sigma)\delta(\omega - \tilde{\nu}_\sigma) + PV\left\{\frac{1}{\omega - \omega'}\right\}\left[PV\left\{\frac{1}{\omega' - \tilde{\nu}_\sigma}\right\} - \frac{1}{\omega' + \tilde{\nu}_\sigma}\right]\right.\\
&\qquad\left. - PV\left\{\frac{1}{\omega - \omega'}\right\}\left[PV\left\{\frac{1}{\omega - \tilde{\nu}_\sigma}\right\} - \frac{1}{\omega + \tilde{\nu}_\sigma}\right]\right)\mathrm{d}\nu\\
&\qquad - \sum_{\sigma}\sum_{ij}q_{ni}(\omega)q_{n'j}^*(\omega')PV\left\{\frac{1}{\omega - \omega'}\right\}\frac{\omega}{\omega_\sigma}x_{ni\sigma}(\omega)g_{i\sigma}(\omega_\sigma)g_{j\sigma}(\omega_\sigma)\\
&\qquad + \sum_{\sigma}\sum_{ij}q_{ni}(\omega)q_{n'j}^*(\omega')PV\left\{\frac{1}{\omega - \omega'}\right\}\frac{\omega'}{\omega_\sigma'}x_{n'j\sigma}^*(\omega')g_{i\sigma}(\omega_\sigma')g_{j\sigma}(\omega_\sigma')\\
&\qquad + \sum_\sigma\sum_{ij}\int_0^\infty q_{ni}(\omega)q_{n'j}^*(\omega')x_{ni\sigma}(\omega)x_{n'j\sigma}^*(\omega')g_{i\sigma}(\nu)g_{j\sigma}(\nu)\delta(\omega - \tilde{\nu}_\sigma)\delta(\omega' - \tilde{\nu}_\sigma)\;\mathrm{d}\nu.
% &\qquad + \sum_\sigma\sum_{ij}q_{\ni}(\omega)q_{n'j}^*(\omega')\frac{\omega}{\omega_\sigma}x_{ni\sigma}(\omega)x_{n'k\sigma}^*(\omega')g_{i\sigma}(\omega_\sigma)g_{j\sigma}(\omega_\sigma)\delta(\omega - \omega').
\end{split}
\end{equation}
Further, with
\begin{equation}
\begin{split}
\int_0^\infty f(\nu)\delta(\omega' - \tilde{\nu}_\sigma)\delta(\omega - \tilde{\nu}_\sigma)\;\mathrm{d}\nu &= \frac{\omega\omega'}{\omega_\sigma\omega_\sigma'}\int_0^\infty f(\nu)[\delta(\nu - \omega_\sigma') + \delta(\nu + \omega_\sigma')][\delta(\nu - \omega_\sigma) + \delta(\nu + \omega_\sigma)]\;\mathrm{d}\nu\\
&= \frac{\omega\omega'}{\omega_\sigma\omega_\sigma'}f(\omega_\sigma)\delta(\omega_\sigma - \omega_\sigma')\\
% &= \frac{\omega\omega'}{\omega_\sigma\omega_\sigma'}f(\omega_\sigma)\frac{\omega_\sigma'}{\omega'}\left[\delta(\omega - \omega') + \delta(\omega + \omega')\right]\\
&= \frac{\omega}{\omega_\sigma}f(\omega_\sigma)\left[\delta(\omega - \omega') + \delta(\omega + \omega')\right]
\end{split}
\end{equation}
% and
% \begin{equation}
% \begin{split}
% \delta(\omega_\sigma - \omega_\sigma') &= \frac{\omega_\sigma'}{\omega'}[\delta(\omega - \omega') + \delta(\omega + \omega')]\\
% &= \frac{\omega_\sigma'}{\omega'}\delta(\omega - \omega')
% \end{split}
% \end{equation}
wherein $\omega_\sigma' = \sqrt{\omega'^2 - \xi_\sigma^2}$, we find
\begin{equation}
\begin{split}
\delta_{nn'}\delta(\omega - \omega') &= \sum_i\frac{(\omega + \omega')}{2\Omega_i}q_{ni}(\omega)q_{n'i}^*(\omega') + \sum_\sigma\sum_{ij}\pi^2\frac{\omega}{\omega_\sigma}\delta(\omega - \omega')g_{i\sigma}(\omega_\sigma)g_{j\sigma}(\omega_\sigma)q_{ni}(\omega)q_{n'j}^*(\omega')\\
& + \sum_\sigma\sum_{ij}q_{ni}(\omega)q_{n'j}^*(\omega')PV\left\{\frac{1}{\omega - \omega'}\right\}\int_0^\infty\left(PV\left\{\frac{1}{\omega' - \tilde{\nu}_\sigma}\right\} - \frac{1}{\omega' + \tilde{\nu}_\sigma}\right)g_{i\sigma}(\nu)g_{j\sigma}(\nu)\;\mathrm{d}\nu\\
&- \sum_\sigma\sum_{ij}q_{ni}(\omega)q_{n'j}^*(\omega')PV\left\{\frac{1}{\omega - \omega'}\right\}\int_0^\infty\left(PV\left\{\frac{1}{\omega - \tilde{\nu}_\sigma}\right\} - \frac{1}{\omega + \tilde{\nu}_\sigma}\right)g_{i\sigma}(\nu)g_{j\sigma}(\nu)\;\mathrm{d}\nu\\
&\qquad + \sum_{\sigma}\sum_{ij}q_{ni}(\omega)q_{n'j}^*(\omega')PV\left\{\frac{1}{\omega - \omega'}\right\}\frac{\omega'}{\omega_\sigma'}x_{n'j\sigma}^*(\omega')g_{i\sigma}(\omega_\sigma')g_{j\sigma}(\omega_\sigma')\\
&\qquad - \sum_{\sigma}\sum_{ij}q_{ni}(\omega)q_{n'j}^*(\omega')PV\left\{\frac{1}{\omega - \omega'}\right\}\frac{\omega}{\omega_\sigma}x_{ni\sigma}(\omega)g_{i\sigma}(\omega_\sigma)g_{j\sigma}(\omega_\sigma)\\
&\qquad + \sum_\sigma\sum_{ij}q_{ni}(\omega)q_{n'j}^*(\omega')x_{ni\sigma}(\omega)x_{n'j\sigma}^*(\omega')\frac{\omega}{\omega_\sigma}g_{i\sigma}(\omega_\sigma)g_{j\sigma}(\omega_\sigma)\delta(\omega - \omega').
\end{split}
\end{equation}
Here, we have let $\delta(\omega + \omega') = 0$ for $\omega,\omega' > 0$. Further, it will make our lives easier if there exists a solution for $q_{ni}(\omega)$ wherein the prefactor $x_{ni\sigma}(\omega)$ is independent of the polariton index $n$. Assuming a solution can be found, we will let $x_{ni\sigma}(\omega)\to x_{i\sigma}(\omega)$. Further defining a new matrix
\begin{equation}
\mathbf{E}_\sigma(\omega) = \sum_{ij}x_{i\sigma}(\omega)x_{j\sigma}^*(\omega)g_{i\sigma}(\omega_\sigma)g_{j\sigma}(\omega_\sigma)\hat{\mathbf{e}}_i\hat{\mathbf{e}}_j,
\end{equation}
we can rearrange the above expression in vector form such that
\begin{equation}
\begin{split}
\delta_{nn'}\delta(\omega - \omega') &= \mathbf{q}_n(\omega)\cdot\sum_{i}\frac{\omega + \omega'}{2\Omega_i}\hat{\mathbf{e}}_i\hat{\mathbf{e}}_i\cdot\mathbf{q}_{n'}^*(\omega')\\
&\qquad + \mathbf{q}_n(\omega)\cdot\pi^2\delta(\omega - \omega')\sum_\sigma\frac{\omega}{\omega_\sigma}\mathbf{D}_\sigma(\omega)\cdot\mathbf{q}_{n'}^*(\omega')\\
&\qquad + \mathbf{q}_n(\omega)\cdot PV\left\{\frac{1}{\omega - \omega'}\right\}\sum_\sigma\left[\mathbf{B}_\sigma(\omega') - \mathbf{B}_\sigma(\omega)\right]\cdot\mathbf{q}_{n'}^*(\omega')\\
&\qquad + \mathbf{q}_{n}(\omega)\cdot PV\left\{\frac{1}{\omega - \omega'}\right\}\sum_\sigma\left[\frac{\omega'}{\omega_\sigma'}\mathbf{C}_\sigma(\omega_\sigma') - \frac{\omega}{\omega_\sigma}\mathbf{C}_\sigma^\top(\omega)\right]\cdot\mathbf{q}_{n'}^*(\omega')\\
&\qquad + \mathbf{q}_n(\omega)\cdot\delta(\omega - \omega')\sum_\sigma\frac{\omega}{\omega_\sigma}\mathbf{E}_\sigma(\omega)\cdot\mathbf{q}_n^*(\omega'),
\end{split}
\end{equation}
wherein we have let $\mathbf{C}_{n\sigma}(\omega)\to\mathbf{C}_\sigma(\omega)$. Use of the identity of Eq. \eqref{eq:ABCmatrixIdentity} provides
\begin{equation}
\begin{split}
\delta_{nn'}\delta(\omega - \omega') &= \mathbf{q}_n(\omega)\cdot\sum_{i}\frac{\omega + \omega'}{2\Omega_i}\hat{\mathbf{e}}_i\hat{\mathbf{e}}_i\cdot\mathbf{q}_{n'}^*(\omega')\\
&\qquad + \mathbf{q}_n(\omega)\cdot\delta(\omega - \omega')\sum_\sigma\frac{\omega}{\omega_\sigma}\left[\pi^2\mathbf{D}_\sigma(\omega) + \mathbf{E}_\sigma(\omega)\right]\cdot\mathbf{q}_{n'}^*(\omega')\\
&\qquad + \mathbf{q}_n(\omega)\cdot PV\left\{\frac{1}{\omega - \omega'}\right\}\sum_\sigma\left[\mathbf{B}_\sigma(\omega') - \mathbf{B}_\sigma(\omega)\right]\cdot\mathbf{q}_{n'}^*(\omega')\\
&\qquad + \mathbf{q}_n(\omega)\cdot PV\left\{\frac{1}{\omega - \omega'}\right\}\left[\mathbf{A}(\omega') - \sum_\sigma\mathbf{B}_\sigma(\omega') - \mathbf{A}(\omega) + \sum_\sigma\mathbf{B}_\sigma(\omega)\right]\\
&= \mathbf{q}_n(\omega)\cdot\sum_i\left(\frac{\omega + \omega'}{2\Omega_i} + PV\left\{\frac{1}{\omega - \omega'}\right\}\left[\frac{\omega'^2 - \Omega_i^2}{2\Omega_i} - \frac{\omega^2 - \Omega_i^2}{2\Omega_i}\right]\right)\hat{\mathbf{e}}_i\hat{\mathbf{e}}_i\cdot\mathbf{q}_{n'}^*(\omega')\\
&\qquad + \mathbf{q}_n(\omega)\cdot\delta(\omega - \omega')\sum_\sigma\frac{\omega}{\omega_\sigma}\left[\pi^2\mathbf{D}_\sigma(\omega) + \mathbf{E}_\sigma(\omega)\right]\cdot\mathbf{q}_{n'}^*(\omega')\\
&= \mathbf{q}_n(\omega)\cdot\sum_i\left(\frac{\omega + \omega'}{2\Omega_i} + PV\left\{\frac{1}{\omega - \omega'}\right\}\frac{(\omega' - \omega)(\omega' + \omega)}{2\Omega_i}\right)\hat{\mathbf{e}}_i\hat{\mathbf{e}}_i\cdot\mathbf{q}_{n'}^*(\omega')\\
&\qquad + \mathbf{q}_n(\omega)\cdot\delta(\omega - \omega')\sum_\sigma\frac{\omega}{\omega_\sigma}\left[\pi^2\mathbf{D}_\sigma(\omega) + \mathbf{E}_\sigma(\omega)\right]\cdot\mathbf{q}_{n'}^*(\omega')\\
&= \delta(\omega - \omega')\mathbf{q}_n(\omega)\cdot\sum_\sigma\frac{\omega}{\omega_\sigma}\left[\pi^2\mathbf{D}_\sigma(\omega) + \mathbf{E}_\sigma(\omega)\right]\cdot\mathbf{q}_{n'}^*(\omega').
\end{split}
\end{equation}

Finally, it is advantageous to define
\begin{equation}
\mathbf{G}(\omega) = \sum_\sigma\frac{\omega}{\omega_\sigma}\left[\pi^2\mathbf{D}_\sigma(\omega) + \mathbf{E}_\sigma(\omega)\right]
\end{equation}
as the characteristic matrix of the polariton decomposition. The bosonic commutator $[\hat{C}_n(\omega,t),\hat{C}_{n'}^\dagger(\omega',t)] = \delta_{nn'}\delta(\omega - \omega')$ is then only satisfied if $\mathbf{q}_n(\omega)$ is an eigenvector of $\mathbf{G}(\omega)$. Letting its corresponding eigenvalue be $\lambda_n(\omega)$, we have, explicitly,
\begin{equation}
\mathbf{G}(\omega)\cdot\mathbf{q}_n(\omega) = \lambda_n(\omega)\mathbf{q}_n(\omega).
\end{equation}
Because the matrices $\mathbf{D}_\sigma(\omega)$ and $\mathbf{E}_\sigma(\omega)$ are Hermitian, $\mathbf{G}(\omega)$ is Hermitian.  Therefore, the eigenvalues $\lambda_n(\omega)$ are real and the eigenvectors $\mathbf{q}_n(\omega)$ are orthogonal such that
\begin{equation}
\mathbf{q}_n(\omega)\cdot\mathbf{q}_{n'}^*(\omega) = |\mathbf{q}_n(\omega)|^2\delta_{nn'}
\end{equation}
and
\begin{equation}
\begin{split}
\delta_{nn'}\delta(\omega - \omega') &= \pi^2\delta(\omega - \omega')\mathbf{q}_n^*(\omega)\cdot\mathbf{G}(\omega)\cdot\mathbf{q}_{n'}(\omega)\\
&= \pi^2\delta(\omega - \omega')\lambda_n(\omega)\mathbf{q}_n^*(\omega)\cdot\mathbf{q}_{n'}(\omega)\\
&= \pi^2\delta(\omega - \omega')\lambda_n(\omega)|\mathbf{q}_n(\omega)|^2\delta_{nn'}.
\end{split}
\end{equation}
We can then see
\begin{equation}
|\mathbf{q}_n(\omega)|^2 = \frac{1}{\pi^2\lambda_n(\omega)}.
\end{equation}
is the normalization condition that sets the absolute magnitude of the elements of $\mathbf{q}_n(\omega)$. 

The inverse Fano transformations used to define the unhybridized operators $\hat{a}_i(t)$ and $\hat{b}_\sigma(\nu,t)$ in terms of the polariton operators can be straightforwardly built from these results and the integral techniques of Appendix \ref{app:fanoDiagonalization}. To begin, we can propose a trial solution
\begin{equation}
\begin{split}
\hat{a}_i(t) &= \frac{1}{\mathcal{N}_i}\sum_n\int_0^\infty P_{ni}(\omega)\left[\frac{\omega + \Omega_i}{2\Omega_i}q_{ni}^*(\omega)\hat{C}_n(\omega,t) - s_{ni}(\omega)\hat{C}_{n}^\dagger(\omega,t)\right]\mathrm{d}\omega\\
&= \frac{1}{\mathcal{N}_i}\left(\sum_j\hat{a}_j(t)\sum_n\int_0^\infty P_{ni}(\omega)\left[\frac{(\omega + \Omega_i)(\omega + \Omega_j)}{4\Omega_i\Omega_j}q_{ni}^*(\omega)q_{nj}(\omega) - s_{ni}(\omega)s_{nj}^*(\omega)\right]\mathrm{d}\omega\right.\\
&\qquad + \sum_j\hat{a}_j^\dagger(t)\sum_n\int_0^\infty P_{ni}(\omega)\left[\frac{\omega + \Omega_i}{2\Omega_i}q_{ni}^*(\omega)s_{nj}(\omega) - \frac{\omega + \Omega_j}{2\Omega_j}s_{ni}(\omega)q_{nj}^*(\omega)\right]\mathrm{d}\omega\\
&\qquad + \sum_\sigma\int_0^\infty\hat{b}_\sigma(\nu,t)\sum_n\int_0^\infty P_{ni}(\omega)\left[\frac{\omega + \Omega_i}{2\Omega_i}q_{ni}^*(\omega)u_{n\sigma}(\omega,\nu) - s_{ni}(\omega)v_{n\sigma}^*(\omega,\nu)\right]\mathrm{d}\omega\,\mathrm{d}\nu\\
&\qquad\left. + \sum_\sigma\int_0^\infty b_\sigma^\dagger(\nu,t)\sum_n\int_0^\infty P_{ni}(\omega)\left[\frac{\omega + \Omega_i}{2\Omega_i}q_{ni}^*(\omega)v_{n\sigma}(\omega,\nu) - s_{ni}(\omega)u_{n\sigma}^*(\omega,\nu)\right]\mathrm{d}\omega\,\mathrm{d}\nu\right)
\end{split}
\end{equation}
where $P_{ni}(\omega)$ is a characteristic prefactor and $\mathcal{N}_i$ is a normalization function. Before we being simplifying these expressions, we can see that, because $\mathbf{G}(\omega)$ is Hermitian, the matrix of its normalized column-eigenvectors,
\begin{equation}
\mathbf{V}(\omega) =
\begin{pmatrix}
\mathbf{v}_1^\top(\omega) & \mathbf{v}_2^\top(\omega) & \cdots & \mathbf{v}_M^\top(\omega) 
\end{pmatrix},
\end{equation}
is unitary. Here $\mathbf{v}_n(\omega) = \pi\sqrt{\lambda_n}\mathbf{q}_n(\omega) = \pi\sqrt{\lambda_n}\left[q_{n1}(\omega)\hat{\mathbf{e}}_1 + q_{n2}(\omega)\hat{\mathbf{e}}_2 + \ldots\right]$. Because both the rows and columns of a unitary matrix form an orthonormal set of vectors, we can define $\mathbf{u}_i(\omega) = \pi\sqrt{\lambda_1}q_{1i}(\omega)\hat{\mathbf{e}}_{1} + \pi\sqrt{\lambda_{2}}q_{2i}(\omega)\hat{\mathbf{e}}_{2} + \ldots$, such that
\begin{equation}
\begin{split}
\mathbf{v}_n^*(\omega)\cdot\mathbf{v}_{n'}(\omega) &= \pi^2\lambda_n\sum_iq_{ni}^*(\omega)q_{n'i}(\omega) = \delta_{nn'},\\
\mathbf{u}_i^*(\omega)\cdot\mathbf{u}_j(\omega) &= \sum_n\pi^2\lambda_nq_{ni}^*(\omega)q_{nj}(\omega) = \delta_{ij}.
\end{split}
\end{equation}
Therefore, looking at the first integral of our trial solution for $\hat{a}_i(t)$, we can see upon substitution of $s_{ni}(\omega) = (\omega - \Omega_i)q_{ni}(\omega)/2\Omega_i$ that
\begin{equation}
\begin{split}
&\sum_n\int_0^\infty P_{ni}(\omega)\left[\frac{(\omega + \Omega_i)(\omega + \Omega_j)}{4\Omega_i\Omega_j}q_{ni}^*(\omega)q_{nj}(\omega) - s_{ni}(\omega)s_{nj}^*(\omega)\right]\mathrm{d}\omega\\
= &\sum_n\int_0^\infty P_{ni}(\omega)\left[\frac{(\omega + \Omega_i)(\omega + \Omega_j)}{4\Omega_i\Omega_j}q_{ni}^*(\omega)q_{nj}(\omega) - \frac{(\omega - \Omega_i)(\omega - \Omega_j)}{4\Omega_i\Omega_j}q_{ni}(\omega)q_{nj}^*(\omega)\right]\mathrm{d}\omega\\
= &\sum_n\int_0^\infty\pi^2\lambda_n\bar{P}_{i}(\omega)\left[\frac{(\omega + \Omega_i)(\omega + \Omega_j)}{4\Omega_i\Omega_j}q_{ni}^*(\omega)q_{nj}(\omega) - \frac{(\omega - \Omega_i)(\omega - \Omega_j)}{4\Omega_i\Omega_j}q_{ni}(\omega)q_{nj}^*(\omega)\right]\mathrm{d}\omega\\
= &\int_0^\infty\bar{P}_{i}(\omega)\delta_{ij}\left[\frac{(\omega + \Omega_i)(\omega + \Omega_j)}{4\Omega_i\Omega_j} - \frac{(\omega - \Omega_i)(\omega - \Omega_j)}{4\Omega_i\Omega_j}\right]\mathrm{d}\omega\\
= &\int_0^\infty\delta_{ij}\frac{\omega(\Omega_i + \Omega_j)}{2\Omega_i\Omega_j}\bar{P}_i(\omega)\;\mathrm{d}\omega,
\end{split}
\end{equation}
where we have let $P_{ni}(\omega) = \pi^2\lambda_n\bar{P}_i(\omega)$. Finally, looking at the form of the integral in Eq. \eqref{eq:qq-minus-ss-integral}, we can propose that $\bar{P}_{i}(\omega) = 4\Omega_i^2 \Sigma_i^2(\omega)/|\omega^2 - \Omega_i^2z_i(\omega)|^2$, wherein
\begin{equation}
\begin{split}
% \Sigma_i^2(\omega) &= \sum_\sigma g_{i\sigma}^2(\omega),\\
z_i(\omega) &= 1 + \lim_{\epsilon\to0}\frac{2}{\Omega_i}\int_{-\infty}^{\infty}\frac{\Sigma_i^2(\nu)}{\omega - \nu - \mathrm{i}\epsilon}\;\mathrm{d}\nu
\end{split}
\end{equation}
and $\Sigma_i^2(\omega)$ is an as-yet-undertermined function with units of coupling constant squared that obeys the property $\Sigma_i^2(-\omega) = -\Sigma_i^2(\omega)$. This provides us with
\begin{equation}
\begin{split}
\sum_j\hat{a}_j(t)\sum_n\int_0^\infty P_{ni}(\omega)&\left[\frac{(\omega + \Omega_i)(\omega + \Omega_j)}{4\Omega_i\Omega_j}q_{ni}^*(\omega)q_{nj}(\omega) - s_{ni}(\omega)s_{nj}^*(\omega)\right]\mathrm{d}\omega\\
&= \sum_j\hat{a}_j(t)\int_0^\infty\delta_{ij}\frac{\omega(\Omega_i + \Omega_j)}{2\Omega_i\Omega_j}\frac{4\Omega_i^2\Sigma_i^2(\omega)}{|\omega^2 - \Omega_i^2z_i(\omega)|^2}\;\mathrm{d}\omega\\
&= \hat{a}_i(t)\int_0^\infty\frac{4\Omega_i\omega\Sigma_i^2(\omega)}{|\omega^2 - \Omega_i^2z_i(\omega)|^2}\;\mathrm{d}\omega\\
&= \hat{a}_i(t)
\end{split}
\end{equation}
such that
\begin{equation}
\begin{split}
P_{ni}(\omega) &= \pi^2\lambda_n\frac{4\Omega_i^2\Sigma_i^2(\omega)}{|\omega^2 - \Omega_i^2z_i(\omega)|^2},\\
\mathcal{N}_i &= 1
\end{split}
\end{equation}
are candidate solutions for the prefactor and normalization constant of $\hat{a}_i(t)$.

Moving forward, we can see that
\begin{equation}
\begin{split}
&\sum_j\hat{a}_j^\dagger(t)\sum_n\int_0^\infty P_{ni}(\omega)\left[\frac{\omega + \Omega_i}{2\Omega_i}q_{ni}^*(\omega)s_{nj}(\omega) - \frac{\omega + \Omega_j}{2\Omega_j}s_{ni}(\omega)q_{nj}^*(\omega)\right]\mathrm{d}\omega\\
= &\sum_j\hat{a}_j^\dagger(t)\sum_n\int_0^\infty \pi^2\lambda_n\frac{4\Omega_i^2\Sigma_i^2(\omega)}{|\omega^2 - \Omega_i^2z_i(\omega)|^2}\\
&\qquad\times\left[\frac{(\omega + \Omega_i)(\omega - \Omega_j)}{4\Omega_i\Omega_j}q_{ni}^*(\omega)q_{nj}(\omega) - \frac{(\omega - \Omega_i)(\omega + \Omega_j)}{4\Omega_i\Omega_j}q_{ni}(\omega)q_{nj}^*(\omega)\right]\;\mathrm{d}\omega\\
&= \sum_j\hat{a}_j^\dagger(t)\int_0^\infty\frac{4\Omega_i^2\Sigma_i^2(\omega)}{|\omega^2 - \Omega_i^2z_i(\omega)|^2}\frac{2\omega(\Omega_i - \Omega_j)}{4\Omega_i\Omega_j}\delta_{ij}\;\mathrm{d}\omega\,\mathrm{d}\nu\\
&= 0.
\end{split}
\end{equation}
Further,
\begin{equation}
\begin{split}
&\sum_\sigma\int_0^\infty\hat{b}_\sigma(\nu,t)\sum_n\int_0^\infty P_{ni}(\omega)\left[\frac{\omega + \Omega_i}{2\Omega_i}q_{ni}^*(\omega)u_{n\sigma}(\omega,\nu) - s_{ni}(\omega)v_{n\sigma}^*(\omega,\nu)\right]\mathrm{d}\omega\,\mathrm{d}\nu\\
= &\sum_\sigma\int_0^\infty\hat{b}_\sigma(\nu,t)\sum_n\int_0^\infty \pi^2\lambda_n\frac{4\Omega_i^2\Sigma_i^2(\omega)}{|\omega^2 - \Omega_i^2z_i(\omega)|^2}\left[\frac{\omega + \Omega_i}{2\Omega_i}q_{ni}^*(\omega)\left(PV\left\{\frac{1}{\omega - \tilde{\nu}_\sigma}\right\}\sum_jg_{j\sigma}(\nu)q_{nj}(\omega)\right.\right.\\
&\qquad\left.\left. + \sum_jg_{j\sigma}(\nu)q_{nj}(\omega)x_{j\sigma}(\omega)\delta(\omega - \tilde{\nu}_\sigma)\right) - \frac{\omega - \Omega_i}{2\Omega_i}q_{ni}(\omega)\frac{1}{\omega + \tilde{\nu}_\sigma}\sum_jg_{j\sigma}(\nu)q_{nj}^*(\omega)\right]\mathrm{d}\omega\,\mathrm{d}\nu.
\end{split}
\end{equation}
To align this expression with the form of Eq. \eqref{eq:qu-minus-sv-integral} such that it evaluates to zero, we choose the form of our final unknown quantity,
\begin{equation}
x_{i\sigma}(\omega) = \mathrm{i}\pi + \frac{\omega^2 - \Omega_i^2z_i(\omega)}{2\Omega_i\Sigma_i^2(\omega)},
\end{equation}
such that
\begin{equation}
u_{n\sigma}(\omega,\nu) = \lim_{\epsilon\to0}\frac{1}{\omega - \tilde{\nu}_\sigma - \mathrm{i}\epsilon}\sum_ig_{i\sigma}(\nu)q_{ni}(\omega) + \sum_ig_{i\sigma}(\nu)q_{ni}(\omega)\frac{\omega^2 - \Omega_i^2z_i(\omega)}{2\Omega_i\Sigma_i^2(\omega)}\delta(\omega - \tilde{\nu}_\sigma)
\end{equation}
and
\begin{equation}
\begin{split}
&\sum_\sigma\int_0^\infty\hat{b}_\sigma(\nu,t)\sum_n\int_0^\infty P_{ni}(\omega)\left[\frac{\omega + \Omega_i}{2\Omega_i}q_{ni}^*(\omega)u_{n\sigma}(\omega,\nu) - s_{ni}(\omega)v_{n\sigma}^*(\omega,\nu)\right]\mathrm{d}\omega\,\mathrm{d}\nu\\
% = &\sum_\sigma\int_0^\infty\hat{b}_\sigma(\nu,t)\sum_n\int_0^\infty \pi^2\lambda_n\frac{4\Omega_i^2\Sigma_i^2(\omega)}{|\omega^2 - \Omega_i^2z_i(\omega)|^2}\left[\frac{\omega + \Omega_i}{2\Omega_i}q_{ni}^*(\omega)\left(PV\left\{\frac{1}{\omega - \tilde{\nu}_\sigma}\right\}\sum_jg_{j\sigma}(\nu)q_{nj}(\omega)\right.\right.\\
% &\qquad\left.\left. + \sum_jg_{j\sigma}(\nu)q_{nj}(\omega)x_{j\sigma}(\omega)\delta(\omega - \tilde{\nu}_\sigma)\right) - \frac{\omega - \Omega_i}{2\Omega_i}q_{ni}(\omega)\frac{1}{\omega + \tilde{\nu}_\sigma}\sum_jg_{j\sigma}(\nu)q_{nj}^*(\omega)\right]\mathrm{d}\omega\,\mathrm{d}\nu\\
% = &\sum_\sigma\int_0^\infty\hat{b}_\sigma(\nu,t)\sum_n\left[\int_0^\infty \pi^2\lambda_n\frac{2\Omega_i\Sigma_i^2(\omega)}{|\omega^2 - \Omega_i^2z_i(\omega)|^2}\lim_{\epsilon\to0}\left(\frac{\omega + \Omega_i}{\omega - \tilde{\nu}_\sigma - \mathrm{i}\epsilon} - \frac{\omega - \Omega_i}{\omega + \tilde{\nu}_\sigma}\right)\right.\\
% &\qquad\times q_{ni}^*(\omega)\sum_jg_{j\sigma}(\nu)q_{nj}(\omega)\;\mathrm{d}\omega + \pi^2\lambda_n\frac{4\Omega_i^2\Sigma_i^2(\tilde{\nu}_\sigma)}{|\tilde{\nu}_\sigma^2 - \Omega_i^2z_i(\tilde{\nu}_\sigma)|^2}\frac{\tilde{\nu}_\sigma + \Omega_i}{2\Omega_i}\\
% &\left.\qquad\times q_{ni}(\tilde{\nu}_\sigma)\sum_jg_{j\sigma}(\nu)q_{nj}^*(\tilde{\nu}_\sigma)\frac{\tilde{\nu}_\sigma^2 - \Omega_i^2z_i(\tilde{\nu}_\sigma)}{2\Omega_i\Sigma_i^2(\tilde{\nu}_\sigma)}\right]\mathrm{d}\nu\\
% = &\sum_\sigma\int_0^\infty\hat{b}_\sigma(\nu,t)\left[\int_0^\infty\frac{2\Omega_i\Sigma_i^2(\omega)}{|\omega^2 - \Omega_i^2z_i(\omega)|^2}\lim_{\epsilon\to0}\left(\frac{\omega + \Omega_i}{\omega - \tilde{\nu}_\sigma - \mathrm{i}\epsilon} - \frac{\omega - \Omega_i}{\omega + \tilde{\nu}_\sigma}\right)\sum_jg_{j\sigma}(\nu)\sum_n\pi^2\lambda_nq_{ni}^*(\omega)q_{nj}(\omega)\;\mathrm{d}\omega\right.\\
% &\qquad\left. + \frac{\tilde{\nu}_\sigma + \Omega_i}{\tilde{\nu}_\sigma^2 - \Omega_i^2z_i^*(\tilde{\nu}_\sigma)}\sum_jg_{j\sigma}(\nu)\sum_n\pi^2\lambda_nq_{ni}(\tilde{\nu}_\sigma)q_{nj}^*(\tilde{\nu}_\sigma)\right]\mathrm{d}\nu\\
% = &\sum_\sigma\int_0^\infty\hat{b}_\sigma(\nu,t)\left[\int_0^\infty\frac{2\Omega_i\Sigma_i^2(\omega)}{|\omega^2 - \Omega_i^2z_i(\omega)|^2}\lim_{\epsilon\to0}\left(\frac{\omega + \Omega_i}{\omega - \tilde{\nu}_\sigma - \mathrm{i}\epsilon} - \frac{\omega - \Omega_i}{\omega + \tilde{\nu}_\sigma}\right)\sum_jg_{j\sigma}(\nu)\delta_{ij}\;\mathrm{d}\omega\right.\\
% &\qquad\left. + \frac{\tilde{\nu}_\sigma + \Omega_i}{\tilde{\nu}_\sigma^2 - \Omega_i^2z_i^*(\tilde{\nu}_\sigma)}\sum_jg_{j\sigma}(\nu)\delta_{ij}\right]\mathrm{d}\nu\\
= &\sum_\sigma\int_0^\infty\hat{b}_\sigma(\nu,t)\left[\int_0^\infty\frac{2\Omega_i\Sigma_i^2(\omega)g_{i\sigma}(\nu)}{|\omega^2 - \Omega_i^2z_i(\omega)|^2}\lim_{\epsilon\to0}\left(\frac{\omega + \Omega_i}{\omega - \tilde{\nu}_\sigma - \mathrm{i}\epsilon} - \frac{\omega - \Omega_i}{\omega + \tilde{\nu}_\sigma}\right)\mathrm{d}\omega + \frac{(\tilde{\nu}_\sigma + \Omega_i)g_{i\sigma}(\nu)}{\tilde{\nu}_\sigma^2 - \Omega_i^2z_i^*(\tilde{\nu}_\sigma)}\right]\mathrm{d}\nu\\
= &\;0.
\end{split}
\end{equation}
The final equality to zero is inferred from the form of the expression in brackets in the second-to-last line above. The bracketed expression can be seen to go to zero via the same process as Eq. \eqref{eq:qu-minus-sv-integral}. An analogous process using the logic of Eq. \eqref{eq:qv-minus-su-integral} can be used to show that 
\begin{equation}
\sum_\sigma\int_0^\infty b_\sigma^\dagger(\nu,t)\sum_n\int_0^\infty P_{ni}(\omega)\left[\frac{\omega + \Omega_i}{2\Omega_i}q_{ni}^*(\omega)v_{n\sigma}(\omega,\nu) - s_{ni}(\omega)u_{n\sigma}^*(\omega,\nu)\right]\mathrm{d}\omega\,\mathrm{d}\nu = 0,
\end{equation}
but we will not show it explicitly here. Therefore, we have confirmed that
\begin{equation}
\hat{a}_i(t) = \sum_n\int_0^\infty \pi^2\lambda_n\frac{4\Omega_i^2\Sigma_i^2(\omega)}{|\omega^2 - \Omega_i^2z_i(\omega)|^2}\left[\frac{\omega + \Omega_i}{2\Omega_i}q_{ni}^*(\omega)\hat{C}_n(\omega,t) - s_{ni}(\omega)\hat{C}_n^\dagger(\omega,t)\right]\mathrm{d}\omega.
\end{equation}
Note that an explicit form for $\Sigma_i^2(\omega)$ has not been determined, such that the above definition for $\hat{a}_i(t)$ is not unique. We can then regard $\Sigma_i(\omega)$ and $z_i(\omega)$ as ``dummy'' functions, with explicit values free to be chosen as is convenient for anyone implementing the model except where they are subject to the constraints listed above.

A trial solution for the bath operators is then
\begin{equation}
\hat{b}_\sigma(\nu,t) = \frac{1}{\mathcal{M}_\sigma(\nu)}\sum_n\int_0^\infty K_{n\sigma}(\omega)\left[u_{n\sigma}^{\prime*}(\omega,\nu)\hat{C}_n'(\omega,t) - v_{n\sigma}(\omega,\nu)\hat{C}_n^{\prime\dagger}(\omega,t)\right]\mathrm{d}\omega.
\end{equation}
The primed expansion coefficient
\begin{equation}
\begin{split}
u_{n\sigma}'(\omega,\nu) &= PV\left\{\frac{1}{\omega - \tilde{\nu}_\sigma}\right\}\sum_ig_{i\sigma}(\nu)q_{ni}(\omega) + \sum_ig_{i\sigma}(\nu)q_{ni}(\omega)[\mathrm{i}\pi + y_{i\sigma}(\omega)]\delta(\omega - \tilde{\nu}_\sigma)\\
&= \lim_{\epsilon\to0}\frac{1}{\omega - \tilde{\nu}_\sigma - \mathrm{i}\epsilon}\sum_ig_{i\sigma}(\nu)q_{ni}(\omega) + \sum_ig_{i\sigma}(\nu)q_{ni}(\omega)y_{i\sigma}(\omega)\delta(\omega - \tilde{\nu}_\sigma)
\end{split}
\end{equation}
is defined such that its Dirac-delta term, which can have any complex coefficient and still allow $u_{n\sigma}'(\omega,\nu)$ to satisfy Eq. \eqref{eq:hybridizationCoefficientsSystemManyModes}, has a prefactor $y_{i\sigma}(\omega) + \mathrm{i}\pi$ that may or may not be equal to $x_{i\sigma}(\omega)$. The primed polariton operators are similarly defined such that $\hat{C}_n'(\omega,t) = \hat{C}_n(\omega,t)|_{u_{n\sigma}\to u'_{n\sigma}}$. 

Expanding, we find
\begin{equation}
\begin{split}
\hat{b}_\sigma(\nu,t) 
% &= \frac{1}{\mathcal{M}_\sigma(\nu)}\sum_n\int_0^\infty K_{n\sigma}(\omega)\left[u_{n\sigma}^{\prime*}(\omega,\nu)\left(\sum_i\left[\frac{\omega + \Omega_i}{2\Omega_i}q_{ni}(\omega)\hat{a}_i(t) + s_{ni}(\omega)\hat{a}_i^\dagger(t)\right]\right.\right.\\
% &\qquad\left. + \sum_{\sigma'}\int_0^\infty\left[u'_{n\sigma'}(\omega,\omega')\hat{b}_{\sigma'}(\omega',t) + v_{n\sigma}(\omega,\omega')\hat{b}_{\sigma'}^\dagger(\omega',t)\right]\mathrm{d}\omega'\right)\\
% &\qquad - v_{n\sigma}(\omega,\nu)\left(\sum_i\left[\frac{\omega + \Omega_i}{2\Omega_i}q_{ni}^*(\omega)\hat{a}_i^\dagger(t) + s_{ni}^*(\omega)\hat{a}_i(t)\right]\right.\\
% &\qquad\left.\left. + \sum_{\sigma'}\int_0^\infty\left[u_{n\sigma'}^{\prime*}(\omega,\omega')\hat{b}_{\sigma'}^\dagger(\omega',t) + v_{n\sigma'}^*(\omega,\omega')\hat{b}_{\sigma'}(\omega',t)\right]\mathrm{d}\omega'\right)\right]\mathrm{d}\omega\\
&= \frac{1}{\mathcal{M}_\sigma(\nu)}\left(\sum_i\hat{a}_i(t)\int_0^\infty\sum_n K_{n\sigma}(\omega)\left[\frac{\omega + \Omega_i}{2\Omega_i}q_{ni}(\omega)u_{n\sigma}^{\prime*}(\omega,\nu) - s_{ni}^*(\omega)v_{n\sigma}(\omega,\nu)\right]\mathrm{d}\omega\right.\\
&\qquad + \sum_i\hat{a}_i^\dagger(t)\int_0^\infty\sum_nK_{n\sigma}(\omega)\left[s_{ni}(\omega)u_{n\sigma}^{\prime*}(\omega,\nu) - \frac{\omega + \Omega_i}{2\Omega_i}q_{ni}^*(\omega)v_{n\sigma}(\omega,\nu)\right]\mathrm{d}\omega\\
&\qquad + \sum_{\sigma'}\int_0^\infty\hat{b}_{\sigma'}(\nu',t)\sum_n\int_0^\infty K_{n\sigma}(\omega)\left[u_{n\sigma}^{\prime*}(\omega,\nu)u'_{n\sigma'}(\omega,\nu') - v_{n\sigma}(\omega,\nu)v_{n\sigma'}^*(\omega,\nu')\right]\mathrm{d}\omega\,\mathrm{d}\nu'\\
&\qquad\left. + \sum_{\sigma'}\int_0^\infty\hat{b}_{\sigma'}^\dagger(\nu',t)\sum_n\int_0^\infty K_{n\sigma}(\omega)\left[u_{n\sigma}^{\prime*}(\omega,\nu)v_{n\sigma'}(\omega,\nu') - v_{n\sigma}(\omega,\nu)u_{n\sigma'}^{\prime*}(\omega,\nu')\right]\mathrm{d}\omega\,\mathrm{d}\nu'\right).
\end{split}
\end{equation}
With
\begin{equation}
\begin{split}
K_{n\sigma}(\omega) &= \pi^2\lambda_n\frac{4\Omega^2\Sigma^2(\omega)}{|\omega^2 - \Omega^2z(\omega)|^2},\\
z(\omega) &= 1 + \lim_{\epsilon\to0}\frac{2}{\Omega}\int_{-\infty}^\infty\frac{\Sigma^2(\nu)}{\omega - \nu - \mathrm{i}\epsilon}\;\mathrm{d}\nu,
\end{split}
\end{equation}
in which $\Omega$ and $\Sigma(\omega)$ are left to be constrained by downstream requirements, we can see that
\begin{equation}
\begin{split}
&\sum_n\int_0^\infty K_{n\sigma}(\omega)\left[u_{n\sigma}^{\prime*}(\omega,\nu)u'_{n\sigma'}(\omega,\nu') - v_{n\sigma}(\omega,\nu)v_{n\sigma'}^*(\omega,\nu')\right]\mathrm{d}\omega\\
% = &\sum_n\int_0^\infty\pi^2\lambda_n\frac{4\Omega^2\Sigma^2(\omega)}{|\omega^2 - \Omega^2z(\omega)|^2}\left[\left(\frac{1}{\omega - \tilde{\nu}_\sigma + \mathrm{i}\epsilon}\sum_ig_{i\sigma}(\nu)q_{ni}(\omega) + \sum_ig_{i\sigma}(\nu)q_{ni}(\omega)y_{i\sigma}^*(\omega)\delta(\omega - \tilde{\nu}_\sigma)\right)\right.\\
% &\qquad \times \left(\frac{1}{\omega - \tilde{\nu}'_{\sigma'} - \mathrm{i}\epsilon}\sum_jg_{j\sigma'}(\nu')q_{nj}(\omega) + \sum_jg_{j\sigma'}(\nu')q_{nj}(\omega)y_{j\sigma'}(\omega)\delta(\omega - \tilde{\nu}'_{\sigma'})\right)\\
% &\left.\qquad - \frac{1}{\omega + \tilde{\nu}_\sigma}\frac{1}{\omega + \tilde{\nu}'_{\sigma'}}\sum_ig_{i\sigma}(\nu)q_{ni}(\omega)\sum_jg_{j\sigma'}(\nu')q_{nj}(\omega)\right]\mathrm{d}\omega\\
% &= \sum_n\pi^2\lambda_n\left[\int_0^\infty\frac{4\Omega^2\Sigma^2(\omega)}{|\omega^2 - \Omega^2z(\omega)|^2}\left(\frac{1}{\omega - \tilde{\nu}_\sigma + \mathrm{i}\epsilon}\frac{1}{\omega - \tilde{\nu}'_{\sigma'} - \mathrm{i}\epsilon} - \frac{1}{\omega + \tilde{\nu}_\sigma}\frac{1}{\omega + \tilde{\nu}'_{\sigma'}}\right)\right.\\
% &\qquad\times\sum_{ij}g_{i\sigma}(\nu)g_{j\sigma'}(\nu')q_{ni}(\omega)q_{nj}(\omega)\;\mathrm{d}\omega\\
% &\qquad + \frac{4\Omega^2\Sigma^2(\tilde{\nu}'_{\sigma'})}{|\tilde{\nu}_{\sigma'}^{\prime2} - \Omega^2z(\tilde{\nu}'_{\sigma'})|^2}\frac{1}{\tilde{\nu}'_{\sigma'} - \tilde{\nu}_\sigma + \mathrm{i}\epsilon}\sum_{ij}y_{j\sigma'}(\tilde{\nu}'_{\sigma'})g_{i\sigma}(\nu)g_{j\sigma'}(\nu')q_{ni}(\tilde{\nu}'_{\sigma'})q_{nj}(\tilde{\nu}'_{\sigma'})\\
% &\qquad + \frac{4\Omega^2\Sigma^2(\tilde{\nu}_\sigma)}{|\tilde{\nu}_\sigma^2 - \Omega^2z(\tilde{\nu}_\sigma)|^2}\frac{1}{\tilde{\nu}_\sigma - \tilde{\nu}'_{\sigma'} - \mathrm{i}\epsilon}\sum_{ij}y_{i\sigma}^*(\tilde{\nu}_\sigma)g_{i\sigma}(\nu)g_{j\sigma'}(\nu')q_{ni}(\tilde{\nu}_{\sigma})q_{nj}(\tilde{\nu}_\sigma)\\
% &\qquad\left. + \frac{4\Omega^2\Sigma^2(\tilde{\nu}_\sigma)}{|\nu_\sigma^2 - \Omega^2z(\tilde{\nu}_\sigma)|^2}\sum_{ij}y^*_{i\sigma}(\tilde{\nu}_\sigma)y_{j\sigma'}(\tilde{\nu}_\sigma)g_{i\sigma}(\nu)g_{j\sigma'}(\nu')q_{ni}(\tilde{\nu}_\sigma)q_{nj}(\tilde{\nu}_\sigma)\delta(\tilde{\nu}_\sigma - \tilde{\nu}'_{\sigma'}) \vphantom{\int_0^\infty} \right]\\
= &\sum_ig_{i\sigma}(\nu)g_{j\sigma'}(\nu')\int_0^\infty\frac{4\Omega^2\Sigma^2(\omega)}{|\omega^2 - \Omega^2z(\omega)|^2}\left(\frac{1}{\omega - \tilde{\nu}_\sigma + \mathrm{i}\epsilon}\frac{1}{\omega - \tilde{\nu}'_{\sigma'} - \mathrm{i}\epsilon} - \frac{1}{\omega + \tilde{\nu}_\sigma}\frac{1}{\omega + \tilde{\nu}'_{\sigma'}}\right)\mathrm{d}\omega\\
&\qquad + \frac{4\Omega^2\Sigma^2(\tilde{\nu}'_{\sigma'})}{|\tilde{\nu}_{\sigma'}^{\prime2} - \Omega^2z(\tilde{\nu}'_{\sigma'})|^2}\frac{1}{\tilde{\nu}'_{\sigma'} - \tilde{\nu}_\sigma + \mathrm{i}\epsilon}\sum_{i}y_{i\sigma'}(\tilde{\nu}'_{\sigma'})g_{i\sigma}(\nu)g_{i\sigma'}(\nu')\\
&\qquad + \frac{4\Omega^2\Sigma^2(\tilde{\nu}_\sigma)}{|\tilde{\nu}_\sigma^2 - \Omega^2z(\tilde{\nu}_\sigma)|^2}\frac{1}{\tilde{\nu}_\sigma - \tilde{\nu}'_{\sigma'} - \mathrm{i}\epsilon}\sum_{i}y_{i\sigma}^*(\tilde{\nu}_\sigma)g_{i\sigma}(\nu)g_{i\sigma'}(\nu')\\
&\qquad + \frac{4\Omega^2\Sigma^2(\tilde{\nu}_\sigma)}{|\nu_\sigma^2 - \Omega^2z(\tilde{\nu}_\sigma)|^2}\sum_{i}y^*_{i\sigma}(\tilde{\nu}_\sigma)y_{i\sigma'}(\tilde{\nu}'_\sigma)g_{i\sigma}(\nu)g_{i\sigma'}(\nu')\delta(\tilde{\nu}_\sigma - \tilde{\nu}'_{\sigma'}).
\end{split}
\end{equation}
From here, we will let
\begin{equation}
y_{i\sigma}(\omega) = \frac{\omega^2 - \Omega^2z(\omega)}{2\Omega\Sigma^2(\omega)}\mathcal{T}_{i\sigma}(\omega)
\end{equation}
where
\begin{equation}
\mathcal{T}_{i\sigma}(\omega)f(\mathbf{x}) = 
\begin{cases}
f(\mathbf{x}), & f(\mathbf{x})\neq f(\omega)\\
\frac{w_{i\sigma}}{g_{i\sigma}(\omega_\sigma)}\sqrt{\frac{\omega_\sigma}{\omega}}\Sigma(\omega)f(\omega), & f(\mathbf{x}) = f(\omega),
\end{cases}
\end{equation}
is a function that takes a value of one when part of a multivariate product of functions and takes a different value when part of a univariate product of functions. Further, $w_{i\sigma}$ are elements of the normalized vector $\mathbf{w}_\sigma$ defined such that $\mathbf{w}_\sigma\cdot\mathbf{w}_{\sigma'} = \delta_{\sigma\sigma'}$. Substitution into the above expression provides
\begin{equation}
\begin{split}
&\sum_n\int_0^\infty K_{n\sigma}(\omega)\left[u_{n\sigma}^{\prime*}(\omega,\nu)u'_{n\sigma'}(\omega,\nu') - v_{n\sigma}(\omega,\nu)v_{n\sigma'}^*(\omega,\nu')\right]\mathrm{d}\omega\\
= &\sum_ig_{i\sigma}(\nu)g_{j\sigma'}(\nu')\int_0^\infty\frac{4\Omega^2\Sigma^2(\omega)}{|\omega^2 - \Omega^2z(\omega)|^2}\left(\frac{1}{\omega - \tilde{\nu}_\sigma + \mathrm{i}\epsilon}\frac{1}{\omega - \tilde{\nu}'_{\sigma'} - \mathrm{i}\epsilon} - \frac{1}{\omega + \tilde{\nu}_\sigma}\frac{1}{\omega + \tilde{\nu}'_{\sigma'}}\right)\mathrm{d}\omega\\
&\qquad + \sum_i\frac{2\Omega g_{i\sigma}(\nu)g_{i\sigma'}(\nu')}{\tilde{\nu}_\sigma^2 - \Omega^2z^*(\tilde{\nu}_\sigma)}\frac{1}{\tilde{\nu}_\sigma - \tilde{\nu}'_{\sigma'} - \mathrm{i}\epsilon} + \sum_i\frac{2\Omega g_{i\sigma}(\nu)g_{i\sigma'}(\nu')}{\tilde{\nu}^{\prime2}_{\sigma'} - \Omega^2z(\tilde{\nu}'_{\sigma'})}\frac{1}{\tilde{\nu}'_{\sigma'} - \tilde{\nu}_\sigma + \mathrm{i}\epsilon}\\
&\qquad + \frac{1}{\Sigma^2(\tilde{\nu}_\sigma)}\sum_i\frac{w_{i\sigma}}{g_{i\sigma}(\nu)}\sqrt{\frac{\nu}{\tilde{\nu}_\sigma}}\Sigma(\tilde{\nu}_\sigma)\frac{w_{i\sigma'}}{g_{i\sigma'}(\nu')}\sqrt{\frac{\nu'}{\tilde{\nu}'_{\sigma'}}}\Sigma(\tilde{\nu}'_{\sigma'})g_{i\sigma}(\nu)g_{i\sigma'}(\nu')\delta(\tilde{\nu}_\sigma - \tilde{\nu}'_{\sigma'})\\
&= 0 + \frac{\Sigma(\tilde{\nu}'_{\sigma'})}{\Sigma(\tilde{\nu}_\sigma)}\sqrt{\frac{\nu\nu'}{\tilde{\nu}_\sigma\tilde{\nu}'_{\sigma'}}}\sum_iw_{i\sigma}w_{i\sigma'}\delta(\tilde{\nu}_\sigma - \tilde{\nu}'_{\sigma'})
\end{split}
\end{equation}
where we have recognized that the last term on the right-hand side is either a function of a single variable ($\nu$ or $\nu'$, equivalently) or zero, as enforced by the Dirac delta. The terms on the right-hand side not proportional to a Dirac delta sum to zero via the process shown in Eq. \eqref{eq:uu-minus-vv-integral} such that
\begin{equation}
\begin{split}
&\sum_n\int_0^\infty K_{n\sigma}(\omega)\left[u_{n\sigma}^{\prime*}(\omega,\nu)u'_{n\sigma'}(\omega,\nu') - v_{n\sigma}(\omega,\nu)v_{n\sigma'}^*(\omega,\nu')\right]\mathrm{d}\omega\\
= & \frac{\Sigma(\tilde{\nu}'_{\sigma'})}{\Sigma(\tilde{\nu}_\sigma)}\sqrt{\frac{\nu\nu'}{\tilde{\nu}_\sigma\tilde{\nu}'_{\sigma'}}}\delta_{\sigma\sigma'}\delta(\tilde{\nu}_\sigma - \tilde{\nu}'_{\sigma'})\\
= & \frac{\Sigma(\tilde{\nu}'_{\sigma'})}{\Sigma(\tilde{\nu}_\sigma)}\sqrt{\frac{\nu\nu'}{\tilde{\nu}_\sigma\tilde{\nu}'_{\sigma'}}}\delta_{\sigma\sigma'}\frac{\tilde{\nu}'_\sigma}{\nu'}\left[\delta(\nu + \nu') + \delta(\nu - \nu')\right]\\
&= \delta_{\sigma\sigma'}\delta(\nu - \nu')
\end{split}
\end{equation}
and
\begin{equation}
\begin{split}
\sum_{\sigma'}\int_0^\infty\hat{b}_{\sigma'}(\nu',t)&\sum_n\int_0^\infty K_{n\sigma}(\omega)\left[u_{n\sigma}^{\prime*}(\omega,\nu)u'_{n\sigma'}(\omega,\nu') - v_{n\sigma}(\omega,\nu)v_{n\sigma'}^*(\omega,\nu')\right]\mathrm{d}\omega\,\mathrm{d}\nu\\
&= \sum_{\sigma'}\int_0^\infty\hat{b}_{\sigma'}(\nu',t)\delta_{\sigma\sigma'}\delta(\nu - \nu')\;\mathrm{d}\nu'\\
&= \hat{b}_\sigma(\nu,t).
\end{split}
\end{equation}

Next, we can look at the integral
\begin{equation}
\begin{split}
&\sum_n\int_0^\infty K_{n\sigma}(\omega)\left[u^{\prime*}_{n\sigma}(\omega,\nu)v_{n\sigma'}(\omega,\nu') - v_{n\sigma}(\omega,\nu)u^{\prime*}_{n\sigma'}(\omega,\nu')\right]\mathrm{d}\omega\\
% = &\sum_n\int_0^\infty\pi^2\lambda_n\frac{4\Omega^2\Sigma^2(\omega)}{|\omega^2 - \Omega^2z(\omega)|^2}\left[\left(\frac{1}{\omega - \tilde{\nu}_\sigma + \mathrm{i}\epsilon}\sum_ig_{i\sigma}(\nu)q_{ni}^*(\omega) + \sum_ig_{i\sigma}(\nu)q_{ni}^*(\omega)\frac{\omega^2 - \Omega^2 z^*(\omega)}{2\Omega\Sigma^2(\omega)}\delta(\omega - \tilde{\nu}_\sigma)\right)\right.\\
% &\qquad\times \frac{1}{\omega + \tilde{\nu}'_{\sigma'}}\sum_jg_{j\sigma'}(\nu')q_{nj}(\omega) - \frac{1}{\omega + \tilde{\nu}_\sigma}\sum_ig_{i\sigma}(\nu)q_{ni}(\omega)\left(\frac{1}{\omega - \tilde{\nu}'_{\sigma'} - \mathrm{i}\epsilon}\sum_jg_{j\sigma'}(\nu')q_{nj}^*(\omega)\right.\\
% &\qquad\left.\left. + \sum_jg_{j\sigma'}(\nu')q_{nj}^*(\omega)\frac{\omega^2 - \Omega^2z(\omega)}{2\Omega\Sigma^2(\omega)}\delta(\omega - \tilde{\nu}'_{\sigma'})\right)\right]\mathrm{d}\omega\\
% = &\sum_n\int_0^\infty\pi^2\lambda_n\frac{4\Omega^2\Sigma^2(\omega)}{|\omega^2 - \Omega^2z(\omega)|^2}\left(\frac{1}{\omega - \tilde{\nu}_\sigma + \mathrm{i}\epsilon}\frac{1}{\omega + \tilde{\nu}'_{\sigma'}}\sum_{ij}g_{i\sigma}(\nu)g_{j\sigma'}(\nu')q_{ni}^*(\omega)q_{nj}(\omega)\right.\\
% &\qquad\left. - \frac{1}{\omega - \tilde{\nu}'_{\sigma'} - \mathrm{i}\epsilon}\frac{1}{\omega + \tilde{\nu}_\sigma}\sum_{ij}g_{i\sigma}(\nu)g_{j\sigma'}(\nu')q_{ni}(\omega)q_{nj}^*(\omega)\right)\mathrm{d}\omega\\
% &\qquad + \sum_n\pi^2\lambda_n\frac{4\Omega^2\Sigma^2(\tilde{\nu}_\sigma)}{|\tilde{\nu}_\sigma^2 - \Omega^2z(\tilde{\nu}_\sigma)|^2}\frac{\tilde{\nu}_\sigma^2 - \Omega^2z^*(\tilde{\nu}_\sigma)}{2\Omega\Sigma^2(\tilde{\nu}_\sigma)}\frac{1}{\tilde{\nu}_\sigma + \tilde{\nu}'_\sigma}\sum_{ij}g_{i\sigma}(\nu)g_{j\sigma'}(\nu')q_{ni}^*(\tilde{\nu}_\sigma)q_{nj}(\tilde{\nu}_\sigma)\\
% &\qquad - \sum_n\pi^2\lambda_n\frac{4\Omega^2\Sigma^2(\tilde{\nu}'_{\sigma'})}{|\tilde{\nu}^{\prime2}_{\sigma'} - \Omega^2z(\tilde{\nu}'_{\sigma'})|^2}\frac{\tilde{\nu}_{\sigma'}^{\prime2} - \Omega^2z(\tilde{\nu}'_{\sigma'})}{2\Omega\Sigma^2(\tilde{\nu}'_{\sigma'})}\frac{1}{\tilde{\nu}'_{\sigma'} + \tilde{\nu}_\sigma}\sum_{ij}g_{i\sigma}(\nu)g_{j\sigma'}(\nu')q_{ni}(\tilde{\nu}'_{\sigma'})q_{nj}^*(\tilde{\nu}'_{\sigma'})\\
= &\int_0^\infty\sum_ig_{i\sigma}(\nu)g_{i\sigma'}(\nu')\frac{4\Omega^2\Sigma^2(\omega)}{|\omega^2 - \Omega^2z(\omega)|^2}\left(\frac{1}{\omega - \tilde{\nu}_\sigma + \mathrm{i}\epsilon}\frac{1}{\omega + \tilde{\nu}'_{\sigma'}} - \frac{1}{\omega - \tilde{\nu}'_{\sigma'} - \mathrm{i}\epsilon}\frac{1}{\omega + \tilde{\nu}_\sigma}\right)\\
&\qquad + \sum_ig_{i\sigma}(\nu)g_{j\sigma'}(\nu')\frac{2\Omega}{\tilde{\nu}_\sigma^2 - \Omega^2z(\tilde{\nu}_\sigma)}\frac{1}{\tilde{\nu}_\sigma + \tilde{\nu}'_{\sigma'}} - \sum_ig_{i\sigma}(\nu)g_{j\sigma'}(\nu')\frac{2\Omega}{\tilde{\nu}_{\sigma'}^{\prime2} - \Omega^2z(\tilde{\nu}'_{\sigma'})}\frac{1}{\tilde{\nu}_\sigma + \tilde{\nu}'_{\sigma'}}\\
= &\;0,
\end{split}
\end{equation}
where the equality with zero is inferred by the analogous forms of the above expression and that of Eq. \eqref{eq:uv-minus-vu-integral}. Finally, we can see that
\begin{equation}
\begin{split}
&\sum_n\int_0^\infty K_{n\sigma}(\omega)\left[\frac{\omega + \Omega_i}{2\Omega_i}q_{ni}(\omega)u^{\prime*}_{n\sigma}(\omega,\nu) - s_{ni}^*(\omega)v_{n\sigma}(\omega,\nu)\right]\mathrm{d}\omega\\
% = &\sum_n\int_0^\infty\pi^2\lambda_n\frac{4\Omega^2\Sigma^2(\omega)}{|\omega^2 - \Omega^2z(\omega)|^2}\left[\frac{\omega + \Omega_i}{2\Omega_i}q_{ni}(\omega)\left(\lim_{\epsilon\to0}\frac{1}{\omega - \tilde{\nu}_\sigma + \mathrm{i}\epsilon}\sum_jg_{j\sigma}(\nu)q_{nj}^*(\omega)\right.\right.\\
% &\qquad\left.\left.  + \sum_jg_{j\sigma}(\nu)q_{nj}^*(\omega)y_{j\sigma}^*(\omega)\delta(\omega - \tilde{\nu}_\sigma)\right) - \frac{\omega - \Omega_i}{2\Omega_i}q_{ni}^*(\omega)\left(\frac{1}{\omega + \tilde{\nu}_\sigma}\sum_jg_{j\sigma}(\nu)q_{nj}(\omega)\right)\right]\\
% = &\sum_n\int_0^\infty\pi^2\lambda_n\frac{4\Omega^2\Sigma^2(\omega)}{|\omega^2 - \Omega^2z(\omega)|^2}\left[\frac{\omega + \Omega_i}{2\Omega_i}\frac{1}{\omega - \tilde{\nu}_\sigma + \mathrm{i}\epsilon}\sum_jg_{j\sigma}q_{ni}(\omega)q_{nj}^*(\omega)\right.\\
% &\qquad - \frac{\omega - \Omega_i}{2\Omega_i}\frac{1}{\omega + \tilde{\nu}_\sigma}\sum_jg_{j\sigma}(\nu)q_{ni}^*(\omega)q_{nj}(\omega)\\
% &\qquad\left. + \frac{\omega + \Omega_i}{2\Omega_i}\sum_jg_{j\sigma}(\nu)q_{ni}(\omega)q_{nj}^*(\omega)\frac{\omega^2 - \Omega^2z^*(\omega)}{2\Omega\Sigma^2(\omega)}\mathcal{T}_{j\sigma}(\omega)\delta(\omega - \tilde{\nu}_\sigma)\right]\mathrm{d}\omega\\
= &\int_0^\infty\frac{4\Omega^2\Sigma^2(\omega)g_{i\sigma}(\nu)}{|\omega^2 - \Omega^2z(\omega)|^2}\left[\frac{\omega + \Omega_i}{2\Omega_i}\frac{1}{\omega - \tilde{\nu}_\sigma + \mathrm{i}\epsilon} - \frac{\omega - \Omega_i}{2\Omega_i}\frac{1}{\omega + \tilde{\nu}_\sigma}\right]\mathrm{d}\omega + \frac{\tilde{\nu}_\sigma + \Omega_i}{2\Omega_i}\frac{2\Omega g_{i\sigma}(\nu)}{\tilde{\nu}_\sigma^2 - \Omega^2z(\tilde{\nu}_\sigma)}\\
= &\;0
\end{split}
\end{equation}
via the process of Eq. \eqref{eq:qu-minus-sv-integral} (note the overall complex conjugate) and
\begin{equation}
\begin{split}
&\sum_n\int_0^\infty K_{n\sigma}(\omega)\left[s_{ni}(\omega)u^{\prime*}_{n\sigma}(\omega,\nu) - \frac{\omega + \Omega_i}{2\Omega_i}q_{ni}^*(\omega)v_{n\sigma}(\omega,\nu)\right]\mathrm{d}\omega\\
% = &\sum_n\int_0^\infty\pi^2\lambda_n\frac{4\Omega^2\Sigma^2(\omega)}{|\omega^2 - \Omega^2z(\omega)|^2}\left[\frac{\omega - \Omega_i}{2\Omega_i}q_{ni}(\omega)\left(\frac{1}{\omega - \tilde{\nu}_\sigma + \mathrm{i}\epsilon}\sum_jg_{j\sigma}(\nu)q_{nj}^*(\omega)\right.\right.\\
% &\left.\left.\qquad + \sum_jg_{j\sigma}(\nu)q_{nj}^*(\omega)y_{j\sigma}^*(\omega)\delta(\omega - \tilde{\nu}_\sigma)\right) - \frac{\omega + \Omega_i}{2\Omega_i}q_{ni}^*(\omega)\frac{1}{\omega + \tilde{\nu}_\sigma}\sum_jg_{j\sigma}(\nu)q_{nj}(\omega)\right]\\
% = &\sum_n\int_0^\infty\pi^2\lambda_n\frac{4\Omega^2\Sigma^2(\omega)}{|\omega^2 - \Omega^2z(\omega)|^2}\left[\frac{\omega - \Omega_i}{2\Omega_i}\frac{1}{\omega - \tilde{\nu}_\sigma + \mathrm{i}\epsilon}\sum_jg_{j\sigma}(\nu)q_{ni}(\omega)q_{nj}^*(\omega)\right.\\
% &\qquad - \frac{\omega + \Omega_i}{2\Omega_i}\frac{1}{\omega + \tilde{\nu}_\sigma}\sum_jg_{j\sigma}(\nu)q_{ni}^*(\omega)q_{nj}(\omega)\\
% &\qquad\left. + \frac{\omega - \Omega_i}{2\Omega_i}\sum_jg_{j\sigma}q_{ni}(\omega)q_{nj}^*(\omega)\frac{\omega^2 - \Omega^2z^*(\omega)}{2\Omega\Sigma^2(\omega)}\delta(\omega - \tilde{\nu}_\sigma)\right]\mathrm{d}\omega\\
= &\int_0^\infty\frac{4\Omega^2\Sigma^2(\omega)g_{i\sigma}(\nu)}{|\omega^2 - \Omega^2z(\omega)|^2}\left[\frac{\omega - \Omega_i}{2\Omega_i}\frac{1}{\omega - \tilde{\nu}_\sigma + \mathrm{i}\epsilon} - \frac{\omega + \Omega_i}{2\Omega_i}\frac{1}{\omega + \tilde{\nu}_\sigma}\right]\mathrm{d}\omega + \frac{\tilde{\nu}_\sigma - \Omega_i}{2\Omega_i}\frac{2\Omega g_{i\sigma}(\nu)}{\tilde{\nu}_\sigma^2 - \Omega^2z(\tilde{\nu}_\sigma)}\\
= &\;0
\end{split}
\end{equation}
via the logic of Eq. \eqref{eq:qv-minus-su-integral} (note the added complex conjugate and negative sign). Therefore, we have shown that, with
\begin{equation}
\mathcal{M}_\sigma(\nu) = 1,
\end{equation}
the reservoir operators can be retrieved via
\begin{equation}
\hat{b}_\sigma(\nu,t) = \sum_n\int_0^\infty\pi^2\lambda_n\frac{4\Omega^2\Sigma^2(\omega)}{|\omega^2 - \Omega^2z(\omega)|^2}\left[u^{\prime*}_{n\sigma}(\omega,\nu)\hat{C}'_n(\omega,t) - v_{n\sigma}(\omega,\nu)\hat{C}^{\prime\dagger}_n(\omega,t)\right]\mathrm{d}\omega.
\end{equation}
Note that, aside from the constraint that $\Sigma^2(\omega)$ has units of coupling strength squared and obeys that property $\Sigma^2(-\omega) = -\Sigma^2(\omega)$, the values of $\Omega$ and $\Sigma^2(\omega)$ can be freely chosen. Therefore, the values of $z(\omega)$ and $\hat{b}_\sigma(\nu,t)$ are not unique.















%%%%%%%%%%%%%%%%%%%%%%%%%%%%%%%%%%%%%%%%%%%%%%%%%%%%%%%%%%%%%%%%%%%%%%%%%%%%
% Helmholtz Expansion of the Dyadic Dirac Delta
%%%%%%%%%%%%%%%%%%%%%%%%%%%%%%%%%%%%%%%%%%%%%%%%%%%%%%%%%%%%%%%%%%%%%%%%%%%%
%!TEX root = finiteMQED.tex

\section{Details of the Helmholtz Expansion of the Dyadic Dirac Delta in Spherical Coordinates}

\subsection{Orthogonality of Transverse and Longitudinal Vector Fields}\label{sec:helmholtzOrthogonality}

A vector field $\mathbf{F}(\mathbf{r})$ can be broken into a transverse part $\mathbf{F}_\perp(\mathbf{r})$ and a longitudinal part $\mathbf{F}_\parallel(\mathbf{r})$ through Helmholtz decomposition. The divergence-free transverse part can be written as the curl of another vector field, such that $\mathbf{F}_\perp(\mathbf{r}) = \nabla\times\mathbf{A}(\mathbf{r})$. The curl-free longitudinal part can be written as the gradient of a scalar field, such that $\mathbf{F}_\parallel(\mathbf{r}) = \nabla\Phi(\mathbf{r})$. Therefore, using the identities $\nabla\cdot\{\mathbf{A}(\mathbf{r})\Phi(\mathbf{r})\} = \mathbf{A}(\mathbf{r})\cdot\nabla\Phi(\mathbf{r}) + \Phi(\mathbf{r})\nabla\cdot\mathbf{A}(\mathbf{r})$ and $\nabla\cdot\nabla\times\mathbf{A}(\mathbf{r}) = 0$, one can use integration by parts and Gauss' law to see that
\begin{equation}
\begin{split}
\int\mathbf{F}_\perp(\mathbf{r})\cdot\mathbf{F}_\parallel(\mathbf{r})\;\mathrm{d}^3\mathbf{r} &= \int\nabla\times\mathbf{A}(\mathbf{r})\cdot\nabla\Phi(\mathbf{r})\;\mathrm{d}^3\mathbf{r}\\
&=\int\left(\nabla\cdot\left\{\Phi(\mathbf{r})\nabla\times\mathbf{A}(\mathbf{r})\right\} - \Phi(\mathbf{r})\nabla\cdot\{\nabla\times\mathbf{A}(\mathbf{r})\}\right)\;\mathrm{d}^3\mathbf{r}\\
&= \oint\Phi(\mathbf{r})\nabla\times\mathbf{A}(\mathbf{r})\cdot\mathrm{d}^2\mathbf{s} + 0.
\end{split}
\end{equation}
Here, $\int\mathrm{d}^3\mathbf{r}$ represents integration over all space and $\oint\mathrm{d}^2\mathbf{s}$ represents integration across the surface at infinity wherein the surface elements $\mathrm{d}^2\mathbf{s}$ have outward-facing normal vectors. The surface integral vanishes for all $\mathbf{A}(\mathbf{r})$ and $\Phi(\mathbf{r})$ that go to zero at infinity, such that any physical vector field $\mathbf{F}(\mathbf{r})$ obeys
\begin{equation}\label{eq:helmholtzOrthogonality}
\int\mathbf{F}_\perp(\mathbf{r})\cdot\mathbf{F}_\parallel(\mathbf{r})\;\mathrm{d}^3\mathbf{r} = 0.
\end{equation}







\subsection{Helmholtz Decomposition of $\bm{1}_2\delta(\mathbf{r} - \mathbf{r}')$ into Spherical Mode Functions}\label{app:helmholtzDelta}

Consider the Laplace equation
\begin{equation}
\nabla^2\psi_{p\ell m}(\mathbf{r},k) = -k^2\psi_{p\ell m}(\mathbf{r},k)
\end{equation}
with the corresponding boundary condition $\lim_{r\to\infty}\psi_{p\ell m}(\mathbf{r},k)\to0$. The solutions, which take the form
\begin{equation}
\psi_{p\ell m}(\mathbf{r},k) = \sqrt{K_{\ell m}}j_\ell(kr)P_{\ell m}(\cos\theta)S_p(m\phi),
\end{equation}
are therefore defined here as eigenfunctions of the Laplacian operator in spherical coordinates with eigenvalues $-k^2$. The individual factors of the eigenfunctions are defined in Appendix \ref{app:harmonicAlgebra}. 

These eigenfunctions possess a special completeness property that is crucial to our ability to reconstruct complicated three-dimensional functions as linear combinations of simpler harmonic functions. To see how this completeness property arises, one can expand the Laplacian in spherical coordinates to find
\begin{equation}
\nabla^2\{\cdot\} = \frac{1}{r^2}\frac{\partial}{\partial r}\left\{r^2\frac{\partial}{\partial r}\left\{\cdot\right\}\right\} + \frac{1}{r^2\sin\theta}\frac{\partial}{\partial \theta}\left\{\sin\theta\frac{\partial}{\partial \theta}\{\cdot\}\right\} + \frac{1}{r^2\sin^2\theta}\frac{\partial^2}{\partial\phi^2}\{\cdot\}.
\end{equation}
Therefore, dropping the indices $(p,\ell,m)$ for a moment, the action of $\nabla^2 + k^2$ on any separable eigenfunction $\psi(\mathbf{r}) = R(r)\Theta(\theta)\Phi(\phi)$ can be seen to produce zero such that
\begin{equation}\label{eq:diffEqExpansion}
\begin{split}
&\left(\nabla^2 + k^2\right)\{\psi(\mathbf{r})\} = \frac{r^2}{\psi(\mathbf{r})}\left(\nabla^2 + k^2\right)\{\psi(\mathbf{r})\},\\
&= \frac{r}{R(r)}\left(k^2rR(r) + 2\frac{\partial R(r)}{\partial r} + r\frac{\partial^2 R(r)}{\partial r^2}\right) + \frac{1}{\tan(\theta)\Theta(\theta)}\left(\frac{\partial\Theta(\theta)}{\partial\theta} + \frac{\partial^2\Theta(\theta)}{\partial\theta^2}\right) + \frac{1}{\sin^2(\theta)\Phi(\phi)}\frac{\partial^2\Phi(\phi)}{\partial\phi^2}\\
&= 0.
\end{split}
\end{equation}
For any fixed values of $\theta$ and $\phi$, the above differential equation in $r$ is a particular form of Sturm-Liouville equation known as the spherical Bessel equation. As is discussed for the one-dimensional case in Titchmarsh,\cite{titchmarsh1946eigenfunction} solutions to Sturm-Liouville equations with specified boundary conditions form complete sets of orthogonal eigenfunctions from which any twice-differentiable function can be exactly reconstructed within the boundaries. Therefore, the set of functions $\{R(r)\}$ that are solutions to the above equation can completely describe any one-dimensional function of $r$ on the interval $(-\infty,\infty)$.

The other two terms of Eq. \eqref{eq:diffEqExpansion} also form Sturm-Liouville equations, the associated Legendre and harmonic differential equations, when only $\theta$ or $\phi$ is allowed to vary, respectively. Therefore, $\{\Theta(\theta)\}$ and $\{\Phi(\phi)\}$ are complete sets on $\theta\in[0,\pi)$ and $\phi\in[0,2\pi)$, respectively, and the eigenfunctions $\{\psi(\mathbf{r})\}$ are guaranteed to be able to reproduce any twice-differentiable scalar field in $\mathbb{R}^3$.

To extend our logic to the reconstruction of vector fields, we can note that three sets of vector eigenfunctions $\hat{\mathbf{e}}_i\psi_{p\ell m}(\mathbf{r},k)$ with $i = \{x,y,z\}$ can be defined that clearly are able to reproduce any vector field on $\mathbb{R}^3$. More precisely, these three sets of vector eigenfunctions are non-collinear\cite{stratton1941electromagnetic} and contain independent eigenfunctions of (the Sturm-Liouville operator) $\nabla^2$. The first property can be deduced from the nonzero cross-products between any two eigenfunctions from two different sets, and the second property can be deduced from the fact that $\nabla^2$ and $\hat{\mathbf{e}}_i$ commute. In other words, $\nabla^2\{\hat{\mathbf{e}}_i\psi_{p\ell m}(\mathbf{r},k)\} = \hat{\mathbf{e}}_i\nabla^2\{\psi_{p\ell m}(\mathbf{r},k)\} = -k^2\hat{\mathbf{e}}_i\psi_{p\ell m}(\mathbf{r},k)$.

Because Cartesian vector functions are inconvenient to use for problems with spherical symmetry, Hansen\cite{hansen1935new} invented, as an alternative, the three functions $\mathbf{L}_{p\ell m}(\mathbf{r},k)$, $\mathbf{M}_{p\ell m}(\mathbf{r},k)$, and $\mathbf{N}_{p\ell m}(\mathbf{r},k)$ as an alternate basis. These three functions satisfy the two criteria above, the latter of which can be directly confirmed via $\nabla^2\{\nabla\psi_{p\ell m}(\mathbf{r},k)\} = \nabla\{\nabla^2\psi_{p\ell m}(\mathbf{r},k)\}$, $\nabla^2\{\nabla\times\{\mathbf{r}\psi_{p\ell m}(\mathbf{r},k)\}\} = \nabla\times\{\mathbf{r}\nabla^2\psi_{p\ell m}(\mathbf{r},k)\}$, and $\nabla^2\{\nabla\times\nabla\times\{\mathbf{r}\psi_{p\ell m}(\mathbf{r},k)\}\} = \nabla\times\nabla\times\{\mathbf{r}\nabla^2\psi_{p\ell m}(\mathbf{r},k)\}$. The non-collinearity of the three sets of functions can be seen by taking the cross-product of any two basis functions and noting that it is nonzero. 

We are thus free to expand in terms of $\mathbf{L}_{p\ell m}(\mathbf{r},k)$, $\mathbf{M}_{p\ell m}(\mathbf{r},k)$, and/or $\mathbf{N}_{p\ell m}(\mathbf{r},k)$ any of the vector fields of our system that are smooth throughout the universe. As a matter of clarity, it is useful to show that we can expand the dyadic Dirac delta in terms of these basis functions as well. Assuming the Dirac delta to be the sharp limit of a smooth bump function, we can let 
\begin{equation}
\bm{1}_2\delta(\mathbf{r} - \mathbf{r}') = \sum_{p\ell m}\int_0^\infty\left[\mathbf{A}_{p\ell m}(\mathbf{r}',k)\mathbf{L}_{p\ell m}(\mathbf{r},k) + \mathbf{B}_{p\ell m}(\mathbf{r}',k)\mathbf{M}_{p\ell m}(\mathbf{r},k) + \mathbf{C}_{p\ell m}(\mathbf{r}',k)\mathbf{N}_{p\ell m}(\mathbf{r},k)\right]\mathrm{d}k,
\end{equation}
wherein $\mathbf{A}_{p\ell m}(\mathbf{r}',k)$, $\mathbf{B}_{p\ell m}(\mathbf{r}',k)$, and $\mathbf{C}_{p\ell m}(\mathbf{r}',k)$ are unknown vector fields. Using the orthogonality condition of Eq. \eqref{eq:longitudinalHarmonicOrthogonality}, we can see that
\begin{equation}
\begin{split}
\int\bm{1}_2\delta(\mathbf{r} - \mathbf{r}')\cdot\mathbf{L}_{p'\ell'm'}(\mathbf{r},k')\;\mathrm{d}^3\mathbf{r} &= \mathbf{L}_{p'\ell'm'}(\mathbf{r}',k')\\
&= \sum_{p\ell m}\int_0^\infty\int\mathbf{A}_{p\ell m}(\mathbf{r}',k)\mathbf{L}_{p\ell m}(\mathbf{r},k)\cdot\mathbf{L}_{p'\ell'm'}(\mathbf{r},k')\;\mathrm{d}^3\mathbf{r}\,\mathrm{d}k\\
&= \sum_{p\ell m}\int_0^\infty\mathbf{A}_{p\ell m}(\mathbf{r}',k)\frac{2\pi^2}{k^2}\delta(k - k')(1 - \delta_{p0}\delta_{m0})\delta_{pp'}\delta_{\ell\ell'}\delta_{mm'}\\
&= \frac{2\pi^2}{k'^2}(1 - \delta_{p'1}\delta_{m'0})\mathbf{A}_{p'\ell'm'}(\mathbf{r}',k').
\end{split}
\end{equation}
Further, with $\mathbf{L}_{p'\ell'm'}(\mathbf{r}',k') = \mathbf{L}_{p'\ell'm'}(\mathbf{r}',k')(1 - \delta_{p'1}\delta_{m'0})$, we can see that
\begin{equation}
\mathbf{A}_{p\ell m}(\mathbf{r}',k) = \frac{k^2}{2\pi^2}\mathbf{L}_{p\ell m}(\mathbf{r}',k).
\end{equation}
Repeating this process for the transverse vector spherical harmonics using \eqref{eq:vectorSphericalHarmonicOrthogonality}, we can see that
\begin{equation}
\begin{split}
\mathbf{B}_{p\ell m}(\mathbf{r}',k) &= \frac{k^2}{2\pi^2}\mathbf{M}_{p\ell m}(\mathbf{r}',k),\\
\mathbf{C}_{p\ell m}(\mathbf{r}',k) &= \frac{k^2}{2\pi^2}\mathbf{N}_{p\ell m}(\mathbf{r}',k),
\end{split}
\end{equation}
such that
\begin{equation}\label{eq:dyadicDiracDelta}
\bm{1}_2\delta(\mathbf{r} - \mathbf{r}') = \sum_{p\ell m}\int_0^\infty\frac{k^2}{2\pi^2}\left[\mathbf{L}_{p\ell m}(\mathbf{r}',k)\mathbf{L}_{p\ell m}(\mathbf{r},k) + \mathbf{M}_{p\ell m}(\mathbf{r}',k)\mathbf{M}_{p\ell m}(\mathbf{r},k) + \mathbf{N}_{p\ell m}(\mathbf{r}',k)\mathbf{N}_{p\ell m}(\mathbf{r},k)\right]\mathrm{d}k.
\end{equation}
This in turn can be separated into its transverse and longitudinal parts as
\begin{equation}\label{eq:dyadicDiracDeltaSeparated}
\begin{split}
\left[\bm{1}_2\delta(\mathbf{r} - \mathbf{r}')\right]_\perp &=  \sum_{p\ell m}\int_0^\infty\frac{k^2}{2\pi^2}\left[\mathbf{M}_{p\ell m}(\mathbf{r}',k)\mathbf{M}_{p\ell m}(\mathbf{r},k) + \mathbf{N}_{p\ell m}(\mathbf{r}',k)\mathbf{N}_{p\ell m}(\mathbf{r},k)\right]\mathrm{d}k,\\
\left[\bm{1}_2\delta(\mathbf{r} - \mathbf{r}')\right]_\parallel &= \sum_{p\ell m}\int_0^\infty\frac{k^2}{2\pi^2}\mathbf{L}_{p\ell m}(\mathbf{r}',k)\mathbf{L}_{p\ell m}(\mathbf{r},k)\;\mathrm{d}k.\\
\end{split}
\end{equation}

Finally, we also deal in this manuscript with vector fields that are smooth within a finite radial boundary $a$ but are discontinuous across the boundary. This problem can be attacked in two ways, both of which will produce the same outcome. The more direct but mathematically complicated route is to represent the sharp boundary as the steep limit of a sigmoid function. This route has nice symmetry with our previous discussion as it involves the expansion of the vector field in the full eigenfunction basis of $\mathbb{R}^3$. However, it will require numerical evaluation of complicated overlap integrals involving spherical Bessel functions and will not, in general, produce expressions with advantageous explanatory power.

The second route begins by setting our Sturm-Liouville boundary conditions such that we only use a subset of the total number vector spherical harmonics. More specifically, we will only use vector spherical harmonics in our expansion that have one or more components that go to zero at $r = a$, with the precise boundary requirements set by physical arguments. Because our boundary condition is a function of $r$, it is the radial index $k$ that will be restricted to certain values, and the completeness of the harmonic expansion of the sphere will depend on whether a set of vector harmonics with the ``allowed'' values of $k$ can be generated from a restricted set of scalar eigenfunctions $\psi_{p\ell m}(\mathbf{r},k)$ that is nonetheless complete within $r < a$.

Conveniently, this set of scalar eigenfunctions is simple to define. The physical boundary conditions on the vector harmonics, as described in the main text, require their boundary-parallel components to go to zero at $r = a$ and their radial components to be nonzero. This behavior can be reproduced from scalar eigenfunctions $\psi_{p\ell m}(\mathbf{r},k_{\ell n})$, where $k_{\ell n}a$ is the $n^\mathrm{th}$ nonzero root of the $\ell^\mathrm{th}$ spherical Bessel function. These scalar eigenfunctions therefore all go to zero at the boundary. This is a valid Sturm-Liouville boundary condition, such that they form a complete set within the sphere. Therefore, we can rebuild our Dirac delta from the discrete set of vector harmonics that have the permitted wavenumbers $k_{\ell n}$. Explicitly,
\begin{equation}
\begin{split}
\bm{1}_2\Theta(a - r)\delta(\mathbf{r} - \mathbf{r}') &= \Theta(a - r)\sum_{p\ell m n}\left[\mathbf{A}_{p\ell mn}(\mathbf{r}')\mathbf{L}_{p\ell m}(\mathbf{r},k_{\ell n})\right.\\
&\qquad\left. + \mathbf{B}_{p\ell mn}(\mathbf{r}')\mathbf{M}_{p\ell m}(\mathbf{r},k_{\ell n}) + \mathbf{C}_{p\ell mn}(\mathbf{r}')\mathbf{N}_{p\ell m}(\mathbf{r},k_{\ell n})\right].
\end{split}
\end{equation}
Projecting as before, we can see that, by the orthonormality conditions of Appendix \ref{app:harmonicAlgebra},
\begin{equation}
\begin{split}
\mathbf{A}_{p\ell mn}(\mathbf{r}') &= \frac{1}{2\pi a^3j_{\ell + 1}^2(k_{\ell n}a)}\mathbf{L}_{p\ell m}(\mathbf{r},k_{\ell n}),\\
\mathbf{B}_{p\ell mn}(\mathbf{r}') &= \frac{1}{2\pi a^3j_{\ell + 1}^2(k_{\ell n}a)}\mathbf{M}_{p\ell m}(\mathbf{r},k_{\ell n}),\\
\mathbf{C}_{p\ell mn}(\mathbf{r}') &= \frac{1}{2\pi a^3j_{\ell + 1}^2(k_{\ell n}a)}\mathbf{N}_{p\ell m}(\mathbf{r},k_{\ell n}).
\end{split}
\end{equation}
Therefore,
\begin{equation}
\begin{split}
&\bm{1}_2\Theta(a - r)\delta(\mathbf{r} - \mathbf{r}') = \Theta(a - r)\sum_{p\ell m n}\frac{1}{2\pi a^3j_{\ell + 1}^2(k_{\ell n}a)}\\
&\qquad\times\left[\mathbf{L}_{p\ell m}(\mathbf{r}',k_{\ell n})\mathbf{L}_{p\ell m}(\mathbf{r},k_{\ell n}) + \mathbf{M}_{p\ell m}(\mathbf{r}',k_{\ell n})\mathbf{M}_{p\ell m}(\mathbf{r},k_{\ell n}) + \mathbf{N}_{p\ell m}(\mathbf{r}',k_{\ell n})\mathbf{N}_{p\ell m}(\mathbf{r},k_{\ell n})\right].
\end{split}
\end{equation}




%%%%%%%%%%%%%%%%%%%%%%%%%%%%%%%%%%%%%%%%%%%%%%%%%%%%%%%%%%%%%%%%%%%%%%%%%%%%
% Derivation of the Dielectric Picture from the EM Lagrangian
%%%%%%%%%%%%%%%%%%%%%%%%%%%%%%%%%%%%%%%%%%%%%%%%%%%%%%%%%%%%%%%%%%%%%%%%%%%%
%!TEX root = finiteMQED.tex

\section{Dependent matter: the dielectric picture}\label{app:dependentMatter}

We have touched on this picture already in the preceding sections, but it is useful to reiterate that the Euler-Lorentz equations derivable from the system Lagrangian $L$ state clearly that the degrees of freedom of the matter and electromagnetic field are coupled. More explicitly, there are four Euler-Lorentz equations of the form
\begin{equation}
\frac{\mathrm{d}}{\mathrm{d}t}\left\{\frac{\partial\mathcal{L}'}{\partial\dot{F}(\mathbf{r},t)}\right\} = \frac{\partial\mathcal{L}'}{\partial F(\mathbf{r},t)} - \sum_{i = 1}^3\frac{\partial}{\partial r_i}\frac{\partial\mathcal{L}'}{\partial\left(\frac{\partial F(\mathbf{r},t)}{\partial r_i}\right)},
\end{equation}
where $F(\mathbf{r},t)$ takes the place of the scalar potential or a component of the vector potential, matter displacement field, or reservoir displacement field. The modified Lagrangian density $\mathcal{L}' = \mathcal{L} - (1/4\pi c)\nabla\Phi(\mathbf{r},t)\cdot\dot{\mathbf{A}}(\mathbf{r},t)$ provides the same Lagrangian, i.e. $\int\mathcal{L}\;\mathrm{d}^3\mathbf{r} = \int\mathcal{L}'\;\mathrm{d}^3\mathbf{r} = L$, due to the orthogonality under integration (see Eq. [\ref{eq:helmholtzOrthogonality}]) of the longitudinal gradient of the scalar potential and transverse (Coulomb gauge) vector potential. Here, the use of $\mathcal{L}'$ provides mathematical simplicity.

The equations of motion the four Euler-Lagrange equations lead to are
\begin{equation}\label{eq:fourEOMs}
\begin{split}
\eta_m(\mathbf{r})\ddot{\mathbf{Q}}_m(\mathbf{r},t) + \int_0^\infty\tilde{v}(\nu)\dot{\tilde{\mathbf{Q}}}_\nu(\mathbf{r},t)\;&\mathrm{d}\nu + \eta_m(\mathbf{r})\omega_0^2\mathbf{Q}_m(\mathbf{r},t)\\
+ \eta_m(\mathbf{r})\sigma_0\mathbf{Q}_m(\mathbf{r},t)\cdot\bm{1}_3&\cdot\mathbf{Q}_m(\mathbf{r},t) = \frac{e}{\Delta}\left(-\nabla\Phi(\mathbf{r},t) - \frac{1}{c}\dot{\mathbf{A}}(\mathbf{r})\right),\\[1.0em]
\eta_\nu(\mathbf{r})\ddot{\tilde{\mathbf{Q}}}_\nu(\mathbf{r},t) + \eta_\nu(\mathbf{r})\nu^2\tilde{\mathbf{Q}}_\nu(\mathbf{r},t) &= \tilde{v}(\nu)\dot{\mathbf{Q}}_m(\mathbf{r},t),\\[0.5em]
-\nabla\cdot\nabla\Phi(\mathbf{r},t) &= 4\pi\rho_f(\mathbf{r},t) - \frac{4\pi e}{\Delta}\nabla\cdot\mathbf{Q}_m(\mathbf{r},t),\\
\nabla\times\nabla\times\mathbf{A}(\mathbf{r},t) - \frac{1}{c^2}\ddot{\mathbf{A}}(\mathbf{r},t) &= \frac{4\pi}{c}\left[\frac{e}{\Delta}\dot{\mathbf{Q}}_m(\mathbf{r},t) + \mathbf{J}_f(\mathbf{r},t)\right] - \frac{\nabla\dot{\Phi}(\mathbf{r},t)}{c}\\
&= \frac{4\pi}{c}\left[\frac{e}{\Delta}\dot{\mathbf{Q}}_m^\perp(\mathbf{r},t) + \mathbf{J}_f^\perp(\mathbf{r},t)\right].
\end{split}
\end{equation}
The most pertient equation to analyze here is the second. Using the Green's function (valid for $r < a$)
\begin{equation}
G(t - t') = \frac{\Theta(t - t')}{\eta_\nu(\mathbf{r})\nu}\sin(\nu[t - t'])
\end{equation}
of the oscillator differential equation
\begin{equation}
\eta_\nu(\mathbf{r})\ddot{G}(t - t') + \eta_\nu(\mathbf{r})\nu^2G(t - t') = 
\begin{cases}
\delta(t - t'), & t > t';\\
0, & t < t';
\end{cases}
\end{equation}
we can see that
\begin{equation}
\begin{split}
\dot{\tilde{\mathbf{Q}}}_\nu(\mathbf{r},t) &= \dot{\tilde{\mathbf{Q}}}_\nu^{(0)}(\mathbf{r},t) + \frac{\mathrm{d}}{\mathrm{d}t}\int_{-\infty}^\infty G(t - t')\tilde{v}(\nu)\dot{\mathbf{Q}}_m(\mathbf{r},t')\;\mathrm{d}t'\\
&= \dot{\tilde{\mathbf{Q}}}_\nu^{(0)}(\mathbf{r},t) + \int_{-\infty}^\infty\frac{\Theta(t - t')}{\eta_\nu(\mathbf{r})}\cos(\nu[t - t'])\tilde{v}(\nu)\dot{\mathbf{Q}}_m(\mathbf{r},t')\;\mathrm{d}t'.
\end{split}
\end{equation}
where $\dot{\tilde{\mathbf{Q}}}^{(0)}_\nu(\mathbf{r},t)$ is the $\nu^\mathrm{th}$ velocity field at $t$ generated by stimuli far in the past (formally, at the initial time $-\infty$). The inverse of the reservoir mass-density, $1/\eta_\nu(\mathbf{r}) = \eta_\nu^{-1}(\mathbf{r})$, is taken to imply use of the function $\eta_\nu^{-1}(\mathbf{r}) = \Theta(a - r)\Delta/\sum_{i = 1}^{N_r}\mu_{ri}\Theta[\mathbf{r}\in\mathbb{V}(\mathbf{r}_i)]$, i.e. is assumed to be equal to $\Delta/\mu_{ri}$ within each small region $\mathbb{V}(\mathbf{r}_i)$ inside the sphere and zero for all $r > a$. We can then define a loss function
\begin{equation}
\gamma_0(t - t') = \frac{1}{\eta_m(\mathbf{r})}\int_0^\infty\frac{\tilde{v}^2(\nu)}{\eta_\nu(\mathbf{r})}\cos(\nu[t - t'])\;\mathrm{d}\nu
\end{equation}
and a thermal noise field (dimensions of force per unit volume)
\begin{equation}
\mathbf{N}_\mathrm{therm}(\mathbf{r},t) = -\int_0^\infty\tilde{v}(\nu)\dot{\tilde{\mathbf{Q}}}^{(0)}_\nu(\mathbf{r},t)\;\mathrm{d}\nu
\end{equation}
using an identical assumption for the inverse of the mass-density of the matter field (simply let $\mu_{ri}\to\mu_{mi}$, etc.) such that the first equation of Eq. \eqref{eq:fourEOMs} becomes
\begin{equation}\label{eq:matterEOM}
\begin{split}
\eta_m(\mathbf{r})\ddot{\mathbf{Q}}_m(\mathbf{r},t) &+ \eta_m(\mathbf{r})\int_{-\infty}^t\gamma_0(t - t')\dot{\mathbf{Q}}_m(\mathbf{r},t')\;\mathrm{d}t' + \eta_m(\mathbf{r})\omega_0^2\mathbf{Q}_m(\mathbf{r},t)\\
&+ \eta_m(\mathbf{r})\sigma_0\mathbf{Q}_m(\mathbf{r},t)\cdot\bm{1}_3\cdot\mathbf{Q}_m(\mathbf{r},t) = \frac{e}{\Delta}\left(-\nabla\Phi(\mathbf{r},t) - \frac{1}{c}\dot{\mathbf{A}}(\mathbf{r})\right) + \mathbf{N}_\mathrm{therm}(\mathbf{r},t).
\end{split}
\end{equation}

Solutions to Eq. \eqref{eq:matterEOM} are simplest in the case where $\sigma_0$ is a ``small'' quantity. We will not belabor the exact definition of ``small'' here, other than to say that we will assume $\sigma_0$ to be linearly proportional to a characteristically small number $\lambda$ that can be used to expand the matter displacement field as a perturbation series:
\begin{equation}
\begin{split}
\mathbf{Q}_m(\mathbf{r},t) &= \mathbf{Q}_m^{(1)}(\mathbf{r},t) + \mathbf{Q}_m^{(2)}(\mathbf{r},t) + \ldots\\
&= \sum_{n = 1}^\infty\lambda^{n-1}\bm{\mathcal{Q}}_m^{(n)}(\mathbf{r},t).
\end{split}
\end{equation}
Here, the functions $\bm{\mathcal{Q}}_m^{(n)}(\mathbf{r},t)$ are of roughly the same order of magnitude at each order $n$ such that it is the different powers of $\lambda$ that separate the terms in the series. Using this expansion, we find that
\begin{equation}\label{eq:matterEOM2}
\begin{split}
\eta_m(\mathbf{r})\ddot{\mathbf{Q}}_m^{(1)}(\mathbf{r},t) + \eta_m(\mathbf{r})\int_{-\infty}^t\gamma_0(t - t')\dot{\mathbf{Q}}_m^{(1)}(\mathbf{r},t')\;\mathrm{d}t' &+ \eta_m(\mathbf{r})\omega_0^2\mathbf{Q}_m^{(1)}(\mathbf{r},t) = \frac{e}{\Delta}\mathbf{E}(\mathbf{r},t) + \mathbf{N}_\mathrm{therm}(\mathbf{r},t),\\
\eta_m(\mathbf{r})\ddot{\mathbf{Q}}_m^{(2)}(\mathbf{r},t) + \eta_m(\mathbf{r})\int_{-\infty}^t\gamma_0(t - t')\dot{\mathbf{Q}}_m^{(2)}(\mathbf{r},t')\;\mathrm{d}t'
&+ \eta_m(\mathbf{r})\omega_0^2\mathbf{Q}_m^{(2)}(\mathbf{r},t)\\
&= -\eta_m(\mathbf{r})\sigma_0\mathbf{Q}_m^{(1)}(\mathbf{r},t)\cdot\bm{1}_3\cdot\mathbf{Q}_m^{(1)}(\mathbf{r},t)
\end{split}
\end{equation}
where we have let $\mathbf{E}(\mathbf{r},t) = -\nabla\Phi(\mathbf{r},t) - \dot{\mathbf{A}}(\mathbf{r},t)/c$.

Further, we can simplify our analysis by choosing the simple loss function $\gamma_0(t - t') = \gamma_0\delta(t - t')$. We can see that this is achievable with the choice of coupling function
\begin{equation}
\tilde{v}(\nu) = \sqrt{\frac{\gamma_0}{\pi}\eta_m(\mathbf{r})\eta_\nu(\mathbf{r})},
\end{equation}
as can be seen through the application of the identity $\delta(t - t') = \int_0^\infty 2\cos(\nu[t - t'])\;\mathrm{d}\nu/2\pi$. Using this simplification, we can take the Fourier transform of both lines of Eq. \eqref{eq:matterEOM2} to see
\begin{equation}\label{eq:matterEOM3}
\begin{split}
\mathbf{Q}_m^{(1)}(\mathbf{r},\omega)\eta_m(\mathbf{r})\left[-\omega^2 - \mathrm{i}\omega\gamma_0 + \omega_0^2\right] &= \frac{e}{\Delta}\mathbf{E}(\mathbf{r},\omega) + \mathbf{N}_\mathrm{therm}(\omega),\\
\mathbf{Q}_m^{(2)}(\mathbf{r},\omega)\eta_m(\mathbf{r})\left[-\omega^2 - \mathrm{i}\omega\gamma_0 + \omega_0^2\right] &= -\eta_m(\mathbf{r})\sigma_0\int_{-\infty}^\infty\mathbf{Q}_m^{(1)}(\mathbf{r},\omega')\cdot\bm{1}_3\cdot\mathbf{Q}_m^{(1)}(\mathbf{r},\omega - \omega')\;\mathrm{d}\omega'.
\end{split}
\end{equation}
This form of the equations of motion of the matter displacement fields is simple to connect back to a dielectric picture through the definition of the polarization field $\mathbf{P}(\mathbf{r},\omega) = (e/\Delta)\mathbf{Q}_m(\mathbf{r},\omega)$. Generally, this is done without thermal fluctuations, so we'll let $\mathbf{N}_\mathrm{therm}(\mathbf{r},\omega)\to0$ for now. The polarization field can be expanded as a perturbation series analogous to that of the displacement field, such that $\mathbf{P}(\mathbf{r},\omega) = \sum_n\mathbf{P}^{(n)}(\mathbf{r},\omega)$. This expansion can further be connected to a dielectric model through the definitions $\mathbf{P}^{(1)}(\mathbf{r},\omega) = \chi^{(1)}(\mathbf{r},\omega)\mathbf{E}(\mathbf{r},\omega)$ and $\mathbf{P}^{(2)}(\mathbf{r},\omega) = \int_{-\infty}^\infty\mathbf{E}(\mathbf{r},\omega')\cdot\bm{\chi}^{(2)}(\mathbf{r};\omega',\omega - \omega')\cdot\mathbf{E}(\mathbf{r},\omega - \omega')\;\mathrm{d}\omega'$. Explicitly, we can use these definitions along with a substitution within Eq. \eqref{eq:matterEOM3} to see that
\begin{equation}
\begin{split}
\chi^{(1)}(\mathbf{r},\omega) &= \frac{e^2}{\Delta^2\eta_m(\mathbf{r})}\frac{1}{\omega_0^2 - \omega^2 - \mathrm{i}\omega\gamma_0},\\
\bm{\chi}^{(2)}(\mathbf{r};\omega',\omega - \omega') &= -\frac{3e^3\sigma_0}{\Delta^3\eta_m^2(\mathbf{r})}\frac{\bm{1}_3}{(\omega_0^2 - \omega^2 - \mathrm{i}\omega\gamma_0)(\omega_0^2 - \omega'^2 - \mathrm{i}\omega'\gamma_0)(\omega_0^2 - [\omega - \omega']^2 - \mathrm{i}[\omega - \omega']\gamma_0)}.
\end{split}
\end{equation}

We are now in a position to connect our microscopic model back to a Green's-function solution to the system. Explicitly, all we need to do is redefine the last two lines of Eq. \eqref{eq:fourEOMs} in Fourier space,
\begin{equation}
\begin{split}
-\nabla\cdot\nabla\Phi(\mathbf{r},\omega) &= 4\pi\rho_f(\mathbf{r},\omega) - 4\pi\nabla\cdot\mathbf{P}(\mathbf{r},\omega),\\
\nabla\times\nabla\times\mathbf{A}(\mathbf{r},\omega) - \frac{\omega^2}{c^2}\mathbf{A}(\mathbf{r},\omega) &= -\frac{4\pi\omega^2}{c}\mathbf{P}^\perp(\mathbf{r},t) - \frac{4\pi\mathrm{i}\omega}{c}\mathbf{J}_f^\perp(\mathbf{r},\omega),
\end{split}
\end{equation}
then use our definitions for the susceptibilities to build the canonical wave Helmholtz equations for inhomogeneous dielectrics. 





%%%%%%%%%%%%%%%%%%%%%%%%%%%%%%%%%%%%%%%%%%%%%%%%%%%%%%%%%%%%%%%%%%%%%%%%%%%%
% Derivation of the Free Charge Picture from the EM Lagrangian
%%%%%%%%%%%%%%%%%%%%%%%%%%%%%%%%%%%%%%%%%%%%%%%%%%%%%%%%%%%%%%%%%%%%%%%%%%%%
% %!TEX root = finiteMQED.tex

\section{No matter: the free-charge picture}\label{app:noMatter}

Allowing the charges is a system to all be free charges is the simplest version of charge-inclusive QED that exists. Rather than deal with the various expansions and specifications of the previous section, we can simply start with Eq. \eqref{eq:lagrangianDensity1} and define current and charge densities $\sum_ie_i\dot{\mathbf{x}}_i(t)\delta[\mathbf{r} - \mathbf{x}_i(t)] = \mathbf{J}(\mathbf{r},t)$ and $\sum_ie_i\delta[\mathbf{r} - \mathbf{x}_i(t)] = \rho(\mathbf{r},t)$, respectively. Further, we can assume the vector potential is transverse (the Coulomb gauge) and drop terms that go like $\mathbf{A}(\mathbf{r},t)\cdot\nabla\Phi(\mathbf{r},t)$ or $\mathbf{J}^\parallel(\mathbf{r},t)\cdot\mathbf{A}(\mathbf{r},t) = [\mathbf{J}(\mathbf{r},t) - \mathbf{J}^\perp(\mathbf{r},t)]\cdot\mathbf{A}(\mathbf{r},t)$ such that
\begin{equation}
\begin{split}
\mathcal{L}' &= \sum_{i = 1}^N\frac{1}{2}\mu_i\dot{\mathbf{x}}_i^2(t)\delta[\mathbf{r} - \mathbf{x}_i(t)] + \frac{1}{8\pi}\left(\left[-\nabla\Phi(\mathbf{r},t)\right]^2 + \frac{1}{c}\dot{\mathbf{A}}^2(\mathbf{r},t) - \left[\nabla\times\mathbf{A}(\mathbf{r},t)\right]^2\right)\\
&+ \mathbf{J}^\perp(\mathbf{r},t)\cdot\mathbf{A}(\mathbf{r},t) - \rho(\mathbf{r},t)\Phi(\mathbf{r},t).
\end{split}
\end{equation}
The Euler-Lagrange equations associated with this Lagrangian density (see Section \ref{app:dependentMatter}) give us
\begin{equation}
\begin{split}
&\nabla\times\nabla\times\mathbf{A}(\mathbf{r},t) + \frac{1}{c^2}\ddot{\mathbf{A}}(\mathbf{r},t) = \frac{4\pi}{c}\mathbf{J}^\perp(\mathbf{r},t),\\
\implies &\mathbf{A}(\mathbf{r},t) = 4\pi\int_{-\infty}^{\infty}\int\mathbf{G}_0(\mathbf{r},\mathbf{r}';\omega)\cdot\frac{\mathbf{J}^\perp(\mathbf{r}',\omega)}{c}\mathrm{e}^{-\mathrm{i}\omega t}\;\mathrm{d}^3\mathbf{r}'\,\frac{\mathrm{d}\omega}{2\pi}
\end{split}
\end{equation}
where the dyadic Green's function $\mathbf{G}_0(\mathbf{r},\mathbf{r}';\omega)$ is given in Eq. \eqref{eq:G0} as
\begin{equation}
\mathbf{G}_0(\mathbf{r},\mathbf{r}';\omega) = \frac{\mathrm{i}k}{4\pi}\sum_{\bm{\alpha}}\left[\bm{\mathcal{X}}_{\bm{\alpha}}(\mathbf{r},k)\mathbf{X}_{\bm{\alpha}}(\mathbf{r}',k)\Theta(r - r') + \mathbf{X}_{\bm{\alpha}}(\mathbf{r},k)\bm{\mathcal{X}}_{\bm{\alpha}}(\mathbf{r}',k)\Theta(r' - r)\right] - \frac{\hat{\mathbf{r}}\hat{\mathbf{r}}}{k^2}\delta(\mathbf{r} - \mathbf{r}').
\end{equation}

The Green's function definition of $\mathbf{A}(\mathbf{r},t)$ tells us that the transverse vector potential is generated by the projection integral of the transverse current $\mathbf{J}^\perp(\mathbf{r},t)$ with the dyadic Green's function. In its usual form, however, the Green's function is not obviously transverse or longitudinal in $\mathbf{r}$, due to the presence of the functions $\Theta(\pm r \mp r')$ and $\hat{\mathbf{r}}\hat{\mathbf{r}}\delta(\mathbf{r} - \mathbf{r}')$ that are not separated into products of functions of $\mathbf{r}$ and $\mathbf{r}'$. We can produce this separation and clarify the role of $\mathbf{G}_0(\mathbf{r},\mathbf{r}';\omega)$, however, through use the analyses of Appendices \ref{sec:truncatedHarmonicExpansion} and \ref{sec:diracDyadExpansion}. First focusing on the terms of $\mathbf{G}_0(\mathbf{r},\mathbf{r}';\omega)$ that appear under the sum in $\bm{\alpha}$, we can let
\begin{equation}
\begin{split}
\bm{\mathcal{X}}_{\bm{\alpha}}(\mathbf{r},k)\Theta(r - r') &= \frac{2}{\pi}(1 - \delta_{p1}\delta_{m0})\int_0^\infty\kappa^2R_{T\ell}^{><}(k,\kappa;r',\infty)\mathbf{X}_{\bm{\alpha}}(\mathbf{r},\kappa)\;\mathrm{d}\kappa\\
&\qquad- (1 - \delta_{p1}\delta_{m0})\delta_{TE}\sqrt{\frac{\ell(\ell + 1)}{2\ell + 1}}\frac{h_\ell(kr')}{kr'}\mathbf{Z}_{\bm{\alpha}}(\mathbf{r};r'),\\[1.0em]
\mathbf{X}_{\bm{\alpha}}(\mathbf{r},k)\Theta(r' - r) &= \frac{2}{\pi}(1 - \delta_{p1}\delta_{m0})\int_0^\infty\kappa^2R_{T\ell}^\ll(k,\kappa;0,r')\mathbf{X}_{\bm{\alpha}}(\mathbf{r},\kappa)\;\mathrm{d}\kappa\\
&\qquad + (1 - \delta_{p1}\delta_{m0})\delta_{TE}\sqrt{\frac{\ell(\ell + 1)}{2\ell + 1}}\frac{j_\ell(kr')}{kr'}\mathbf{Z}_{\bm{\alpha}}(\mathbf{r};r'),
\end{split}
\end{equation}
where in both equations the first term on the right-hand side is transverse in $\mathbf{r}$ and the second is longitudinal. Although the longitudinal terms involve the non-separable longitudinal harmonics $\mathbf{Z}_{\bm{\alpha}}(\mathbf{r};r') = \nabla\left\{r'^{-\ell + 1} f_{\bm{\alpha}}^<(\mathbf{r})\Theta(r' - r) + r'^{\ell + 2} f_{\bm{\alpha}}^>(\mathbf{r})\Theta(r - r')\right\}$ and are more unwieldy to work with, one can see that any integration in $\mathbf{r}'$ involving these harmonics can be carried out under the gradient in $\mathbf{r}$, thus guaranteeing that the resulting contributions to the vector potential will be curl-free.

The ``self-field'' term of $\mathbf{G}_0(\mathbf{r},\mathbf{r}';\omega)$, i.e. the Driac-delta dyad $\hat{\mathbf{r}}\hat{\mathbf{r}}\delta(\mathbf{r} - \mathbf{r}')/k^2$, can be expanded similarly, providing
\begin{equation}
\begin{split}
-\frac{\hat{\mathbf{r}}\hat{\mathbf{r}}}{k^2}\delta(\mathbf{r} - \mathbf{r}') &= -\frac{1}{2\pi^2k^2}\sum_{\bm{\alpha}}\int_0^\infty\mathbf{X}_{\bm{\alpha}}(\mathbf{r},\kappa)\left[\hat{\mathbf{r}}'\cdot\mathbf{X}_{\bm{\alpha}}(\mathbf{r}',\kappa)\right]\hat{\mathbf{r}}'\kappa^2\;\mathrm{d}\kappa\\
&\qquad-\frac{1}{4\pi k^2}\sum_{\bm{\alpha}}\nabla\left\{\frac{\partial f_{\bm{\alpha}}^>(\mathbf{r}')}{\partial r'}f_{\bm{\alpha}}^<(\mathbf{r})\Theta(r' - r) + \frac{\partial f_{\bm{\alpha}}^<(\mathbf{r}')}{\partial r'}f_{\bm{\alpha}}^>(\mathbf{r})\Theta(r - r')\right\}\hat{\mathbf{r}}'.
\end{split}
\end{equation}
As expected, only the radially-oriented components of the current contribute to the vector potential through the self-field term, and both longitudinal and transverse self-fields exist, in general. We can see now that the Green's function is decomposable as $\mathbf{G}_0(\mathbf{r},\mathbf{r}';\omega) = \mathbf{G}_0^\perp(\mathbf{r},\mathbf{r}';\omega) + \mathbf{G}_0^\parallel(\mathbf{r},\mathbf{r}';\omega)$, wherein the transverse ($\perp$) and longitudinal ($\parallel$) components of the Green's function are
\begin{equation}\label{eq:helmholtzDecompositionFreeSpaceGreenFunction}
\begin{split}
\mathbf{G}_0^\perp(\mathbf{r},\mathbf{r}';\omega) &= \frac{\mathrm{i}k}{2\pi^2}\sum_{\bm{\alpha}}\int_0^\infty\left( \vphantom{\frac{1}{4\pi k^2}} \mathbf{X}_{\bm{\alpha}}(\mathbf{r},\kappa)\mathbf{X}_{\bm{\alpha}}(\mathbf{r}',k)R_{T\ell}^{><}(k,\kappa;r',\infty)\right.\\
&\left.+ \mathbf{X}_{\bm{\alpha}}(\mathbf{r},\kappa)\bm{\mathcal{X}}_{\bm{\alpha}}(\mathbf{r}',k)R_{T\ell}^\ll(k,\kappa;0,r') - \frac{1}{\mathrm{i} k^3}\mathbf{X}_{\bm{\alpha}}(\mathbf{r},\kappa)\left[\hat{\mathbf{r}}'\cdot\mathbf{X}_{\bm{\alpha}}(\mathbf{r}',\kappa)\right]\hat{\mathbf{r}}'\right)\kappa^2\mathrm{d}\kappa,\\[1.0em]
\mathbf{G}_0^\parallel(\mathbf{r},\mathbf{r}';\omega) &= \frac{\mathrm{i}k}{4\pi}\sum_{\bm{\alpha}}\left(\delta_{TE}\sqrt{\frac{\ell(\ell + 1)}{2\ell + 1}}\mathbf{Z}_{\bm{\alpha}}(\mathbf{r};r')\left[-\frac{h_\ell(kr')}{kr'}\mathbf{N}_{\bm{\alpha}}(\mathbf{r}',k) + \frac{j_\ell(kr')}{kr'}\bm{\mathcal{N}}_{\bm{\alpha}}(\mathbf{r}',k)\right]\right.\\
&\left.\qquad-\frac{1}{\mathrm{i} k^3}\nabla\left\{\frac{\partial f_{\bm{\alpha}}^>(\mathbf{r}')}{\partial r'}f_{\bm{\alpha}}^<(\mathbf{r})\Theta(r' - r) + \frac{\partial f_{\bm{\alpha}}^<(\mathbf{r}')}{\partial r'}f_{\bm{\alpha}}^>(\mathbf{r})\Theta(r - r')\right\}\hat{\mathbf{r}}' \vphantom{\sqrt{\frac{\ell(\ell + 1)}{2\ell + 1}}} \right).
\end{split}
\end{equation}
We have dropped factors of $1 - \delta_{p1}\delta_{m0}$ as they are redundant when multiplied by vector harmonics that are already zero at $(p,m) = (1,0)$. The $\mathbf{r}$-transverse and -longitudinal nature of these functions is explicit from the properties of $\mathbf{X}_{\bm{\alpha}}(\mathbf{r},\kappa)$, $\mathbf{Z}_{\bm{\alpha}}(\mathbf{r};r')$, and $\nabla f_{\bm{\alpha}}^{>,<}(\mathbf{r})$. It is less obvious but, nonetheless, true that these components are also transverse and longitudinal, respectively, in the source coordinate $\mathbf{r}'$ as long as $\mathbf{r}'\neq\mathbf{r}$ and $r' > 0$. The proofs of these latter conditions are given in Appendix \ref{sec:helmholtzDecompositionFreeSpaceGreenFunction}.

It is now clear that we can rewrite the integral definition of the vector potential as
\begin{equation}
\begin{split}
\mathbf{A}(\mathbf{r},t) &= 4\pi\int_{-\infty}^{\infty}\int\mathbf{G}_0^\perp(\mathbf{r},\mathbf{r}';\omega)\cdot\frac{\mathbf{J}^\perp(\mathbf{r}',\omega)}{c}\mathrm{e}^{-\mathrm{i}\omega t}\;\mathrm{d}^3\mathbf{r}'\,\frac{\mathrm{d}\omega}{2\pi}\\
&= 4\pi\int_{-\infty}^\infty\int\left[\frac{\mathrm{i}k}{2\pi^2}\sum_{\bm{\alpha}}\int_0^\infty\mathbf{X}_{\bm{\alpha}}(\mathbf{r},\kappa)\left( \vphantom{\frac{1}{4\pi k^2}} \mathbf{X}_{\bm{\alpha}}(\mathbf{r}',k)R_{T\ell}^{><}(k,\kappa;r',\infty) + \bm{\mathcal{X}}_{\bm{\alpha}}(\mathbf{r}',k)R_{T\ell}^\ll(k,\kappa;0,r')\right.\right.\\
&\quad\left.\left.- \frac{1}{\mathrm{i} k^3}\left[\hat{\mathbf{r}}'\cdot\mathbf{X}_{\bm{\alpha}}(\mathbf{r}',\kappa)\right]\hat{\mathbf{r}}'\right)\kappa^2\mathrm{d}\kappa \vphantom{\int_0^\infty} \right]\cdot\frac{\mathbf{J}^\perp(\mathbf{r}',\omega)}{c}\mathrm{e}^{-\mathrm{i}\omega t}\;\mathrm{d}^3\mathbf{r}'\,\frac{\mathrm{d}\omega}{2\pi}\\
&= \sum_{\bm{\alpha}}\int_{-\infty}^\infty\int_0^\infty\mathcal{A}_{\bm{\alpha}}(\kappa,k)\mathbf{X}_{\bm{\alpha}}(\mathbf{r},\kappa)\mathrm{e}^{-\mathrm{i}\omega t}\;\mathrm{d}\kappa\,\frac{\mathrm{d}\omega}{2\pi},
\end{split}
\end{equation}
where
\begin{equation}\label{eq:AconnectionFunction}
\begin{split}
\mathcal{A}_{\bm{\alpha}}(\kappa,k) &= \frac{2\mathrm{i}k\kappa^2}{\pi c}\int\left( \vphantom{\frac{1}{k^3}} \mathbf{X}_{\bm{\alpha}}(\mathbf{r}',k)R_{T\ell}^{><}(k,\kappa;r',\infty) + \bm{\mathcal{X}}_{\bm{\alpha}}(\mathbf{r}',k)R^\ll_{T\ell}(k,\kappa;0,r')\right.\\
&\qquad\left.- \frac{1}{\mathrm{i}k^3}\left[\hat{\mathbf{r}}'\cdot\mathbf{X}_{\bm{\alpha}}(\mathbf{r}',\kappa)\right]\hat{\mathbf{r}}'\right)\cdot\mathbf{J}^\perp(\mathbf{r}',\omega)\;\mathrm{d}^3\mathbf{r}'
\end{split}
\end{equation}
are the amplitudes of the modes of the field that have angular indices $\bm{\alpha}$, a radial index $\kappa$, and oscillate at $\omega$. In contrast to waves moving in free space, the excited modes here do not display a fixed dispersion relation between $\kappa$ and $k = \omega/c$: modes of radial index (spatial wavenumber?) $\kappa$ can oscillate in time at any frequency $\omega$ (temporal wavenumber $k$?). However, both $R_{T\ell}^><(k,\kappa;r',\infty)$ and $R_{T\ell}^\ll(k,\kappa;0,r')$ vary as $1/(k^2 - \kappa^2)$ (see Appendix \ref{app:harmonicAlgebra}) such that modes in which $|k|\approx\kappa$ are more readily excited than those in which $|k|$ and $\kappa$ are dissimilar. Finally, these amplitudes have dimensions of $[\mathrm{charge}][\mathrm{length}][\mathrm{time}]/[\mathrm{length}] = [\mathrm{charge}][\mathrm{time}]$.

The transverse and longitudinal electric fields are also straightforward to calculate from the Helmholtz components of $\mathbf{G}_0(\mathbf{r},\mathbf{r}';\omega)$. Specifically, 
\begin{equation}
\mathbf{E}^{\perp,\parallel}(\mathbf{r},t) = \int_{-\infty}^\infty\int\mathbf{G}_0^{\perp,\parallel}(\mathbf{r},\mathbf{r}';\omega)\cdot\frac{4\pi\mathrm{i}\omega}{c^2}\mathbf{J}^{\perp,\parallel}(\mathbf{r}',\omega)\mathrm{e}^{-\mathrm{i}\omega t}\;\mathrm{d}^3\mathbf{r}'\,\frac{\mathrm{d}\omega}{2\pi},
\end{equation}
such that
\begin{equation}
\mathbf{E}^\perp(\mathbf{r},t) = \sum_{\bm{\alpha}}\int_{-\infty}^\infty\int_0^\infty\mathrm{i}k\mathcal{A}(\kappa,k)\mathbf{X}_{\bm{\alpha}}(\mathbf{r},\kappa)\mathrm{e}^{-\mathrm{i}\omega t}\;\mathrm{d}\kappa\,\frac{\mathrm{d}\omega}{2\pi}
\end{equation}
in agreement with the condition $\mathbf{E}^\perp(\mathbf{r},t) = -\dot{\mathbf{A}}(\mathbf{r},t)/c$ and
\begin{equation}\label{eq:Elongitudinal1}
\begin{split}
\mathbf{E}&^\parallel(\mathbf{r},t) = \int_{-\infty}^\infty\int\frac{\mathrm{i}k}{4\pi}\sum_{\bm{\alpha}}\left(\delta_{TE}\sqrt{\frac{\ell(\ell + 1)}{2\ell + 1}}\mathbf{Z}_{\bm{\alpha}}(\mathbf{r};r')\left[-\frac{h_\ell(kr')}{kr'}\mathbf{N}_{\bm{\alpha}}(\mathbf{r}',k) + \frac{j_\ell(kr')}{kr'}\bm{\mathcal{N}}_{\bm{\alpha}}(\mathbf{r}',k)\right]\right.\\
&\left.-\frac{1}{\mathrm{i} k^3}\nabla\left\{\frac{\partial f_{\bm{\alpha}}^>(\mathbf{r}')}{\partial r'}f_{\bm{\alpha}}^<(\mathbf{r})\Theta(r' - r) + \frac{\partial f_{\bm{\alpha}}^<(\mathbf{r}')}{\partial r'}f_{\bm{\alpha}}^>(\mathbf{r})\Theta(r - r')\right\}\hat{\mathbf{r}}' \vphantom{\sqrt{\frac{\ell(\ell + 1)}{2\ell + 1}}} \right) \cdot\frac{4\pi\mathrm{i}\omega}{c^2}\mathbf{J}^\parallel(\mathbf{r}',\omega)\mathrm{e}^{-\mathrm{i}\omega t}\;\mathrm{d}^3\mathbf{r}'\,\frac{\mathrm{d}\omega}{2\pi}.
\end{split}
\end{equation}
The longitudinal electric field can be simplified through the expansion of the harmonics $\mathbf{Z}_{\bm{\alpha}}(\mathbf{r};r')$ and the application of the gradient in the second line above. In more detail, use of the closure relation
\begin{equation}
\sum_{p\ell m}(2 - \delta_{m0})(2\ell + 1)\frac{(\ell - m)!}{(\ell + m)!}P_{\ell m}(\cos\theta)P_{\ell m}(\cos\theta')S_p(m\phi)S_p(m\phi') = \frac{4\pi}{\sin\theta}\delta(\theta - \theta')\delta(\phi - \phi')
\end{equation}
in addition to the product rule and the Dirac delta definitions $\nabla\Theta(\pm r \mp a) = \pm\hat{\mathbf{r}}\delta(r - a)$ and $\delta(\mathbf{r} - \mathbf{r}') = \delta(r - r')\delta(\theta - \theta')\delta(\phi - \phi')/r^2\sin\theta$ provides
\begin{equation}
\begin{split}
&\sum_{\bm{\alpha}}\nabla\left\{\frac{\partial f_{\bm{\alpha}}^>(\mathbf{r}')}{\partial r'}f_{\bm{\alpha}}^<(\mathbf{r})\Theta(r' - r) + \frac{\partial f_{\bm{\alpha}}^<(\mathbf{r}')}{\partial r'}f_{\bm{\alpha}}^>(\mathbf{r})\Theta(r - r')\right\}\hat{\mathbf{r}}'\\
=&\sum_{\bm{\alpha}}\left[\frac{\partial f_{\bm{\alpha}}^>(\mathbf{r}')}{\partial r'}\nabla f_{\bm{\alpha}}^<(\mathbf{r})\Theta(r' - r) + \frac{\partial f_{\bm{\alpha}}^<(\mathbf{r}')}{\partial r'} \nabla f_{\bm{\alpha}}^>(\mathbf{r})\Theta(r - r')\right]\hat{\mathbf{r}}' + 4\pi\delta(\mathbf{r} - \mathbf{r}')\hat{\mathbf{r}}\hat{\mathbf{r}}'.
\end{split}
\end{equation}
Therefore,
\begin{equation}\label{eq:Elongitudinal2}
\mathbf{E}^\parallel(\mathbf{r},t) = \sum_{\bm{\alpha}}\int_{-\infty}^\infty\left[\mathcal{E}_{\bm{\alpha}}^<(k;r)\nabla f_{\bm{\alpha}}^<(\mathbf{r}) + \mathcal{E}_{\bm{\alpha}}^>(k;r)\nabla f_{\bm{\alpha}}^>(\mathbf{r})\right]\mathrm{e}^{-\mathrm{i}\omega t}\;\frac{\mathrm{d}\omega}{2\pi} - \int_{-\infty}^\infty\frac{4\pi\mathrm{i}}{\omega}\left[\mathbf{J}^\parallel(\mathbf{r},\omega)\cdot\hat{\mathbf{r}}\right]\hat{\mathbf{r}}\mathrm{e}^{-\mathrm{i}\omega t}\;\frac{\mathrm{d}\omega}{2\pi},
\end{equation}
wherein
\begin{equation}
\begin{split}
\mathcal{E}_{\bm{\alpha}}^<(k;r) &= -\frac{k^2}{c}\int\left[\delta_{TE}\sqrt{\frac{\ell(\ell + 1)}{2\ell + 1}}r'^{-\ell + 1}\left(\frac{j_\ell(kr')}{kr'}\bm{\mathcal{N}}_{p\ell m}(\mathbf{r}',k) - \frac{h_\ell(kr')}{kr'}\mathbf{N}_{p\ell m}(\mathbf{r}',k)\right)\right.\\
&\qquad+\left.\frac{\mathrm{i}}{k^3}\frac{\partial f_{\bm{\alpha}}^>(\mathbf{r}')}{\partial r'}\hat{\mathbf{r}}' 
\vphantom{\sqrt{\frac{\ell}{\ell}}} \right]\cdot\mathbf{J}^\parallel(\mathbf{r}',\omega)\Theta(r' - r)\;\mathrm{d}^3\mathbf{r}',\\
\mathcal{E}_{\bm{\alpha}}^>(k;r) &= -\frac{k^2}{c}\int\left[\delta_{TE}\sqrt{\frac{\ell(\ell + 1)}{2\ell + 1}}r'^{\ell + 2}\left(\frac{j_\ell(kr')}{kr'}\bm{\mathcal{N}}_{p\ell m}(\mathbf{r}',k) - \frac{h_\ell(kr')}{kr'}\mathbf{N}_{p\ell m}(\mathbf{r}',k)\right)\right.\\
&\qquad+\left.\frac{\mathrm{i}}{k^3}\frac{\partial f_{\bm{\alpha}}^<(\mathbf{r}')}{\partial r'}\hat{\mathbf{r}}' 
\vphantom{\sqrt{\frac{\ell}{\ell}}} \right]\cdot\mathbf{J}^\parallel(\mathbf{r}',\omega)\Theta(r - r')\;\mathrm{d}^3\mathbf{r}'.
\end{split}
\end{equation}
The scalar potential defined by $-\nabla \Phi(\mathbf{r},t) = \mathbf{E}^\parallel(\mathbf{r},t)$ can be found straightforwardly from Eq. \eqref{eq:Elongitudinal1} by removing, rather than applying, the gradient operators in $\mathbf{r}$. Explicitly,
\begin{equation}
\begin{split}
\Phi(\mathbf{r},t) &= -\int_{-\infty}^\infty\int\frac{\mathrm{i}k}{4\pi}\sum_{\bm{\alpha}}\left(\delta_{TE}\sqrt{\frac{\ell(\ell + 1)}{2\ell + 1}}\left[r'^{-\ell + 1}f_{\bm{\alpha}}^<(\mathbf{r})\Theta(r' - r) + r'^{\ell + 2}f_{\bm{\alpha}}^>(\mathbf{r})\Theta(r - r')\right]\right.\\
&\qquad\times\left[-\frac{h_\ell(kr')}{kr'}\mathbf{N}_{\bm{\alpha}}(\mathbf{r}',k) + \frac{j_\ell(kr')}{kr'}\bm{\mathcal{N}}_{\bm{\alpha}}(\mathbf{r}',k)\right] - \frac{1}{\mathrm{i} k^3}\left[\frac{\partial f_{\bm{\alpha}}^>(\mathbf{r}')}{\partial r'}f_{\bm{\alpha}}^<(\mathbf{r})\Theta(r' - r)\right.\\
&\qquad\left.\left. + \frac{\partial f_{\bm{\alpha}}^<(\mathbf{r}')}{\partial r'}f_{\bm{\alpha}}^>(\mathbf{r})\Theta(r - r')\right]\hat{\mathbf{r}}' \vphantom{\sqrt{\frac{\ell(\ell + 1)}{2\ell + 1}}} \right)\cdot\frac{4\pi\mathrm{i}\omega}{c^2}\mathbf{J}^\parallel(\mathbf{r}',\omega)\mathrm{e}^{-\mathrm{i}\omega t}\;\mathrm{d}^3\mathbf{r'}\,\frac{\mathrm{d}\omega}{2\pi}.
\end{split}
\end{equation}
Finally, the magnetic fields can be calculated from $\mathbf{B}(\mathbf{r},t) = \nabla\times\mathbf{A}(\mathbf{r},t)$ and give
\begin{equation}
\mathbf{B}(\mathbf{r},t) = \sum_{\bm{\alpha}}\int_{-\infty}^\infty\int_0^\infty\kappa\mathcal{A}_{\bm{\alpha}}(\kappa,k)\mathbf{Y}_{\bm{\alpha}}(\mathbf{r},\kappa)\mathrm{e}^{-\mathrm{i}\omega t}\;\mathrm{d}\kappa\,\frac{\mathrm{d}\omega}{2\pi}.
\end{equation}





%%%%%%%%%%%%%%%%%%%%%%%%%%%%%%%%%%%%%%%%%%%%%%%%%%%%%%%%%%%%%%%%%%%%%%%%%%%%
% PZW Hamiltonian Derivation
%%%%%%%%%%%%%%%%%%%%%%%%%%%%%%%%%%%%%%%%%%%%%%%%%%%%%%%%%%%%%%%%%%%%%%%%%%%%
% %!TEX root = finiteMQED.tex

\section{The Power-Zienau-Woolley Lagrangian and Hamiltonian}\label{sec:PZWderivation}

We begin with a set of $N$ classical particles with positions $\mathbf{x}_i(t)$, velocities $\dot{\mathbf{x}}_i(t)$, and charges $q_i$ that are well-separated from a second set of $N_0$ particles with positions $\mathbf{x}_{0j}(t)$, velocities $\dot{\mathbf{x}}_{0j}(t)$, and charges $q_{0j}$. The first set of particles will be assumed to be sensitive to the influence of the system's electromagnetic fields, while the second set will be assumed to have their trajectories fixed, to good approximation, by forces outside of the system. The simplest Lagrangian density that defines such a system is
\begin{equation}
\begin{split}
\mathcal{L}&\left[\mathbf{x}_1,\ldots,\mathbf{x}_N,\dot{\mathbf{x}}_1,\ldots,\dot{\mathbf{x}}_N;\mathbf{A},\dot{\mathbf{A}},\Phi,\dot{\Phi};\mathbf{A}_0,\dot{\mathbf{A}}_0,\Phi_0,\dot{\Phi}_0;\mathbf{r},t\right] = \sum_{i = 1}^N\frac{1}{2}m_i\dot{\mathbf{x}}_i^2(t)\delta[\mathbf{r} - \mathbf{x}_i(t)]\\
& + \frac{1}{8\pi}\left[\mathbf{E}^2(\mathbf{r},t) - \mathbf{B}^2(\mathbf{r},t)\right] + \frac{1}{8\pi}\left[\mathbf{E}_0^2(\mathbf{r},t) - \mathbf{B}_0^2(\mathbf{r},t)\right]\\
&+ \sum_{i = 1}^N\frac{q_i}{c}\dot{\mathbf{x}}_i(t)\delta[\mathbf{r} - \mathbf{x}_i(t)]\cdot\left[\mathbf{A}(\mathbf{r},t) + \mathbf{A}_0(\mathbf{r},t)\right] - \sum_{i = 1}^Nq_i\delta[\mathbf{r} - \mathbf{x}_i(t)]\left[\Phi(\mathbf{r},t) + \Phi_0(\mathbf{r},t)\right]\\
&+\sum_{j = 1}^{N_0}\frac{q_{0j}}{c}\dot{\mathbf{x}}_{0j}(t)\delta[\mathbf{r} - \mathbf{x}_{0j}(t)]\cdot\mathbf{A}_0(\mathbf{r},t) - \sum_{j = 1}^{N_0}q_{0j}\Phi_0(\mathbf{r},t).
%+ \sum_{j = 1}^{N_0}\frac{1}{2}m_{0j}\dot{\mathbf{x}}_{0j}^2(t)\delta[\mathbf{r} - \mathbf{x}_{0j}(t)]
\end{split}
\end{equation}
Here, $\mathbf{A}(\mathbf{r},t)$ and $\Phi(\mathbf{r},t)$ are the vector and scalar potentials of the dynamical set of particles ($\{i\}$), respectively, with associated electric and magnetic fields $\mathbf{E}(\mathbf{r},t) = -\nabla\Phi(\mathbf{r},t) - (1/c)\dot{\mathbf{A}}(\mathbf{r},t)$ and $\mathbf{B}(\mathbf{r},t) = \nabla\times\mathbf{A}(\mathbf{r},t)$. The analogous fields of the nondynamical particles ($\{j\}$) are $\mathbf{A}_0(\mathbf{r},t)$, $\Phi_0(\mathbf{r},t)$, $\mathbf{E}_0(\mathbf{r},t) = -\nabla\Phi_0(\mathbf{r},t) - (1/c)\dot{\mathbf{A}}_0(\mathbf{r},t)$, and $\mathbf{B}_0(\mathbf{r},t) = \nabla\times\mathbf{A}_0(\mathbf{r},t)$. Importantly, however, in contrast to the fields of the nondynamical particles, the particles' positions and momenta $\mathbf{x}_{0j}(t)$ and $\dot{\mathbf{x}}_{0j}(t)$ are not included in the Lagrangian density's list of independent variables. Therefore, $\partial\mathcal{L}/\partial\mathbf{x}_{0j}(t) = 0$ and $\partial\mathcal{L}/\partial\dot{\mathbf{x}}_{0j}(t) = 0$.

These last two properties are important for understanding how the equations of motion for the various quantities of the system are constructed. In particular, the Maxwell-Lorentz equations for the dynamical particles can be found via the Lagrangian $L[\ldots;t] = \int\mathcal{L}[\ldots;\mathbf{r},t]\;\mathrm{d}^3\mathbf{r}$ via the Euler-Lagrange equations
\begin{equation}
\frac{\mathrm{d}}{\mathrm{d}t}\left\{\frac{\partial L}{\partial \dot{\mathbf{x}}_i(t)}\right\} = \frac{\partial L}{\partial \mathbf{x}_i(t)}.
\end{equation}
Explicitly, these imply
\begin{equation}\label{eq:ELcoord1}
\begin{split}
\frac{\mathrm{d}}{\mathrm{d}t}\left\{m_i\dot{\mathbf{x}}_i(t) + \frac{q_i}{c}\left(\mathbf{A}[\mathbf{x}_i(t),t] + \mathbf{A}_0[\mathbf{x}_i(t),t]\right)\right\} &= \frac{q_i}{c}\frac{\partial}{\partial\mathbf{x}_i(t)}\left\{\dot{\mathbf{x}}_i(t)\cdot\left(\mathbf{A}[\mathbf{x}_i(t),t] + \mathbf{A}_0[\mathbf{x}_i(t),t]\right)\right\}\\
&- q_i\frac{\partial}{\partial\mathbf{x}_i(t)}\left\{\Phi[\mathbf{x}_i(t),t] + \Phi_0[\mathbf{x}_i(t),t]\right\},
\end{split}
\end{equation}
where $\partial/\partial\mathbf{z}$ is simply another way to write $\nabla_{\mathbf{z}} = \sum_{k = 1}^3\hat{\mathbf{e}}_k\partial/\partial z_k$. Further, we have used the property $(\partial/\partial\mathbf{x})\{\mathbf{a}\cdot\mathbf{x}\} = \mathbf{a}$. Using the identities
\begin{equation}
\begin{split}
\nabla\left\{\mathbf{a}\cdot\mathbf{F}(\mathbf{r})\right\} &= \mathbf{a}\times\left\{\nabla\times\mathbf{F}(\mathbf{r})\right\} + (\mathbf{a}\cdot\nabla)\left\{\mathbf{F}(\mathbf{r})\right\},\\
\frac{\mathrm{d}}{\mathrm{d}t}\left\{\mathbf{A}[\mathbf{x}_i(t),t]\right\} &= \dot{\mathbf{A}}[\mathbf{x}_i(t),t] + \left(\dot{\mathbf{x}}_i(t)\cdot\nabla_{\mathbf{x}_i(t)}\right)\mathbf{A}[\mathbf{x}_i(t),t],
\end{split}
\end{equation}
Eq. \eqref{eq:ELcoord1} simplifies to
\begin{equation}
\begin{split}
m_i\ddot{\mathbf{x}}_i(t) &+ \frac{q_i}{c}\left(\dot{\mathbf{A}}[\mathbf{x}_i(t),t] + \dot{\mathbf{A}}_0[\mathbf{x}_i(t),t]\right) = -\frac{q_i}{c}\left([\dot{\mathbf{x}}_i(t)\cdot\nabla_{\mathbf{x}_i(t)}]\left\{\mathbf{A}[\mathbf{x}_i(t),t] + \mathbf{A}_0[\mathbf{x}_i(t),t]\right\}\right)\\
&+ \frac{q_i}{c}\nabla_{\mathbf{x}_i(t)}\left\{\dot{\mathbf{x}}_i(t)\cdot\left(\mathbf{A}[\mathbf{x}_i(t),t] + \mathbf{A}_0[\mathbf{x}_i(t),t]\right)\right\} - q_i\nabla_{\mathbf{x}_i(t)}\left\{\Phi[\mathbf{x}_i(t),t] + \Phi_0[\mathbf{x}_i(t),t]\right\}\\
&= \frac{q_i}{c}\dot{\mathbf{x}}_i(t)\times\left(\nabla_{\mathbf{x}_i(t)}\times\left\{\mathbf{A}[\mathbf{x}_i(t),t] + \mathbf{A}_0[\mathbf{x}_i(t),t]\right\}\right) - q_i\nabla_{\mathbf{x}_i(t)}\left\{\Phi[\mathbf{x}_i(t),t] + \Phi_0[\mathbf{x}_i(t),t]\right\}
\end{split}
\end{equation}
such that, after rearranging to construct the electric and magnetic fields
\begin{equation}
\begin{split}
-\nabla_{\mathbf{x}_i(t)}\left\{\Phi[\mathbf{x}_i(t),t] + \Phi_0[\mathbf{x}_i(t),t]\right\} - \frac{1}{c}\left(\dot{\mathbf{A}}[\mathbf{x}_i(t),t] + \dot{\mathbf{A}}_0[\mathbf{x}_i(t),t]\right) &= \mathbf{E}[\mathbf{x}_i(t),t] + \mathbf{E}_0[\mathbf{x}_i(t),t],\\
\nabla_{\mathbf{x}_i(t)}\times\left\{\mathbf{A}[\mathbf{x}_i(t),t] + \mathbf{A}_0[\mathbf{x}_i(t),t]\right\} &= \mathbf{B}[\mathbf{x}_i(t),t] + \mathbf{B}_0[\mathbf{x}_i(t),t],
\end{split}
\end{equation}
one finds the correct Maxwell-Lorentz force for the dynamical particle $i$:
\begin{equation}
m_i\ddot{\mathbf{x}}_i(t) = q_i\left(\mathbf{E}[\mathbf{x}_i(t),t] + \frac{\dot{\mathbf{x}}_i(t)}{c}\times\mathbf{B}[\mathbf{x}_i(t),t]\right).
\end{equation}
In contrast, the Euler-Lagrange equations return the tautology $0 = 0$ when searching for the equation of motion of $\mathbf{x}_{0j}(t)$ such that the motion of the nondynamical particles is beyond the scope of our Lagrangian.

The motion of the nondynamical particles' fields, however, \textit{is} described by the Lagrangian. To see this, we can use the Euler-Lagrange equations for a general field $\mathbf{F}(\mathbf{r},t)$,
\begin{equation}
\frac{\mathrm{d}}{\mathrm{d}t}\left\{\frac{\partial \mathcal{L}}{\partial \dot{F}_\mu(\mathbf{r},t)}\right\} = \frac{\partial \mathcal{L}}{\partial F_\mu(\mathbf{r},t)} - \sum_{\nu = 1}^3\frac{\partial}{\partial r_\nu}\frac{\partial \mathcal{L}}{\partial\!\left(\frac{\partial F_\mu(\mathbf{r},t)}{\partial r_\nu}\right)},
\end{equation}
where $\mu\in\{1,2,3\}$ is a Cartesian index. The Euler-Lagrange equations for the components of $\mathbf{A}_0(\mathbf{r},t)$ give
\begin{equation}
\begin{split}
\sum_\mu&\frac{\mathrm{d}}{\mathrm{d}t}\left(\frac{\partial}{\partial\dot{A}_{0\mu}(\mathbf{r},t)}\left\{\frac{1}{8\pi}\left[\frac{2}{c}\nabla\Phi_0(\mathbf{r},t)\cdot\dot{\mathbf{A}}_0(\mathbf{r},t) + \frac{1}{c^2}\dot{\mathbf{A}}_0^2(\mathbf{r},t)\right]\right\}\right)\hat{\mathbf{e}}_\mu = \sum_{i}\frac{q_i}{c}\dot{\mathbf{x}}_i(t)\delta[\mathbf{r} - \mathbf{x}_i(t)]\\
&+ \sum_j\frac{q_{0j}}{c}\dot{\mathbf{x}}_{0j}(t)\delta[\mathbf{r} - \mathbf{x}_{0j}(t)] - \sum_{\mu,\nu}\frac{\partial}{\partial r_\nu}\frac{\partial}{\partial\!\left(\frac{\partial A_{0\mu}(\mathbf{r},t)}{\partial r_\nu}\right)}\left\{-\frac{1}{8\pi}\left[\nabla\times\mathbf{A}(\mathbf{r},t)\right]^2\right\}\hat{\mathbf{e}}_\mu.
\end{split}
\end{equation}
The equations simplify with the identity
\begin{equation}
\begin{split}
\sum_{\mu,\nu}\frac{\partial}{\partial r_\nu}\frac{\partial}{\partial\!\left(\frac{\partial F_\mu(\mathbf{r},t)}{\partial r_\nu}\right)}\left\{\left[\nabla\times\mathbf{F}(\mathbf{r},t)\right]^2\right\}\hat{\mathbf{e}}_\mu &= -2\nabla\times\nabla\times\mathbf{F}(\mathbf{r},t),
\end{split}
\end{equation}
such that
\begin{equation}\label{eq:ELfield1}
\frac{1}{4\pi c}\nabla\dot{\Phi}_0(\mathbf{r},t) + \frac{1}{4\pi c^2}\ddot{\mathbf{A}}_0(\mathbf{r},t) = \sum_{i}\frac{q_i}{c}\dot{\mathbf{x}}_i(t)\delta[\mathbf{r} - \mathbf{x}_i(t)] + \sum_j\frac{q_{0j}}{c}\dot{\mathbf{x}}_{0j}(t)\delta[\mathbf{r} - \mathbf{x}_{0j}(t)] - \frac{1}{4\pi}\nabla\times\nabla\times\mathbf{A}_0(\mathbf{r},t).
\end{equation}
With the definitions of the bound and free currents
\begin{equation}
\begin{split}
\mathbf{J}_b(\mathbf{r},t) &= \sum_iq_i\dot{\mathbf{x}}_i(t)\delta[\mathbf{r} - \mathbf{x}_i(t)],\\
\mathbf{J}_0(\mathbf{r},t) &= \sum_jq_{0j}\dot{\mathbf{x}}_{0j}(t)\delta[\mathbf{r} - \mathbf{x}_{0j}(t)],
\end{split}
\end{equation}
one can rearrange Eq. \eqref{eq:ELfield1} to find the Amp\`{e}re-Maxwell equation
\begin{equation}
\nabla\times\mathbf{B}_0(\mathbf{r},t) = \frac{4\pi}{c}\left[\mathbf{J}_b(\mathbf{r},t) + \mathbf{J}_0(\mathbf{r},t)\right] + \frac{1}{c}\dot{\mathbf{E}}_0(\mathbf{r},t).
\end{equation}
We will, in what follows, use the approximation $\mathbf{J}_0(\mathbf{r},t)\gg\mathbf{J}_b(\mathbf{r},t)$ in the driving term of the fields of the nondynamical particles. In other words, we will assume that these fields are influenced minimally by the dynamical charges of the system such that
\begin{equation}
\nabla\times\mathbf{B}_0(\mathbf{r},t) \approx \frac{4\pi}{c}\mathbf{J}_0(\mathbf{r},t) + \frac{1}{c}\dot{\mathbf{E}}_0(\mathbf{r},t).
\end{equation}
Further, we will call thee fields of nondynamical particles ``free fields'' from here on to emphasize their independence from the complicated system motion.

The Euler-Lagrange equation for the free scalar potential is simpler than the equations of the free vector potential:
\begin{equation}
\frac{\mathrm{d}}{\mathrm{d}t}\left\{\frac{\partial \mathcal{L}}{\partial \dot{\Phi}_0(\mathbf{r},t)}\right\} = \frac{\partial \mathcal{L}}{\partial \Phi_0(\mathbf{r},t)} - \sum_{\nu = 1}^3\frac{\partial}{\partial r_\nu}\frac{\partial \mathcal{L}}{\partial\!\left(\frac{\partial \Phi_0(\mathbf{r},t)}{\partial r_\nu}\right)}.
\end{equation}
It simplifies to
\begin{equation}
\begin{split}
0 &= -\sum_iq_i\delta[\mathbf{r} - \mathbf{x}_i(t)] - \sum_jq_{0j}\delta[\mathbf{r} - \mathbf{x}_{0j}(t)] - \nabla\cdot\frac{\partial}{\partial\!\left(\nabla\Phi_0(\mathbf{r},t)\right)}\left\{\frac{1}{8\pi}\left([\nabla\Phi_0(\mathbf{r},t)]^2 + \frac{2}{c}\nabla\Phi_0(\mathbf{r},t)\cdot\dot{\mathbf{A}}_0(\mathbf{r},t)\right)\right\}\\
&= -\sum_iq_i\delta[\mathbf{r} - \mathbf{x}_i(t)] - \sum_jq_{0j}\delta[\mathbf{r} - \mathbf{x}_{0j}(t)] - \nabla\cdot\left\{\frac{1}{4\pi}\nabla\Phi_0(\mathbf{r},t) + \frac{1}{4\pi c}\dot{\mathbf{A}}_0(\mathbf{r},t)\right\}
\end{split}
\end{equation}
which, using the identities of the bound and free charge densities
\begin{equation}
\begin{split}
\rho_b(\mathbf{r},t) &= \sum_iq_i\delta[\mathbf{r} - \mathbf{x}_i(t)],\\
\rho_0(\mathbf{r},t) &= \sum_jq_{0j}\delta[\mathbf{r} - \mathbf{x}_{0j}(t)],\\
\end{split}
\end{equation}
becomes Gauss' law,
\begin{equation}
\nabla\cdot\mathbf{E}_0(\mathbf{r},t) = 4\pi\left[\rho_b(\mathbf{r},t) + \rho_0(\mathbf{r},t)\right].
\end{equation}
Under the same approximation as above, the bound charges are assumed to contribute negligibly to the free field's divergence such that
\begin{equation}
\nabla\cdot\mathbf{E}_0(\mathbf{r},t) \approx 4\pi\rho_0(\mathbf{r},t).
\end{equation}

The Euler-Lagrange equations for the system potentials provides through an entirely analogous process the Amp\`{e}re-Maxwell and Gauss laws for the system fields,
\begin{equation}
\begin{split}
\nabla\times\mathbf{B}(\mathbf{r},t) &= \frac{4\pi}{c}\mathbf{J}_b(\mathbf{r},t) + \frac{1}{c}\dot{\mathbf{E}}(\mathbf{r},t),\\
\nabla\cdot\mathbf{E}(\mathbf{r},t) &= 4\pi\rho_b(\mathbf{r},t).
\end{split}
\end{equation}
As opposed to the free fields, we will allow the system fields to be driven both the bound and free (dynamical and nondynamical) sources. The other two Maxwell equations, i.e. Faraday's law and the magnetic nondivergence law, are satisfied identically by our choice of potentials, so we can omit their derivations. 

In systems where the total charge of all of the particles is zero, i.e. where $\sum_iq_i = 0$, it is often convenient to represent the bound current density as a polarization density $\mathbf{P}(\mathbf{r},t)$ rather than through the individual motion of each charge. In general, this can be done through
\begin{equation}
\mathbf{J}_b(\mathbf{r},t) = \dot{\mathbf{P}}(\mathbf{r},t) + c\nabla\times\mathbf{M}(\mathbf{r},t),
\end{equation}
however in nonmagnetic systems the last term containing the magnetization density $\mathbf{M}(\mathbf{r},t)$ is often neglected. To rewrite our Lagrangian in terms of a polarization density, it is simplest to use the (slightly modified) Power-Zienau-Woolley transformation
\begin{equation}
L_{PZW} = L - \frac{\mathrm{d}}{\mathrm{d}t}\left\{\int\mathbf{P}(\mathbf{r},t)\cdot\left[\mathbf{A}(\mathbf{r},t) + \mathbf{A}_0(\mathbf{r},t)\right]\mathrm{d}^3\mathbf{r}\right\}.
\end{equation}
Because adding a total time derivative to a Lagrangian leaves the action invariant and does not modify the Euler-Lagrange equations or equations of motion of the system, we can be confident that both the Maxwell-Lorentz forces and Maxwell equations implied by the Lagrangian are unchanged by the new term. Using the relationships between $\dot{\mathbf{x}}_i(t)$, $\mathbf{J}_b(\mathbf{r},t)$, and $\mathbf{P}(\mathbf{r},t)$, the PZW Lagrangian density simplifies to
\begin{equation}
\begin{split}
\mathcal{L}_{PZW} &= \sum_i\frac{1}{2}m_i\dot{\mathbf{x}}_i^2(t)\delta[\mathbf{r} - \mathbf{x}_i(t)] + \frac{1}{8\pi}\left[\mathbf{E}^2(\mathbf{r},t) - \mathbf{B}^2(\mathbf{r},t)\right] + \frac{1}{8\pi}\left[\mathbf{E}_0^2(\mathbf{r},t) - \mathbf{B}_0^2(\mathbf{r},t)\right]\\
& - \rho_b(\mathbf{r},t)\left[\Phi(\mathbf{r},t) + \Phi_0(\mathbf{r},t)\right] + \frac{\mathbf{J}_0(\mathbf{r},t)}{c}\cdot\mathbf{A}_0(\mathbf{r},t) - \rho_0(\mathbf{r},t)\Phi_0(\mathbf{r},t)\\
& - \frac{1}{c}\mathbf{P}(\mathbf{r},t)\cdot\left[\dot{\mathbf{A}}(\mathbf{r},t) + \dot{\mathbf{A}}_0(\mathbf{r},t)\right]
%+ \sum_j\frac{1}{2}m_{0j}\dot{\mathbf{x}}_{0j}^2(t)\delta[\mathbf{r} - \mathbf{x}_{0j}(t)]
\end{split}
\end{equation}
with the assumption that $\mathbf{M}(\mathbf{r},t) = 0$.

The conjugate momenta associated with $L_{PZW}$ are
\begin{equation}
\begin{split}
\mathbf{p}_i(t) &= \frac{\partial L_{PZW}}{\partial \dot{\mathbf{x}}(t)} = m_i\dot{\mathbf{x}}_i(t),\\
\bm{\pi}(\mathbf{r},t) &= \frac{\partial \mathcal{L}_{PZW}}{\partial \dot{\mathbf{A}}(\mathbf{r},t)} = -\frac{1}{4\pi c}\mathbf{E}(\mathbf{r},t) - \frac{1}{c}\mathbf{P}(\mathbf{r},t),\\
\bm{\pi}_0(\mathbf{r},t) &= \frac{\partial \mathcal{L}_{PZW}}{\partial \dot{\mathbf{A}}_0(\mathbf{r},t)} = -\frac{1}{4\pi c}\mathbf{E}_0(\mathbf{r},t) - \frac{1}{c}\mathbf{P}(\mathbf{r},t),
\end{split}
\end{equation}
such that, using the identities $\dot{\mathbf{A}}(\mathbf{r},t) = -c[\mathbf{E}(\mathbf{r},t) + \nabla\Phi(\mathbf{r},t)]$ and $\dot{\mathbf{A}}_0(\mathbf{r},t) = -c[\mathbf{E}_0(\mathbf{r},t) + \nabla\Phi_0(\mathbf{r},t)]$, the associated Hamiltonian is
\begin{equation}
\begin{split}
H_{PZW} &= \sum_i\mathbf{p}_i(t)\cdot\dot{\mathbf{x}}_i(t) + \int\bm{\pi}(\mathbf{r},t)\cdot\dot{\mathbf{A}}(\mathbf{r},t)\;\mathrm{d}^3\mathbf{r} + \int\bm{\pi}_0(\mathbf{r},t)\cdot\dot{\mathbf{A}}_0(\mathbf{r},t)\;\mathrm{d}^3\mathbf{r} - L_{PZW}\\
&= \sum_i\frac{1}{2}m_i\dot{\mathbf{x}}_i^2(t) + \frac{1}{8\pi}\int\left[\mathbf{E}^2(\mathbf{r},t) + \mathbf{B}^2(\mathbf{r},t)\right]\mathrm{d}^3\mathbf{r} + \frac{1}{8\pi}\int\left[\mathbf{E}_0^2(\mathbf{r},t) + \mathbf{B}_0^2(\mathbf{r},t)\right]\mathrm{d}^3\mathbf{r}\\
&\qquad+\int\mathbf{P}(\mathbf{r},t)\cdot\left[\mathbf{E}(\mathbf{r},t) + \mathbf{E}_0(\mathbf{r},t) + \nabla\Phi(\mathbf{r},t) + \nabla\Phi_0(\mathbf{r},t)\right]\mathrm{d}^3\mathbf{r}\\
&\qquad+ \frac{1}{4\pi}\int\mathbf{E}(\mathbf{r},t)\cdot\nabla\Phi(\mathbf{r},t)\;\mathrm{d}^3\mathbf{r} + \frac{1}{4\pi}\int\mathbf{E}_0(\mathbf{r},t)\cdot\nabla\Phi_0(\mathbf{r},t)\;\mathrm{d}^3\mathbf{r}\\
&\qquad+ \int\rho_b(\mathbf{r},t)\left[\Phi(\mathbf{r},t) + \Phi_0(\mathbf{r},t)\right]\mathrm{d}^3\mathbf{r} + \int\rho_0(\mathbf{r},t)\Phi_0(\mathbf{r},t)\;\mathrm{d}^3\mathbf{r}\\
&\qquad - \frac{1}{c}\int\mathbf{J}_0(\mathbf{r},t)\cdot\mathbf{A}_0(\mathbf{r},t)\;\mathrm{d}^3\mathbf{r} + \frac{1}{c}\int\mathbf{P}(\mathbf{r},t)\cdot\left[\dot{\mathbf{A}}(\mathbf{r},t) + \dot{\mathbf{A}}_0(\mathbf{r},t)\right]\mathrm{d}^3\mathbf{r}.
\end{split}
\end{equation}
One can see through substitution of the potential form of the electric field that the fourth term of the second sum above cancels with the last term, i.e.
\begin{equation}
\int\mathbf{P}(\mathbf{r},t)\cdot\left[\mathbf{E}(\mathbf{r},t) + \mathbf{E}_0(\mathbf{r},t) + \nabla\Phi(\mathbf{r},t) + \nabla\Phi_0(\mathbf{r},t)\right]\mathrm{d}^3\mathbf{r} = -\frac{1}{c}\int\mathbf{P}(\mathbf{r},t)\cdot\left[\dot{\mathbf{A}}(\mathbf{r},t) + \dot{\mathbf{A}}_0(\mathbf{r},t)\right]\mathrm{d}^3\mathbf{r}.
\end{equation}
Further, integration by parts and use of the identities $\nabla\cdot\mathbf{E}(\mathbf{r},t) = 4\pi\rho_b(\mathbf{r},t)$ and $\nabla\cdot\mathbf{E}_0(\mathbf{r},t)\approx4\pi\rho_0(\mathbf{r},t)$ can be used to show that
\begin{equation}
\begin{split}
\frac{1}{4\pi}\int\mathbf{E}(\mathbf{r},t)\cdot\nabla\Phi(\mathbf{r},t)\;\mathrm{d}^3\mathbf{r} &= -\int\rho_b(\mathbf{r},t)\Phi(\mathbf{r},t)\;\mathrm{d}^3\mathbf{r},\\
\frac{1}{4\pi}\int\mathbf{E}_0(\mathbf{r},t)\cdot\nabla\Phi_0(\mathbf{r},t)\;\mathrm{d}^3\mathbf{r} &= -\int\rho_0(\mathbf{r},t)\Phi_0(\mathbf{r},t)\;\mathrm{d}^3\mathbf{r}.
\end{split}
\end{equation}
Finally, with $\mathbf{E}^2(\mathbf{r},t) = \mathbf{E}(\mathbf{r},t)\cdot\mathbf{D}(\mathbf{r},t) - 4\pi\mathbf{E}(\mathbf{r},t)\cdot\mathbf{P}(\mathbf{r},t)$, the Hamiltonian simplifies to
\begin{equation}\label{eq:Hpzw}
\begin{split}
H_{PZW} &= \sum_i\frac{1}{2}m_i\dot{\mathbf{x}}_i^2(t) - \frac{1}{2}\int\mathbf{E}(\mathbf{r},t)\cdot\mathbf{P}(\mathbf{r},t)\;\mathrm{d}^3\mathbf{r}\\
& + \frac{1}{8\pi}\int\left[\mathbf{E}(\mathbf{r},t)\cdot\mathbf{D}(\mathbf{r},t) + \mathbf{B}^2(\mathbf{r},t)\right]\mathrm{d}^3\mathbf{r} + \frac{1}{8\pi}\int\left[\mathbf{E}_0^2(\mathbf{r},t) + \mathbf{B}_0^2(\mathbf{r},t)\right]\mathrm{d}^3\mathbf{r}\\
& + \int\rho_b(\mathbf{r},t)\Phi_0(\mathbf{r},t)\;\mathrm{d}^3\mathbf{r} - \frac{1}{c}\int\mathbf{J}_0(\mathbf{r},t)\cdot\mathbf{A}_0(\mathbf{r},t)\;\mathrm{d}^3\mathbf{r}.
\end{split}
\end{equation}

This form of the Hamiltonian is conceptually convenient because it includes the energy bound in the macroscopic electromagnetic fields, $(1/8\pi)\int[\mathbf{E}(\mathbf{r,t})\cdot\mathbf{D}(\mathbf{r},t) + \mathbf{B}^2(\mathbf{r},t)]\;\mathrm{d}^3\mathbf{r}$, of a system that includes materials. However, as the conjugate momentum of the system vector potential is proportional to the displacement field, $\bm{\pi}(\mathbf{r},t) = -(1/4\pi c)\mathbf{D}(\mathbf{r},t)$, we need to eliminate the extraneous factor of $\mathbf{E}(\mathbf{r},t)$ inside the integral to bring the Hamiltonian into its canonical form.






%%%%%%%%%%%%%%%%%%%%%%%%%%%%%%%%%%%%%%%%%%%%%%%%%%%%%%%%%%%%%%%%%%%%%%%%%%%%
% Green's Function Helmholtz Expansion
%%%%%%%%%%%%%%%%%%%%%%%%%%%%%%%%%%%%%%%%%%%%%%%%%%%%%%%%%%%%%%%%%%%%%%%%%%%%
% %!TEX root = finiteMQED.tex

\section{Helmholtz Expansion of the Dyadic Green's Function}

\subsection{Helmholtz expansion of Dirac-delta dyads}\label{sec:diracDyadExpansion}

Beginning with the ``self-field'' term of the dyadic Green's function, $\hat{\mathbf{r}}\hat{\mathbf{r}}\delta(\mathbf{r} - \mathbf{r}')/k^2$, we can conduct an expansion similar to that of Appendix \ref{sec:truncatedHarmonicExpansion} to produce
\begin{equation}
\begin{split}
\frac{\hat{\mathbf{r}}\hat{\mathbf{r}}}{k^2}\delta(\mathbf{r} - \mathbf{r}') &= \int\frac{\hat{\mathbf{s}}\hat{\mathbf{s}}}{k^2}\delta(\mathbf{s} - \mathbf{r}')\cdot\bm{1}_2\delta(\mathbf{s} - \mathbf{r})\;\mathrm{d}^3\mathbf{s}\\
&= \int\frac{\hat{\mathbf{s}}\hat{\mathbf{s}}}{k^2}\delta(\mathbf{s} - \mathbf{r}')\cdot\left[\frac{1}{2\pi^2}\sum_{\bm{\alpha}}\int_0^\infty\mathbf{X}_{\bm{\alpha}}(\mathbf{s},\kappa)\mathbf{X}_{\bm{\alpha}}(\mathbf{r},\kappa)\kappa^2\;\mathrm{d}\kappa\right.\\
&\qquad\left.+\frac{1}{4\pi}\sum_{\bm{\alpha}}\nabla_{\mathbf{s}}\nabla\left\{f_{\bm{\alpha}}^>(\mathbf{s})f_{\bm{\alpha}}^<(\mathbf{r})\Theta(s - r) + f_{\bm{\alpha}}^<(\mathbf{s})f_{\bm{\alpha}}^>(\mathbf{r})\Theta(r - s)\right\} \vphantom{\int_0^\infty} \right]\mathrm{d}^3\mathbf{s}\\
&= \frac{1}{2\pi^2k^2}\sum_{\bm{\alpha}}\int_0^\infty\mathbf{X}_{\bm{\alpha}}(\mathbf{r},\kappa)\left[\hat{\mathbf{r}}'\cdot\mathbf{X}_{\bm{\alpha}}(\mathbf{r}',\kappa)\right]\hat{\mathbf{r}}'\kappa^2\;\mathrm{d}\kappa\\
&\qquad+\frac{1}{4\pi k^2}\sum_{\bm{\alpha}}\nabla\left\{\frac{\partial f_{\bm{\alpha}}^>(\mathbf{r}')}{\partial r'}f_{\bm{\alpha}}^<(\mathbf{r})\Theta(r' - r) + \frac{\partial f_{\bm{\alpha}}^<(\mathbf{r}')}{\partial r'}f_{\bm{\alpha}}^>(\mathbf{r})\Theta(r - r')\right\}\hat{\mathbf{r}}'.
\end{split}
\end{equation}







\subsection{Finite radial integrals of products of spherical Bessel functions}\label{sec:finiteRadialIntegrals}

Letting $z_\ell(x)$ be a spherical Bessel ($j_\ell(x)$, $y_\ell(x)$) or Hankel ($h_\ell^{(1)}(x)$, $h_\ell^{(2)}(x)$) function, one can see that the Sturm-Liouville equation
\begin{equation}
\frac{\mathrm{d}}{\mathrm{d}r}\left\{r^2\frac{\mathrm{d}z_\ell(kr)}{\mathrm{d}r}\right\} - \ell(\ell + 1)z_\ell(kr) = -k^2r^2z_\ell(kr). 
\end{equation} 
One can then see that a function
\begin{equation}
v_\ell(r) = r^2\left[\frac{\mathrm{d}z_{1\ell}(kr)}{\mathrm{d}r}z_{2\ell}(k'r) - z_{1\ell}(kr)\frac{\mathrm{d}z_{2\ell}(k'r)}{\mathrm{d}r}\right]
\end{equation}
has a derivative
\begin{equation}
\begin{split}
\frac{\mathrm{d}v_\ell(r)}{\mathrm{d}r} &= \frac{\mathrm{d}}{\mathrm{d}r}\left\{r^2\frac{\mathrm{d}z_{1\ell}(kr)}{\mathrm{d}r}\right\}z_{2\ell}(k'r) - z_{1\ell}(kr)\frac{\mathrm{d}}{\mathrm{d}r}\left\{r^2\frac{\mathrm{d}z_{2\ell}(k'r)}{\mathrm{d}r}\right\}\\
&= r^2(k'^2 - k^2)z_{1\ell}(kr)z_{2\ell}(k'r)
\end{split}
\end{equation}
where $z_{1\ell}$ and $z_{2\ell}$ can be any two spherical Bessel or Hankel functions. Therefore,
\begin{equation}
\begin{split}
\int_a^bz_{1\ell}(kr)z_{2\ell}(k'r)r^2 &= \left.\frac{1}{k'^2 - k^2}v_\ell(r)\right|_a^b\\
&= \left.\frac{r^2}{k'^2 - k^2}\left[\frac{\mathrm{d}z_{1\ell}(kr)}{\mathrm{d}r}z_{2\ell}(k'r) - z_{1\ell}(kr)\frac{\mathrm{d}z_{2\ell}(k'r)}{\mathrm{d}r}\right]\right|_a^b.
\end{split}
\end{equation}
We can use the identity 
\begin{equation}
\frac{\mathrm{d}z_\ell(kr)}{\mathrm{d}r} = k\left[z_{\ell-1}(kr) - \frac{\ell + 1}{kr}z_\ell(kr)\right]
\end{equation}
to see that
\begin{equation}
\begin{split}
\frac{\mathrm{d}z_{1\ell}(kr)}{\mathrm{d}r}z_{2\ell}(k'r) - z_{1\ell}(kr)\frac{\mathrm{d}z_{2\ell}(k'r)}{\mathrm{d}r} &= k\left[z_{1,\ell-1}(kr) - \frac{\ell + 1}{kr}z_{1\ell}(kr)\right]z_{2\ell}(k'r)\\
&\qquad- z_{1\ell}(kr)k'\left[z_{2,\ell-1}(k'r) - \frac{\ell + 1}{k'r}z_{2\ell}(k'r)\right]\\[0.5em]
&= kz_{1,\ell-1}(kr)z_{2\ell}(k'r) - k'z_{2,\ell-1}(k'r)z_{1\ell}(kr).
\end{split}
\end{equation}
Therefore, we can simplify our integral to
\begin{equation}
\int_a^bz_{1\ell}(kr)z_{2\ell}(k'r)r^2 = \left.\frac{r^2}{k'^2 - k^2}\left[kz_{1,\ell - 1}(kr)z_{2\ell}(k'r) - k'z_{2,\ell - 1}(k'r)z_{1\ell}(kr)\right]\right|_a^b.
\end{equation}








\subsection{Helmholtz expansion of piecewise-truncated vector spherical harmonics}\label{sec:truncatedHarmonicExpansion}

A useful mathematical exercise that appears in the treatment of quantum optical problems using dyadic Green's functions in spherical coordinates is the expansion of functions of the types
\begin{equation}
\begin{split}
&\mathbf{X}_{\bm{\alpha}}(\mathbf{r},k)\Theta(\pm r\mp r'),\\
&\bm{\mathcal{X}}_{\bm{\alpha}}(\mathbf{r},k)\Theta(\pm r\mp r')
\end{split}
\end{equation}
into separated products of functions of the variables $r$ and $r'$. To do so, we can use the expansion of the dyadic Dirac delta $\bm{1}_2\delta(\mathbf{r} - \mathbf{r}')$ from Appendix \ref{app:helmholtzDelta}. This allows us to say
\begin{equation}\label{eq:truncatedVectorHarmonicExpansion1}
\begin{split}
\mathbf{X}_{\bm{\alpha}}(\mathbf{r},k)\Theta(r - r') &= \int\mathbf{X}_{\bm{\alpha}}(\mathbf{s},k)\Theta(s - r')\cdot\bm{1}_2\delta(\mathbf{s} - \mathbf{r})\;\mathrm{d}^3\mathbf{s}\\
&= \int\mathbf{X}_{\bm{\alpha}}(\mathbf{s},k)\Theta(s - r')\cdot\left[\frac{1}{2\pi^2}\int_0^\infty\sum_{\bm{\beta}}\kappa^2\mathbf{X}_{\bm{\beta}}(\mathbf{s},\kappa)\mathbf{X}_{\bm{\beta}}(\mathbf{r},\kappa)\;\mathrm{d}\kappa\right.\\
&\qquad\left.+ \frac{1}{4\pi}\sum_{\bm{\beta}}\nabla_\mathbf{r}\nabla_\mathbf{s}\left\{ f_{\bm{\beta}}^>(\mathbf{s})f_{\bm{\beta}}^<(\mathbf{r})\Theta(s - r) +  f_{\bm{\beta}}^<(\mathbf{s})f_{\bm{\beta}}^>(\mathbf{r})\Theta(r - s)\right\}\right]\mathrm{d}^3\mathbf{s}.\\
\end{split}
\end{equation}
We can evaluate the projection integrals using the orthogonality identities of Eqs. \eqref{eq:vectorSphericalHarmonicOrthogonalityConditionFinite} and \eqref{eq:angularMixedOrthogonalityNf} as well as the generalizations of the product rule and divergence theorem
\begin{equation}\label{eq:generalizedProductRule}
g(\mathbf{r})\nabla\cdot\left\{\mathbf{F}(\mathbf{r})f(\mathbf{r})\mathbf{c}\right\} = \left[\mathbf{F}(\mathbf{r})\cdot\nabla \left\{f(\mathbf{r})\mathbf{c}\right\} + \nabla\cdot\mathbf{F}(\mathbf{r})f(\mathbf{r})\mathbf{c}\right]g(\mathbf{r})
\end{equation}
and
\begin{equation}\label{eq:generalizedDivergenceTheorem}
\begin{split}
\int_\mathbb{V}\nabla\cdot\left\{\mathbf{F}(\mathbf{r})f(\mathbf{r})\mathbf{c}\right\}\;\mathrm{d}^3\mathbf{r} &= \mathbf{c}\int_\mathbb{V}\nabla\cdot\left\{\mathbf{F}(\mathbf{r})f(\mathbf{r})\right\}\;\mathrm{d}^3\mathbf{r}\\
&= \mathbf{c}\oint_{\partial\mathbb{V}} f(\mathbf{r})\mathbf{F}(\mathbf{r})\cdot\mathrm{d}^2\mathbf{a},
\end{split}
\end{equation} 
respectively. Explicitly, we can see that
\begin{equation}
\int\mathbf{X}_{\bm{\alpha}}(\mathbf{s},k)\cdot\mathbf{X}_{\bm{\beta}}(\mathbf{s},\kappa)\Theta(s - r')\;\mathrm{d}^3\mathbf{s} = 4\pi(1 - \delta_{p1}\delta_{m0})\delta_{\bm{\alpha}\bm{\beta}}R^\ll_{T\ell}(k,\kappa;r',\infty)
\end{equation}
such that the transverse part of the truncated vector harmonics are
\begin{equation}
\begin{split}
\left[\mathbf{X}_{\bm{\alpha}}(\mathbf{r},k)\Theta(r - r')\right]^\perp &= \int\mathbf{X}_{\bm{\alpha}}(\mathbf{s},k)\Theta(s - r')\cdot\frac{1}{2\pi^2}\sum_{\bm{\beta}}\int_0^\infty\kappa^2\mathbf{X}_{\bm{\beta}}(\mathbf{s},\kappa)\mathbf{X}_{\bm{\beta}}(\mathbf{r},\kappa)\;\mathrm{d}\kappa\,\mathrm{d}^3\mathbf{s}\\
&= \sum_{\bm{\beta}}\int_0^\infty \frac{\kappa^2}{2\pi^2}4\pi(1 - \delta_{p1}\delta_{m0})\delta_{\bm{\alpha}\bm{\beta}}R_{T\ell}^\ll(k,\kappa;r',\infty)\mathbf{X}_{\bm{\beta}}(\mathbf{r},\kappa)\;\mathrm{d}\kappa\\
&= \frac{2}{\pi}(1 - \delta_{p1}\delta_{m0})\int_0^\infty \kappa^2R_{T\ell}^\ll(k,\kappa;r',\infty)\mathbf{X}_{\bm{\alpha}}(\mathbf{r},\kappa)\;\mathrm{d}\kappa.
\end{split}
\end{equation}


The longitudinal parts are more difficult to calculate, but we can start by passing through the $\mathbf{r}$-gradient $\nabla_{\mathbf{r}}$ in eq. \eqref{eq:truncatedVectorHarmonicExpansion1} such that
\begin{equation}
\begin{split}
&\left[\mathbf{X}_{\bm{\alpha}}(\mathbf{r},k)\Theta(r - r')\right]^\parallel\\
&\quad = \int\mathbf{X}_{\bm{\alpha}}(\mathbf{s},k)\Theta(s - r')\cdot\frac{1}{4\pi}\sum_{\bm{\beta}}\nabla_{\mathbf{s}}\nabla_{\mathbf{r}}\left\{f_{\bm{\beta}}^>(\mathbf{s}) f_{\bm{\beta}}^<(\mathbf{r})\Theta(s - r) + f_{\bm{\beta}}^<(\mathbf{s})f_{\bm{\beta}}^>(\mathbf{r})\Theta(r - s)\right\}\mathrm{d}^3\mathbf{s}\\
&\quad = \int\mathbf{X}_{\bm{\alpha}}(\mathbf{s},k)\Theta(s - r')\cdot\frac{1}{4\pi}\sum_{\bm{\beta}}\nabla_{\mathbf{s}}\left\{f_{\bm{\beta}}^>(\mathbf{s})\nabla_{\mathbf{r}} f_{\bm{\beta}}^<(\mathbf{r})\Theta(s - r) + f_{\bm{\beta}}^<(\mathbf{s})\nabla_{\mathbf{r}}f^>(\mathbf{r})\Theta(r - s)\right\}\mathrm{d}^3\mathbf{s},
\end{split}
\end{equation}
wherein terms proportional to gradients of the Heaviside functions cancel as shown in Appendix \ref{app:helmholtzDelta}. From here, we can use the generalized product rule of Eq. \eqref{eq:generalizedProductRule} to replace terms that go like $\mathbf{X}_{\bm{\alpha}}(\mathbf{s},k)\cdot\nabla_{\mathbf{s}}\{f_{\bm{\beta}}^i(\mathbf{s})\nabla_{\mathbf{r}}f^j(\mathbf{r})\Theta(\pm r \mp s)\}$ with terms that involve divergences, such that
\begin{equation}
\begin{split}
\int&\mathbf{X}_{\bm{\alpha}}(\mathbf{s},k)\Theta(s - r')\cdot\frac{1}{4\pi}\sum_{\bm{\beta}}\nabla_{\mathbf{s}}\left\{f_{\bm{\beta}}^>(\mathbf{s})\nabla_{\mathbf{r}} f_{\bm{\beta}}^<(\mathbf{r})\Theta(s - r) + f_{\bm{\beta}}^<(\mathbf{s})\nabla_{\mathbf{r}}f_{\bm{\beta}}^>(\mathbf{r})\Theta(r - s)\right\}\mathrm{d}^3\mathbf{s}\\
&= \frac{1}{4\pi}\sum_{\bm{\beta}}\int\Theta(s - r')\nabla_{\mathbf{s}}\cdot\left\{\mathbf{X}_{\bm{\alpha}}(\mathbf{s},k)\left[f_{\bm{\beta}}^>(\mathbf{s})\nabla_{\mathbf{r}}f_{\bm{\beta}}^<(\mathbf{r})\Theta(s - r) + f_{\bm{\beta}}^<(\mathbf{s})\nabla_{\mathbf{r}}f_{\bm{\beta}}^>(\mathbf{r})\Theta(r - s)\right]\right\}\mathrm{d}^3\mathbf{s}\\
&\qquad- \frac{1}{4\pi}\sum_{\bm{\beta}}\int\Theta(s - r')\nabla_{\mathbf{s}}\cdot\mathbf{X}_{\bm{\alpha}}(\mathbf{s},k)\left[f_{\bm{\beta}}^>(\mathbf{s})\nabla_{\mathbf{r}}f_{\bm{\beta}}^<(\mathbf{r})\Theta(s - r) + f_{\bm{\beta}}^<(\mathbf{s})\nabla_{\mathbf{r}}f_{\bm{\beta}}^>(\mathbf{r})\Theta(r - s)\right]\mathrm{d}^3\mathbf{s}.
\end{split}
\end{equation}
Because the transverse vector spherical harmonics $\mathbf{X}_{\bm{\alpha}}(\mathbf{s},k)$ are transverse in $\mathbf{s}$, the second term on the right-hand side above is zero. We can then use the generalized divergence theorem of Eq. \eqref{eq:generalizedDivergenceTheorem} to replace the first term of the right-hand side with an integral across the boundaries of the region set by $\Theta(s - r')$:
\begin{equation}
\begin{split}
\frac{1}{4\pi}&\sum_{\bm{\beta}}\int\Theta(s - r')\nabla_{\mathbf{s}}\cdot\left\{\mathbf{X}_{\bm{\alpha}}(\mathbf{s},k)\left[f_{\bm{\beta}}^>(\mathbf{s})\nabla_{\mathbf{r}}f_{\bm{\beta}}^<(\mathbf{r})\Theta(s - r) + f_{\bm{\beta}}^<(\mathbf{s})\nabla_{\mathbf{r}}f_{\bm{\beta}}^>(\mathbf{r})\Theta(r - s)\right]\right\}\mathrm{d}^3\mathbf{s}\\
&= \frac{1}{4\pi}\sum_{\bm{\beta}}\oint_{\substack{\partial\mathbb{V}_{s\to r'}\\ + \partial\mathbb{V}_{s\to\infty}}}\mathrm{d}^2\mathbf{a}\cdot\mathbf{X}_{\bm{\alpha}}(\mathbf{s},k)\left[f_{\bm{\beta}}^>(\mathbf{s})\nabla_{\mathbf{r}}f_{\bm{\beta}}^<(\mathbf{r})\Theta(s - r) + f_{\bm{\beta}}^<(\mathbf{s})\nabla_{\mathbf{r}}f_{\bm{\beta}}^>(\mathbf{r})\Theta(r - s)\right]
\end{split}
\end{equation}
One of these boundaries ($\partial\mathbb{V}_{s\to\infty}$) is at infinity, at which the integrand above is zero. In more detail, $\Theta(r - s)$ is clearly zero for all finite $r$ and infinite $s$ as $\Theta(r - s) = 0$ for $s > r$. Additionally, it is zero at infinite $r$ due to the $\sim r^{-\ell_{\bm{\beta}} - 2}$ fall-off of $\nabla_{\mathbf{r}}f_{\bm{\beta}}^>(\mathbf{r})$ (see Eq. \eqref{eq:scalarHarmonics} for details). 

Therefore, the only surface integral that contributes to the total is the integral over $\partial\mathbb{V}_{s\to r'}$. Along this surface, $\mathrm{d}^2\mathbf{a} = -\hat{\mathbf{r}}'r'^2\sin\theta\mathrm{d}\theta\,\mathrm{d}\phi$ such that
\begin{equation}
\begin{split}
\left[\mathbf{X}_{\bm{\alpha}}(\mathbf{r},k)\Theta(r - r')\right]^\parallel &= 
-\frac{1}{4\pi}\sum_{\bm{\beta}} \int_0^{2\pi}\int_0^\pi\hat{\mathbf{r}}'\cdot\mathbf{X}_{\bm{\alpha}}(r',\theta,\phi;k) \left[f_{\bm{\beta}}^>(r',\theta,\phi)\nabla f_{\bm{\beta}}^<(\mathbf{r})\Theta(r' - r)\right.\\
&\qquad\left.+ f_{\bm{\beta}}^<(r',\theta,\phi)\nabla f_{\bm{\beta}}^>(\mathbf{r})\Theta(r - r')\right]r'^2\sin\theta\mathrm{d}\theta\,\mathrm{d}\phi.
\end{split}
\end{equation}
We have dopped the subscript $\mathbf{r}$ from the gradient symbol as the meaning of $\nabla$ now clearly implies differentiation in $\mathbf{r}$. Further, we can use the orthogonality condition of Eq. \eqref{eq:angularMixedOrthogonalityNf} to substitute
\begin{equation}
\begin{split}
\int_0^{2\pi}\int_0^\pi\hat{\mathbf{r}}'\cdot\mathbf{X}_{\bm{\alpha}}(r',\theta,\phi;k)f_{\bm{\beta}}^>(r',\theta,\phi)\sin\theta\;\mathrm{d}\theta\,\mathrm{d}\phi &= 4\pi(1 - \delta_{p1}\delta_{m0})\delta_{TE}\delta_{\bm{\alpha}\bm{\beta}}\sqrt{\frac{\ell(\ell + 1)}{2\ell + 1}}\frac{j_\ell(kr')}{kr'}\frac{1}{r'^{\ell + 1}},\\
\int_0^{2\pi}\int_0^\pi\hat{\mathbf{r}}'\cdot\mathbf{X}_{\bm{\alpha}}(r',\theta,\phi;k)f_{\bm{\beta}}^<(r',\theta,\phi)\sin\theta\;\mathrm{d}\theta\,\mathrm{d}\phi &= 4\pi(1 - \delta_{p1}\delta_{m0})\delta_{TE}\delta_{\bm{\alpha}\bm{\beta}}\sqrt{\frac{\ell(\ell + 1)}{2\ell + 1}}\frac{j_\ell(kr')}{kr'}r'^\ell,
\end{split}
\end{equation}
such that
\begin{equation}
\begin{split}
\left[\mathbf{X}_{\bm{\alpha}}(\mathbf{r},k)\Theta(r - r')\right]^\parallel &= -(1 - \delta_{p1}\delta_{m0})\delta_{TE}\sqrt{\frac{\ell(\ell + 1)}{2\ell + 1}}\frac{j_\ell(kr')}{kr'}\\
&\qquad\times\left[r'^{-\ell + 1}\nabla f_{\bm{\alpha}}^<(\mathbf{r})\Theta(r' - r) + r'^{\ell + 2}\nabla f_{\bm{\alpha}}^>(\mathbf{r})\Theta(r - r')\right]\\[0.5em]
&= -(1 - \delta_{p1}\delta_{m0})\delta_{TE}\sqrt{\frac{\ell(\ell + 1)}{2\ell + 1}}\frac{j_\ell(kr')}{kr'}\mathbf{Z}_{\bm{\alpha}}(\mathbf{r};r').
\end{split}
\end{equation}

Here, we have defined the regularized longitudinal spherical vector harmonics 
\begin{equation}\label{eq:regularizedLongitudinalHarmonics}
\begin{split}
\mathbf{Z}_{\bm{\alpha}}(\mathbf{r};s) &= \left[s^{-\ell + 1}\nabla f_{\bm{\alpha}}^<(\mathbf{r})\Theta(s - r) + s^{\ell + 2}\nabla f_{\bm{\alpha}}^>(\mathbf{r})\Theta(r - s)\right]\\
&= \nabla\left\{s^{-\ell + 1} f_{\bm{\alpha}}^<(\mathbf{r})\Theta(s - r) + s^{\ell + 2} f_{\bm{\alpha}}^>(\mathbf{r})\Theta(r - s)\right\}.
\end{split}
\end{equation}
These obey the orthogonality relation
\begin{equation}
\int\mathbf{Z}_{\bm{\alpha}}(\mathbf{r};s)\cdot\mathbf{Z}_{\bm{\beta}}(\mathbf{r};s)\;\mathrm{d}^3\mathbf{r} = 4\pi s^3\delta_{TE}(1 - \delta_{p1}\delta_{m0})\delta_{\bm{\alpha}\bm{\beta}}
\end{equation}
and, due to the null Laplacian of the scalar harmonics, $\nabla\cdot\nabla f_{\bm{\alpha}}^i(\mathbf{r}) = 0$, have the useful divergence
\begin{equation}
\begin{split}
\nabla\cdot\mathbf{Z}_{\bm{\alpha}}(\mathbf{r};s) &= \nabla\cdot\left\{a^{\ell + 2}\nabla f_{\bm{\alpha}}^>(\mathbf{r})\Theta(r - s) + s^{-\ell + 1}\nabla f_{\bm{\alpha}}^<(\mathbf{r})\Theta(s - r)\right\}\\[0.5em]
&= \left[s^{\ell + 2}\nabla f_{\bm{\alpha}}^>(\mathbf{r}) - s^{-\ell + 1}\nabla f_{\bm{\alpha}}^<(\mathbf{r})\right]\cdot\hat{\mathbf{r}}\delta(r - s)\\
&= \nabla\left\{s^{\ell + 2} f_{\bm{\alpha}}^>(\mathbf{r}) - s^{-\ell + 1} f_{\bm{\alpha}}^<(\mathbf{r})\right\}\cdot\hat{\mathbf{r}}\delta(r - s)
\end{split}
\end{equation}

In conclusion, we have
\begin{equation}
\begin{split}
&\mathbf{X}_{\bm{\alpha}}(\mathbf{r},k)\Theta(r - r') = \left[\mathbf{X}_{\bm{\alpha}}(\mathbf{r},k)\Theta(r - r')\right]^\perp + \left[\mathbf{X}_{\bm{\alpha}}(\mathbf{r},k)\Theta(r - r')\right]^\parallel\\
&\quad= \frac{2}{\pi}(1 - \delta_{p1}\delta_{m0})\int_0^\infty\kappa^2R_{T\ell}^\ll(k,\kappa;r',\infty)\mathbf{X}_{\bm{\alpha}}(\mathbf{r},\kappa)\;\mathrm{d}\kappa - (1 - \delta_{p1}\delta_{m0})\delta_{TE}\sqrt{\frac{\ell(\ell + 1)}{2\ell + 1}}\frac{j_\ell(kr')}{kr'}\mathbf{Z}_{\bm{\alpha}}(\mathbf{r};r').
\end{split}
\end{equation}
We can repeat this process for both truncated exterior transverse spherical harmonics and truncations using $\Theta(r' - r)$. The mathematics is cumbersome but straightforward, and results in
\begin{equation}
\begin{split}
\mathbf{X}_{\bm{\alpha}}(\mathbf{r},k)\Theta(r' - r) &= \frac{2}{\pi}(1 - \delta_{p1}\delta_{m0})\int_0^\infty\kappa^2R_{T\ell}^\ll(k,\kappa;0,r')\mathbf{X}_{\bm{\alpha}}(\mathbf{r},\kappa)\;\mathrm{d}\kappa\\
&\qquad + (1 - \delta_{p1}\delta_{m0})\delta_{TE}\sqrt{\frac{\ell(\ell + 1)}{2\ell + 1}}\frac{j_\ell(kr')}{kr'}\mathbf{Z}_{\bm{\alpha}}(\mathbf{r};r'),\\[1.0em]
\bm{\mathcal{X}}_{\bm{\alpha}}(\mathbf{r},k)\Theta(r - r') &= \frac{2}{\pi}(1 - \delta_{p1}\delta_{m0})\int_0^\infty\kappa^2R_{T\ell}^{><}(k,\kappa;r',\infty)\mathbf{X}_{\bm{\alpha}}(\mathbf{r},\kappa)\;\mathrm{d}\kappa\\
&\qquad- (1 - \delta_{p1}\delta_{m0})\delta_{TE}\sqrt{\frac{\ell(\ell + 1)}{2\ell + 1}}\frac{h_\ell(kr')}{kr'}\mathbf{Z}_{\bm{\alpha}}(\mathbf{r};r'),\\[1.0em]
\bm{\mathcal{X}}_{\bm{\alpha}}(\mathbf{r},k)\Theta(r' - r) &= \frac{2}{\pi}(1 - \delta_{p1}\delta_{m0})\int_0^\infty\kappa^2R_{T\ell}^{><}(k,\kappa;0,r')\mathbf{X}_{\bm{\alpha}}(\mathbf{r},\kappa)\;\mathrm{d}\kappa\\
&\qquad+ (1 - \delta_{p1}\delta_{m0})\delta_{TE}\sqrt{\frac{\ell(\ell + 1)}{2\ell + 1}}\frac{h_\ell(kr')}{kr'}\mathbf{Z}_{\bm{\alpha}}(\mathbf{r};r').
\end{split}
\end{equation}







\subsection{Helmholtz expansion of the dyadic Green's function of free space}\label{sec:helmholtzDecompositionFreeSpaceGreenFunction}

In this subsection, we provide the identity given in Eq. \eqref{eq:helmholtzDecompositionFreeSpaceGreenFunction}. Using the results of Appendices \ref{sec:truncatedHarmonicExpansion} and \ref{sec:diracDyadExpansion}, we can see that
\begin{equation}
\begin{split}
\mathbf{G}_0(\mathbf{r},\mathbf{r}';\omega) &= \frac{\mathrm{i}k}{4\pi}\sum_{\bm{\alpha}}\left[\bm{\mathcal{X}}_{\bm{\alpha}}(\mathbf{r},k)\mathbf{X}_{\bm{\alpha}}(\mathbf{r}',k)\Theta(r - r') + \mathbf{X}_{\bm{\alpha}}(\mathbf{r},k) \bm{\mathcal{X}}_{\bm{\alpha}}(\mathbf{r}',k)\Theta(r' - r)\right] - \frac{\hat{\mathbf{r}}\hat{\mathbf{r}}}{k^2}\delta(\mathbf{r} - \mathbf{r}')\\[1.0em]
&= \frac{\mathrm{i}k}{4\pi}\sum_{\bm{\alpha}}\left[\mathbf{X}_{\bm{\alpha}}(\mathbf{r}',k)\frac{2}{\pi}(1 - \delta_{p1}\delta_{m0})\int_0^\infty\kappa^2R_{T\ell}^{><}(k,\kappa;r',\infty)\mathbf{X}_{\bm{\alpha}}(\mathbf{r},\kappa)\;\mathrm{d}\kappa\right.\\
&\qquad- \mathbf{X}_{\bm{\alpha}}(\mathbf{r}',k)(1 - \delta_{p1}\delta_{m0})\delta_{TE}\sqrt{\frac{\ell(\ell + 1)}{2\ell + 1}}\frac{h_\ell(kr')}{kr'}\mathbf{Z}_{\bm{\alpha}}(\mathbf{r};r')\\
&\qquad+ \bm{\mathcal{X}}_{\bm{\alpha}}(\mathbf{r}',k)\frac{2}{\pi}(1 - \delta_{p1}\delta_{m0})\int_0^\infty\kappa^2R_{T\ell}^\ll(k,\kappa;0,r')\mathbf{X}_{\bm{\alpha}}(\mathbf{r},\kappa)\;\mathrm{d}\kappa\\
&\qquad+\left.\bm{\mathcal{X}}_{\bm{\alpha}}(\mathbf{r}',k)(1 - \delta_{p1}\delta_{m0})\delta_{TE}\sqrt{\frac{\ell(\ell + 1)}{2\ell + 1}}\frac{j_\ell(kr')}{kr'}\mathbf{Z}_{\bm{\alpha}}(\mathbf{r};r')\right]\\
&- \frac{1}{2\pi^2k^2}\sum_{\bm{\alpha}}\int_0^\infty\mathbf{X}_{\bm{\alpha}}(\mathbf{r},\kappa)\left[\hat{\mathbf{r}}'\cdot\mathbf{X}_{\bm{\alpha}}(\mathbf{r}',\kappa)\right]\hat{\mathbf{r}}'\kappa^2\;\mathrm{d}\kappa\\
&\qquad - \frac{1}{4\pi k^2}\sum_{\bm{\alpha}}\nabla\left\{\frac{\partial f_{\bm{\alpha}}^>(\mathbf{r}')}{\partial r'}f_{\bm{\alpha}}^<(\mathbf{r})\Theta(r' - r) + \frac{\partial f_{\bm{\alpha}}^<(\mathbf{r}')}{\partial r'}f_{\bm{\alpha}}^>(\mathbf{r})\Theta(r - r')\right\}\hat{\mathbf{r}}'.
\end{split}
\end{equation}
We can use the identity $\sum_{\bm{\alpha}}\delta_{TE}\mathbf{X}_{\bm{\alpha}}(\mathbf{r},k) = \sum_{p\ell m}\mathbf{N}_{p\ell m}(\mathbf{r},k)$ and group terms to rewrite this as
\begin{equation}
\begin{split}
\mathbf{G}_0(&\mathbf{r},\mathbf{r}';\omega) = \frac{\mathrm{i}k}{2\pi^2}\sum_{\bm{\alpha}}\int_0^\infty\mathbf{X}_{\bm{\alpha}}(\mathbf{r},\kappa)\left( \vphantom{\frac{1}{4\pi k^2}} \mathbf{X}_{\bm{\alpha}}(\mathbf{r}',k)R_{T\ell}^{><}(k,\kappa;r',\infty) + \bm{\mathcal{X}}_{\bm{\alpha}}(\mathbf{r}',k)R_{T\ell}^\ll(k,\kappa;0,r')\right.\\
&\left. - \frac{1}{\mathrm{i} k^3}\left[\hat{\mathbf{r}}'\cdot\mathbf{X}_{\bm{\alpha}}(\mathbf{r}',\kappa)\right]\hat{\mathbf{r}}'\right)\kappa^2\mathrm{d}\kappa + \frac{\mathrm{i}k}{4\pi}\sum_{\bm{\alpha}}\left(\delta_{TE}\sqrt{\frac{\ell(\ell + 1)}{2\ell + 1}}\mathbf{Z}_{\bm{\alpha}}(\mathbf{r},r')\left[-\frac{h_\ell(kr')}{kr'}\mathbf{N}_{p\ell m}(\mathbf{r}',k)\right.\right.\\
&\left.\left.+ \frac{j_\ell(kr')}{kr'}\bm{\mathcal{N}}_{p\ell m}(\mathbf{r}',k)\right] -\frac{1}{\mathrm{i} k^3}\nabla\left\{\frac{\partial f_{\bm{\alpha}}^>(\mathbf{r}')}{\partial r'}f_{\bm{\alpha}}^<(\mathbf{r})\Theta(r' - r) + \frac{\partial f_{\bm{\alpha}}^<(\mathbf{r}')}{\partial r'}f_{\bm{\alpha}}^>(\mathbf{r})\Theta(r - r')\right\}\hat{\mathbf{r}}' \vphantom{\sqrt{\frac{\ell(\ell + 1)}{2\ell + 1}}} \right).
\end{split}
\end{equation}
The first sum over $\bm{\alpha}$ (accompanied by a $\kappa$-integral) is a sum over functions that vary in the observer coordinate $\mathbf{r}$ like $\mathbf{X}_{\bm{\alpha}}(\mathbf{r},\kappa)$ while the second sum (notably integral-free) is a sum over functions that vary as $\nabla f_{\bm{\alpha}}^<(\mathbf{r})$ or $\nabla f_{\bm{\alpha}}^>(\mathbf{r})$, depending on the relative magnitudes of $r$ and $r'$. Therefore, it is clear that the first sum is divergence-free in $\mathbf{r}$ while the second is curl free. What is nontrivial to show, however, is that the first sum is also divergence-free in $\mathbf{r}'$ and the second is analogously also curl-free in $\mathbf{r}'$.

Showing that the $\mathbf{r}'$-behavior mirrors the $\mathbf{r}$ is important, as we must reproduce the Helmholtz decomposition
\begin{equation}\label{eq:waveEquationDerivativeHelmholtzDecomp}
\begin{split}
&\nabla\times\nabla\times\mathbf{A}(\mathbf{r},\omega) - \frac{\omega^2}{c^2}\mathbf{A}(\mathbf{r},\omega) = \frac{4\pi}{c}\mathbf{J}(\mathbf{r},\omega)\\[0.5em]
\implies&
\begin{dcases}
\nabla\times\nabla\times\mathbf{A}^\perp(\mathbf{r},\omega) - \frac{\omega^2}{c^2}\mathbf{A}^\perp(\mathbf{r},\omega) = \frac{4\pi}{c}\mathbf{J}^\perp(\mathbf{r},\omega),\\
\nabla\times\nabla\times\mathbf{A}^\parallel(\mathbf{r},\omega) - \frac{\omega^2}{c^2}\mathbf{A}^\parallel(\mathbf{r},\omega) = \frac{4\pi}{c}\mathbf{J}^\parallel(\mathbf{r},\omega)
\end{dcases}
\end{split}
\end{equation}
within the integral formulation of the wave equation. In other words, we require
\begin{equation}
\mathbf{A}(\mathbf{r},\omega) = \int\mathbf{G}_0(\mathbf{r},\mathbf{r}';\omega)\cdot\frac{\mathbf{J}(\mathbf{r}',\omega)}{c}\;\mathrm{d}^3\mathbf{r}'
\end{equation}
to imply that
\begin{equation}\label{eq:waveEquationIntegralHelmholtzDecomp}
\begin{split}
\mathbf{A}^\perp(\mathbf{r},\omega) &= \int\mathbf{G}_0^\perp(\mathbf{r},\mathbf{r}';\omega)\cdot\frac{\mathbf{J}^\perp(\mathbf{r}',\omega)}{c}\;\mathrm{d}^3\mathbf{r}',\\
\mathbf{A}^\parallel(\mathbf{r},\omega) &= \int\mathbf{G}_0^\parallel(\mathbf{r},\mathbf{r}';\omega)\cdot\frac{\mathbf{J}^\parallel(\mathbf{r}',\omega)}{c}\;\mathrm{d}^3\mathbf{r}',
\end{split}
\end{equation}
where $\mathbf{G}_0^{\perp,\parallel}(\mathbf{r},\mathbf{r}';\omega)$ are transverse and longitudinal in $\mathbf{r}'$, respectively, such that the integrals in the source coordinate are not null, and are also transverse and longitudinal in $\mathbf{r}$ such that the transverse (longitudinal) part of the vector potential can only be driven by the transverse (longitudinal) part of the current density.

The following proofs of these required properties are long and arduous but necessary. Beginning with the $\mathbf{r}$-transverse part of the Green's function of free space, we can use the product rule $\nabla\cdot\left\{\mathbf{c}\mathbf{F}(\mathbf{r})g(\mathbf{r})\right\} = \mathbf{c}\left[g(\mathbf{r})\nabla\cdot\mathbf{F}(\mathbf{r}) + \mathbf{F}(\mathbf{r})\cdot\nabla g(\mathbf{r})\right]$ and the fundamental theorem of calculus $(\partial/\partial x)\int_x^a f(y)\;\mathrm{d}y = -f(x)$ to say
\begin{equation}
\begin{split}
\nabla'\cdot&\left\{\mathbf{X}_{\bm{\alpha}}(\mathbf{r},\kappa)\mathbf{X}_{\bm{\alpha}}(\mathbf{r}',k)R_{T\ell}^{><}(k,\kappa;r',\infty)\right\}\\[0.5em]
&= \mathbf{X}_{\bm{\alpha}}(\mathbf{r},\kappa)\left[\nabla'\cdot\mathbf{X}_{\bm{\alpha}}(\mathbf{r}',k)R_{T\ell}^{><}(k,\kappa;r',\infty) + \mathbf{X}_{\bm{\alpha}}(\mathbf{r}',k)\cdot\nabla'R_{M\ell}^{><}(k,\kappa;r',\infty)\right]\\
&= \mathbf{X}_{\bm{\alpha}}(\mathbf{r},\kappa)\left[ \vphantom{\frac{\ell}{2\ell + 1}} 0 - \delta_{TM}\mathbf{M}_{p\ell m}(\mathbf{r}',k)\cdot\hat{\mathbf{r}}'h_\ell(kr')j_\ell(\kappa r')r'^2\right.\\
&\left.\qquad- \delta_{TE}\mathbf{N}_{p\ell m}(\mathbf{r}',k)\cdot\hat{\mathbf{r}}'\left(\frac{\ell + 1}{2\ell + 1}h_{\ell - 1}(kr')j_{\ell - 1}(\kappa r') + \frac{\ell}{2\ell + 1}h_{\ell + 1}(kr')j_{\ell + 1}(\kappa r')\right)r'^2\right]\\
&= -\mathbf{X}_{\bm{\alpha}}(\mathbf{r},\kappa)\delta_{TE}\left(\sqrt{K_{\ell m}}\frac{\ell(\ell + 1)}{2\ell + 1}\left[j_{\ell - 1}(kr') + j_{\ell + 1}(kr')\right]P_{\ell m}(\cos\theta')S_p(m\phi')\right)\\
&\qquad\times\left(\frac{\ell + 1}{2\ell + 1}h_{\ell - 1}(kr')j_{\ell - 1}(\kappa r') + \frac{\ell}{2\ell + 1}h_{\ell + 1}(kr')j_{\ell + 1}(\kappa r')\right)r'^2,
\end{split}
\end{equation}
wherein we have also used the fact that the magnetic-type transverse vector spherical harmonics have no radial component and that the electric-type transverse vector spherical harmonics have a radial component that goes like $j_{\ell}(kr')/kr' = [j_{\ell - 1}(kr') + j_{\ell + 1}(kr')]/(2\ell + 1)$. Similarly,
\begin{equation}
\begin{split}
\nabla'\cdot&\left\{\mathbf{X}_{\bm{\alpha}}(\mathbf{r},\kappa)\bm{\mathcal{X}}_{\bm{\alpha}}(\mathbf{r}',k)R_{T\ell}^\ll(k,\kappa;0,r')\right\}\\
&= \mathbf{X}_{\bm{\alpha}}(\mathbf{r},\kappa)\delta_{TE}\left(\sqrt{K_{\ell m}}\frac{\ell(\ell + 1)}{2\ell + 1}\left[h_{\ell - 1}(kr') + h_{\ell + 1}(kr')\right]P_{\ell m}(\cos\theta')S_p(m\phi')\right)\\
&\qquad\times\left(\frac{\ell + 1}{2\ell + 1}j_{\ell - 1}(kr')j_{\ell - 1}(\kappa r') + \frac{\ell}{2\ell + 1}j_{\ell + 1}(kr')j_{\ell + 1}(\kappa r')\right)r'^2.
\end{split}
\end{equation}
The third derivative identity we will need is the divergence of the term of the (putatively) transverse Green's function that arises from the dyadic Dirac delta. We can succinctly write this as
\begin{equation}
\begin{split}
\nabla'\cdot&\left\{-\frac{1}{\mathrm{i}k^3}\mathbf{X}_{\bm{\alpha}}(\mathbf{r},\kappa)\left[\hat{\mathbf{r}}'\cdot\mathbf{X}_{\bm{\alpha}}(\mathbf{r}',\kappa)\right]\hat{\mathbf{r}}'\right\}\\
&= \delta_{TE}\nabla'\cdot\left\{-\frac{1}{\mathrm{i}k^3}\mathbf{X}_{\bm{\alpha}}(\mathbf{r},\kappa)\left[\hat{\mathbf{r}}'\cdot\mathbf{N}_{p\ell m}(\mathbf{r}',\kappa)\right]\hat{\mathbf{r}}'\right\}\\
&= -\delta_{TE}\frac{1}{\mathrm{i}k^3}\mathbf{X}_{\bm{\alpha}}(\mathbf{r},\kappa)\frac{\partial}{\partial r'}\left\{\sqrt{K_{\ell m}}\frac{j_\ell(\kappa r')}{\kappa r'}P_{\ell m}(\cos\theta')S_p(m\phi')\right\}\\
&= \delta_{TE}\frac{\mathrm{i}}{k^3}\mathbf{X}_{\bm{\alpha}}(\mathbf{r},\kappa)\sqrt{K_{\ell m}}\ell(\ell + 1)P_{\ell m}(\cos\theta')S_p(m\phi')\left[\frac{1}{\kappa r'}\frac{\partial j_\ell(\kappa r')}{\partial r'} - \frac{j_\ell(\kappa r')}{\kappa r'^2}\right]\\
&= \delta_{TE}\mathrm{i}\mathbf{N}_{p\ell m}(\mathbf{r},\kappa)\sqrt{K_{\ell m}}\ell(\ell + 1)P_{\ell m}(\cos\theta')S_p(m\phi')\frac{1}{k^3\kappa r'^2}\left[r'\frac{\partial j_\ell(\kappa r')}{\partial r'} - j_\ell(\kappa r')\right]
\end{split}
\end{equation}
using the identity $\mathbf{X}_{\bm{\alpha}}(\mathbf{r}',\kappa)\cdot\hat{\mathbf{r}}' = \delta_{TE}\mathbf{N}_{p\ell m}(\mathbf{r}',\kappa)$. 

Combining the first two derivatives above and using the identity
\begin{equation}
j_{\ell - 1}(kr')h_{\ell + 1}(kr') - j_{\ell + 1}(kr')h_{\ell - 1}(kr') = -\mathrm{i}\frac{2\ell + 1}{(kr')^3},
\end{equation}
one finds, after distribution and reorganization of the products of spherical Bessel and Hankel functions, 
\begin{equation}
\begin{split}
\nabla'\cdot&\left\{\mathbf{X}_{\bm{\alpha}}(\mathbf{r},\kappa)\mathbf{X}_{\bm{\alpha}}(\mathbf{r}',k)R_{T\ell}^{><}(k,\kappa;r',\infty)\right\} + \nabla'\cdot\left\{\mathbf{X}_{\bm{\alpha}}(\mathbf{r},\kappa)\bm{\mathcal{X}}_{\bm{\alpha}}(\mathbf{r}',k)R_{T\ell}^\ll(k,\kappa;0,r')\right\}\\
&=\delta_{TE}\mathbf{X}_{\bm{\alpha}}(\mathbf{r},\kappa)\sqrt{K_{\ell m}}\frac{\ell(\ell + 1)}{2\ell + 1}P_{\ell m}(\cos\theta')S_p(m\phi')r'^2\\
&\qquad\times\left[j_{\ell - 1}(kr')h_{\ell + 1}(kr')\left(\frac{\ell + 1}{2\ell + 1}j_{\ell - 1}(\kappa r') - \frac{\ell}{2\ell + 1}j_{\ell + 1}(\kappa r')\right)\right.\\
&\qquad-\left.j_{\ell + 1}(kr')h_{\ell - 1}(kr')\left(\frac{\ell + 1}{2\ell + 1}j_{\ell - 1}(\kappa r') - \frac{\ell}{2\ell + 1}j_{\ell + 1}(\kappa r')\right)\right]\\
&= \delta_{TE}\mathbf{X}_{\bm{\alpha}}(\mathbf{r},\kappa)\sqrt{K_{\ell m}}\frac{\ell(\ell + 1)}{2\ell + 1}P_{\ell m}(\cos\theta')S_p(m\phi')r'^2\\
&\qquad\times\left[j_{\ell - 1}(kr')h_{\ell + 1}(kr') - j_{\ell + 1}(kr')h_{\ell - 1}(kr')\right]\frac{1}{\kappa r'}\frac{\partial\{r'j_\ell(\kappa r')\}}{\partial r'}\\
&= -\delta_{TE}\mathrm{i}\mathbf{X}_{\bm{\alpha}}(\mathbf{r},\kappa)\sqrt{K_{\ell m}}\ell(\ell + 1)P_{\ell m}(\cos\theta')S_p(m\phi')\frac{r'^2}{(kr')^3\kappa r'}\left[j_\ell(\kappa r') + r'\frac{\partial\{j_\ell(\kappa r')\}}{\partial r'}\right]
\end{split}
\end{equation}
Therefore, one can see that the sum of the three terms is
\begin{equation}
\begin{split}
\nabla'\cdot&\left\{ \vphantom{\frac{1}{k^3}} \mathbf{X}_{\bm{\alpha}}(\mathbf{r},\kappa)\mathbf{X}_{\bm{\alpha}}(\mathbf{r}',k)R_{T\ell}^{><}(k,\kappa;r',\infty)\right.\\
&\left.+ \mathbf{X}_{\bm{\alpha}}(\mathbf{r},\kappa)\bm{\mathcal{X}}_{\bm{\alpha}}(\mathbf{r}',k)R_{T\ell}^\ll(k,\kappa;0,r') - \frac{1}{\mathrm{i}k^3}\mathbf{X}_{\bm{\alpha}}(\mathbf{r},\kappa)\left[\hat{\mathbf{r}}'\cdot\mathbf{X}_{\bm{\alpha}}(\mathbf{r}',\kappa)\right]\hat{\mathbf{r}}'\right\}\\
&= \delta_{TE}\mathrm{i}\mathbf{X}_{\bm{\alpha}}(\mathbf{r},\kappa)\sqrt{K_{\ell m}}\ell(\ell + 1)P_{\ell m}(\cos\theta')S_p(m\phi')\frac{1}{k^3\kappa r'^2}\left[r'\frac{\partial j_\ell(\kappa r')}{\partial r'} - j_\ell(\kappa r') - j_\ell(\kappa r') - r'\frac{\partial\{j_\ell(\kappa r')\}}{\partial r'}\right]\\
&= -\delta_{TE}\frac{2\mathrm{i}}{k^3r'}\mathbf{X}_{\bm{\alpha}}(\mathbf{r},\kappa)\sqrt{K_{\ell m}}\ell(\ell + 1)\frac{j_\ell(\kappa r')}{\kappa r'}P_{\ell m}(\cos\theta')S_p(m\phi')\\
&= -\delta_{TE}\frac{2\mathrm{i}}{k^3r'}\mathbf{N}_{p\ell m}(\mathbf{r},\kappa)\mathbf{N}_{p\ell m}(\mathbf{r}',\kappa)\cdot\hat{\mathbf{r}}'.
\end{split}
\end{equation}
Further, from Appendices \ref{app:helmholtzDelta} and \ref{sec:diracDyadExpansion}, we can see that
\begin{equation}
\begin{split}
\hat{\mathbf{r}}'\delta(\mathbf{r} - \mathbf{r}') &= \left[\hat{\mathbf{r}}'\hat{\mathbf{r}}'\delta(\mathbf{r} - \mathbf{r}')\right]\cdot\hat{\mathbf{r}}'\\
&= \left(\left[\bm{1}_2\delta(\mathbf{r} - \mathbf{r}')\right]_\perp\cdot\hat{\mathbf{r}}'\hat{\mathbf{r}}'\right)\cdot\hat{\mathbf{r}}'\\
&= \left(\frac{1}{2\pi^2}\int_0^\infty\sum_{\bm{\alpha}}k^2\mathbf{X}_{\bm{\alpha}}(\mathbf{r},k)\mathbf{X}_{\bm{\alpha}}(\mathbf{r}',k)\;\mathrm{d}k\cdot\hat{\mathbf{r}}'\hat{\mathbf{r}}'\right)\cdot\hat{\mathbf{r}}'\\
&= \frac{1}{2\pi^2}\int_0^\infty\sum_{p\ell m}\mathbf{N}_{p\ell m}(\mathbf{r},k)\mathbf{N}_{p\ell m}(\mathbf{r}',k)\cdot\hat{\mathbf{r}}'k^2\;\mathrm{d}k,
\end{split}
\end{equation}
such that we can see that the $\mathbf{r}'$-divergence of the $\mathbf{r}$-transverse part of $\mathbf{G}_0(\mathbf{r},\mathbf{r}';\omega)$ is 
\begin{equation}
\begin{split}
&\nabla'\cdot\frac{\mathrm{i}k}{2\pi^2}\sum_{\bm{\alpha}}\int_0^\infty\mathbf{X}_{\bm{\alpha}}(\mathbf{r},\kappa)\left( \vphantom{\frac{1}{4\pi k^2}} \mathbf{X}_{\bm{\alpha}}(\mathbf{r}',k)R_{T\ell}^{><}(k,\kappa;r',\infty)\right.\\
&\qquad\left.+ \bm{\mathcal{X}}_{\bm{\alpha}}(\mathbf{r}',k)R_{T\ell}^\ll(k,\kappa;0,r') - \frac{1}{\mathrm{i} k^3}\left[\hat{\mathbf{r}}'\cdot\mathbf{X}_{\bm{\alpha}}(\mathbf{r}',\kappa)\right]\hat{\mathbf{r}}'\right)\kappa^2\mathrm{d}\kappa\\
&= -\frac{2\mathrm{i}}{k^3r'}\frac{\mathrm{i}k}{2\pi^2}\sum_{p\ell m}\int_0^\infty\mathbf{N}_{p\ell m}(\mathbf{r},\kappa)\mathbf{N}_{p\ell m}(\mathbf{r}',\kappa)\cdot\hat{\mathbf{r}}'\kappa^2\;\mathrm{d}\kappa\\
&= \frac{2\hat{\mathbf{r}}'}{k^2r'}\delta(\mathbf{r} - \mathbf{r}'),
\end{split}
\end{equation}
such that the $\mathbf{r}$-transverse part of $\mathbf{G}_0(\mathbf{r},\mathbf{r}';\omega)$ is also transverse in $\mathbf{r}'$ for $\mathbf{r}'\neq\mathbf{r}$. We can also see that the $\mathbf{r}'$-divergence of the $\mathbf{r}$-transverse part is irregular at $r' = 0$. 

Both of these cases need to be handled with care, however neither causes insurmountable problems. In the case $\mathbf{r}'\to\mathbf{r}$, the fields are, in general, irregular, and we can safely ignore these points as our solution is meaningless anyway. For $r'\to0$, our expansion into mode functions $\mathbf{X}_{\bm{\alpha}}(\mathbf{r},\kappa)$ and $f_{\bm{\alpha}}^{>,<}(\mathbf{r})$ breaks due to the irregularity of the function $R_{T\ell}^{><}(k,\kappa;r',\infty)$ as $r'$ approaches the origin. However, in this case the Green's function simply becomes
\begin{equation}
\lim_{r'\to0}\mathbf{G}(\mathbf{r},\mathbf{r}';\omega) = \lim_{r'\to0}\left(\frac{ik}{4\pi}\sum_{\bm{\alpha}}\bm{\mathcal{X}}_{\bm{\alpha}}(\mathbf{r},k)\mathbf{X}_{\bm{\alpha}}(\mathbf{r}',k) - \frac{\hat{\mathbf{r}}\hat{\mathbf{r}}}{k^2}\delta(\mathbf{r} - \mathbf{r}')\right),
\end{equation}
such that (if we allow derivatives to pass through the limit operator) $\nabla'\cdot\lim_{r'\to0}\mathbf{G}(\mathbf{r},\mathbf{r}';\omega) = \lim_{r'\to0}\nabla'\cdot\hat{\mathbf{r}}'\hat{\mathbf{r}}\delta(\mathbf{r} - \mathbf{r}')/k^2 = \lim_{r'\to0}\hat{\mathbf{r}}(\partial/\partial r')\{\delta(\mathbf{r} - \mathbf{r}')\}/k^2$ as $\mathbf{X}_{\bm{\alpha}}(\mathbf{r}',k)$ is divergence-free everywhere. We have also let one factor of $\hat{\mathbf{r}}$ become $\hat{\mathbf{r}}'$ due to the equivalence enforced by the Dirac delta $\delta(\mathbf{r} - \mathbf{r}')$. The resulting $r'$-derivative operator acting on $\delta(\mathbf{r} - \mathbf{r}')$ produces more expressions proportional to $\delta(\mathbf{r} - \mathbf{r}')$, such that, even for $r'\to 0$, the $\mathbf{r}'$-divergence of the Green's function is zero except when the observer and source are co-located. Therefore, for $\mathbf{r} \neq \mathbf{r}'$ and $r'>0$, we can see that
\begin{equation}\label{eq:G0transverse}
\begin{split}
\mathbf{G}_0^\perp(\mathbf{r},\mathbf{r}';\omega) &= \frac{\mathrm{i}k}{2\pi^2}\sum_{\bm{\alpha}}\int_0^\infty\mathbf{X}_{\bm{\alpha}}(\mathbf{r},\kappa)\left( \vphantom{\frac{1}{4\pi k^2}} \mathbf{X}_{\bm{\alpha}}(\mathbf{r}',k)R_{T\ell}^{><}(k,\kappa;r',\infty)\right.\\
&\qquad\left.+ \bm{\mathcal{X}}_{\bm{\alpha}}(\mathbf{r}',k)R_{T\ell}^\ll(k,\kappa;0,r') - \frac{1}{\mathrm{i} k^3}\left[\hat{\mathbf{r}}'\cdot\mathbf{X}_{\bm{\alpha}}(\mathbf{r}',\kappa)\right]\hat{\mathbf{r}}'\right)\kappa^2\mathrm{d}\kappa
\end{split}
\end{equation}
is divergence-free in both $\mathbf{r}$ and $\mathbf{r}'$, and the case of $r' = 0$ requires more thought but produces similarly null divergences.

The $\mathbf{r}$-longitudinal part of the free-space Green's function is easier to analyze than $\mathbf{G}_0^\perp(\mathbf{r},\mathbf{r}';\omega)$. We can quickly use the vector calculus identity $\nabla'\times\left\{\nabla f(\mathbf{r})g(\mathbf{r}')\mathbf{F}(\mathbf{r}')\right\} = \nabla f(\mathbf{r})\left[g(\mathbf{r}')\nabla'\times\mathbf{F}(\mathbf{r}') + \nabla'g(\mathbf{r}')\times\mathbf{F}(\mathbf{r}')\right]$ to analyze its three main terms:
\begin{equation}
\begin{split}
\nabla'&\times\left\{\mathbf{Z}_{\bm{\alpha}}(\mathbf{r},r')\left(-\frac{h_\ell(kr')}{kr'}\right)\mathbf{N}_{p\ell m}(\mathbf{r}',k)\right\}\\
&= \left[r'^{-\ell + 1}\nabla f_{\bm{\alpha}}^<(\mathbf{r})\Theta(r' - r) + r'^{\ell + 2}\nabla f_{\bm{\alpha}}^>(\mathbf{r})\Theta(r - r')\right]\left(-\frac{h_\ell(kr')}{kr'}\right)\nabla'\times\mathbf{N}_{p\ell m}(\mathbf{r}',k)\\
&\qquad+ \nabla'\left\{\left[r'^{-\ell + 1}\nabla f_{\bm{\alpha}}^<(\mathbf{r})\Theta(r' - r) + r'^{\ell + 2}\nabla f_{\bm{\alpha}}^>(\mathbf{r})\Theta(r - r')\right]\left(-\frac{h_\ell(kr')}{kr'}\right)\right\}\times\mathbf{N}_{p\ell m}(\mathbf{r}',k)\\
&= -\left[r'^{-\ell + 1}\nabla f_{\bm{\alpha}}^<(\mathbf{r})\Theta(r' - r) + r'^{\ell + 2}\nabla f_{\bm{\alpha}}^>(\mathbf{r})\Theta(r - r')\right]\frac{h_\ell(kr')}{kr'}k\mathbf{M}_{p\ell m}(\mathbf{r}',k)\\
&\qquad - \left[r'^{-\ell + 1}\nabla f_{\bm{\alpha}}^<(\mathbf{r})\Theta(r' - r) + r'^{\ell + 2}\nabla f_{\bm{\alpha}}^>(\mathbf{r})\Theta(r - r')\right]\frac{\partial}{\partial r'}\left\{\frac{h_\ell(kr')}{kr'}\right\}\hat{\mathbf{r}}'\times\mathbf{N}_{p\ell m}(\mathbf{r}',k)\\
&\qquad - \left[(-\ell + 1)r'^{-\ell}\nabla f_{\bm{\alpha}}^<(\mathbf{r})\Theta(r' - r) + (\ell + 2)r'^{\ell + 1}\nabla f_{\bm{\alpha}}^>(\mathbf{r})\Theta(r - r')\right]\frac{h_\ell(kr')}{kr'}\hat{\mathbf{r}}'\times\mathbf{N}_{p\ell m}(\mathbf{r}',k)\\
&= -\left[r'^{-\ell + 1}\nabla f_{\bm{\alpha}}^<(\mathbf{r})\Theta(r' - r) + r'^{\ell + 2}\nabla f_{\bm{\alpha}}^>(\mathbf{r})\Theta(r - r')\right]\frac{h_\ell(kr')}{kr'}k\mathbf{M}_{p\ell m}(\mathbf{r}',k)\\
&\qquad - \left[r'^{-\ell}\nabla f_{\bm{\alpha}}^<(\mathbf{r})\Theta(r' - r) + r'^{\ell + 1}\nabla f_{\bm{\alpha}}^>(\mathbf{r})\Theta(r - r')\right]\\
&\qquad\qquad\times\frac{1}{kr'}\left(-2h_\ell(kr') + \frac{\partial\left\{r'h_\ell(kr')\right\}}{\partial r'}\right)\hat{\mathbf{r}}'\times\mathbf{N}_{p\ell m}(\mathbf{r}',k)\\
&\qquad - \left[(-\ell + 1)r'^{-\ell}\nabla f_{\bm{\alpha}}^<(\mathbf{r})\Theta(r' - r) + (\ell + 2)r'^{\ell + 1}\nabla f_{\bm{\alpha}}^>(\mathbf{r})\Theta(r - r')\right]\\
&\qquad\qquad\times\frac{1}{kr'}\left(\frac{\partial}{\partial r'}\left\{r'h_\ell(kr')\right\} - r'\frac{\partial h_\ell(kr')}{\partial r'}\right)\hat{\mathbf{r}}'\times\mathbf{N}_{p\ell m}(\mathbf{r}',k),
\end{split}
\end{equation}
and
\begin{equation}
\begin{split}
\nabla'&\times\left\{\mathbf{Z}_{\bm{\alpha}}(\mathbf{r},r')\left(\frac{j_\ell(kr')}{kr'}\right)\bm{\mathcal{N}}_{p\ell m}(\mathbf{r}',k)\right\}\\
&= \left[r'^{-\ell + 1}\nabla f_{\bm{\alpha}}^<(\mathbf{r})\Theta(r' - r) + r'^{\ell + 2}\nabla f_{\bm{\alpha}}^>(\mathbf{r})\Theta(r - r')\right]\frac{j_\ell(kr')}{kr'}k\bm{\mathcal{M}}_{p\ell m}(\mathbf{r}',k)\\
&\qquad + \left[r'^{-\ell}\nabla f_{\bm{\alpha}}^<(\mathbf{r})\Theta(r' - r) + r'^{\ell + 1}\nabla f_{\bm{\alpha}}^>(\mathbf{r})\Theta(r - r')\right]\\
&\qquad\qquad\times\frac{1}{kr'}\left(-2j_\ell(kr') + \frac{\partial\left\{r'j_\ell(kr')\right\}}{\partial r'}\right)\hat{\mathbf{r}}'\times\bm{\mathcal{N}}_{p\ell m}(\mathbf{r}',k)\\
&\qquad + \left[(-\ell + 1)r'^{-\ell}\nabla f_{\bm{\alpha}}^<(\mathbf{r})\Theta(r' - r) + (\ell + 2)r'^{\ell + 1}\nabla f_{\bm{\alpha}}^>(\mathbf{r})\Theta(r - r')\right]\\
&\qquad\qquad\times\frac{1}{kr'}\left(\frac{\partial}{\partial r'}\left\{r'j_\ell(kr')\right\} - r'\frac{\partial j_\ell(kr')}{\partial r'}\right)\hat{\mathbf{r}}'\times\bm{\mathcal{N}}_{p\ell m}(\mathbf{r}',k),
\end{split}
\end{equation}
and
\begin{equation}\label{eq:term3}
\begin{split}
&\nabla'\times\left\{\nabla\left\{\frac{\partial f_{\bm{\alpha}}^>(\mathbf{r}')}{\partial r'}f_{\bm{\alpha}}^<(\mathbf{r})\Theta(r' - r) + \frac{\partial f_{\bm{\alpha}}^<(\mathbf{r}')}{\partial r'}f_{\bm{\alpha}}^>(\mathbf{r})\Theta(r - r')\right\}\hat{\mathbf{r}}'\right\}\\
&= -\hat{\mathbf{r}}'\times\nabla\nabla'\left\{\frac{\partial f_{\bm{\alpha}}^>(\mathbf{r}')}{\partial r'}f_{\bm{\alpha}}^<(\mathbf{r})\Theta(r' - r) + \frac{\partial f_{\bm{\alpha}}^<(\mathbf{r}')}{\partial r'}f_{\bm{\alpha}}^>(\mathbf{r})\Theta(r - r')\right\}\\
&= \delta_{TE}\sqrt{(2 - \delta_{m0})\frac{(\ell - m)!}{(\ell + m)!}}\left(-\hat{\bm{\phi}}'\frac{1}{r'}\frac{\partial}{\partial\theta'} + \hat{\bm{\theta}}'\frac{1}{r'\sin\theta'}\frac{\partial}{\partial \phi'}\right)\\
&\quad\times\nabla\left\{(-\ell - 1)\frac{1}{r'^{\ell + 2}}P_{\ell m}(\cos\theta')S_p(m\phi')f_{\bm{\alpha}}^<(\mathbf{r})\Theta(r' - r) + \ell r'^{\ell - 1}P_{\ell m}(\cos\theta')S_p(m\phi')f_{\bm{\alpha}}^>(\mathbf{r})\Theta(r - r')\right\}\\
&= \delta_{TE}\sqrt{(2 - \delta_{m0})\frac{(\ell - m)!}{(\ell + m)!}}\left(-\frac{\partial P_{\ell m}(\cos\theta')}{\partial \theta'}S_p(m\phi')\hat{\bm{\phi}}' + (-1)^{p+1}m\frac{P_{\ell m}(\cos\theta')}{\sin\theta'}S_{p+1}(m\phi')\hat{\bm{\theta}}'\right)\\
&\quad\times\nabla\left\{-\frac{\ell + 1}{r'^{\ell + 3}}f_{\bm{\alpha}}^<(\mathbf{r})\Theta(r' - r) + \ell r'^{\ell - 2}f_{\bm{\alpha}}^>(\mathbf{r})\Theta(r - r')\right\}.
\end{split}
\end{equation}
The first two of these can be combined to give
\begin{equation}
\begin{split}
&\frac{\mathrm{i}k}{4\pi}\delta_{TE}\sqrt{\frac{\ell(\ell + 1)}{2\ell + 1}}\nabla'\times\left\{\mathbf{Z}_{\bm{\alpha}}(\mathbf{r};r')\left(\frac{j_\ell(kr')}{kr'}\bm{\mathcal{N}}_{p\ell m}(\mathbf{r}',k) - \frac{h_\ell(kr')}{kr'}\mathbf{N}_{p\ell m}(\mathbf{r}',k)\right)\right\}\\
=& \frac{\mathrm{i}k}{4\pi}\delta_{TE}\sqrt{\frac{\ell(\ell + 1)}{2\ell + 1}}\left( \vphantom{\frac{1}{kr'}} \left[r'^{-\ell + 1}\nabla f_{\bm{\alpha}}^<(\mathbf{r})\Theta(r' - r) + r'^{\ell + 2}\nabla f_{\bm{\alpha}}^>(\mathbf{r})\Theta(r - r')\right]\right.\\
&\qquad\times\left[-\frac{h_\ell(kr')}{kr'}k\mathbf{M}_{p\ell m}(\mathbf{r}',k) + \frac{j_\ell(kr')}{kr'}k\bm{\mathcal{M}}_{p\ell m}(\mathbf{r}',k)\right]\\
&\;+ \left[r'^{-\ell}\nabla f_{\bm{\alpha}}^<(\mathbf{r})\Theta(r' - r) + r'^{\ell + 1}\nabla f_{\bm{\alpha}}^>(\mathbf{r})\Theta(r - r')\right]\\
&\qquad\times\left[-\frac{1}{kr'}\frac{\partial\{r'h_\ell(kr')\}}{\partial r'}\hat{\mathbf{r}}'\times\mathbf{N}_{p\ell m}(\mathbf{r}',k) + \frac{1}{kr'}\frac{\partial\{r'j_\ell(kr')\}}{\partial r'}\hat{\mathbf{r}}'\times\bm{\mathcal{N}}_{p\ell m}(\mathbf{r}',k)\right]\\
&\;+ \left[(-\ell + 1)r'^{-\ell}\nabla f_{\bm{\alpha}}^<(\mathbf{r})\Theta(r' - r) + (\ell + 2)r'^{\ell + 1}\nabla f_{\bm{\alpha}}^>(\mathbf{r})\Theta(r - r')\right]\\
&\qquad\times \left[-\frac{1}{kr'}\frac{\partial\{r'h_\ell(kr')\}}{\partial r'}\hat{\mathbf{r}}'\times\mathbf{N}_{p\ell m}(\mathbf{r}',k) + \frac{1}{kr'}\frac{\partial\{r'j_\ell(kr')\}}{\partial r'}\hat{\mathbf{r}}'\times\bm{\mathcal{N}}_{p\ell m}(\mathbf{r}',k)\right]\\
&\;+ \left[r'^{-\ell}\nabla f_{\bm{\alpha}}^<(\mathbf{r})\Theta(r' - r) + r'^{\ell + 1}\nabla f_{\bm{\alpha}}^>(\mathbf{r})\Theta(r - r')\right]\\
&\qquad\times \left[\left(\frac{2}{kr'}h_\ell(kr') + \frac{2}{kr'}r'\frac{\partial h_\ell(kr')}{\partial r'}\right)\hat{\mathbf{r}}'\times\mathbf{N}_{p\ell m}(\mathbf{r}',k)\right.\\
&\qquad\qquad\left.- \left(\frac{2}{kr'}j_\ell(kr') + \frac{2}{kr'}r'\frac{\partial j_\ell(kr')}{\partial r'}\right)\hat{\mathbf{r}}'\times\bm{\mathcal{N}}_{p\ell m}(\mathbf{r}',k)\right]\\
&\;+\left[(-\ell - 1)r'^{-\ell}\nabla f_{\bm{\alpha}}^<(\mathbf{r})\Theta(r' - r) + \ell r'^{\ell + 1}\nabla f_{\bm{\alpha}}^>(\mathbf{r})\Theta(r - r')\right]\\
&\qquad\times\left.\left[\frac{1}{kr'}r'\frac{h_\ell(kr')}{\partial r'}\hat{\mathbf{r}}'\times\mathbf{N}_{p\ell m}(\mathbf{r}',k) - \frac{1}{kr'}r'\frac{j_\ell(kr')}{\partial r'}\hat{\mathbf{r}}'\times\bm{\mathcal{N}}_{p\ell m}(\mathbf{r}',k)\right]\right).
\end{split}
\end{equation}
The first term in the sum on the right-hand side is zero as $\mathbf{M}_{p\ell m}(\mathbf{r}',k)$ and $\bm{\mathcal{M}}_{p\ell m}(\mathbf{r}',k)$ vary in $r'$ as $j_\ell(kr')$ and $h_\ell(kr')$, respectively, and have identical $\theta'$ and $\phi'$ dependence. The second and third terms can be seen to zero via a similar cancellation, as the $\hat{\bm{\theta}}'$ and $\hat{\bm{\phi}}'$ terms of $\mathbf{N}_{p\ell m}(\mathbf{r}',k)$ and $\bm{\mathcal{N}}_{p\ell m}(\mathbf{r}',k)$, i.e. the terms that survive the cross-product with $\hat{\mathbf{r}}'$, vary in $r'$ as $(\partial\{r' j_\ell(kr')\}/\partial r')/kr'$ and $(\partial\{r' h_\ell(kr')\}/\partial r')/kr'$, respectively, and have the same angular dependence. The fourth term is zero by the same logic, as can be seen using the identities $\partial\{r'j_\ell(kr')\}\partial r' = j_\ell(kr') + r'\partial j_\ell(kr')/\partial r'$ and $\partial\{r'h_\ell(kr')\}\partial r' = h_\ell(kr') + r'\partial h_\ell(kr')/\partial r'$ to simplify the factors in parentheses. The fifth term is the only nonzero contribution to the total and gives, after expansion of the vector harmonics, use of the identities $\partial j_\ell(kr')/\partial r' = -kj_{\ell + 1}(kr') + \ell k j_\ell(kr')/kr'$ and $\partial h_\ell(kr')/\partial r' = -kh_{\ell + 1}(kr') + \ell k h_\ell(kr')/kr'$, and some cancellation,
\begin{equation}
\begin{split}
&\frac{\mathrm{i}k}{4\pi}\delta_{TE}\sqrt{\frac{\ell(\ell + 1)}{2\ell + 1}}\nabla'\times\left\{\mathbf{Z}_{\bm{\alpha}}(\mathbf{r};r')\left(\frac{j_\ell(kr')}{kr'}\bm{\mathcal{N}}_{p\ell m}(\mathbf{r}',k) - \frac{h_\ell(kr')}{kr'}\mathbf{N}_{p\ell m}(\mathbf{r}',k)\right)\right\}\\
=& -\frac{1}{4\pi k^2r'^3}\delta_{TE}\sqrt{(2 - \delta_{m0})\frac{(\ell - m)!}{(\ell + m)!}}\left[(-\ell - 1)r'^{-\ell}\nabla f_{\bm{\alpha}}^<(\mathbf{r})\Theta(r - r') + \ell r'^{\ell + 1}\nabla f_{\bm{\alpha}}^>(\mathbf{r})\Theta(r' - r)\right]\\
&\qquad\times\left(\frac{\partial P_{\ell m}(\cos\theta')}{\partial \theta'}S_p(m\phi')\hat{\bm{\phi}}' - \frac{(-1)^{p+1}m}{\sin\theta'}P_{\ell m}(\cos\theta')S_{p+1}(m\phi')\hat{\bm{\theta}}'\right).
\end{split}
\end{equation}
We can see that the above sum is the negative of Eq. \eqref{eq:term3} times a factor of $-1/4\pi k^2$, such that
\begin{equation}
\begin{split}
\nabla'&\times\frac{\mathrm{i}k}{4\pi}\sum_{\bm{\alpha}}\left(\delta_{TE}\sqrt{\frac{\ell(\ell + 1)}{2\ell + 1}}\mathbf{Z}_{\bm{\alpha}}(\mathbf{r},r')\left[-\frac{h_\ell(kr')}{kr'}\mathbf{N}_{p\ell m}(\mathbf{r}',k) + \frac{j_\ell(kr')}{kr'}\bm{\mathcal{N}}_{p\ell m}(\mathbf{r}',k)\right]\right.\\
&\left. -\frac{1}{\mathrm{i} k^3}\nabla\left\{\frac{\partial f_{\bm{\alpha}}^>(\mathbf{r}')}{\partial r'}f_{\bm{\alpha}}^<(\mathbf{r})\Theta(r' - r) + \frac{\partial f_{\bm{\alpha}}^<(\mathbf{r}')}{\partial r'}f_{\bm{\alpha}}^>(\mathbf{r})\Theta(r - r')\right\}\hat{\mathbf{r}}' \vphantom{\sqrt{\frac{\ell(\ell + 1)}{2\ell + 1}}} \right) = 0
\end{split}
\end{equation}
and the $\mathbf{r}$-longitudinal part of the Green's function is also $\mathbf{r}'$-longitudinal. Therefore, we can easily define
\begin{equation}\label{eq:G0longitudinal}
\begin{split}
\mathbf{G}_0^\parallel(\mathbf{r},\mathbf{r}';\omega) &= \frac{\mathrm{i}k}{4\pi}\sum_{\bm{\alpha}}\left(\delta_{TE}\sqrt{\frac{\ell(\ell + 1)}{2\ell + 1}}\mathbf{Z}_{\bm{\alpha}}(\mathbf{r},r')\left[-\frac{h_\ell(kr')}{kr'}\mathbf{N}_{p\ell m}(\mathbf{r}',k) + \frac{j_\ell(kr')}{kr'}\bm{\mathcal{N}}_{p\ell m}(\mathbf{r}',k)\right]\right.\\
&\left. -\frac{1}{\mathrm{i} k^3}\nabla\left\{\frac{\partial f_{\bm{\alpha}}^>(\mathbf{r}')}{\partial r'}f_{\bm{\alpha}}^<(\mathbf{r})\Theta(r' - r) + \frac{\partial f_{\bm{\alpha}}^<(\mathbf{r}')}{\partial r'}f_{\bm{\alpha}}^>(\mathbf{r})\Theta(r - r')\right\}\hat{\mathbf{r}}' \vphantom{\sqrt{\frac{\ell(\ell + 1)}{2\ell + 1}}} \right),
\end{split}
\end{equation}
which, alongside the definition of the transverse Green's function in Eq. \eqref{eq:G0transverse}, provides
\begin{equation}
\mathbf{G}_0(\mathbf{r},\mathbf{r}';\omega) = \mathbf{G}_0^\perp(\mathbf{r},\mathbf{r}';\omega) + \mathbf{G}_0^\parallel(\mathbf{r},\mathbf{r}';\omega)
\end{equation}
which satisfies the originally desired conditions of Eqs. \eqref{eq:waveEquationDerivativeHelmholtzDecomp} and \eqref{eq:waveEquationIntegralHelmholtzDecomp}.







%%%%%%%%%%%%%%%%%%%%%%%%%%%%%%%%%%%%%%%%%%%%%%%%%%%%%%%%%%%%%%%%%%%%%%%%%%%%
% Polarization Wave Equation Derivation
%%%%%%%%%%%%%%%%%%%%%%%%%%%%%%%%%%%%%%%%%%%%%%%%%%%%%%%%%%%%%%%%%%%%%%%%%%%%
% %!TEX root = finiteMQED.tex

\section{Derivation of the Wave Equation Satisfied by the Polarization Density}\label{app:polarizationWaveEquation}

Before we can quantize a theory of light-matter interactions in a rigorous way, we need to describe the classical behavior of the matter we are working with. to do so, we begin with Maxwell's equations in the Fourier regime,
\begin{equation}
\begin{split}
\nabla\cdot\mathbf{E}(\mathbf{r},\omega) &= 4\pi\rho(\mathbf{r},\omega),\\
\nabla\cdot\mathbf{D}(\mathbf{r},\omega) &= 4\pi\rho_f(\mathbf{r},\omega),\\
\nabla\cdot\mathbf{B}(\mathbf{r},\omega) &= 0,\\
\nabla\times\mathbf{E}(\mathbf{r},\omega) &= \frac{\mathrm{i}\omega}{c}\mathbf{B}(\mathbf{r},\omega),\\
\nabla\times\mathbf{B}(\mathbf{r},\omega) &= \frac{4\pi}{c}\mathbf{J}(\mathbf{r},\omega) - \frac{\mathrm{i}\omega}{c}\mathbf{E}(\mathbf{r},\omega),\\
\nabla\times\mathbf{H}(\mathbf{r},\omega) &= \frac{4\pi}{c}\mathbf{J}_f(\mathbf{r},\omega) - \frac{\mathrm{i}\omega}{c}\mathbf{D}(\mathbf{r},\omega),
\end{split}
\end{equation}
where
\begin{equation}
\begin{split}
\mathbf{D}(\mathbf{r},\omega) &= \epsilon(\mathbf{r},\omega)\mathbf{E}(\mathbf{r},\omega),\\
\mathbf{B}(\mathbf{r},\omega) &= \mu(\mathbf{r},\omega)\mathbf{H}(\mathbf{r},\omega),
\end{split}
\end{equation}
are the electric and magnetic constitutive relations, respectively, and the charge and current distributions have been separated into bound and free contributions such that $\rho(\mathbf{r},\omega) = \rho_b(\mathbf{r},\omega) + \rho_f(\mathbf{r},\omega)$ and $\mathbf{J}(\mathbf{r},\omega) = \mathbf{J}_b(\mathbf{r},\omega) + \mathbf{J}_f(\mathbf{r},\omega)$. Alternatively, the electric and magnetic constitutive relations can be written as
\begin{equation}
\begin{split}
\mathbf{D}(\mathbf{r},\omega) &= \mathbf{E}(\mathbf{r},\omega) + 4\pi\mathbf{P}(\mathbf{r},\omega),\\
\mathbf{B}(\mathbf{r},\omega) &= \mathbf{H}(\mathbf{r},\omega) + 4\pi\mathbf{M}(\mathbf{r},\omega),
\end{split}
\end{equation}
with $\mathbf{P}(\mathbf{r},\omega)$ the polarization density and $\mathbf{M}(\mathbf{r},\omega)$ the magnetization density of the matter. These lead straightforwardly to the definitions of the bound charge and current:
\begin{equation}
\begin{split}
\rho_b(\mathbf{r},\omega) &= -\nabla\cdot\mathbf{P}(\mathbf{r},\omega),\\
\mathbf{J}_b(\mathbf{r},\omega) &= -\mathrm{i}\omega\mathbf{P}(\mathbf{r},\omega) + c\nabla\times\mathbf{M}(\mathbf{r},\omega).
\end{split}
\end{equation}

The constitutive relations are useful for defining a corollary to Faraday's law for the $\mathbf{D}(\mathbf{r},\omega)$- and $\mathbf{H}(\mathbf{r},\omega)$-fields:
\begin{equation}
\begin{split}
\nabla\times\mathbf{D}(\mathbf{r},\omega) &= \epsilon(\mathbf{r},\omega)\nabla\times\mathbf{E}(\mathbf{r},\omega) + \nabla\epsilon(\mathbf{r},\omega)\times\mathbf{E}(\mathbf{r},\omega)\\
&= \frac{\mathrm{i}\omega}{c}\epsilon(\mathbf{r},\omega)\mu(\mathbf{r},\omega)\mathbf{H}(\mathbf{r},\omega) + \nabla\epsilon(\mathbf{r},\omega)\times\mathbf{E}(\mathbf{r},\omega),
\end{split}
\end{equation}
wherein we have used the identity $\nabla\times\left\{f(\mathbf{r})\mathbf{F}(\mathbf{r})\right\} = f(\mathbf{r})\nabla\times\mathbf{F}(\mathbf{r}) + \nabla f(\mathbf{r})\times\mathbf{F}(\mathbf{r})$. Application of the curl operator to Faraday's law and its corollary produces
\begin{equation}\label{eq:waveEquationsEandD}
\begin{split}
\nabla\times\nabla\times\mathbf{E}(\mathbf{r},\omega) &= \frac{4\pi\mathrm{i}\omega}{c^2}\mu(\mathbf{r},\omega)\mathbf{J}_f(\mathbf{r},\omega) + \epsilon(\mathbf{r},\omega)\mu(\mathbf{r},\omega)\frac{\omega^2}{c^2}\mathbf{E}(\mathbf{r},\omega) + \frac{\mathrm{i}\omega}{c}\nabla\mu(\mathbf{r},\omega)\times\mathbf{H}(\mathbf{r},\omega),\\
\nabla\times\nabla\times\mathbf{D}(\mathbf{r},\omega) &= \frac{4\pi\mathrm{i}\omega}{c^2}\epsilon(\mathbf{r},\omega)\mu(\mathbf{r},\omega)\mathbf{J}_f(\mathbf{r},\omega) + \epsilon(\mathbf{r},\omega)\mu(\mathbf{r},\omega)\frac{\omega^2}{c^2}\mathbf{D}(\mathbf{r},\omega)\\
&\qquad + \frac{\mathrm{i}\omega}{c}\nabla\left\{\epsilon(\mathbf{r},\omega)\mu(\mathbf{r},\omega)\right\}\times\mathbf{H}(\mathbf{r},\omega) + \nabla\times\left\{\nabla\epsilon(\mathbf{r},\omega)\times\mathbf{E}(\mathbf{r},\omega)\right\}.
\end{split}
\end{equation}
The last term on the right-hand side of the first equation and the last two terms on the right-hand side of the second equation are nonzero for any system with space-dependent (i.e. interesting) material properties. These terms also act as sources that depend on the state of the field at any given frequency, such that, on the surface, they appear to make our lives quite complicated. However, in the special case where the dielectric and diamagnetic functions are piece-wise constant in space, these terms disappear. To see this, we can begin with the first line of Eq. \eqref{eq:waveEquationsEandD} and let
\begin{equation}
\mathbf{E}(\mathbf{r},\omega) = \sum_i\Theta(\mathbf{r}\in\mathbb{V}_i)\mathbf{E}_i(\mathbf{r},\omega).
\end{equation}
Here, space is divided into a number of regions $i$ that each contain a set of points $\mathbb{V}_i$. Each volume $\mathbb{V}_i$ borders any other number neighboring of volumes $\mathbb{V}_j$ and possibly infinity. Each volume $\mathbb{V}_i$ shares a set of points along the surface $\partial\mathbb{V}_{ij}$ with its neighbor $\mathbb{V}_j$ along which the dielectric functions, diamagnetic functions, and fields are undefined. Letting $\partial\mathbb{V}_i$ be the total set of surface points surrounding the volume $\mathbb{V}_i$, we can define the Heaviside function
\begin{equation}
\Theta(\mathbf{r}\in\mathbb{V}_i) = 
\begin{cases}
\mathbf{r}\in\mathbb{V}_i\setminus\partial\mathbb{V}_i, & 1,\\
\mathbf{r}\notin\mathbb{V}_i, & 0,\\
\mathbf{r}\in\partial\mathbb{V}_i, & \mathrm{undefined}.
\end{cases}
\end{equation}
Further, 
\begin{equation}
\mathbf{E}_i(\mathbf{r},\omega) = 
\begin{cases}
\mathbf{E}(\mathbf{r},\omega), & \mathbf{r}\in\mathbb{V}_i,\\
0, & \mathrm{otherwise},
\end{cases}
\end{equation}
such that our piece-wise definition of the electric field serves to ``stitch together'' the electric field in each separate region of space by connecting it along the system's boundaries to its neighbors. The dielectric and diamagnetic functions follow as
\begin{equation}
\begin{split}
\epsilon(\mathbf{r},\omega) &= \sum_i\epsilon_i(\omega)\Theta(\mathbf{r}\in\mathbb{V}_i),\\
\mu(\mathbf{r},\omega) &= \sum_i\mu_i(\omega)\Theta(\mathbf{r}\in\mathbb{V}_i),
\end{split}
\end{equation}
but, in contrast with the definition of the electric field, are space-independent within each region.

Using these definitions, we can apply the double-curl operator on the left-hand side of the first line of Eq. \eqref{eq:waveEquationsEandD} to find
\begin{equation}\label{eq:waveEqEseparated}
\begin{split}
&\sum_i\Theta(\mathbf{r}\in\mathbb{V}_i)\nabla\times\nabla\times\mathbf{E}_i(\mathbf{r},\omega) + \sum_i\delta(\mathbf{r}\in\partial\mathbb{V}_i)\hat{\mathbf{n}}_i(\mathbf{r})\times\nabla\times\mathbf{E}_i(\mathbf{r},\omega)\\
&\qquad + \sum_i\nabla\times\left\{\delta(\mathbf{r}\in\mathbb{V}_i)\hat{\mathbf{n}}_i(\mathbf{r})\times\mathbf{E}_i(\mathbf{r},\omega)\right\} = \sum_i\frac{4\pi\mathrm{i}\omega}{c}\mu_i(\omega)\mathbf{J}_{fi}(\mathbf{r},\omega)\Theta(\mathbf{r}\in\mathbb{V}_i)\\
&\qquad + \frac{\omega^2}{c^2}\sum_i\mu_i(\omega)\epsilon_i(\omega)\mathbf{E}_i(\mathbf{r},\omega)\Theta(\mathbf{r}\in\mathbb{V}_i) + \sum_i\frac{\mathrm{i}\omega}{c}\mu_i(\omega)\delta(\mathbf{r}\in\partial\mathbb{V}_i)\hat{\mathbf{n}}_i(\mathbf{r})\times\mathbf{H}_i(\mathbf{r},\omega).
\end{split}
\end{equation}
Here, we have used the identity $\nabla\Theta(\mathbf{r}\in\mathbb{V}_i) = \hat{\mathbf{n}}_i(\mathbf{r})\delta(\mathbf{r}\in\partial\mathbb{V}_i)$ to evaluate the derivatives, where $\hat{\mathbf{n}}_i(\mathbf{r})$ is the outward-facing surface-normal unit vector of the boundary of $\mathbb{V}_i$. Because the fields and material functions are not defined on the boundary exactly, we will take the Dirac deltas to imply limits as one approaches a point on a boundary from inside the associated bounded volume. In this case, we have let $\Theta(\mathbf{r}\in\mathbb{V}_i)\delta(\mathbf{r}\in\partial\mathbb{V}_i) = \delta(\mathbf{r}\in\partial\mathbb{V}_i)$. Moreover, we can see that along each shared boundary $\partial\mathbb{V}_{ij}$,
\begin{equation}
\hat{\mathbf{n}}_i(\mathbf{r}\in\partial\mathbb{V}_{ij})\times\mathbf{E}_i(\mathbf{r}\in\partial\mathbb{V}_{ij},\omega) = -\hat{\mathbf{n}}_j(\mathbf{r}\in\partial\mathbb{V}_{ij})\times\mathbf{E}_j(\mathbf{r}\in\partial\mathbb{V}_{ij},\omega)
\end{equation}
as the surface-parallel components of the electric fields are equal and the surface-normal vectors are equal and opposite. Therefore, we can see that
\begin{equation}
\sum_i\delta(\mathbf{r}\in\mathbb{V}_i)\hat{\mathbf{n}}_i(\mathbf{r})\times\mathbf{E}_i(\mathbf{r},\omega) = 0
\end{equation}
and, accordingly, $\sum_i\nabla\times\left\{\delta(\mathbf{r}\in\mathbb{V}_i)\hat{\mathbf{n}}_i(\mathbf{r})\times\mathbf{E}_i(\mathbf{r},\omega)\right\} = 0$. Moreover, we can see from Faraday's law that
\begin{equation}
\sum_i\Theta(\mathbf{r}\in\mathbb{V}_i)\nabla\times\mathbf{E}_i(\mathbf{r},\omega) + \sum_i\delta(\mathbf{r}\in\partial\mathbb{V}_i)\hat{\mathbf{n}}_i(\mathbf{r})\times\mathbf{E}_i(\mathbf{r},\omega) = \sum_i\frac{\mathrm{i}\omega}{c}\mathbf{B}_i(\mathbf{r},\omega)\Theta(\mathbf{r}\in\mathbb{V}_i).
\end{equation}
Since the second term on the left-hand side is zero, we can equate the remaining quantities point-wise within their respective regions:
\begin{equation}
\nabla\times\mathbf{E}_i(\mathbf{r},\omega) = \frac{\mathrm{i}\omega}{c}\mathbf{B}_i(\mathbf{r},\omega).
\end{equation}
\sloppy Combined with the definition $\mu_i(\omega)\mathbf{H}_i(\mathbf{r},\omega) = \mathbf{B}_i(\mathbf{r},\omega)$, we can use the region-separated form of Faraday's law to see that $\nabla\times\mathbf{E}_i(\mathbf{r},\omega) = \mathrm{i}\omega\mu_i(\omega)\mathbf{H}_i(\mathbf{r},\omega)/c$. We can then cancel the remaining terms on either side of Eq. \eqref{eq:waveEqEseparated} proportional to Dirac deltas. Further, as the rest of the terms feature sums of independent quantities, we can drop the sums and Heaviside functions as we did with Faraday's law, such that the double curl of $\mathbf{E}(\mathbf{r},\omega)$ provides
\begin{equation}\label{eq:waveEqEseparated2}
\nabla\times\nabla\times\mathbf{E}_i(\mathbf{r},\omega) = \frac{4\pi\mathrm{i}\omega}{c}\mu_i(\omega)\mathbf{J}_{fi}(\mathbf{r},\omega) + \frac{\omega^2}{c^2}\mu_i(\omega)\epsilon_i(\omega)\mathbf{E}_i(\mathbf{r},\omega).
\end{equation}
This is simply the wave equation for the electric field within region $i$. Therefore, we have proved that the wave equation for the electric field in Eq. \eqref{eq:waveEquationsEandD} can be decomposed into a series of wave equations in each region that must be solved individually and do \textit{not} contain any anomalous boundary terms proportional to $\nabla\mu(\mathbf{r},\omega)$.

A similar process can be used to find the region-separated form of the wave equation of the $\mathbf{D}$-field. Letting
\begin{equation}
\begin{split}
\mathbf{D}(\mathbf{r},\omega) &= \sum_i\Theta(\mathbf{r}\in\mathbb{V}_i)\epsilon_i(\omega)\mathbf{E}_i(\mathbf{r},\omega),\\
\epsilon(\mathbf{r},\omega)\mu(\mathbf{r},\omega) &= \sum_i\Theta(\mathbf{r}\in\mathbb{V}_i)\epsilon_i(\omega)\mu_i(\omega),
\end{split}
\end{equation}
we arrive at
\begin{equation}
\begin{split}
&\sum_i\Theta(\mathbf{r}\in\mathbb{V}_i)\nabla\times\nabla\times\mathbf{D}_i(\mathbf{r},\omega) + \sum_i\epsilon_i(\omega)\delta(\mathbf{r}\in\partial\mathbb{V}_i)\hat{\mathbf{n}}_i(\mathbf{r})\times\nabla\times\mathbf{E}_i(\mathbf{r},\omega)\\
&\qquad + \nabla\times\left\{\sum_i\epsilon_i(\omega)\delta(\mathbf{r}\in\partial\mathbb{V}_i)\hat{\mathbf{n}}_i(\mathbf{r})\times\mathbf{E}_i(\mathbf{r},\omega)\right\} = \sum_i\Theta(\mathbf{r}\in\mathbb{V}_i)\frac{4\pi\mathrm{i}\omega}{c^2}\epsilon_i(\omega)\mu_i(\omega)\mathbf{J}_{fi}(\mathbf{r},\omega)\\
&\qquad + \sum_i\Theta(\mathbf{r}\in\mathbb{V}_i)\epsilon_i(\omega)\mu_i(\omega)\frac{\omega^2}{c^2}\mathbf{D}_i(\mathbf{r},\omega) + \sum_i\frac{\mathrm{i}\omega}{c}\epsilon_i(\omega)\mu_i(\omega)\delta(\mathbf{r}\in\delta\mathbb{V}_i)\hat{\mathbf{n}}_i(\mathbf{r})\times\mathbf{H}_i(\mathbf{r},\omega)\\
&\qquad + \nabla\times\left\{\sum_i\epsilon_i(\omega)\delta(\mathbf{r}\in\partial\mathbb{V}_i)\hat{\mathbf{n}}_i(\mathbf{r})\times\mathbf{E}_i(\mathbf{r},\omega)\right\},
\end{split}
\end{equation}
wherein $\mathbf{D}_i(\mathbf{r},\omega) = \epsilon_i(\omega)\mathbf{E}_i(\mathbf{r},\omega)$. We can immediately see that the last terms on both the left- and right-hand sides of the equation cancel. The second-to-last terms on either side of the equation can be seen to cancel using Faraday's law as was done in the previous paragraph, such that, after dropping sums and Heaviside functions, we find
\begin{equation}\label{eq:waveEqDseparated2}
\nabla\times\nabla\times\mathbf{D}_i(\mathbf{r},\omega) = \frac{4\pi\mathrm{i}\omega}{c^2}\epsilon_i(\omega)\mu_i(\omega)\mathbf{J}_{fi}(\mathbf{r},\omega) + \epsilon_i(\omega)\mu_i(\omega)\frac{\omega^2}{c^2}\mathbf{D}_i(\mathbf{r},\omega).
\end{equation}

We can now define the polarization field in each region as $\mathbf{P}_i(\mathbf{r},\omega)$ such that
\begin{equation}
\mathbf{P}(\mathbf{r},\omega) = \sum_i\Theta(\mathbf{r}\in\mathbb{V}_i)\mathbf{P}_i(\mathbf{r},\omega).
\end{equation}
This paves the wave for the identity
\begin{equation}
\mathbf{P}_i(\mathbf{r},\omega) = \frac{\mathbf{D}_i(\mathbf{r},\omega) - \mathbf{E}_i(\mathbf{r},\omega)}{4\pi}.
\end{equation}
Therefore, subtracting Eq. \eqref{eq:waveEqEseparated2} from Eq. \eqref{eq:waveEqDseparated2} and dividing by $4\pi$, we find
\begin{equation}
\nabla\times\nabla\times\mathbf{P}_i(\mathbf{r},\omega) - \epsilon_i(\omega)\mu_i(\omega)\frac{\omega^2}{c^2}\mathbf{P}_i(\mathbf{r},\omega) = \frac{\mathrm{i}\omega}{c}\mu_i(\omega)\left[\epsilon_i(\omega) - 1\right]\mathbf{J}_{fi}(\mathbf{r},\omega).
\end{equation}
Therefore, the polarization field in each region obeys the same wave equation as do the electric and $\mathbf{D}$-fields, modulo a space-independent prefactor in front of the free current source term on the right-hand side.











% \section{Miscellaneous Proofs and Derivations}

% \subsection{Analysis of the Divergence of the Matter Displacement Field}\label{app:polarizationDivergence}

% The divergence of the matter displacement field within the sphere is given by
% \begin{equation}
% \begin{split}
% \nabla&\cdot\mathbf{Q}(\mathbf{r},t) = \sum_{\beta = x,y,z}\frac{\partial}{\partial r_\beta}\left\{\hat{\mathbf{e}}_\beta\cdot\Theta(a - r)\sum_i\mathbf{q}_{mi}(t)\Theta[\mathbf{r}\in\mathbb{V}(\mathbf{r}_i)]\right\}\\
% % &= \Theta(a - r)\sum_{\beta i}\hat{\mathbf{e}}_\beta\cdot\mathbf{q}_{mi}(t)\frac{\partial}{\partial r_\beta}\prod_\gamma\Theta\!\left(r_\gamma - \left[r_{i\gamma} - \frac{\Delta^\frac{1}{3}}{2}\right]\right)\Theta\!\left(\left[r_{i\gamma} + \frac{\Delta^\frac{1}{3}}{2}\right] - r_\gamma\right)\\
% % &\qquad + \sum_\beta\frac{\partial}{\partial r_\beta}\left\{\Theta(a - r)\right\}\hat{\mathbf{e}}_\beta\cdot\sum_i\mathbf{q}_{mi}(t)\Theta[\mathbf{r}\in\mathbb{V}(\mathbf{r}_i)]\\
% &= \Theta(a - r)\sum_{\beta i}\hat{\mathbf{e}}_\beta\cdot\mathbf{q}_{mi}(t)\left[\delta(r_\beta - [r_{i\beta} - \Delta^{1/3}/2]) - \delta(r_\beta - [r_{i\beta} + \Delta^{1/3}/2])\right]\\
% &\times\prod_{\gamma\neq\beta}\Theta(r_\gamma - [r_{i\gamma} - \Delta^{1/3}/2])\Theta([r_{i\gamma} + \Delta^{1/3}/2] - r_\gamma) - \hat{\mathbf{r}}\delta(r - a)\cdot\sum_{i = 1}^{N_m}\mathbf{q}_{mi}(t)\Theta[\mathbf{r}\in\mathbb{V}(\mathbf{r}_i)].
% \end{split}
% \end{equation}
% The first term on the right-hand side highlights that unit cells within the sphere's bulk that have relative charge displacements $\mathbf{q}_{mi}(t)$ unequal to those of their neighbors will give rise to coordinate field divergences. To see that, one can analyze two neighboring terms $j$ and $k$ in the sums over $i$ above. These terms will have opposite signs attached to the Dirac deltas $\delta(r_\beta - [r_{j\beta} \mp \Delta^{1/3}/2])$ and $\delta(r_\beta - [r_{k\beta} \pm \Delta^{1/3}/2])$ that are otherwise equal along the boundary shared by the corresponding volumes $\mathbb{V}(\mathbf{r}_j)$ and $\mathbb{V}(\mathbf{r}_k)$. Therefore, if they have identical displacement projections $\hat{\mathbf{e}}_\beta\cdot\mathbf{q}_{mj}(t) = \hat{\mathbf{e}}_\beta\cdot\mathbf{q}_{mk}(t)$, they will cancel, such that the divergence of the matter field is only nonzero where the ``head-to-tail'' cancellation of neighboring displacement vectors is broken.

% This logic seems to suggest that only a uniform matter displacement field will be free of divergences. Since the polarization field inside the sphere is not uniform in general, such a limitation would present real difficulties. However, one can see that smooth variations in $\mathbf{Q}(\mathbf{r},t)$ will not produce divergences and that our model for $\mathbf{Q}(\mathbf{r},t)$ is more flexible. To see this, we can recognize that, as the side-length $\Delta^{1/3}$ of each unit-cell goes to zero, infinitesimal differences between the displacement vectors of each cell proportional to this length will also go to zero. For example, if $\hat{\mathbf{e}}_\beta\cdot\mathbf{q}_{mj}(t) = \hat{\mathbf{e}}_\beta\cdot\mathbf{q}_{mk}(t) + a\Delta^{1/3}$ with $a$ some constant, the cancellation condition described above will still be satisfied as $\Delta\to0$ and $\mathbf{Q}(\mathbf{r},t)$ will remain divergence-free.

% The second term on the right-hand side describes a similar broken-cancellation effect that occurs at the particle's surface. In particular, projections of the displacement vectors $\mathbf{q}_{mi}(t)$ along a given axis represent the infinitesimal flow of charge along that axis. When this projection occurs across the surface, as is represented by $\hat{\mathbf{r}}\cdot\mathbf{q}_{mi}(t)\delta(r - a)$, this flow produces infinitesimally thin layers of unbalanced positive or negative charge along the particle's surface. Therefore, as divergences of the matter displacement field within the sphere's bulk represent regions of nonzero bulk bound charge density, the surface-divergences represent regions of nonzero surface bound charge density.










% \section{Miscellaneous Work In-Progress}

% Generally, it is easiest to construct this picture in a Hamiltonian formalism, especially if we are to quantize it later. We must therefore begin by identifying the momenta and momentum densities associated with our particle coordinates and fields, respectively. Following the usual procedure, we find
% \begin{equation}
% \begin{split}
% \mathbf{p}_{fi}(t) &= \sum_k\frac{\partial L}{\partial\left(\dot{\mathbf{x}}_{fi}(t)\cdot\hat{\mathbf{e}}_k\right)}\hat{\mathbf{e}}_k = \mu_{fi}\dot{\mathbf{x}}_{fi}(t) + \frac{e_i}{c}\mathbf{A}[\mathbf{x}_{fi}(t),t],\\
% \mathbf{P}(\mathbf{r},t) &= \sum_k\frac{\partial\mathcal{L}}{\partial\left(\dot{\mathbf{Q}}(\mathbf{r},t)\cdot\hat{\mathbf{e}}_k\right)}\hat{\mathbf{e}}_k = \eta_m(\mathbf{r})\dot{\mathbf{Q}}(\mathbf{r},t) + \frac{e}{c\Delta}\mathbf{A}(\mathbf{r},t),\\
% \tilde{\mathbf{P}}_\nu(\mathbf{r},t) &= \sum_k\frac{\partial\mathcal{L}}{\partial\left(\dot{\mathbf{Q}}_\nu(\mathbf{r},t)\cdot\hat{\mathbf{e}}_k\right)}\hat{\mathbf{e}}_k = \eta_\nu(\mathbf{r})\dot{\mathbf{Q}}_\nu(\mathbf{r},t) -  v(\nu)\mathbf{Q}(\mathbf{r},t),\\
% \mathbf{\Pi}(\mathbf{r},t) &= \sum_k\frac{\partial\mathcal{L}}{\partial\left(\dot{\mathbf{A}}(\mathbf{r},t)\cdot\hat{\mathbf{e}}_k\right)}\hat{\mathbf{e}}_k = \frac{1}{4\pi c}\left[\frac{1}{c}\dot{\mathbf{A}}(\mathbf{r},t) + \nabla\Phi(\mathbf{r},t)\right],
% \end{split}
% \end{equation}
% such that
% \begin{equation}
% \begin{split}
% H &= \sum_i\mathbf{p}_{fi}(t)\cdot\dot{\mathbf{x}}_{fi}(t) + \int\left[\mathbf{P}(\mathbf{r},t)\cdot\dot{\mathbf{Q}}(\mathbf{r},t) + \int_0^\infty\tilde{\mathbf{P}}_\nu(\mathbf{r},t)\cdot\dot{\mathbf{Q}}_\nu(\mathbf{r},t)\;\mathrm{d}\nu + \bm{\Pi}(\mathbf{r},t)\cdot\dot{\mathbf{A}}(\mathbf{r},t)\right]\mathrm{d}^3\mathbf{r} - L\\
% &= \sum_i\left(\mu_{fi}\dot{\mathbf{x}}_{fi}^2(t) + \frac{e_i}{c}\dot{\mathbf{x}}_{fi}(t)\cdot\mathbf{A}\left[\mathbf{x}_{fi}(t),t\right]\right) + \int\left[\eta_m(\mathbf{r})\dot{\mathbf{Q}}^2(\mathbf{r},t) + \frac{e}{c\Delta}\dot{\mathbf{Q}}(\mathbf{r},t)\cdot\mathbf{A}(\mathbf{r},t)\right]\mathrm{d}^3\mathbf{r}\\
% &+ \int\int_0^\infty\left[\eta_\nu(\mathbf{r})\dot{\mathbf{Q}}^2_\nu(\mathbf{r},t) -  v(\nu)\mathbf{Q}(\mathbf{r},t)\cdot\dot{\mathbf{Q}}_\nu(\mathbf{r},t)\right]\mathrm{d}\nu\,\mathrm{d}^3\mathbf{r} + \frac{1}{4\pi c}\int\left[\frac{1}{c}\dot{\mathbf{A}}^2(\mathbf{r},t) + \nabla\Phi(\mathbf{r},t)\cdot\dot{\mathbf{A}}(\mathbf{r},t)\right]\;\mathrm{d}^3\mathbf{r}\\
% &- \sum_i\frac{1}{2}\mu_{fi}\dot{\mathbf{x}}_{fi}^2(t) + \int\frac{1}{2}\eta_m(\mathbf{r})\left(-\dot{\mathbf{Q}}^2(\mathbf{r},t) + \omega_0^2\mathbf{Q}^2(\mathbf{r},t) + \frac{2}{3}\sigma_0\mathbf{Q}(\mathbf{r},t)\cdot\left[\mathbf{Q}(\mathbf{r},t)\cdot\bm{1}_3\cdot\mathbf{Q}(\mathbf{r},t)\right]\right)\mathrm{d}^3\mathbf{r}\\
% &+ \int\int_0^\infty\left(-\frac{1}{2}\eta_\nu(\mathbf{r})\dot{\mathbf{Q}}_\nu^2(\mathbf{r},t) + \frac{1}{2}\eta_\nu(\mathbf{r})\nu^2\mathbf{Q}_\nu^2(\mathbf{r},t)\right)\mathrm{d}\nu\,\mathrm{d}^3\mathbf{r} + \int\int_0^\infty v(\nu)\mathbf{Q}(\mathbf{r},t)\cdot\dot{\mathbf{Q}}_\nu(\mathbf{r},t)\;\mathrm{d}\nu\,\mathrm{d}^3\mathbf{r}\\
% &+ \int\left(-\frac{e}{\Delta}\nabla\cdot\mathbf{Q}(\mathbf{r},t) + \sum_ie_i\delta[\mathbf{x}_{fi}(t) - \mathbf{r}]\right)\Phi(\mathbf{r},t)\;\mathrm{d}^3\mathbf{r}\\
% &- \int\left(\frac{e}{c\Delta}\dot{\mathbf{Q}}(\mathbf{r},t) + \sum_i\frac{e_i}{c}\dot{\mathbf{x}}_{fi}(t)\delta[\mathbf{x}_{fi}(t) - \mathbf{r}]\right)\cdot\mathbf{A}(\mathbf{r},t)\;\mathrm{d}^3\mathbf{r} + \sum_{\alpha,\beta = m,b,r}\left[u_{\alpha\beta}^\infty(\mathbf{r}) + u_{\beta\alpha}^\infty(\mathbf{r})\right]\\
% &+ \sum_{\alpha = m,b,r,f}u_\alpha^\mathrm{self} - \frac{1}{8\pi}\int\left(\left[\nabla\Phi(\mathbf{r},t)\right]^2 + \frac{1}{c^2}\dot{\mathbf{A}}^2(\mathbf{r},t) - \left[\nabla\times\mathbf{A}(\mathbf{r},t)\right]^2 + \frac{2}{c}\nabla\Phi(\mathbf{r},t)\cdot\dot{\mathbf{A}}(\mathbf{r},t)\right)\mathrm{d}^3\mathbf{r}.
% \end{split}
% \end{equation}
% After eliminating any terms that include integration of Helmholtz-orthogonal fields and cancelling any reverse duplicates, the Hamiltonian simplifies to
% \begin{equation}
% \begin{split}
% H &= \sum_i\frac{1}{2}\mu_{fi}\dot{\mathbf{x}}_{fi}^2(t) + \int\frac{1}{2}\eta_m(\mathbf{r})\left[\dot{\mathbf{Q}}^2(\mathbf{r},t) + \omega_0^2\mathbf{Q}^2(\mathbf{r},t)\right]\mathrm{d}^3\mathbf{r}\\
% &+ \int\frac{1}{3}\eta_m(\mathbf{r})\sigma_0 \mathbf{Q}(\mathbf{r},t)\cdot\left[\mathbf{Q}(\mathbf{r},t)\cdot\bm{1}_3\cdot\mathbf{Q}(\mathbf{r},t)\right]\mathrm{d}^3\mathbf{r} + \int\int_0^\infty\frac{1}{2}\eta_\nu(\mathbf{r})\left[\dot{\mathbf{Q}}_\nu^2(\mathbf{r},t) + \nu^2\mathbf{Q}_\nu^2(\mathbf{r},t)\right]\mathrm{d}\nu\,\mathrm{d}^3\mathbf{r}\\
% &+ \frac{1}{8\pi}\int\left(\frac{1}{c^2}\dot{\mathbf{A}}^2(\mathbf{r},t) + \left[\nabla\times\mathbf{A}(\mathbf{r},t)\right]^2\right)\mathrm{d}^3\mathbf{r} - \frac{1}{8\pi}\int\left[\nabla\Phi(\mathbf{r},t)\right]^2\mathrm{d}^3\mathbf{r} - \int\frac{e}{\Delta}\nabla\cdot\mathbf{Q}(\mathbf{r},t)\Phi(\mathbf{r},t)\;\mathrm{d}^3\mathbf{r}\\
% &+ \sum_ie_i\Phi[\mathbf{x}_{fi}(t),t] + \sum_{\alpha,\beta = m,b,r}\left[u_{\alpha\beta}^\infty(\mathbf{r}) + u_{\beta\alpha}^\infty(\mathbf{r})\right] + \sum_{\alpha = m,b,r,f}u_\alpha^\mathrm{self}.
% \end{split}
% \end{equation}
% Eliminating the velocities and velocity densities in favor of momenta and momentum densities, one finds
% \begin{equation}
% \begin{split}
% H &= \sum_i\frac{1}{2}\mu_{fi}\left(\frac{\mathbf{p}_{fi}(t)}{\mu_{fi}} - \frac{e_i}{\mu_{fi}c}\mathbf{A}\left[\mathbf{x}_{fi}(t),t\right]\right)^2\\
% &+ \int\frac{1}{2}\eta_m(\mathbf{r})\left[\left(\frac{\mathbf{P}(\mathbf{r},t)}{\eta_m(\mathbf{r})} - \frac{e\mathbf{A}(\mathbf{r},t)}{\eta_m(\mathbf{r})c\Delta}\right)^2 + \omega_0^2\mathbf{Q}^2(\mathbf{r},t)\right]\mathrm{d}^3\mathbf{r}\\
% &+ \int\frac{1}{3}\eta_m(\mathbf{r})\sigma_0 \mathbf{Q}(\mathbf{r},t)\cdot\left[\mathbf{Q}(\mathbf{r},t)\cdot\bm{1}_3\cdot\mathbf{Q}(\mathbf{r},t)\right]\mathrm{d}^3\mathbf{r}\\
% &+ \int\int_0^\infty\frac{1}{2}\eta_\nu(\mathbf{r})\left[\left(\frac{\tilde{\mathbf{P}}_\nu(\mathbf{r},t)}{\eta_\nu(\mathbf{r})} + \frac{ v(\nu)}{\eta_\nu(\mathbf{r})}\mathbf{Q}(\mathbf{r},t)\right)^2 + \nu^2\mathbf{Q}_\nu^2(\mathbf{r},t)\right]\mathrm{d}\nu\,\mathrm{d}^3\mathbf{r}\\
% &+ \frac{1}{8\pi}\int\left(\frac{1}{c^2}\left[4\pi c^2\bm{\Pi}(\mathbf{r},t) - c\nabla\Phi(\mathbf{r},t)\right]^2 + \left[\nabla\times\mathbf{A}(\mathbf{r},t)\right]^2\right)\mathrm{d}^3\mathbf{r} - \frac{1}{8\pi}\int\left[\nabla\Phi(\mathbf{r},t)\right]^2\mathrm{d}^3\mathbf{r}\\
% &- \int\frac{e}{\Delta}\nabla\cdot\mathbf{Q}(\mathbf{r},t)\Phi(\mathbf{r},t)\;\mathrm{d}^3\mathbf{r} + \sum_ie_i\Phi[\mathbf{x}_{fi}(t),t] + \sum_{\alpha,\beta = m,b,r}\left[u_{\alpha\beta}^\infty(\mathbf{r}) + u_{\beta\alpha}^\infty(\mathbf{r})\right] + \sum_{\alpha = m,b,r,f}u_\alpha^\mathrm{self}.
% \end{split}
% \end{equation}
% From here, we can simplify by separating the scalar potentials of the free and bound charges. Since $\partial\mathcal{L}/\partial\dot{\Phi}(\mathbf{r},t) = 0$, the scalar potentials are not dynamical variables and do not carry their own degrees of freedom. Therefore, the scalar potentials can be considered to instantaneously track the motion of the free and bound charges such that
% \begin{equation}
% \Phi(\mathbf{r},t) = \Phi_f(\mathbf{r},t) + \Phi_m(\mathbf{r},t)
% \end{equation}
% where $\Phi_\alpha(\mathbf{r},t) = \int\rho_\alpha(\mathbf{r}',t)/|\mathbf{r} - \mathbf{r}'|\;\mathrm{d}^3\mathbf{r}'$. The 
% % first two terms in the last line of $H$ above can then be expanded as
% % \begin{equation}
% % \begin{split}
% % - \int\frac{e}{\Delta}&\nabla\cdot\mathbf{Q}(\mathbf{r},t)\Phi(\mathbf{r},t)\;\mathrm{d}^3\mathbf{r} + \sum_ie_i\Phi[\mathbf{x}_{fi}(t),t] = - \int\frac{e}{\Delta}\nabla\cdot\mathbf{Q}(\mathbf{r},t)\Phi_m(\mathbf{r},t)\;\mathrm{d}^3\mathbf{r}\\
% % &+ \sum_ie_i\Phi_m[\mathbf{x}_{fi}(t),t] - \int\frac{e}{\Delta}\nabla\cdot\mathbf{Q}(\mathbf{r},t)\Phi_f(\mathbf{r},t)\;\mathrm{d}^3\mathbf{r} + \sum_ie_i\Phi_f[\mathbf{x}_{fi}(t),t]
% % \end{split}
% % \end{equation}
% % and the 
% total Hamiltonian can then be expanded as
% \begin{equation}
% H = H_f + H_m + H_r + H_\mathrm{EM} + H_\mathrm{int} + H_\mathrm{drive} + H_\mathrm{bind} + H_\mathrm{self} + H_\mathrm{NL}.
% \end{equation}
% The first term is the Hamiltonian of the free particles,
% \begin{equation}
% H_f = \sum_i\frac{\mathbf{p}_{fi}^2(t)}{2\mu_{fi}} + \sum_ie_i\Phi_f[\mathbf{x}_{fi}(t),t].
% \end{equation}
% The second is the Hamiltonian of the matter resonance,
% \begin{equation}
% \begin{split}
% H_m &= \int\left[\frac{\mathbf{P}^2(\mathbf{r},t)}{2\eta_m(\mathbf{r})} + \frac{1}{2}\eta_m(\mathbf{r})\omega_0^2\mathbf{Q}^2(\mathbf{r},t)\right]\mathrm{d}^3\mathbf{r} + \int\int_0^\infty\frac{ v^2(\nu)}{2\eta_\nu(\mathbf{r})}\mathbf{Q}^2(\mathbf{r},t)\;\mathrm{d}\nu\,\mathrm{d}^3\mathbf{r}\\
% &- \int\frac{e}{\Delta}\nabla\cdot\mathbf{Q}(\mathbf{r},t)\Phi_m(\mathbf{r},t)\;\mathrm{d}^3\mathbf{r},
% \end{split}
% \end{equation}
% the third is the Hamiltonian of the bath,
% \begin{equation}
% H_r = \int\int_0^\infty\left[\frac{\tilde{\mathbf{P}}_\nu^2(\mathbf{r},t)}{2\eta_\nu(\mathbf{r})} + \frac{1}{2}\eta_\nu(\mathbf{r})\nu^2\mathbf{Q}_\nu^2(\mathbf{r},t)\right]\mathrm{d}\nu\,\mathrm{d}^3\mathbf{r},
% \end{equation}
% and the fourth is the Hamiltonian of the transverse electromagnetic field,
% \begin{equation}
% \begin{split}
% H_\mathrm{EM} &= \int\left(2\pi c^2\bm{\Pi}^2(\mathbf{r},t) + \frac{1}{8\pi}\left[\nabla\times\mathbf{A}(\mathbf{r},t)\right]^2\right)\mathrm{d}^3\mathbf{r} + \int\frac{e^2}{2\eta_m(\mathbf{r})c^2\Delta^2}\mathbf{A}^2(\mathbf{r},t)\;\mathrm{d}^3\mathbf{r}\\&+ \sum_i\frac{e_i^2}{2\mu_{fi}c^2}\mathbf{A}^2[\mathbf{x}_{fi}(t),t].
% \end{split}
% \end{equation}
% The fifth is the interaction Hamiltonian that couples the matter, bath, and EM field together,
% \begin{equation}
% \begin{split}
% H_\mathrm{int} &= -\int\frac{e}{\eta_m(\mathbf{r})c\Delta}\mathbf{P}(\mathbf{r},t)\cdot\mathbf{A}(\mathbf{r},t)\;\mathrm{d}^3\mathbf{r} + \int\int_0^\infty\frac{ v(\nu)}{\eta_\nu(\mathbf{r})}\tilde{\mathbf{P}}_\nu(\mathbf{r},t)\cdot\mathbf{Q}(\mathbf{r},t)\;\mathrm{d}\nu\,\mathrm{d}^3\mathbf{r}\\
% & - c\int\bm{\Pi}(\mathbf{r},t)\cdot\nabla\Phi_m(\mathbf{r},t)\;\mathrm{d}^3\mathbf{r},
% \end{split}
% \end{equation}
% and the sixth is the driving Hamiltonian that details the interactions of the free charges with the other resonances of the system,
% \begin{equation}
% \begin{split}
% H_\mathrm{drive} &= -\sum_i\frac{e_i}{\mu_{fi}c}\mathbf{p}_{fi}(t)\cdot\mathbf{A}[\mathbf{x}_{fi}(t),t] - c\int\bm{\Pi}(\mathbf{r},t)\cdot\nabla\Phi_f(\mathbf{r},t)\;\mathrm{d}^3\mathbf{r}\\&- \int\frac{e}{\Delta}\nabla\cdot\mathbf{Q}(\mathbf{r},t)\Phi_f(\mathbf{r},t)\;\mathrm{d}^3\mathbf{r} + \sum_ie_i\Phi_m[\mathbf{x}_{fi}(t),t].
% \end{split}
% \end{equation}
% The seventh and eighth terms are the binding and self-energy Hamiltonians
% \begin{equation}
% \begin{split}
% H_\mathrm{bind} &= \sum_{\alpha,\beta = m,b,r}\left[u_{\alpha\beta}^\infty(\mathbf{r}) + u_{\beta\alpha}^\infty(\mathbf{r})\right],\\
% H_\mathrm{self} &= \sum_{\alpha = m,b,r,f}u_\alpha^\mathrm{self},
% \end{split}
% \end{equation}
% while the ninth is the nonlinear term in the potential energy of the matter resonance,
% \begin{equation}
% H_\mathrm{NL} = \int\frac{1}{3}\eta_m(\mathbf{r})\sigma_0 \mathbf{Q}(\mathbf{r},t)\cdot\left[\mathbf{Q}(\mathbf{r},t)\cdot\bm{1}_3\cdot\mathbf{Q}(\mathbf{r},t)\right]\mathrm{d}^3\mathbf{r}.
% \end{equation}
















%%%%%%%%%%%%%%%%%%%%%%%%%%%%%%%% Bibliography %%%%%%%%%%%%%%%%%%%%%%%%%%%%%%%%%%
\newpage
\bibliography{mqedRefs}
\bibliographystyle{unsrtnat}



%%%%%%%%%%%%%%%%%%%%%%%%%%%%%%%% End document %%%%%%%%%%%%%%%%%%%%%%%%%%%%%%%%%%

\end{document}